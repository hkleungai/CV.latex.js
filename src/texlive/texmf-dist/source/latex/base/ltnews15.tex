% \iffalse meta-comment
%
% Copyright 1993 1994 1995 1996 1997 1998 1999 2000 2001 2002 2003 2004 2005 2006 2007 2008 2009
% The LaTeX3 Project and any individual authors listed elsewhere
% in this file. 
% 
% This file is part of the LaTeX base system.
% -------------------------------------------
% 
% It may be distributed and/or modified under the
% conditions of the LaTeX Project Public License, either version 1.3c
% of this license or (at your option) any later version.
% The latest version of this license is in
%    http://www.latex-project.org/lppl.txt
% and version 1.3c or later is part of all distributions of LaTeX 
% version 2005/12/01 or later.
% 
% This file has the LPPL maintenance status "maintained".
% 
% The list of all files belonging to the LaTeX base distribution is
% given in the file `manifest.txt'. See also `legal.txt' for additional
% information.
% 
% The list of derived (unpacked) files belonging to the distribution 
% and covered by LPPL is defined by the unpacking scripts (with 
% extension .ins) which are part of the distribution.
% 
% \fi
% Filename: ltnews15.tex 
% 
% This is issue 15 of LaTeX News.

\documentclass
%    [lw35fonts]     % uncomment this line to get Palatino
   {ltnews}[2004/02/28]

% \usepackage[T1]{fontenc}

\publicationmonth{December}  
\publicationyear{2003}
\publicationissue{15}

\begin{document}

\maketitle

%\raisefirstsection

\section{Anniversary release}

Yes, it's now 10~years since the first release in this series and, for
Knuthists, this release also contains \textit{Issue 16}\,!

Meanwhile this \textit{Issue~15} describes the major new features in
the current release whilst \textit{Issue~16} looks\newline
a little way into the future of \LaTeX{}.


\section{LPPL -- new version}

Most importantly, there is now a new version, 1.3, of the \LaTeX{}
Project Public Licence.   Many of you will\newline
be thrilled to know that, following the exchange of over 1600 e-mail
messages dissecting various aspects of its philosophy such as `how
many angels can appear in the name of a file before it becomes
non-free', this version\newline
is now officially a DFSG (Debian Free Software Guidelines) approved license.
The discussions start at
\url{http://lists.debian.org/debian-legal/2002/debian-legal-200207/threads.html}
with high traffic throughout August to October~2002 and further
heated discussions starting in April~2003 and concluding
around June at
\url{http://lists.debian.org/debian-legal/2003/debian-legal-200306/msg00206.html}.

The important features of the new version are useful clarifications in
the wording, and revised procedures\newline
for making a change to the Current Maintainer of a package.  Special
thanks to all those people from\newline Debian Legal who worked
constructively with us\newline
on this onerous task, especially but not exclusively\newline
Jeff Licquia and Branden Robinson.


\section{Small updates to varioref}

The English has been corrected in \verb|\reftextbefore|
(an incompatible change).  There are other extensions
such as \verb|\labelformat|, \verb|\Ref|, \verb|\Vref| and \verb|\vpagerefnum|. 
Some Dutch text has also been changed and two\newline
new options added: \package{slovak} and \package{slovene}.


\section{New and more robust commands}

Many of the math mode commands for compound symbols have been made
robust and a new robust command has been added: \verb|\nobreakdashes|.
This last is a low-level command, borrowed from the \package{amsmath}
package, for use only before hyphens or dashes.  It prevents the line
break that is normally allowed\newline
after the following sequence of dashes.


\section{Fixing font sizes}

The new \package{fix-cm} package, by Walter Schmidt, changes the CM font
definition (\texttt{.fd}) files so that similar design sizes are used
in both the \texttt{OT1} and \texttt{T1} encodings.


\section{Font encodings}

A number of options have been added to the \package{textcomp} package,
enabling only available glyphs to be used.
Also, the `NFSS font families' are now divided into five different groups
according to the subset of glyphs each provides from the full
collection of symbols in the TS1 encoding. 
Given sufficient information about a font family 
\package{textcomp} will use this in order to limit the\newline
typesetting to those glyphs that are available.

Use of this mechanism has also enhanced \verb|\oldstylenums|
to use the current font if possible.


\section{Displaying font tables}

With the \package{nfssfont} package you can now
specify the font to display by giving its `NFSS classification',
rather than needing to know its external font file's name.
It is also now possible to generate large collections of font
tables in batch mode by providing a suitable input file.


\section{New input encodings}

The \package{inputenc} package has been extended as follows: 
\package{macce} input encoding
(Apple Central European),
thanks to Radek Tryc and Marcin Wolinski; \package{cp1257}\newline
for Baltic languages; \package{latin10},
thanks to Ionel Ciob\^{i}c\u{a}.\newline
The euro symbol has by
now been added to several encodings:
\package{ansinew}, \package{cp1250}
and \package{cp1252} (which also\newline
has another addition), whilst \package{cp858}
adds it to \package{cp850}.


\section{Unicode input}

Partial, experimental support for text files that use the Unicode
encoding form UTF-8 is now provided by the option \package{utf8} for
the \package{inputenc} package.

The only Unicode text file characters supported by the current version
are those based on the most common inputs for glyphs from the small
collection\newline
of standard \LaTeX{} Latin encodings.


\section{And finally \ldots\ pict2e}

The old, non-functional version of this package has been removed 
 as there is now a fully working version from Hubert G\"a{\ss}lein
 and Rolf Niepraschk.  It is described
 in \textit{The \LaTeX{} Manual}.

\end{document}
