% \iffalse meta-comment
%
% File: listingsutf8.dtx
% Version: 2011/11/10 v1.2
% Info: Allow UTF-8 in listings input
%
% Copyright (C) 2007, 2011 by
%    Heiko Oberdiek <heiko.oberdiek at googlemail.com>
%
% This work may be distributed and/or modified under the
% conditions of the LaTeX Project Public License, either
% version 1.3c of this license or (at your option) any later
% version. This version of this license is in
%    http://www.latex-project.org/lppl/lppl-1-3c.txt
% and the latest version of this license is in
%    http://www.latex-project.org/lppl.txt
% and version 1.3 or later is part of all distributions of
% LaTeX version 2005/12/01 or later.
%
% This work has the LPPL maintenance status "maintained".
%
% This Current Maintainer of this work is Heiko Oberdiek.
%
% This work consists of the main source file listingsutf8.dtx
% and the derived files
%    listingsutf8.sty, listingsutf8.pdf, listingsutf8.ins, listingsutf8.drv,
%    listingsutf8-test1.tex, listingsutf8-test2.tex,
%    listingsutf8-test3.tex, listingsutf8-test4.tex,
%    listingsutf8-test5.tex.
%
% Distribution:
%    CTAN:macros/latex/contrib/oberdiek/listingsutf8.dtx
%    CTAN:macros/latex/contrib/oberdiek/listingsutf8.pdf
%
% Unpacking:
%    (a) If listingsutf8.ins is present:
%           tex listingsutf8.ins
%    (b) Without listingsutf8.ins:
%           tex listingsutf8.dtx
%    (c) If you insist on using LaTeX
%           latex \let\install=y% \iffalse meta-comment
%
% File: listingsutf8.dtx
% Version: 2011/11/10 v1.2
% Info: Allow UTF-8 in listings input
%
% Copyright (C) 2007, 2011 by
%    Heiko Oberdiek <heiko.oberdiek at googlemail.com>
%
% This work may be distributed and/or modified under the
% conditions of the LaTeX Project Public License, either
% version 1.3c of this license or (at your option) any later
% version. This version of this license is in
%    http://www.latex-project.org/lppl/lppl-1-3c.txt
% and the latest version of this license is in
%    http://www.latex-project.org/lppl.txt
% and version 1.3 or later is part of all distributions of
% LaTeX version 2005/12/01 or later.
%
% This work has the LPPL maintenance status "maintained".
%
% This Current Maintainer of this work is Heiko Oberdiek.
%
% This work consists of the main source file listingsutf8.dtx
% and the derived files
%    listingsutf8.sty, listingsutf8.pdf, listingsutf8.ins, listingsutf8.drv,
%    listingsutf8-test1.tex, listingsutf8-test2.tex,
%    listingsutf8-test3.tex, listingsutf8-test4.tex,
%    listingsutf8-test5.tex.
%
% Distribution:
%    CTAN:macros/latex/contrib/oberdiek/listingsutf8.dtx
%    CTAN:macros/latex/contrib/oberdiek/listingsutf8.pdf
%
% Unpacking:
%    (a) If listingsutf8.ins is present:
%           tex listingsutf8.ins
%    (b) Without listingsutf8.ins:
%           tex listingsutf8.dtx
%    (c) If you insist on using LaTeX
%           latex \let\install=y% \iffalse meta-comment
%
% File: listingsutf8.dtx
% Version: 2011/11/10 v1.2
% Info: Allow UTF-8 in listings input
%
% Copyright (C) 2007, 2011 by
%    Heiko Oberdiek <heiko.oberdiek at googlemail.com>
%
% This work may be distributed and/or modified under the
% conditions of the LaTeX Project Public License, either
% version 1.3c of this license or (at your option) any later
% version. This version of this license is in
%    http://www.latex-project.org/lppl/lppl-1-3c.txt
% and the latest version of this license is in
%    http://www.latex-project.org/lppl.txt
% and version 1.3 or later is part of all distributions of
% LaTeX version 2005/12/01 or later.
%
% This work has the LPPL maintenance status "maintained".
%
% This Current Maintainer of this work is Heiko Oberdiek.
%
% This work consists of the main source file listingsutf8.dtx
% and the derived files
%    listingsutf8.sty, listingsutf8.pdf, listingsutf8.ins, listingsutf8.drv,
%    listingsutf8-test1.tex, listingsutf8-test2.tex,
%    listingsutf8-test3.tex, listingsutf8-test4.tex,
%    listingsutf8-test5.tex.
%
% Distribution:
%    CTAN:macros/latex/contrib/oberdiek/listingsutf8.dtx
%    CTAN:macros/latex/contrib/oberdiek/listingsutf8.pdf
%
% Unpacking:
%    (a) If listingsutf8.ins is present:
%           tex listingsutf8.ins
%    (b) Without listingsutf8.ins:
%           tex listingsutf8.dtx
%    (c) If you insist on using LaTeX
%           latex \let\install=y% \iffalse meta-comment
%
% File: listingsutf8.dtx
% Version: 2011/11/10 v1.2
% Info: Allow UTF-8 in listings input
%
% Copyright (C) 2007, 2011 by
%    Heiko Oberdiek <heiko.oberdiek at googlemail.com>
%
% This work may be distributed and/or modified under the
% conditions of the LaTeX Project Public License, either
% version 1.3c of this license or (at your option) any later
% version. This version of this license is in
%    http://www.latex-project.org/lppl/lppl-1-3c.txt
% and the latest version of this license is in
%    http://www.latex-project.org/lppl.txt
% and version 1.3 or later is part of all distributions of
% LaTeX version 2005/12/01 or later.
%
% This work has the LPPL maintenance status "maintained".
%
% This Current Maintainer of this work is Heiko Oberdiek.
%
% This work consists of the main source file listingsutf8.dtx
% and the derived files
%    listingsutf8.sty, listingsutf8.pdf, listingsutf8.ins, listingsutf8.drv,
%    listingsutf8-test1.tex, listingsutf8-test2.tex,
%    listingsutf8-test3.tex, listingsutf8-test4.tex,
%    listingsutf8-test5.tex.
%
% Distribution:
%    CTAN:macros/latex/contrib/oberdiek/listingsutf8.dtx
%    CTAN:macros/latex/contrib/oberdiek/listingsutf8.pdf
%
% Unpacking:
%    (a) If listingsutf8.ins is present:
%           tex listingsutf8.ins
%    (b) Without listingsutf8.ins:
%           tex listingsutf8.dtx
%    (c) If you insist on using LaTeX
%           latex \let\install=y\input{listingsutf8.dtx}
%        (quote the arguments according to the demands of your shell)
%
% Documentation:
%    (a) If listingsutf8.drv is present:
%           latex listingsutf8.drv
%    (b) Without listingsutf8.drv:
%           latex listingsutf8.dtx; ...
%    The class ltxdoc loads the configuration file ltxdoc.cfg
%    if available. Here you can specify further options, e.g.
%    use A4 as paper format:
%       \PassOptionsToClass{a4paper}{article}
%
%    Programm calls to get the documentation (example):
%       pdflatex listingsutf8.dtx
%       makeindex -s gind.ist listingsutf8.idx
%       pdflatex listingsutf8.dtx
%       makeindex -s gind.ist listingsutf8.idx
%       pdflatex listingsutf8.dtx
%
% Installation:
%    TDS:tex/latex/oberdiek/listingsutf8.sty
%    TDS:doc/latex/oberdiek/listingsutf8.pdf
%    TDS:doc/latex/oberdiek/test/listingsutf8-test1.tex
%    TDS:doc/latex/oberdiek/test/listingsutf8-test2.tex
%    TDS:doc/latex/oberdiek/test/listingsutf8-test3.tex
%    TDS:doc/latex/oberdiek/test/listingsutf8-test4.tex
%    TDS:doc/latex/oberdiek/test/listingsutf8-test5.tex
%    TDS:source/latex/oberdiek/listingsutf8.dtx
%
%<*ignore>
\begingroup
  \catcode123=1 %
  \catcode125=2 %
  \def\x{LaTeX2e}%
\expandafter\endgroup
\ifcase 0\ifx\install y1\fi\expandafter
         \ifx\csname processbatchFile\endcsname\relax\else1\fi
         \ifx\fmtname\x\else 1\fi\relax
\else\csname fi\endcsname
%</ignore>
%<*install>
\input docstrip.tex
\Msg{************************************************************************}
\Msg{* Installation}
\Msg{* Package: listingsutf8 2011/11/10 v1.2 Allow UTF-8 in listings input (HO)}
\Msg{************************************************************************}

\keepsilent
\askforoverwritefalse

\let\MetaPrefix\relax
\preamble

This is a generated file.

Project: listingsutf8
Version: 2011/11/10 v1.2

Copyright (C) 2007, 2011 by
   Heiko Oberdiek <heiko.oberdiek at googlemail.com>

This work may be distributed and/or modified under the
conditions of the LaTeX Project Public License, either
version 1.3c of this license or (at your option) any later
version. This version of this license is in
   http://www.latex-project.org/lppl/lppl-1-3c.txt
and the latest version of this license is in
   http://www.latex-project.org/lppl.txt
and version 1.3 or later is part of all distributions of
LaTeX version 2005/12/01 or later.

This work has the LPPL maintenance status "maintained".

This Current Maintainer of this work is Heiko Oberdiek.

This work consists of the main source file listingsutf8.dtx
and the derived files
   listingsutf8.sty, listingsutf8.pdf, listingsutf8.ins, listingsutf8.drv,
   listingsutf8-test1.tex, listingsutf8-test2.tex,
   listingsutf8-test3.tex, listingsutf8-test4.tex,
   listingsutf8-test5.tex.

\endpreamble
\let\MetaPrefix\DoubleperCent

\generate{%
  \file{listingsutf8.ins}{\from{listingsutf8.dtx}{install}}%
  \file{listingsutf8.drv}{\from{listingsutf8.dtx}{driver}}%
  \usedir{tex/latex/oberdiek}%
  \file{listingsutf8.sty}{\from{listingsutf8.dtx}{package}}%
  \usedir{doc/latex/oberdiek/test}%
  \file{listingsutf8-test1.tex}{\from{listingsutf8.dtx}{test1}}%
  \file{listingsutf8-test2.tex}{\from{listingsutf8.dtx}{test2,utf8}}%
  \file{listingsutf8-test3.tex}{\from{listingsutf8.dtx}{test3,utf8x}}%
  \file{listingsutf8-test4.tex}{\from{listingsutf8.dtx}{test4,utf8,noetex}}%
  \file{listingsutf8-test5.tex}{\from{listingsutf8.dtx}{test5,utf8x,noetex}}%
  \nopreamble
  \nopostamble
  \usedir{source/latex/oberdiek/catalogue}%
  \file{listingsutf8.xml}{\from{listingsutf8.dtx}{catalogue}}%
}

\catcode32=13\relax% active space
\let =\space%
\Msg{************************************************************************}
\Msg{*}
\Msg{* To finish the installation you have to move the following}
\Msg{* file into a directory searched by TeX:}
\Msg{*}
\Msg{*     listingsutf8.sty}
\Msg{*}
\Msg{* To produce the documentation run the file `listingsutf8.drv'}
\Msg{* through LaTeX.}
\Msg{*}
\Msg{* Happy TeXing!}
\Msg{*}
\Msg{************************************************************************}

\endbatchfile
%</install>
%<*ignore>
\fi
%</ignore>
%<*driver>
\NeedsTeXFormat{LaTeX2e}
\ProvidesFile{listingsutf8.drv}%
  [2011/11/10 v1.2 Allow UTF-8 in listings input (HO)]%
\documentclass{ltxdoc}
\usepackage{holtxdoc}[2011/11/22]
\begin{document}
  \DocInput{listingsutf8.dtx}%
\end{document}
%</driver>
% \fi
%
% \CheckSum{311}
%
% \CharacterTable
%  {Upper-case    \A\B\C\D\E\F\G\H\I\J\K\L\M\N\O\P\Q\R\S\T\U\V\W\X\Y\Z
%   Lower-case    \a\b\c\d\e\f\g\h\i\j\k\l\m\n\o\p\q\r\s\t\u\v\w\x\y\z
%   Digits        \0\1\2\3\4\5\6\7\8\9
%   Exclamation   \!     Double quote  \"     Hash (number) \#
%   Dollar        \$     Percent       \%     Ampersand     \&
%   Acute accent  \'     Left paren    \(     Right paren   \)
%   Asterisk      \*     Plus          \+     Comma         \,
%   Minus         \-     Point         \.     Solidus       \/
%   Colon         \:     Semicolon     \;     Less than     \<
%   Equals        \=     Greater than  \>     Question mark \?
%   Commercial at \@     Left bracket  \[     Backslash     \\
%   Right bracket \]     Circumflex    \^     Underscore    \_
%   Grave accent  \`     Left brace    \{     Vertical bar  \|
%   Right brace   \}     Tilde         \~}
%
% \GetFileInfo{listingsutf8.drv}
%
% \title{The \xpackage{listingsutf8} package}
% \date{2011/11/10 v1.2}
% \author{Heiko Oberdiek\\\xemail{heiko.oberdiek at googlemail.com}}
%
% \maketitle
%
% \begin{abstract}
% Package \xpackage{listings} does not support files with multi-byte
% encodings such as UTF-8. In case of \cs{lstinputlisting} a simple
% workaround is possible if an one-byte encoding exists that the file
% can be converted to. Also \eTeX\ and \pdfTeX\ regardless of its mode
% are required.
% \end{abstract}
%
% \tableofcontents
%
% \section{Documentation}
%
% \subsection{User interface}
%
% Load this package after or instead of package \xpackage{listings}
% \cite{listings}.
% The package does not define own options and passes given options to
% package \xpackage{listings}.
%
% The syntax of package \xpackage{listings}' key \xoption{inputencoding}
% is extended:
% \begin{quote}
%   |inputencoding=utf8/|\meta{one-byte-encoding}\\
%   Example: |inputencoding=utf8/latin1|
% \end{quote}
% That means the file is encoded in UTF-8 and can
% be converted to the given \meta{one-byte-encoding}.
% The available encodings for \meta{one-byte-encoding} are
% listed in section ``1.2 Supported encodings'' of
% package \xpackage{stringenc}'s documentation \cite{stringenc}.
% Of course, the encoding must encode its characters with
% one byte exactly. This excludes the unicode encodings
% (\xoption{utf8}, \xoption{utf16}, \dots).
%
% Only \cs{lstinputlisting} is supported by the syntax extension
% of key \xoption{inputencoding}.
%
% Internally package \xpackage{listingsutf8} reads the file as binary file
% via primitives of \pdfTeX\ (\cs{pdffiledump}). Then the file
% contents is converted as string using package \xpackage{stringenc} and
% finally the string is read as virtual file by \eTeX's \cs{scantokens}.
%
% \subsection{Future}
%
% Workarounds are not provided for
% \begin{itemize}
% \item \cs{lstinline}
% \item Environment |lstlisting|.
% \item Environments defined by \cs{lstnewenvironment}.
% \end{itemize}
% Perhaps someone will find time to extend package \xpackage{listings}
% with full native support for UTF-8. Then this package would become obsolete.
%
% \StopEventually{
% }
%
% \section{Implementation}
%
%    \begin{macrocode}
%<*package>
%    \end{macrocode}
%
% \subsection{Catcodes and identification}
%
%    \begin{macrocode}
\begingroup\catcode61\catcode48\catcode32=10\relax%
  \catcode13=5 % ^^M
  \endlinechar=13 %
  \catcode123=1 % {
  \catcode125=2 % }
  \catcode64=11 % @
  \def\x{\endgroup
    \expandafter\edef\csname lstU@AtEnd\endcsname{%
      \endlinechar=\the\endlinechar\relax
      \catcode13=\the\catcode13\relax
      \catcode32=\the\catcode32\relax
      \catcode35=\the\catcode35\relax
      \catcode61=\the\catcode61\relax
      \catcode64=\the\catcode64\relax
      \catcode123=\the\catcode123\relax
      \catcode125=\the\catcode125\relax
    }%
  }%
\x\catcode61\catcode48\catcode32=10\relax%
\catcode13=5 % ^^M
\endlinechar=13 %
\catcode35=6 % #
\catcode64=11 % @
\catcode123=1 % {
\catcode125=2 % }
\def\TMP@EnsureCode#1#2{%
  \edef\lstU@AtEnd{%
    \lstU@AtEnd
    \catcode#1=\the\catcode#1\relax
  }%
  \catcode#1=#2\relax
}
\TMP@EnsureCode{10}{12}% ^^J
\TMP@EnsureCode{33}{12}% !
\TMP@EnsureCode{36}{3}% $
\TMP@EnsureCode{38}{4}% &
\TMP@EnsureCode{39}{12}% '
\TMP@EnsureCode{40}{12}% (
\TMP@EnsureCode{41}{12}% )
\TMP@EnsureCode{42}{12}% *
\TMP@EnsureCode{43}{12}% +
\TMP@EnsureCode{44}{12}% ,
\TMP@EnsureCode{45}{12}% -
\TMP@EnsureCode{46}{12}% .
\TMP@EnsureCode{47}{12}% /
\TMP@EnsureCode{58}{12}% :
\TMP@EnsureCode{60}{12}% <
\TMP@EnsureCode{62}{12}% >
\TMP@EnsureCode{91}{12}% [
\TMP@EnsureCode{93}{12}% ]
\TMP@EnsureCode{94}{7}% ^ (superscript)
\TMP@EnsureCode{95}{8}% _ (subscript)
\TMP@EnsureCode{96}{12}% `
\TMP@EnsureCode{124}{12}% |
\TMP@EnsureCode{126}{13}% ~ (active)
\edef\lstU@AtEnd{\lstU@AtEnd\noexpand\endinput}
%    \end{macrocode}
%
%    Package identification.
%    \begin{macrocode}
\NeedsTeXFormat{LaTeX2e}
\ProvidesPackage{listingsutf8}%
  [2011/11/10 v1.2 Allow UTF-8 in listings input (HO)]
%    \end{macrocode}
%
% \subsection{Package options}
%
% Just pass options to package listings.
%
%    \begin{macrocode}
\DeclareOption*{%
  \PassOptionsToPackage\CurrentOption{listings}%
}
\ProcessOptions*
%    \end{macrocode}
%    Key \xoption{inputencoding} was introduced in version
%    2002/04/01 v1.0 of package \xpackage{listings}.
%    \begin{macrocode}
\RequirePackage{listings}[2002/04/01]
%    \end{macrocode}
%    Ensure that \cs{inputencoding} is provided.
%    \begin{macrocode}
\AtBeginDocument{%
  \@ifundefined{inputencoding}{%
    \RequirePackage{inputenc}%
  }{}%
}
%    \end{macrocode}
%
% \subsection{Check prerequisites}
%
%    \begin{macrocode}
\RequirePackage{pdftexcmds}[2011/04/22]
%    \end{macrocode}
%
%    \begin{macrocode}
\def\lstU@temp#1#2{%
  \begingroup\expandafter\expandafter\expandafter\endgroup
  \expandafter\ifx\csname #1\endcsname\relax
    \PackageWarningNoLine{listingsutf8}{%
      Package loading is aborted because of missing %
      \@backslashchar#1.\MessageBreak
      #2%
    }%
    \expandafter\lstU@AtEnd
  \fi
}
\lstU@temp{scantokens}{It is provided by e-TeX}%
\lstU@temp{pdf@unescapehex}{It is provided by pdfTeX >= 1.30}%
\lstU@temp{pdf@filedump}{It is provided by pdfTeX >= 1.30}%
\lstU@temp{pdf@filesize}{It is provided by pdfTeX >= 1.30}%
%    \end{macrocode}
%
%    \begin{macrocode}
\RequirePackage{stringenc}[2010/03/01]
%    \end{macrocode}
%
% \subsection{Add support for UTF-8}
%
%    \begin{macro}{\iflstU@utfviii}
%    \begin{macrocode}
\newif\iflstU@utfviii
%    \end{macrocode}
%    \end{macro}
%
%    \begin{macro}{\lstU@inputenc}
%    \begin{macrocode}
\def\lstU@inputenc#1{%
  \expandafter\lstU@@inputenc#1utf8/utf8/\@nil
}
%    \end{macrocode}
%    \end{macro}
%    \begin{macro}{\lstU@@inputenc}
\def\lstU@@inputenc#1utf8/#2utf8/#3\@nil{%
  \ifx\\#1\\%
    \lstU@utfviiitrue
    \def\lst@inputenc{#2}%
  \else
    \lstU@utfviiifalse
  \fi
}
%    \begin{macrocode}
%    \end{macrocode}
%    \end{macro}
%
%    \begin{macrocode}
\lst@Key{inputencoding}\relax{%
  \def\lst@inputenc{#1}%
  \lstU@inputenc{#1}%
}
%    \end{macrocode}
%
% \subsubsection{Conversion}
%
%    \begin{macro}{\lstU@input}
%    \begin{macrocode}
\def\lstU@input#1{%
  \iflstU@utfviii
    \edef\lstU@text{%
      \pdf@unescapehex{%
        \pdf@filedump{0}{\pdf@filesize{#1}}{#1}%
      }%
    }%
    \lstU@CRLFtoLF\lstU@text
    \StringEncodingConvert\lstU@text\lstU@text{utf8}\lst@inputenc
    \def\lstU@temp{%
      \scantokens\expandafter{\lstU@text}%
    }%
  \else
    \def\lstU@temp{%
      \input{#1}%
    }%
  \fi
  \lstU@temp
}
%    \end{macrocode}
%    \end{macro}
%
% \subsubsection{Convert CR/LF pairs to LF}
%
%    \begin{macro}{\lstU@CRLFtoLF}
%    \begin{macrocode}
\begingroup
  \endlinechar=-1 %
  \@makeother\^^J %
  \@makeother\^^M %
  \gdef\lstU@CRLFtoLF#1{%
    \edef#1{%
      \expandafter\lstU@CRLFtoLF@aux#1^^M^^J\@nil
    }%
  }%
  \gdef\lstU@CRLFtoLF@aux#1^^M^^J#2\@nil{%
    #1%
    \ifx\relax#2\relax
      \@car
    \fi
    ^^J%
    \lstU@CRLFtoLF@aux#2\@nil
  }%
\endgroup %
%    \end{macrocode}
%    \end{macro}
%
% \subsubsection{Patch \cs{lst@InputListing}}
%
%    \begin{macrocode}
\def\lstU@temp#1\def\lst@next#2#3\@nil{%
  \def\lst@InputListing##1{%
    #1%
    \def\lst@next{\lstU@input{##1}}%
    #3%
  }%
}
\expandafter\lstU@temp\lst@InputListing{#1}\@nil
%    \end{macrocode}
%
%    \begin{macrocode}
\lstU@AtEnd%
%</package>
%    \end{macrocode}
%
% \section{Test}
%
% \subsection{Catcode checks for loading}
%
%    \begin{macrocode}
%<*test1>
%    \end{macrocode}
%    \begin{macrocode}
\NeedsTeXFormat{LaTeX2e}
\documentclass{minimal}
\makeatletter
\def\RestoreCatcodes{}
\count@=0 %
\loop
  \edef\RestoreCatcodes{%
    \RestoreCatcodes
    \catcode\the\count@=\the\catcode\count@\relax
  }%
\ifnum\count@<255 %
  \advance\count@\@ne
\repeat

\def\RangeCatcodeInvalid#1#2{%
  \count@=#1\relax
  \loop
    \catcode\count@=15 %
  \ifnum\count@<#2\relax
    \advance\count@\@ne
  \repeat
}
\def\Test{%
  \RangeCatcodeInvalid{0}{47}%
  \RangeCatcodeInvalid{58}{64}%
  \RangeCatcodeInvalid{91}{96}%
  \RangeCatcodeInvalid{123}{127}%
  \catcode`\@=12 %
  \catcode`\\=0 %
  \catcode`\{=1 %
  \catcode`\}=2 %
  \catcode`\#=6 %
  \catcode`\[=12 %
  \catcode`\]=12 %
  \catcode`\%=14 %
  \catcode`\ =10 %
  \catcode13=5 %
  \RequirePackage{listingsutf8}[2011/11/10]\relax
  \RestoreCatcodes
}
\Test
\csname @@end\endcsname
\end
%    \end{macrocode}
%    \begin{macrocode}
%</test1>
%    \end{macrocode}
%
% \subsection{Test example for latin1}
%
%    \begin{macrocode}
%<*test2>
%    \end{macrocode}
%    \begin{macrocode}
\NeedsTeXFormat{LaTeX2e}
\documentclass{minimal}
\usepackage{filecontents}
\def\do#1{%
  \ifx#1\^%
  \else
    \noexpand\do\noexpand#1%
  \fi
}
\expandafter\let\expandafter\dospecials\expandafter\empty
\expandafter\edef\expandafter\dospecials\expandafter{\dospecials}
\begin{filecontents*}{ExampleUTF8.java}
public class ExampleUTF8 {
    public static String testString =
        "Umlauts: " +
        "^^c3^^84^^c3^^96^^c3^^9c^^c3^^a4^^c3^^b6^^c3^^bc^^c3^^9f";
    public static void main(String[] args) {
        System.out.println(testString);
    }
}
\end{filecontents*}
\usepackage{listingsutf8}[2011/11/10]
\def\Text{%
  Umlauts: %
  ^^c3^^84^^c3^^96^^c3^^9c^^c3^^a4^^c3^^b6^^c3^^bc^^c3^^9f%
}
\begin{document}
\lstinputlisting[%
  language=Java,%
  inputencoding=utf8/latin1,%
]{ExampleUTF8.java}
\end{document}
%</test2>
%    \end{macrocode}
%
% \section{Installation}
%
% \subsection{Download}
%
% \paragraph{Package.} This package is available on
% CTAN\footnote{\url{ftp://ftp.ctan.org/tex-archive/}}:
% \begin{description}
% \item[\CTAN{macros/latex/contrib/oberdiek/listingsutf8.dtx}] The source file.
% \item[\CTAN{macros/latex/contrib/oberdiek/listingsutf8.pdf}] Documentation.
% \end{description}
%
%
% \paragraph{Bundle.} All the packages of the bundle `oberdiek'
% are also available in a TDS compliant ZIP archive. There
% the packages are already unpacked and the documentation files
% are generated. The files and directories obey the TDS standard.
% \begin{description}
% \item[\CTAN{install/macros/latex/contrib/oberdiek.tds.zip}]
% \end{description}
% \emph{TDS} refers to the standard ``A Directory Structure
% for \TeX\ Files'' (\CTAN{tds/tds.pdf}). Directories
% with \xfile{texmf} in their name are usually organized this way.
%
% \subsection{Bundle installation}
%
% \paragraph{Unpacking.} Unpack the \xfile{oberdiek.tds.zip} in the
% TDS tree (also known as \xfile{texmf} tree) of your choice.
% Example (linux):
% \begin{quote}
%   |unzip oberdiek.tds.zip -d ~/texmf|
% \end{quote}
%
% \paragraph{Script installation.}
% Check the directory \xfile{TDS:scripts/oberdiek/} for
% scripts that need further installation steps.
% Package \xpackage{attachfile2} comes with the Perl script
% \xfile{pdfatfi.pl} that should be installed in such a way
% that it can be called as \texttt{pdfatfi}.
% Example (linux):
% \begin{quote}
%   |chmod +x scripts/oberdiek/pdfatfi.pl|\\
%   |cp scripts/oberdiek/pdfatfi.pl /usr/local/bin/|
% \end{quote}
%
% \subsection{Package installation}
%
% \paragraph{Unpacking.} The \xfile{.dtx} file is a self-extracting
% \docstrip\ archive. The files are extracted by running the
% \xfile{.dtx} through \plainTeX:
% \begin{quote}
%   \verb|tex listingsutf8.dtx|
% \end{quote}
%
% \paragraph{TDS.} Now the different files must be moved into
% the different directories in your installation TDS tree
% (also known as \xfile{texmf} tree):
% \begin{quote}
% \def\t{^^A
% \begin{tabular}{@{}>{\ttfamily}l@{ $\rightarrow$ }>{\ttfamily}l@{}}
%   listingsutf8.sty & tex/latex/oberdiek/listingsutf8.sty\\
%   listingsutf8.pdf & doc/latex/oberdiek/listingsutf8.pdf\\
%   test/listingsutf8-test1.tex & doc/latex/oberdiek/test/listingsutf8-test1.tex\\
%   test/listingsutf8-test2.tex & doc/latex/oberdiek/test/listingsutf8-test2.tex\\
%   test/listingsutf8-test3.tex & doc/latex/oberdiek/test/listingsutf8-test3.tex\\
%   test/listingsutf8-test4.tex & doc/latex/oberdiek/test/listingsutf8-test4.tex\\
%   test/listingsutf8-test5.tex & doc/latex/oberdiek/test/listingsutf8-test5.tex\\
%   listingsutf8.dtx & source/latex/oberdiek/listingsutf8.dtx\\
% \end{tabular}^^A
% }^^A
% \sbox0{\t}^^A
% \ifdim\wd0>\linewidth
%   \begingroup
%     \advance\linewidth by\leftmargin
%     \advance\linewidth by\rightmargin
%   \edef\x{\endgroup
%     \def\noexpand\lw{\the\linewidth}^^A
%   }\x
%   \def\lwbox{^^A
%     \leavevmode
%     \hbox to \linewidth{^^A
%       \kern-\leftmargin\relax
%       \hss
%       \usebox0
%       \hss
%       \kern-\rightmargin\relax
%     }^^A
%   }^^A
%   \ifdim\wd0>\lw
%     \sbox0{\small\t}^^A
%     \ifdim\wd0>\linewidth
%       \ifdim\wd0>\lw
%         \sbox0{\footnotesize\t}^^A
%         \ifdim\wd0>\linewidth
%           \ifdim\wd0>\lw
%             \sbox0{\scriptsize\t}^^A
%             \ifdim\wd0>\linewidth
%               \ifdim\wd0>\lw
%                 \sbox0{\tiny\t}^^A
%                 \ifdim\wd0>\linewidth
%                   \lwbox
%                 \else
%                   \usebox0
%                 \fi
%               \else
%                 \lwbox
%               \fi
%             \else
%               \usebox0
%             \fi
%           \else
%             \lwbox
%           \fi
%         \else
%           \usebox0
%         \fi
%       \else
%         \lwbox
%       \fi
%     \else
%       \usebox0
%     \fi
%   \else
%     \lwbox
%   \fi
% \else
%   \usebox0
% \fi
% \end{quote}
% If you have a \xfile{docstrip.cfg} that configures and enables \docstrip's
% TDS installing feature, then some files can already be in the right
% place, see the documentation of \docstrip.
%
% \subsection{Refresh file name databases}
%
% If your \TeX~distribution
% (\teTeX, \mikTeX, \dots) relies on file name databases, you must refresh
% these. For example, \teTeX\ users run \verb|texhash| or
% \verb|mktexlsr|.
%
% \subsection{Some details for the interested}
%
% \paragraph{Attached source.}
%
% The PDF documentation on CTAN also includes the
% \xfile{.dtx} source file. It can be extracted by
% AcrobatReader 6 or higher. Another option is \textsf{pdftk},
% e.g. unpack the file into the current directory:
% \begin{quote}
%   \verb|pdftk listingsutf8.pdf unpack_files output .|
% \end{quote}
%
% \paragraph{Unpacking with \LaTeX.}
% The \xfile{.dtx} chooses its action depending on the format:
% \begin{description}
% \item[\plainTeX:] Run \docstrip\ and extract the files.
% \item[\LaTeX:] Generate the documentation.
% \end{description}
% If you insist on using \LaTeX\ for \docstrip\ (really,
% \docstrip\ does not need \LaTeX), then inform the autodetect routine
% about your intention:
% \begin{quote}
%   \verb|latex \let\install=y\input{listingsutf8.dtx}|
% \end{quote}
% Do not forget to quote the argument according to the demands
% of your shell.
%
% \paragraph{Generating the documentation.}
% You can use both the \xfile{.dtx} or the \xfile{.drv} to generate
% the documentation. The process can be configured by the
% configuration file \xfile{ltxdoc.cfg}. For instance, put this
% line into this file, if you want to have A4 as paper format:
% \begin{quote}
%   \verb|\PassOptionsToClass{a4paper}{article}|
% \end{quote}
% An example follows how to generate the
% documentation with pdf\LaTeX:
% \begin{quote}
%\begin{verbatim}
%pdflatex listingsutf8.dtx
%makeindex -s gind.ist listingsutf8.idx
%pdflatex listingsutf8.dtx
%makeindex -s gind.ist listingsutf8.idx
%pdflatex listingsutf8.dtx
%\end{verbatim}
% \end{quote}
%
% \section{Catalogue}
%
% The following XML file can be used as source for the
% \href{http://mirror.ctan.org/help/Catalogue/catalogue.html}{\TeX\ Catalogue}.
% The elements \texttt{caption} and \texttt{description} are imported
% from the original XML file from the Catalogue.
% The name of the XML file in the Catalogue is \xfile{listingsutf8.xml}.
%    \begin{macrocode}
%<*catalogue>
<?xml version='1.0' encoding='us-ascii'?>
<!DOCTYPE entry SYSTEM 'catalogue.dtd'>
<entry datestamp='$Date$' modifier='$Author$' id='listingsutf8'>
  <name>listingsutf8</name>
  <caption>Allow UTF-8 in listings input.</caption>
  <authorref id='auth:oberdiek'/>
  <copyright owner='Heiko Oberdiek' year='2007,2011'/>
  <license type='lppl1.3'/>
  <version number='1.2'/>
  <description>
    Package <xref refid='listings'>listings</xref> does not support files
    with multi-byte encodings such as UTF-8.  In the case of
    <tt>\lstinputlisting</tt>, a simple workaround is possible if a
    one-byte encoding exists that the file can be converted to.  The
    package requires the e-TeX extensions under pdfTeX (in either PDF
    or DVI output mode).
    <p/>
    The package is part of the <xref refid='oberdiek'>oberdiek</xref> bundle.
  </description>
  <documentation details='Package documentation'
      href='ctan:/macros/latex/contrib/oberdiek/listingsutf8.pdf'/>
  <ctan file='true' path='/macros/latex/contrib/oberdiek/listingsutf8.dtx'/>
  <miktex location='oberdiek'/>
  <texlive location='oberdiek'/>
  <install path='/macros/latex/contrib/oberdiek/oberdiek.tds.zip'/>
</entry>
%</catalogue>
%    \end{macrocode}
%
% \begin{thebibliography}{9}
%
% \bibitem{inputenc}
%   Alan Jeffrey, Frank Mittelbach,
%   \textit{inputenc.sty}, 2006/05/05 v1.1b.
%   \CTAN{macros/latex/base/inputenc.dtx}
%
% \bibitem{listings}
%   Carsten Heinz, Brooks Moses:
%  \textit{The \xpackage{listings} package};
%   2007/02/22;\\
%   \CTAN{macros/latex/contrib/listings/}.
%
% \bibitem{stringenc}
%   Heiko Oberdiek:
%   \textit{The \xpackage{stringenc} package};
%   2007/10/22;\\
%   \CTAN{macros/latex/contrib/oberdiek/stringenc.pdf}.
%
% \end{thebibliography}
%
% \begin{History}
%   \begin{Version}{2007/10/22 v1.0}
%   \item
%     First version.
%   \end{Version}
%   \begin{Version}{2007/11/11 v1.1}
%   \item
%     Use of package \xpackage{pdftexcmds}.
%   \end{Version}
%   \begin{Version}{2011/11/10 v1.2}
%   \item
%     DOS line ends CR/LF normalized to LF to avoid empty lines
%     (Bug report of Thomas Benkert in de.comp.text.tex).
%   \end{Version}
% \end{History}
%
% \PrintIndex
%
% \Finale
\endinput

%        (quote the arguments according to the demands of your shell)
%
% Documentation:
%    (a) If listingsutf8.drv is present:
%           latex listingsutf8.drv
%    (b) Without listingsutf8.drv:
%           latex listingsutf8.dtx; ...
%    The class ltxdoc loads the configuration file ltxdoc.cfg
%    if available. Here you can specify further options, e.g.
%    use A4 as paper format:
%       \PassOptionsToClass{a4paper}{article}
%
%    Programm calls to get the documentation (example):
%       pdflatex listingsutf8.dtx
%       makeindex -s gind.ist listingsutf8.idx
%       pdflatex listingsutf8.dtx
%       makeindex -s gind.ist listingsutf8.idx
%       pdflatex listingsutf8.dtx
%
% Installation:
%    TDS:tex/latex/oberdiek/listingsutf8.sty
%    TDS:doc/latex/oberdiek/listingsutf8.pdf
%    TDS:doc/latex/oberdiek/test/listingsutf8-test1.tex
%    TDS:doc/latex/oberdiek/test/listingsutf8-test2.tex
%    TDS:doc/latex/oberdiek/test/listingsutf8-test3.tex
%    TDS:doc/latex/oberdiek/test/listingsutf8-test4.tex
%    TDS:doc/latex/oberdiek/test/listingsutf8-test5.tex
%    TDS:source/latex/oberdiek/listingsutf8.dtx
%
%<*ignore>
\begingroup
  \catcode123=1 %
  \catcode125=2 %
  \def\x{LaTeX2e}%
\expandafter\endgroup
\ifcase 0\ifx\install y1\fi\expandafter
         \ifx\csname processbatchFile\endcsname\relax\else1\fi
         \ifx\fmtname\x\else 1\fi\relax
\else\csname fi\endcsname
%</ignore>
%<*install>
\input docstrip.tex
\Msg{************************************************************************}
\Msg{* Installation}
\Msg{* Package: listingsutf8 2011/11/10 v1.2 Allow UTF-8 in listings input (HO)}
\Msg{************************************************************************}

\keepsilent
\askforoverwritefalse

\let\MetaPrefix\relax
\preamble

This is a generated file.

Project: listingsutf8
Version: 2011/11/10 v1.2

Copyright (C) 2007, 2011 by
   Heiko Oberdiek <heiko.oberdiek at googlemail.com>

This work may be distributed and/or modified under the
conditions of the LaTeX Project Public License, either
version 1.3c of this license or (at your option) any later
version. This version of this license is in
   http://www.latex-project.org/lppl/lppl-1-3c.txt
and the latest version of this license is in
   http://www.latex-project.org/lppl.txt
and version 1.3 or later is part of all distributions of
LaTeX version 2005/12/01 or later.

This work has the LPPL maintenance status "maintained".

This Current Maintainer of this work is Heiko Oberdiek.

This work consists of the main source file listingsutf8.dtx
and the derived files
   listingsutf8.sty, listingsutf8.pdf, listingsutf8.ins, listingsutf8.drv,
   listingsutf8-test1.tex, listingsutf8-test2.tex,
   listingsutf8-test3.tex, listingsutf8-test4.tex,
   listingsutf8-test5.tex.

\endpreamble
\let\MetaPrefix\DoubleperCent

\generate{%
  \file{listingsutf8.ins}{\from{listingsutf8.dtx}{install}}%
  \file{listingsutf8.drv}{\from{listingsutf8.dtx}{driver}}%
  \usedir{tex/latex/oberdiek}%
  \file{listingsutf8.sty}{\from{listingsutf8.dtx}{package}}%
  \usedir{doc/latex/oberdiek/test}%
  \file{listingsutf8-test1.tex}{\from{listingsutf8.dtx}{test1}}%
  \file{listingsutf8-test2.tex}{\from{listingsutf8.dtx}{test2,utf8}}%
  \file{listingsutf8-test3.tex}{\from{listingsutf8.dtx}{test3,utf8x}}%
  \file{listingsutf8-test4.tex}{\from{listingsutf8.dtx}{test4,utf8,noetex}}%
  \file{listingsutf8-test5.tex}{\from{listingsutf8.dtx}{test5,utf8x,noetex}}%
  \nopreamble
  \nopostamble
  \usedir{source/latex/oberdiek/catalogue}%
  \file{listingsutf8.xml}{\from{listingsutf8.dtx}{catalogue}}%
}

\catcode32=13\relax% active space
\let =\space%
\Msg{************************************************************************}
\Msg{*}
\Msg{* To finish the installation you have to move the following}
\Msg{* file into a directory searched by TeX:}
\Msg{*}
\Msg{*     listingsutf8.sty}
\Msg{*}
\Msg{* To produce the documentation run the file `listingsutf8.drv'}
\Msg{* through LaTeX.}
\Msg{*}
\Msg{* Happy TeXing!}
\Msg{*}
\Msg{************************************************************************}

\endbatchfile
%</install>
%<*ignore>
\fi
%</ignore>
%<*driver>
\NeedsTeXFormat{LaTeX2e}
\ProvidesFile{listingsutf8.drv}%
  [2011/11/10 v1.2 Allow UTF-8 in listings input (HO)]%
\documentclass{ltxdoc}
\usepackage{holtxdoc}[2011/11/22]
\begin{document}
  \DocInput{listingsutf8.dtx}%
\end{document}
%</driver>
% \fi
%
% \CheckSum{311}
%
% \CharacterTable
%  {Upper-case    \A\B\C\D\E\F\G\H\I\J\K\L\M\N\O\P\Q\R\S\T\U\V\W\X\Y\Z
%   Lower-case    \a\b\c\d\e\f\g\h\i\j\k\l\m\n\o\p\q\r\s\t\u\v\w\x\y\z
%   Digits        \0\1\2\3\4\5\6\7\8\9
%   Exclamation   \!     Double quote  \"     Hash (number) \#
%   Dollar        \$     Percent       \%     Ampersand     \&
%   Acute accent  \'     Left paren    \(     Right paren   \)
%   Asterisk      \*     Plus          \+     Comma         \,
%   Minus         \-     Point         \.     Solidus       \/
%   Colon         \:     Semicolon     \;     Less than     \<
%   Equals        \=     Greater than  \>     Question mark \?
%   Commercial at \@     Left bracket  \[     Backslash     \\
%   Right bracket \]     Circumflex    \^     Underscore    \_
%   Grave accent  \`     Left brace    \{     Vertical bar  \|
%   Right brace   \}     Tilde         \~}
%
% \GetFileInfo{listingsutf8.drv}
%
% \title{The \xpackage{listingsutf8} package}
% \date{2011/11/10 v1.2}
% \author{Heiko Oberdiek\\\xemail{heiko.oberdiek at googlemail.com}}
%
% \maketitle
%
% \begin{abstract}
% Package \xpackage{listings} does not support files with multi-byte
% encodings such as UTF-8. In case of \cs{lstinputlisting} a simple
% workaround is possible if an one-byte encoding exists that the file
% can be converted to. Also \eTeX\ and \pdfTeX\ regardless of its mode
% are required.
% \end{abstract}
%
% \tableofcontents
%
% \section{Documentation}
%
% \subsection{User interface}
%
% Load this package after or instead of package \xpackage{listings}
% \cite{listings}.
% The package does not define own options and passes given options to
% package \xpackage{listings}.
%
% The syntax of package \xpackage{listings}' key \xoption{inputencoding}
% is extended:
% \begin{quote}
%   |inputencoding=utf8/|\meta{one-byte-encoding}\\
%   Example: |inputencoding=utf8/latin1|
% \end{quote}
% That means the file is encoded in UTF-8 and can
% be converted to the given \meta{one-byte-encoding}.
% The available encodings for \meta{one-byte-encoding} are
% listed in section ``1.2 Supported encodings'' of
% package \xpackage{stringenc}'s documentation \cite{stringenc}.
% Of course, the encoding must encode its characters with
% one byte exactly. This excludes the unicode encodings
% (\xoption{utf8}, \xoption{utf16}, \dots).
%
% Only \cs{lstinputlisting} is supported by the syntax extension
% of key \xoption{inputencoding}.
%
% Internally package \xpackage{listingsutf8} reads the file as binary file
% via primitives of \pdfTeX\ (\cs{pdffiledump}). Then the file
% contents is converted as string using package \xpackage{stringenc} and
% finally the string is read as virtual file by \eTeX's \cs{scantokens}.
%
% \subsection{Future}
%
% Workarounds are not provided for
% \begin{itemize}
% \item \cs{lstinline}
% \item Environment |lstlisting|.
% \item Environments defined by \cs{lstnewenvironment}.
% \end{itemize}
% Perhaps someone will find time to extend package \xpackage{listings}
% with full native support for UTF-8. Then this package would become obsolete.
%
% \StopEventually{
% }
%
% \section{Implementation}
%
%    \begin{macrocode}
%<*package>
%    \end{macrocode}
%
% \subsection{Catcodes and identification}
%
%    \begin{macrocode}
\begingroup\catcode61\catcode48\catcode32=10\relax%
  \catcode13=5 % ^^M
  \endlinechar=13 %
  \catcode123=1 % {
  \catcode125=2 % }
  \catcode64=11 % @
  \def\x{\endgroup
    \expandafter\edef\csname lstU@AtEnd\endcsname{%
      \endlinechar=\the\endlinechar\relax
      \catcode13=\the\catcode13\relax
      \catcode32=\the\catcode32\relax
      \catcode35=\the\catcode35\relax
      \catcode61=\the\catcode61\relax
      \catcode64=\the\catcode64\relax
      \catcode123=\the\catcode123\relax
      \catcode125=\the\catcode125\relax
    }%
  }%
\x\catcode61\catcode48\catcode32=10\relax%
\catcode13=5 % ^^M
\endlinechar=13 %
\catcode35=6 % #
\catcode64=11 % @
\catcode123=1 % {
\catcode125=2 % }
\def\TMP@EnsureCode#1#2{%
  \edef\lstU@AtEnd{%
    \lstU@AtEnd
    \catcode#1=\the\catcode#1\relax
  }%
  \catcode#1=#2\relax
}
\TMP@EnsureCode{10}{12}% ^^J
\TMP@EnsureCode{33}{12}% !
\TMP@EnsureCode{36}{3}% $
\TMP@EnsureCode{38}{4}% &
\TMP@EnsureCode{39}{12}% '
\TMP@EnsureCode{40}{12}% (
\TMP@EnsureCode{41}{12}% )
\TMP@EnsureCode{42}{12}% *
\TMP@EnsureCode{43}{12}% +
\TMP@EnsureCode{44}{12}% ,
\TMP@EnsureCode{45}{12}% -
\TMP@EnsureCode{46}{12}% .
\TMP@EnsureCode{47}{12}% /
\TMP@EnsureCode{58}{12}% :
\TMP@EnsureCode{60}{12}% <
\TMP@EnsureCode{62}{12}% >
\TMP@EnsureCode{91}{12}% [
\TMP@EnsureCode{93}{12}% ]
\TMP@EnsureCode{94}{7}% ^ (superscript)
\TMP@EnsureCode{95}{8}% _ (subscript)
\TMP@EnsureCode{96}{12}% `
\TMP@EnsureCode{124}{12}% |
\TMP@EnsureCode{126}{13}% ~ (active)
\edef\lstU@AtEnd{\lstU@AtEnd\noexpand\endinput}
%    \end{macrocode}
%
%    Package identification.
%    \begin{macrocode}
\NeedsTeXFormat{LaTeX2e}
\ProvidesPackage{listingsutf8}%
  [2011/11/10 v1.2 Allow UTF-8 in listings input (HO)]
%    \end{macrocode}
%
% \subsection{Package options}
%
% Just pass options to package listings.
%
%    \begin{macrocode}
\DeclareOption*{%
  \PassOptionsToPackage\CurrentOption{listings}%
}
\ProcessOptions*
%    \end{macrocode}
%    Key \xoption{inputencoding} was introduced in version
%    2002/04/01 v1.0 of package \xpackage{listings}.
%    \begin{macrocode}
\RequirePackage{listings}[2002/04/01]
%    \end{macrocode}
%    Ensure that \cs{inputencoding} is provided.
%    \begin{macrocode}
\AtBeginDocument{%
  \@ifundefined{inputencoding}{%
    \RequirePackage{inputenc}%
  }{}%
}
%    \end{macrocode}
%
% \subsection{Check prerequisites}
%
%    \begin{macrocode}
\RequirePackage{pdftexcmds}[2011/04/22]
%    \end{macrocode}
%
%    \begin{macrocode}
\def\lstU@temp#1#2{%
  \begingroup\expandafter\expandafter\expandafter\endgroup
  \expandafter\ifx\csname #1\endcsname\relax
    \PackageWarningNoLine{listingsutf8}{%
      Package loading is aborted because of missing %
      \@backslashchar#1.\MessageBreak
      #2%
    }%
    \expandafter\lstU@AtEnd
  \fi
}
\lstU@temp{scantokens}{It is provided by e-TeX}%
\lstU@temp{pdf@unescapehex}{It is provided by pdfTeX >= 1.30}%
\lstU@temp{pdf@filedump}{It is provided by pdfTeX >= 1.30}%
\lstU@temp{pdf@filesize}{It is provided by pdfTeX >= 1.30}%
%    \end{macrocode}
%
%    \begin{macrocode}
\RequirePackage{stringenc}[2010/03/01]
%    \end{macrocode}
%
% \subsection{Add support for UTF-8}
%
%    \begin{macro}{\iflstU@utfviii}
%    \begin{macrocode}
\newif\iflstU@utfviii
%    \end{macrocode}
%    \end{macro}
%
%    \begin{macro}{\lstU@inputenc}
%    \begin{macrocode}
\def\lstU@inputenc#1{%
  \expandafter\lstU@@inputenc#1utf8/utf8/\@nil
}
%    \end{macrocode}
%    \end{macro}
%    \begin{macro}{\lstU@@inputenc}
\def\lstU@@inputenc#1utf8/#2utf8/#3\@nil{%
  \ifx\\#1\\%
    \lstU@utfviiitrue
    \def\lst@inputenc{#2}%
  \else
    \lstU@utfviiifalse
  \fi
}
%    \begin{macrocode}
%    \end{macrocode}
%    \end{macro}
%
%    \begin{macrocode}
\lst@Key{inputencoding}\relax{%
  \def\lst@inputenc{#1}%
  \lstU@inputenc{#1}%
}
%    \end{macrocode}
%
% \subsubsection{Conversion}
%
%    \begin{macro}{\lstU@input}
%    \begin{macrocode}
\def\lstU@input#1{%
  \iflstU@utfviii
    \edef\lstU@text{%
      \pdf@unescapehex{%
        \pdf@filedump{0}{\pdf@filesize{#1}}{#1}%
      }%
    }%
    \lstU@CRLFtoLF\lstU@text
    \StringEncodingConvert\lstU@text\lstU@text{utf8}\lst@inputenc
    \def\lstU@temp{%
      \scantokens\expandafter{\lstU@text}%
    }%
  \else
    \def\lstU@temp{%
      \input{#1}%
    }%
  \fi
  \lstU@temp
}
%    \end{macrocode}
%    \end{macro}
%
% \subsubsection{Convert CR/LF pairs to LF}
%
%    \begin{macro}{\lstU@CRLFtoLF}
%    \begin{macrocode}
\begingroup
  \endlinechar=-1 %
  \@makeother\^^J %
  \@makeother\^^M %
  \gdef\lstU@CRLFtoLF#1{%
    \edef#1{%
      \expandafter\lstU@CRLFtoLF@aux#1^^M^^J\@nil
    }%
  }%
  \gdef\lstU@CRLFtoLF@aux#1^^M^^J#2\@nil{%
    #1%
    \ifx\relax#2\relax
      \@car
    \fi
    ^^J%
    \lstU@CRLFtoLF@aux#2\@nil
  }%
\endgroup %
%    \end{macrocode}
%    \end{macro}
%
% \subsubsection{Patch \cs{lst@InputListing}}
%
%    \begin{macrocode}
\def\lstU@temp#1\def\lst@next#2#3\@nil{%
  \def\lst@InputListing##1{%
    #1%
    \def\lst@next{\lstU@input{##1}}%
    #3%
  }%
}
\expandafter\lstU@temp\lst@InputListing{#1}\@nil
%    \end{macrocode}
%
%    \begin{macrocode}
\lstU@AtEnd%
%</package>
%    \end{macrocode}
%
% \section{Test}
%
% \subsection{Catcode checks for loading}
%
%    \begin{macrocode}
%<*test1>
%    \end{macrocode}
%    \begin{macrocode}
\NeedsTeXFormat{LaTeX2e}
\documentclass{minimal}
\makeatletter
\def\RestoreCatcodes{}
\count@=0 %
\loop
  \edef\RestoreCatcodes{%
    \RestoreCatcodes
    \catcode\the\count@=\the\catcode\count@\relax
  }%
\ifnum\count@<255 %
  \advance\count@\@ne
\repeat

\def\RangeCatcodeInvalid#1#2{%
  \count@=#1\relax
  \loop
    \catcode\count@=15 %
  \ifnum\count@<#2\relax
    \advance\count@\@ne
  \repeat
}
\def\Test{%
  \RangeCatcodeInvalid{0}{47}%
  \RangeCatcodeInvalid{58}{64}%
  \RangeCatcodeInvalid{91}{96}%
  \RangeCatcodeInvalid{123}{127}%
  \catcode`\@=12 %
  \catcode`\\=0 %
  \catcode`\{=1 %
  \catcode`\}=2 %
  \catcode`\#=6 %
  \catcode`\[=12 %
  \catcode`\]=12 %
  \catcode`\%=14 %
  \catcode`\ =10 %
  \catcode13=5 %
  \RequirePackage{listingsutf8}[2011/11/10]\relax
  \RestoreCatcodes
}
\Test
\csname @@end\endcsname
\end
%    \end{macrocode}
%    \begin{macrocode}
%</test1>
%    \end{macrocode}
%
% \subsection{Test example for latin1}
%
%    \begin{macrocode}
%<*test2>
%    \end{macrocode}
%    \begin{macrocode}
\NeedsTeXFormat{LaTeX2e}
\documentclass{minimal}
\usepackage{filecontents}
\def\do#1{%
  \ifx#1\^%
  \else
    \noexpand\do\noexpand#1%
  \fi
}
\expandafter\let\expandafter\dospecials\expandafter\empty
\expandafter\edef\expandafter\dospecials\expandafter{\dospecials}
\begin{filecontents*}{ExampleUTF8.java}
public class ExampleUTF8 {
    public static String testString =
        "Umlauts: " +
        "^^c3^^84^^c3^^96^^c3^^9c^^c3^^a4^^c3^^b6^^c3^^bc^^c3^^9f";
    public static void main(String[] args) {
        System.out.println(testString);
    }
}
\end{filecontents*}
\usepackage{listingsutf8}[2011/11/10]
\def\Text{%
  Umlauts: %
  ^^c3^^84^^c3^^96^^c3^^9c^^c3^^a4^^c3^^b6^^c3^^bc^^c3^^9f%
}
\begin{document}
\lstinputlisting[%
  language=Java,%
  inputencoding=utf8/latin1,%
]{ExampleUTF8.java}
\end{document}
%</test2>
%    \end{macrocode}
%
% \section{Installation}
%
% \subsection{Download}
%
% \paragraph{Package.} This package is available on
% CTAN\footnote{\url{ftp://ftp.ctan.org/tex-archive/}}:
% \begin{description}
% \item[\CTAN{macros/latex/contrib/oberdiek/listingsutf8.dtx}] The source file.
% \item[\CTAN{macros/latex/contrib/oberdiek/listingsutf8.pdf}] Documentation.
% \end{description}
%
%
% \paragraph{Bundle.} All the packages of the bundle `oberdiek'
% are also available in a TDS compliant ZIP archive. There
% the packages are already unpacked and the documentation files
% are generated. The files and directories obey the TDS standard.
% \begin{description}
% \item[\CTAN{install/macros/latex/contrib/oberdiek.tds.zip}]
% \end{description}
% \emph{TDS} refers to the standard ``A Directory Structure
% for \TeX\ Files'' (\CTAN{tds/tds.pdf}). Directories
% with \xfile{texmf} in their name are usually organized this way.
%
% \subsection{Bundle installation}
%
% \paragraph{Unpacking.} Unpack the \xfile{oberdiek.tds.zip} in the
% TDS tree (also known as \xfile{texmf} tree) of your choice.
% Example (linux):
% \begin{quote}
%   |unzip oberdiek.tds.zip -d ~/texmf|
% \end{quote}
%
% \paragraph{Script installation.}
% Check the directory \xfile{TDS:scripts/oberdiek/} for
% scripts that need further installation steps.
% Package \xpackage{attachfile2} comes with the Perl script
% \xfile{pdfatfi.pl} that should be installed in such a way
% that it can be called as \texttt{pdfatfi}.
% Example (linux):
% \begin{quote}
%   |chmod +x scripts/oberdiek/pdfatfi.pl|\\
%   |cp scripts/oberdiek/pdfatfi.pl /usr/local/bin/|
% \end{quote}
%
% \subsection{Package installation}
%
% \paragraph{Unpacking.} The \xfile{.dtx} file is a self-extracting
% \docstrip\ archive. The files are extracted by running the
% \xfile{.dtx} through \plainTeX:
% \begin{quote}
%   \verb|tex listingsutf8.dtx|
% \end{quote}
%
% \paragraph{TDS.} Now the different files must be moved into
% the different directories in your installation TDS tree
% (also known as \xfile{texmf} tree):
% \begin{quote}
% \def\t{^^A
% \begin{tabular}{@{}>{\ttfamily}l@{ $\rightarrow$ }>{\ttfamily}l@{}}
%   listingsutf8.sty & tex/latex/oberdiek/listingsutf8.sty\\
%   listingsutf8.pdf & doc/latex/oberdiek/listingsutf8.pdf\\
%   test/listingsutf8-test1.tex & doc/latex/oberdiek/test/listingsutf8-test1.tex\\
%   test/listingsutf8-test2.tex & doc/latex/oberdiek/test/listingsutf8-test2.tex\\
%   test/listingsutf8-test3.tex & doc/latex/oberdiek/test/listingsutf8-test3.tex\\
%   test/listingsutf8-test4.tex & doc/latex/oberdiek/test/listingsutf8-test4.tex\\
%   test/listingsutf8-test5.tex & doc/latex/oberdiek/test/listingsutf8-test5.tex\\
%   listingsutf8.dtx & source/latex/oberdiek/listingsutf8.dtx\\
% \end{tabular}^^A
% }^^A
% \sbox0{\t}^^A
% \ifdim\wd0>\linewidth
%   \begingroup
%     \advance\linewidth by\leftmargin
%     \advance\linewidth by\rightmargin
%   \edef\x{\endgroup
%     \def\noexpand\lw{\the\linewidth}^^A
%   }\x
%   \def\lwbox{^^A
%     \leavevmode
%     \hbox to \linewidth{^^A
%       \kern-\leftmargin\relax
%       \hss
%       \usebox0
%       \hss
%       \kern-\rightmargin\relax
%     }^^A
%   }^^A
%   \ifdim\wd0>\lw
%     \sbox0{\small\t}^^A
%     \ifdim\wd0>\linewidth
%       \ifdim\wd0>\lw
%         \sbox0{\footnotesize\t}^^A
%         \ifdim\wd0>\linewidth
%           \ifdim\wd0>\lw
%             \sbox0{\scriptsize\t}^^A
%             \ifdim\wd0>\linewidth
%               \ifdim\wd0>\lw
%                 \sbox0{\tiny\t}^^A
%                 \ifdim\wd0>\linewidth
%                   \lwbox
%                 \else
%                   \usebox0
%                 \fi
%               \else
%                 \lwbox
%               \fi
%             \else
%               \usebox0
%             \fi
%           \else
%             \lwbox
%           \fi
%         \else
%           \usebox0
%         \fi
%       \else
%         \lwbox
%       \fi
%     \else
%       \usebox0
%     \fi
%   \else
%     \lwbox
%   \fi
% \else
%   \usebox0
% \fi
% \end{quote}
% If you have a \xfile{docstrip.cfg} that configures and enables \docstrip's
% TDS installing feature, then some files can already be in the right
% place, see the documentation of \docstrip.
%
% \subsection{Refresh file name databases}
%
% If your \TeX~distribution
% (\teTeX, \mikTeX, \dots) relies on file name databases, you must refresh
% these. For example, \teTeX\ users run \verb|texhash| or
% \verb|mktexlsr|.
%
% \subsection{Some details for the interested}
%
% \paragraph{Attached source.}
%
% The PDF documentation on CTAN also includes the
% \xfile{.dtx} source file. It can be extracted by
% AcrobatReader 6 or higher. Another option is \textsf{pdftk},
% e.g. unpack the file into the current directory:
% \begin{quote}
%   \verb|pdftk listingsutf8.pdf unpack_files output .|
% \end{quote}
%
% \paragraph{Unpacking with \LaTeX.}
% The \xfile{.dtx} chooses its action depending on the format:
% \begin{description}
% \item[\plainTeX:] Run \docstrip\ and extract the files.
% \item[\LaTeX:] Generate the documentation.
% \end{description}
% If you insist on using \LaTeX\ for \docstrip\ (really,
% \docstrip\ does not need \LaTeX), then inform the autodetect routine
% about your intention:
% \begin{quote}
%   \verb|latex \let\install=y% \iffalse meta-comment
%
% File: listingsutf8.dtx
% Version: 2011/11/10 v1.2
% Info: Allow UTF-8 in listings input
%
% Copyright (C) 2007, 2011 by
%    Heiko Oberdiek <heiko.oberdiek at googlemail.com>
%
% This work may be distributed and/or modified under the
% conditions of the LaTeX Project Public License, either
% version 1.3c of this license or (at your option) any later
% version. This version of this license is in
%    http://www.latex-project.org/lppl/lppl-1-3c.txt
% and the latest version of this license is in
%    http://www.latex-project.org/lppl.txt
% and version 1.3 or later is part of all distributions of
% LaTeX version 2005/12/01 or later.
%
% This work has the LPPL maintenance status "maintained".
%
% This Current Maintainer of this work is Heiko Oberdiek.
%
% This work consists of the main source file listingsutf8.dtx
% and the derived files
%    listingsutf8.sty, listingsutf8.pdf, listingsutf8.ins, listingsutf8.drv,
%    listingsutf8-test1.tex, listingsutf8-test2.tex,
%    listingsutf8-test3.tex, listingsutf8-test4.tex,
%    listingsutf8-test5.tex.
%
% Distribution:
%    CTAN:macros/latex/contrib/oberdiek/listingsutf8.dtx
%    CTAN:macros/latex/contrib/oberdiek/listingsutf8.pdf
%
% Unpacking:
%    (a) If listingsutf8.ins is present:
%           tex listingsutf8.ins
%    (b) Without listingsutf8.ins:
%           tex listingsutf8.dtx
%    (c) If you insist on using LaTeX
%           latex \let\install=y\input{listingsutf8.dtx}
%        (quote the arguments according to the demands of your shell)
%
% Documentation:
%    (a) If listingsutf8.drv is present:
%           latex listingsutf8.drv
%    (b) Without listingsutf8.drv:
%           latex listingsutf8.dtx; ...
%    The class ltxdoc loads the configuration file ltxdoc.cfg
%    if available. Here you can specify further options, e.g.
%    use A4 as paper format:
%       \PassOptionsToClass{a4paper}{article}
%
%    Programm calls to get the documentation (example):
%       pdflatex listingsutf8.dtx
%       makeindex -s gind.ist listingsutf8.idx
%       pdflatex listingsutf8.dtx
%       makeindex -s gind.ist listingsutf8.idx
%       pdflatex listingsutf8.dtx
%
% Installation:
%    TDS:tex/latex/oberdiek/listingsutf8.sty
%    TDS:doc/latex/oberdiek/listingsutf8.pdf
%    TDS:doc/latex/oberdiek/test/listingsutf8-test1.tex
%    TDS:doc/latex/oberdiek/test/listingsutf8-test2.tex
%    TDS:doc/latex/oberdiek/test/listingsutf8-test3.tex
%    TDS:doc/latex/oberdiek/test/listingsutf8-test4.tex
%    TDS:doc/latex/oberdiek/test/listingsutf8-test5.tex
%    TDS:source/latex/oberdiek/listingsutf8.dtx
%
%<*ignore>
\begingroup
  \catcode123=1 %
  \catcode125=2 %
  \def\x{LaTeX2e}%
\expandafter\endgroup
\ifcase 0\ifx\install y1\fi\expandafter
         \ifx\csname processbatchFile\endcsname\relax\else1\fi
         \ifx\fmtname\x\else 1\fi\relax
\else\csname fi\endcsname
%</ignore>
%<*install>
\input docstrip.tex
\Msg{************************************************************************}
\Msg{* Installation}
\Msg{* Package: listingsutf8 2011/11/10 v1.2 Allow UTF-8 in listings input (HO)}
\Msg{************************************************************************}

\keepsilent
\askforoverwritefalse

\let\MetaPrefix\relax
\preamble

This is a generated file.

Project: listingsutf8
Version: 2011/11/10 v1.2

Copyright (C) 2007, 2011 by
   Heiko Oberdiek <heiko.oberdiek at googlemail.com>

This work may be distributed and/or modified under the
conditions of the LaTeX Project Public License, either
version 1.3c of this license or (at your option) any later
version. This version of this license is in
   http://www.latex-project.org/lppl/lppl-1-3c.txt
and the latest version of this license is in
   http://www.latex-project.org/lppl.txt
and version 1.3 or later is part of all distributions of
LaTeX version 2005/12/01 or later.

This work has the LPPL maintenance status "maintained".

This Current Maintainer of this work is Heiko Oberdiek.

This work consists of the main source file listingsutf8.dtx
and the derived files
   listingsutf8.sty, listingsutf8.pdf, listingsutf8.ins, listingsutf8.drv,
   listingsutf8-test1.tex, listingsutf8-test2.tex,
   listingsutf8-test3.tex, listingsutf8-test4.tex,
   listingsutf8-test5.tex.

\endpreamble
\let\MetaPrefix\DoubleperCent

\generate{%
  \file{listingsutf8.ins}{\from{listingsutf8.dtx}{install}}%
  \file{listingsutf8.drv}{\from{listingsutf8.dtx}{driver}}%
  \usedir{tex/latex/oberdiek}%
  \file{listingsutf8.sty}{\from{listingsutf8.dtx}{package}}%
  \usedir{doc/latex/oberdiek/test}%
  \file{listingsutf8-test1.tex}{\from{listingsutf8.dtx}{test1}}%
  \file{listingsutf8-test2.tex}{\from{listingsutf8.dtx}{test2,utf8}}%
  \file{listingsutf8-test3.tex}{\from{listingsutf8.dtx}{test3,utf8x}}%
  \file{listingsutf8-test4.tex}{\from{listingsutf8.dtx}{test4,utf8,noetex}}%
  \file{listingsutf8-test5.tex}{\from{listingsutf8.dtx}{test5,utf8x,noetex}}%
  \nopreamble
  \nopostamble
  \usedir{source/latex/oberdiek/catalogue}%
  \file{listingsutf8.xml}{\from{listingsutf8.dtx}{catalogue}}%
}

\catcode32=13\relax% active space
\let =\space%
\Msg{************************************************************************}
\Msg{*}
\Msg{* To finish the installation you have to move the following}
\Msg{* file into a directory searched by TeX:}
\Msg{*}
\Msg{*     listingsutf8.sty}
\Msg{*}
\Msg{* To produce the documentation run the file `listingsutf8.drv'}
\Msg{* through LaTeX.}
\Msg{*}
\Msg{* Happy TeXing!}
\Msg{*}
\Msg{************************************************************************}

\endbatchfile
%</install>
%<*ignore>
\fi
%</ignore>
%<*driver>
\NeedsTeXFormat{LaTeX2e}
\ProvidesFile{listingsutf8.drv}%
  [2011/11/10 v1.2 Allow UTF-8 in listings input (HO)]%
\documentclass{ltxdoc}
\usepackage{holtxdoc}[2011/11/22]
\begin{document}
  \DocInput{listingsutf8.dtx}%
\end{document}
%</driver>
% \fi
%
% \CheckSum{311}
%
% \CharacterTable
%  {Upper-case    \A\B\C\D\E\F\G\H\I\J\K\L\M\N\O\P\Q\R\S\T\U\V\W\X\Y\Z
%   Lower-case    \a\b\c\d\e\f\g\h\i\j\k\l\m\n\o\p\q\r\s\t\u\v\w\x\y\z
%   Digits        \0\1\2\3\4\5\6\7\8\9
%   Exclamation   \!     Double quote  \"     Hash (number) \#
%   Dollar        \$     Percent       \%     Ampersand     \&
%   Acute accent  \'     Left paren    \(     Right paren   \)
%   Asterisk      \*     Plus          \+     Comma         \,
%   Minus         \-     Point         \.     Solidus       \/
%   Colon         \:     Semicolon     \;     Less than     \<
%   Equals        \=     Greater than  \>     Question mark \?
%   Commercial at \@     Left bracket  \[     Backslash     \\
%   Right bracket \]     Circumflex    \^     Underscore    \_
%   Grave accent  \`     Left brace    \{     Vertical bar  \|
%   Right brace   \}     Tilde         \~}
%
% \GetFileInfo{listingsutf8.drv}
%
% \title{The \xpackage{listingsutf8} package}
% \date{2011/11/10 v1.2}
% \author{Heiko Oberdiek\\\xemail{heiko.oberdiek at googlemail.com}}
%
% \maketitle
%
% \begin{abstract}
% Package \xpackage{listings} does not support files with multi-byte
% encodings such as UTF-8. In case of \cs{lstinputlisting} a simple
% workaround is possible if an one-byte encoding exists that the file
% can be converted to. Also \eTeX\ and \pdfTeX\ regardless of its mode
% are required.
% \end{abstract}
%
% \tableofcontents
%
% \section{Documentation}
%
% \subsection{User interface}
%
% Load this package after or instead of package \xpackage{listings}
% \cite{listings}.
% The package does not define own options and passes given options to
% package \xpackage{listings}.
%
% The syntax of package \xpackage{listings}' key \xoption{inputencoding}
% is extended:
% \begin{quote}
%   |inputencoding=utf8/|\meta{one-byte-encoding}\\
%   Example: |inputencoding=utf8/latin1|
% \end{quote}
% That means the file is encoded in UTF-8 and can
% be converted to the given \meta{one-byte-encoding}.
% The available encodings for \meta{one-byte-encoding} are
% listed in section ``1.2 Supported encodings'' of
% package \xpackage{stringenc}'s documentation \cite{stringenc}.
% Of course, the encoding must encode its characters with
% one byte exactly. This excludes the unicode encodings
% (\xoption{utf8}, \xoption{utf16}, \dots).
%
% Only \cs{lstinputlisting} is supported by the syntax extension
% of key \xoption{inputencoding}.
%
% Internally package \xpackage{listingsutf8} reads the file as binary file
% via primitives of \pdfTeX\ (\cs{pdffiledump}). Then the file
% contents is converted as string using package \xpackage{stringenc} and
% finally the string is read as virtual file by \eTeX's \cs{scantokens}.
%
% \subsection{Future}
%
% Workarounds are not provided for
% \begin{itemize}
% \item \cs{lstinline}
% \item Environment |lstlisting|.
% \item Environments defined by \cs{lstnewenvironment}.
% \end{itemize}
% Perhaps someone will find time to extend package \xpackage{listings}
% with full native support for UTF-8. Then this package would become obsolete.
%
% \StopEventually{
% }
%
% \section{Implementation}
%
%    \begin{macrocode}
%<*package>
%    \end{macrocode}
%
% \subsection{Catcodes and identification}
%
%    \begin{macrocode}
\begingroup\catcode61\catcode48\catcode32=10\relax%
  \catcode13=5 % ^^M
  \endlinechar=13 %
  \catcode123=1 % {
  \catcode125=2 % }
  \catcode64=11 % @
  \def\x{\endgroup
    \expandafter\edef\csname lstU@AtEnd\endcsname{%
      \endlinechar=\the\endlinechar\relax
      \catcode13=\the\catcode13\relax
      \catcode32=\the\catcode32\relax
      \catcode35=\the\catcode35\relax
      \catcode61=\the\catcode61\relax
      \catcode64=\the\catcode64\relax
      \catcode123=\the\catcode123\relax
      \catcode125=\the\catcode125\relax
    }%
  }%
\x\catcode61\catcode48\catcode32=10\relax%
\catcode13=5 % ^^M
\endlinechar=13 %
\catcode35=6 % #
\catcode64=11 % @
\catcode123=1 % {
\catcode125=2 % }
\def\TMP@EnsureCode#1#2{%
  \edef\lstU@AtEnd{%
    \lstU@AtEnd
    \catcode#1=\the\catcode#1\relax
  }%
  \catcode#1=#2\relax
}
\TMP@EnsureCode{10}{12}% ^^J
\TMP@EnsureCode{33}{12}% !
\TMP@EnsureCode{36}{3}% $
\TMP@EnsureCode{38}{4}% &
\TMP@EnsureCode{39}{12}% '
\TMP@EnsureCode{40}{12}% (
\TMP@EnsureCode{41}{12}% )
\TMP@EnsureCode{42}{12}% *
\TMP@EnsureCode{43}{12}% +
\TMP@EnsureCode{44}{12}% ,
\TMP@EnsureCode{45}{12}% -
\TMP@EnsureCode{46}{12}% .
\TMP@EnsureCode{47}{12}% /
\TMP@EnsureCode{58}{12}% :
\TMP@EnsureCode{60}{12}% <
\TMP@EnsureCode{62}{12}% >
\TMP@EnsureCode{91}{12}% [
\TMP@EnsureCode{93}{12}% ]
\TMP@EnsureCode{94}{7}% ^ (superscript)
\TMP@EnsureCode{95}{8}% _ (subscript)
\TMP@EnsureCode{96}{12}% `
\TMP@EnsureCode{124}{12}% |
\TMP@EnsureCode{126}{13}% ~ (active)
\edef\lstU@AtEnd{\lstU@AtEnd\noexpand\endinput}
%    \end{macrocode}
%
%    Package identification.
%    \begin{macrocode}
\NeedsTeXFormat{LaTeX2e}
\ProvidesPackage{listingsutf8}%
  [2011/11/10 v1.2 Allow UTF-8 in listings input (HO)]
%    \end{macrocode}
%
% \subsection{Package options}
%
% Just pass options to package listings.
%
%    \begin{macrocode}
\DeclareOption*{%
  \PassOptionsToPackage\CurrentOption{listings}%
}
\ProcessOptions*
%    \end{macrocode}
%    Key \xoption{inputencoding} was introduced in version
%    2002/04/01 v1.0 of package \xpackage{listings}.
%    \begin{macrocode}
\RequirePackage{listings}[2002/04/01]
%    \end{macrocode}
%    Ensure that \cs{inputencoding} is provided.
%    \begin{macrocode}
\AtBeginDocument{%
  \@ifundefined{inputencoding}{%
    \RequirePackage{inputenc}%
  }{}%
}
%    \end{macrocode}
%
% \subsection{Check prerequisites}
%
%    \begin{macrocode}
\RequirePackage{pdftexcmds}[2011/04/22]
%    \end{macrocode}
%
%    \begin{macrocode}
\def\lstU@temp#1#2{%
  \begingroup\expandafter\expandafter\expandafter\endgroup
  \expandafter\ifx\csname #1\endcsname\relax
    \PackageWarningNoLine{listingsutf8}{%
      Package loading is aborted because of missing %
      \@backslashchar#1.\MessageBreak
      #2%
    }%
    \expandafter\lstU@AtEnd
  \fi
}
\lstU@temp{scantokens}{It is provided by e-TeX}%
\lstU@temp{pdf@unescapehex}{It is provided by pdfTeX >= 1.30}%
\lstU@temp{pdf@filedump}{It is provided by pdfTeX >= 1.30}%
\lstU@temp{pdf@filesize}{It is provided by pdfTeX >= 1.30}%
%    \end{macrocode}
%
%    \begin{macrocode}
\RequirePackage{stringenc}[2010/03/01]
%    \end{macrocode}
%
% \subsection{Add support for UTF-8}
%
%    \begin{macro}{\iflstU@utfviii}
%    \begin{macrocode}
\newif\iflstU@utfviii
%    \end{macrocode}
%    \end{macro}
%
%    \begin{macro}{\lstU@inputenc}
%    \begin{macrocode}
\def\lstU@inputenc#1{%
  \expandafter\lstU@@inputenc#1utf8/utf8/\@nil
}
%    \end{macrocode}
%    \end{macro}
%    \begin{macro}{\lstU@@inputenc}
\def\lstU@@inputenc#1utf8/#2utf8/#3\@nil{%
  \ifx\\#1\\%
    \lstU@utfviiitrue
    \def\lst@inputenc{#2}%
  \else
    \lstU@utfviiifalse
  \fi
}
%    \begin{macrocode}
%    \end{macrocode}
%    \end{macro}
%
%    \begin{macrocode}
\lst@Key{inputencoding}\relax{%
  \def\lst@inputenc{#1}%
  \lstU@inputenc{#1}%
}
%    \end{macrocode}
%
% \subsubsection{Conversion}
%
%    \begin{macro}{\lstU@input}
%    \begin{macrocode}
\def\lstU@input#1{%
  \iflstU@utfviii
    \edef\lstU@text{%
      \pdf@unescapehex{%
        \pdf@filedump{0}{\pdf@filesize{#1}}{#1}%
      }%
    }%
    \lstU@CRLFtoLF\lstU@text
    \StringEncodingConvert\lstU@text\lstU@text{utf8}\lst@inputenc
    \def\lstU@temp{%
      \scantokens\expandafter{\lstU@text}%
    }%
  \else
    \def\lstU@temp{%
      \input{#1}%
    }%
  \fi
  \lstU@temp
}
%    \end{macrocode}
%    \end{macro}
%
% \subsubsection{Convert CR/LF pairs to LF}
%
%    \begin{macro}{\lstU@CRLFtoLF}
%    \begin{macrocode}
\begingroup
  \endlinechar=-1 %
  \@makeother\^^J %
  \@makeother\^^M %
  \gdef\lstU@CRLFtoLF#1{%
    \edef#1{%
      \expandafter\lstU@CRLFtoLF@aux#1^^M^^J\@nil
    }%
  }%
  \gdef\lstU@CRLFtoLF@aux#1^^M^^J#2\@nil{%
    #1%
    \ifx\relax#2\relax
      \@car
    \fi
    ^^J%
    \lstU@CRLFtoLF@aux#2\@nil
  }%
\endgroup %
%    \end{macrocode}
%    \end{macro}
%
% \subsubsection{Patch \cs{lst@InputListing}}
%
%    \begin{macrocode}
\def\lstU@temp#1\def\lst@next#2#3\@nil{%
  \def\lst@InputListing##1{%
    #1%
    \def\lst@next{\lstU@input{##1}}%
    #3%
  }%
}
\expandafter\lstU@temp\lst@InputListing{#1}\@nil
%    \end{macrocode}
%
%    \begin{macrocode}
\lstU@AtEnd%
%</package>
%    \end{macrocode}
%
% \section{Test}
%
% \subsection{Catcode checks for loading}
%
%    \begin{macrocode}
%<*test1>
%    \end{macrocode}
%    \begin{macrocode}
\NeedsTeXFormat{LaTeX2e}
\documentclass{minimal}
\makeatletter
\def\RestoreCatcodes{}
\count@=0 %
\loop
  \edef\RestoreCatcodes{%
    \RestoreCatcodes
    \catcode\the\count@=\the\catcode\count@\relax
  }%
\ifnum\count@<255 %
  \advance\count@\@ne
\repeat

\def\RangeCatcodeInvalid#1#2{%
  \count@=#1\relax
  \loop
    \catcode\count@=15 %
  \ifnum\count@<#2\relax
    \advance\count@\@ne
  \repeat
}
\def\Test{%
  \RangeCatcodeInvalid{0}{47}%
  \RangeCatcodeInvalid{58}{64}%
  \RangeCatcodeInvalid{91}{96}%
  \RangeCatcodeInvalid{123}{127}%
  \catcode`\@=12 %
  \catcode`\\=0 %
  \catcode`\{=1 %
  \catcode`\}=2 %
  \catcode`\#=6 %
  \catcode`\[=12 %
  \catcode`\]=12 %
  \catcode`\%=14 %
  \catcode`\ =10 %
  \catcode13=5 %
  \RequirePackage{listingsutf8}[2011/11/10]\relax
  \RestoreCatcodes
}
\Test
\csname @@end\endcsname
\end
%    \end{macrocode}
%    \begin{macrocode}
%</test1>
%    \end{macrocode}
%
% \subsection{Test example for latin1}
%
%    \begin{macrocode}
%<*test2>
%    \end{macrocode}
%    \begin{macrocode}
\NeedsTeXFormat{LaTeX2e}
\documentclass{minimal}
\usepackage{filecontents}
\def\do#1{%
  \ifx#1\^%
  \else
    \noexpand\do\noexpand#1%
  \fi
}
\expandafter\let\expandafter\dospecials\expandafter\empty
\expandafter\edef\expandafter\dospecials\expandafter{\dospecials}
\begin{filecontents*}{ExampleUTF8.java}
public class ExampleUTF8 {
    public static String testString =
        "Umlauts: " +
        "^^c3^^84^^c3^^96^^c3^^9c^^c3^^a4^^c3^^b6^^c3^^bc^^c3^^9f";
    public static void main(String[] args) {
        System.out.println(testString);
    }
}
\end{filecontents*}
\usepackage{listingsutf8}[2011/11/10]
\def\Text{%
  Umlauts: %
  ^^c3^^84^^c3^^96^^c3^^9c^^c3^^a4^^c3^^b6^^c3^^bc^^c3^^9f%
}
\begin{document}
\lstinputlisting[%
  language=Java,%
  inputencoding=utf8/latin1,%
]{ExampleUTF8.java}
\end{document}
%</test2>
%    \end{macrocode}
%
% \section{Installation}
%
% \subsection{Download}
%
% \paragraph{Package.} This package is available on
% CTAN\footnote{\url{ftp://ftp.ctan.org/tex-archive/}}:
% \begin{description}
% \item[\CTAN{macros/latex/contrib/oberdiek/listingsutf8.dtx}] The source file.
% \item[\CTAN{macros/latex/contrib/oberdiek/listingsutf8.pdf}] Documentation.
% \end{description}
%
%
% \paragraph{Bundle.} All the packages of the bundle `oberdiek'
% are also available in a TDS compliant ZIP archive. There
% the packages are already unpacked and the documentation files
% are generated. The files and directories obey the TDS standard.
% \begin{description}
% \item[\CTAN{install/macros/latex/contrib/oberdiek.tds.zip}]
% \end{description}
% \emph{TDS} refers to the standard ``A Directory Structure
% for \TeX\ Files'' (\CTAN{tds/tds.pdf}). Directories
% with \xfile{texmf} in their name are usually organized this way.
%
% \subsection{Bundle installation}
%
% \paragraph{Unpacking.} Unpack the \xfile{oberdiek.tds.zip} in the
% TDS tree (also known as \xfile{texmf} tree) of your choice.
% Example (linux):
% \begin{quote}
%   |unzip oberdiek.tds.zip -d ~/texmf|
% \end{quote}
%
% \paragraph{Script installation.}
% Check the directory \xfile{TDS:scripts/oberdiek/} for
% scripts that need further installation steps.
% Package \xpackage{attachfile2} comes with the Perl script
% \xfile{pdfatfi.pl} that should be installed in such a way
% that it can be called as \texttt{pdfatfi}.
% Example (linux):
% \begin{quote}
%   |chmod +x scripts/oberdiek/pdfatfi.pl|\\
%   |cp scripts/oberdiek/pdfatfi.pl /usr/local/bin/|
% \end{quote}
%
% \subsection{Package installation}
%
% \paragraph{Unpacking.} The \xfile{.dtx} file is a self-extracting
% \docstrip\ archive. The files are extracted by running the
% \xfile{.dtx} through \plainTeX:
% \begin{quote}
%   \verb|tex listingsutf8.dtx|
% \end{quote}
%
% \paragraph{TDS.} Now the different files must be moved into
% the different directories in your installation TDS tree
% (also known as \xfile{texmf} tree):
% \begin{quote}
% \def\t{^^A
% \begin{tabular}{@{}>{\ttfamily}l@{ $\rightarrow$ }>{\ttfamily}l@{}}
%   listingsutf8.sty & tex/latex/oberdiek/listingsutf8.sty\\
%   listingsutf8.pdf & doc/latex/oberdiek/listingsutf8.pdf\\
%   test/listingsutf8-test1.tex & doc/latex/oberdiek/test/listingsutf8-test1.tex\\
%   test/listingsutf8-test2.tex & doc/latex/oberdiek/test/listingsutf8-test2.tex\\
%   test/listingsutf8-test3.tex & doc/latex/oberdiek/test/listingsutf8-test3.tex\\
%   test/listingsutf8-test4.tex & doc/latex/oberdiek/test/listingsutf8-test4.tex\\
%   test/listingsutf8-test5.tex & doc/latex/oberdiek/test/listingsutf8-test5.tex\\
%   listingsutf8.dtx & source/latex/oberdiek/listingsutf8.dtx\\
% \end{tabular}^^A
% }^^A
% \sbox0{\t}^^A
% \ifdim\wd0>\linewidth
%   \begingroup
%     \advance\linewidth by\leftmargin
%     \advance\linewidth by\rightmargin
%   \edef\x{\endgroup
%     \def\noexpand\lw{\the\linewidth}^^A
%   }\x
%   \def\lwbox{^^A
%     \leavevmode
%     \hbox to \linewidth{^^A
%       \kern-\leftmargin\relax
%       \hss
%       \usebox0
%       \hss
%       \kern-\rightmargin\relax
%     }^^A
%   }^^A
%   \ifdim\wd0>\lw
%     \sbox0{\small\t}^^A
%     \ifdim\wd0>\linewidth
%       \ifdim\wd0>\lw
%         \sbox0{\footnotesize\t}^^A
%         \ifdim\wd0>\linewidth
%           \ifdim\wd0>\lw
%             \sbox0{\scriptsize\t}^^A
%             \ifdim\wd0>\linewidth
%               \ifdim\wd0>\lw
%                 \sbox0{\tiny\t}^^A
%                 \ifdim\wd0>\linewidth
%                   \lwbox
%                 \else
%                   \usebox0
%                 \fi
%               \else
%                 \lwbox
%               \fi
%             \else
%               \usebox0
%             \fi
%           \else
%             \lwbox
%           \fi
%         \else
%           \usebox0
%         \fi
%       \else
%         \lwbox
%       \fi
%     \else
%       \usebox0
%     \fi
%   \else
%     \lwbox
%   \fi
% \else
%   \usebox0
% \fi
% \end{quote}
% If you have a \xfile{docstrip.cfg} that configures and enables \docstrip's
% TDS installing feature, then some files can already be in the right
% place, see the documentation of \docstrip.
%
% \subsection{Refresh file name databases}
%
% If your \TeX~distribution
% (\teTeX, \mikTeX, \dots) relies on file name databases, you must refresh
% these. For example, \teTeX\ users run \verb|texhash| or
% \verb|mktexlsr|.
%
% \subsection{Some details for the interested}
%
% \paragraph{Attached source.}
%
% The PDF documentation on CTAN also includes the
% \xfile{.dtx} source file. It can be extracted by
% AcrobatReader 6 or higher. Another option is \textsf{pdftk},
% e.g. unpack the file into the current directory:
% \begin{quote}
%   \verb|pdftk listingsutf8.pdf unpack_files output .|
% \end{quote}
%
% \paragraph{Unpacking with \LaTeX.}
% The \xfile{.dtx} chooses its action depending on the format:
% \begin{description}
% \item[\plainTeX:] Run \docstrip\ and extract the files.
% \item[\LaTeX:] Generate the documentation.
% \end{description}
% If you insist on using \LaTeX\ for \docstrip\ (really,
% \docstrip\ does not need \LaTeX), then inform the autodetect routine
% about your intention:
% \begin{quote}
%   \verb|latex \let\install=y\input{listingsutf8.dtx}|
% \end{quote}
% Do not forget to quote the argument according to the demands
% of your shell.
%
% \paragraph{Generating the documentation.}
% You can use both the \xfile{.dtx} or the \xfile{.drv} to generate
% the documentation. The process can be configured by the
% configuration file \xfile{ltxdoc.cfg}. For instance, put this
% line into this file, if you want to have A4 as paper format:
% \begin{quote}
%   \verb|\PassOptionsToClass{a4paper}{article}|
% \end{quote}
% An example follows how to generate the
% documentation with pdf\LaTeX:
% \begin{quote}
%\begin{verbatim}
%pdflatex listingsutf8.dtx
%makeindex -s gind.ist listingsutf8.idx
%pdflatex listingsutf8.dtx
%makeindex -s gind.ist listingsutf8.idx
%pdflatex listingsutf8.dtx
%\end{verbatim}
% \end{quote}
%
% \section{Catalogue}
%
% The following XML file can be used as source for the
% \href{http://mirror.ctan.org/help/Catalogue/catalogue.html}{\TeX\ Catalogue}.
% The elements \texttt{caption} and \texttt{description} are imported
% from the original XML file from the Catalogue.
% The name of the XML file in the Catalogue is \xfile{listingsutf8.xml}.
%    \begin{macrocode}
%<*catalogue>
<?xml version='1.0' encoding='us-ascii'?>
<!DOCTYPE entry SYSTEM 'catalogue.dtd'>
<entry datestamp='$Date$' modifier='$Author$' id='listingsutf8'>
  <name>listingsutf8</name>
  <caption>Allow UTF-8 in listings input.</caption>
  <authorref id='auth:oberdiek'/>
  <copyright owner='Heiko Oberdiek' year='2007,2011'/>
  <license type='lppl1.3'/>
  <version number='1.2'/>
  <description>
    Package <xref refid='listings'>listings</xref> does not support files
    with multi-byte encodings such as UTF-8.  In the case of
    <tt>\lstinputlisting</tt>, a simple workaround is possible if a
    one-byte encoding exists that the file can be converted to.  The
    package requires the e-TeX extensions under pdfTeX (in either PDF
    or DVI output mode).
    <p/>
    The package is part of the <xref refid='oberdiek'>oberdiek</xref> bundle.
  </description>
  <documentation details='Package documentation'
      href='ctan:/macros/latex/contrib/oberdiek/listingsutf8.pdf'/>
  <ctan file='true' path='/macros/latex/contrib/oberdiek/listingsutf8.dtx'/>
  <miktex location='oberdiek'/>
  <texlive location='oberdiek'/>
  <install path='/macros/latex/contrib/oberdiek/oberdiek.tds.zip'/>
</entry>
%</catalogue>
%    \end{macrocode}
%
% \begin{thebibliography}{9}
%
% \bibitem{inputenc}
%   Alan Jeffrey, Frank Mittelbach,
%   \textit{inputenc.sty}, 2006/05/05 v1.1b.
%   \CTAN{macros/latex/base/inputenc.dtx}
%
% \bibitem{listings}
%   Carsten Heinz, Brooks Moses:
%  \textit{The \xpackage{listings} package};
%   2007/02/22;\\
%   \CTAN{macros/latex/contrib/listings/}.
%
% \bibitem{stringenc}
%   Heiko Oberdiek:
%   \textit{The \xpackage{stringenc} package};
%   2007/10/22;\\
%   \CTAN{macros/latex/contrib/oberdiek/stringenc.pdf}.
%
% \end{thebibliography}
%
% \begin{History}
%   \begin{Version}{2007/10/22 v1.0}
%   \item
%     First version.
%   \end{Version}
%   \begin{Version}{2007/11/11 v1.1}
%   \item
%     Use of package \xpackage{pdftexcmds}.
%   \end{Version}
%   \begin{Version}{2011/11/10 v1.2}
%   \item
%     DOS line ends CR/LF normalized to LF to avoid empty lines
%     (Bug report of Thomas Benkert in de.comp.text.tex).
%   \end{Version}
% \end{History}
%
% \PrintIndex
%
% \Finale
\endinput
|
% \end{quote}
% Do not forget to quote the argument according to the demands
% of your shell.
%
% \paragraph{Generating the documentation.}
% You can use both the \xfile{.dtx} or the \xfile{.drv} to generate
% the documentation. The process can be configured by the
% configuration file \xfile{ltxdoc.cfg}. For instance, put this
% line into this file, if you want to have A4 as paper format:
% \begin{quote}
%   \verb|\PassOptionsToClass{a4paper}{article}|
% \end{quote}
% An example follows how to generate the
% documentation with pdf\LaTeX:
% \begin{quote}
%\begin{verbatim}
%pdflatex listingsutf8.dtx
%makeindex -s gind.ist listingsutf8.idx
%pdflatex listingsutf8.dtx
%makeindex -s gind.ist listingsutf8.idx
%pdflatex listingsutf8.dtx
%\end{verbatim}
% \end{quote}
%
% \section{Catalogue}
%
% The following XML file can be used as source for the
% \href{http://mirror.ctan.org/help/Catalogue/catalogue.html}{\TeX\ Catalogue}.
% The elements \texttt{caption} and \texttt{description} are imported
% from the original XML file from the Catalogue.
% The name of the XML file in the Catalogue is \xfile{listingsutf8.xml}.
%    \begin{macrocode}
%<*catalogue>
<?xml version='1.0' encoding='us-ascii'?>
<!DOCTYPE entry SYSTEM 'catalogue.dtd'>
<entry datestamp='$Date$' modifier='$Author$' id='listingsutf8'>
  <name>listingsutf8</name>
  <caption>Allow UTF-8 in listings input.</caption>
  <authorref id='auth:oberdiek'/>
  <copyright owner='Heiko Oberdiek' year='2007,2011'/>
  <license type='lppl1.3'/>
  <version number='1.2'/>
  <description>
    Package <xref refid='listings'>listings</xref> does not support files
    with multi-byte encodings such as UTF-8.  In the case of
    <tt>\lstinputlisting</tt>, a simple workaround is possible if a
    one-byte encoding exists that the file can be converted to.  The
    package requires the e-TeX extensions under pdfTeX (in either PDF
    or DVI output mode).
    <p/>
    The package is part of the <xref refid='oberdiek'>oberdiek</xref> bundle.
  </description>
  <documentation details='Package documentation'
      href='ctan:/macros/latex/contrib/oberdiek/listingsutf8.pdf'/>
  <ctan file='true' path='/macros/latex/contrib/oberdiek/listingsutf8.dtx'/>
  <miktex location='oberdiek'/>
  <texlive location='oberdiek'/>
  <install path='/macros/latex/contrib/oberdiek/oberdiek.tds.zip'/>
</entry>
%</catalogue>
%    \end{macrocode}
%
% \begin{thebibliography}{9}
%
% \bibitem{inputenc}
%   Alan Jeffrey, Frank Mittelbach,
%   \textit{inputenc.sty}, 2006/05/05 v1.1b.
%   \CTAN{macros/latex/base/inputenc.dtx}
%
% \bibitem{listings}
%   Carsten Heinz, Brooks Moses:
%  \textit{The \xpackage{listings} package};
%   2007/02/22;\\
%   \CTAN{macros/latex/contrib/listings/}.
%
% \bibitem{stringenc}
%   Heiko Oberdiek:
%   \textit{The \xpackage{stringenc} package};
%   2007/10/22;\\
%   \CTAN{macros/latex/contrib/oberdiek/stringenc.pdf}.
%
% \end{thebibliography}
%
% \begin{History}
%   \begin{Version}{2007/10/22 v1.0}
%   \item
%     First version.
%   \end{Version}
%   \begin{Version}{2007/11/11 v1.1}
%   \item
%     Use of package \xpackage{pdftexcmds}.
%   \end{Version}
%   \begin{Version}{2011/11/10 v1.2}
%   \item
%     DOS line ends CR/LF normalized to LF to avoid empty lines
%     (Bug report of Thomas Benkert in de.comp.text.tex).
%   \end{Version}
% \end{History}
%
% \PrintIndex
%
% \Finale
\endinput

%        (quote the arguments according to the demands of your shell)
%
% Documentation:
%    (a) If listingsutf8.drv is present:
%           latex listingsutf8.drv
%    (b) Without listingsutf8.drv:
%           latex listingsutf8.dtx; ...
%    The class ltxdoc loads the configuration file ltxdoc.cfg
%    if available. Here you can specify further options, e.g.
%    use A4 as paper format:
%       \PassOptionsToClass{a4paper}{article}
%
%    Programm calls to get the documentation (example):
%       pdflatex listingsutf8.dtx
%       makeindex -s gind.ist listingsutf8.idx
%       pdflatex listingsutf8.dtx
%       makeindex -s gind.ist listingsutf8.idx
%       pdflatex listingsutf8.dtx
%
% Installation:
%    TDS:tex/latex/oberdiek/listingsutf8.sty
%    TDS:doc/latex/oberdiek/listingsutf8.pdf
%    TDS:doc/latex/oberdiek/test/listingsutf8-test1.tex
%    TDS:doc/latex/oberdiek/test/listingsutf8-test2.tex
%    TDS:doc/latex/oberdiek/test/listingsutf8-test3.tex
%    TDS:doc/latex/oberdiek/test/listingsutf8-test4.tex
%    TDS:doc/latex/oberdiek/test/listingsutf8-test5.tex
%    TDS:source/latex/oberdiek/listingsutf8.dtx
%
%<*ignore>
\begingroup
  \catcode123=1 %
  \catcode125=2 %
  \def\x{LaTeX2e}%
\expandafter\endgroup
\ifcase 0\ifx\install y1\fi\expandafter
         \ifx\csname processbatchFile\endcsname\relax\else1\fi
         \ifx\fmtname\x\else 1\fi\relax
\else\csname fi\endcsname
%</ignore>
%<*install>
\input docstrip.tex
\Msg{************************************************************************}
\Msg{* Installation}
\Msg{* Package: listingsutf8 2011/11/10 v1.2 Allow UTF-8 in listings input (HO)}
\Msg{************************************************************************}

\keepsilent
\askforoverwritefalse

\let\MetaPrefix\relax
\preamble

This is a generated file.

Project: listingsutf8
Version: 2011/11/10 v1.2

Copyright (C) 2007, 2011 by
   Heiko Oberdiek <heiko.oberdiek at googlemail.com>

This work may be distributed and/or modified under the
conditions of the LaTeX Project Public License, either
version 1.3c of this license or (at your option) any later
version. This version of this license is in
   http://www.latex-project.org/lppl/lppl-1-3c.txt
and the latest version of this license is in
   http://www.latex-project.org/lppl.txt
and version 1.3 or later is part of all distributions of
LaTeX version 2005/12/01 or later.

This work has the LPPL maintenance status "maintained".

This Current Maintainer of this work is Heiko Oberdiek.

This work consists of the main source file listingsutf8.dtx
and the derived files
   listingsutf8.sty, listingsutf8.pdf, listingsutf8.ins, listingsutf8.drv,
   listingsutf8-test1.tex, listingsutf8-test2.tex,
   listingsutf8-test3.tex, listingsutf8-test4.tex,
   listingsutf8-test5.tex.

\endpreamble
\let\MetaPrefix\DoubleperCent

\generate{%
  \file{listingsutf8.ins}{\from{listingsutf8.dtx}{install}}%
  \file{listingsutf8.drv}{\from{listingsutf8.dtx}{driver}}%
  \usedir{tex/latex/oberdiek}%
  \file{listingsutf8.sty}{\from{listingsutf8.dtx}{package}}%
  \usedir{doc/latex/oberdiek/test}%
  \file{listingsutf8-test1.tex}{\from{listingsutf8.dtx}{test1}}%
  \file{listingsutf8-test2.tex}{\from{listingsutf8.dtx}{test2,utf8}}%
  \file{listingsutf8-test3.tex}{\from{listingsutf8.dtx}{test3,utf8x}}%
  \file{listingsutf8-test4.tex}{\from{listingsutf8.dtx}{test4,utf8,noetex}}%
  \file{listingsutf8-test5.tex}{\from{listingsutf8.dtx}{test5,utf8x,noetex}}%
  \nopreamble
  \nopostamble
  \usedir{source/latex/oberdiek/catalogue}%
  \file{listingsutf8.xml}{\from{listingsutf8.dtx}{catalogue}}%
}

\catcode32=13\relax% active space
\let =\space%
\Msg{************************************************************************}
\Msg{*}
\Msg{* To finish the installation you have to move the following}
\Msg{* file into a directory searched by TeX:}
\Msg{*}
\Msg{*     listingsutf8.sty}
\Msg{*}
\Msg{* To produce the documentation run the file `listingsutf8.drv'}
\Msg{* through LaTeX.}
\Msg{*}
\Msg{* Happy TeXing!}
\Msg{*}
\Msg{************************************************************************}

\endbatchfile
%</install>
%<*ignore>
\fi
%</ignore>
%<*driver>
\NeedsTeXFormat{LaTeX2e}
\ProvidesFile{listingsutf8.drv}%
  [2011/11/10 v1.2 Allow UTF-8 in listings input (HO)]%
\documentclass{ltxdoc}
\usepackage{holtxdoc}[2011/11/22]
\begin{document}
  \DocInput{listingsutf8.dtx}%
\end{document}
%</driver>
% \fi
%
% \CheckSum{311}
%
% \CharacterTable
%  {Upper-case    \A\B\C\D\E\F\G\H\I\J\K\L\M\N\O\P\Q\R\S\T\U\V\W\X\Y\Z
%   Lower-case    \a\b\c\d\e\f\g\h\i\j\k\l\m\n\o\p\q\r\s\t\u\v\w\x\y\z
%   Digits        \0\1\2\3\4\5\6\7\8\9
%   Exclamation   \!     Double quote  \"     Hash (number) \#
%   Dollar        \$     Percent       \%     Ampersand     \&
%   Acute accent  \'     Left paren    \(     Right paren   \)
%   Asterisk      \*     Plus          \+     Comma         \,
%   Minus         \-     Point         \.     Solidus       \/
%   Colon         \:     Semicolon     \;     Less than     \<
%   Equals        \=     Greater than  \>     Question mark \?
%   Commercial at \@     Left bracket  \[     Backslash     \\
%   Right bracket \]     Circumflex    \^     Underscore    \_
%   Grave accent  \`     Left brace    \{     Vertical bar  \|
%   Right brace   \}     Tilde         \~}
%
% \GetFileInfo{listingsutf8.drv}
%
% \title{The \xpackage{listingsutf8} package}
% \date{2011/11/10 v1.2}
% \author{Heiko Oberdiek\\\xemail{heiko.oberdiek at googlemail.com}}
%
% \maketitle
%
% \begin{abstract}
% Package \xpackage{listings} does not support files with multi-byte
% encodings such as UTF-8. In case of \cs{lstinputlisting} a simple
% workaround is possible if an one-byte encoding exists that the file
% can be converted to. Also \eTeX\ and \pdfTeX\ regardless of its mode
% are required.
% \end{abstract}
%
% \tableofcontents
%
% \section{Documentation}
%
% \subsection{User interface}
%
% Load this package after or instead of package \xpackage{listings}
% \cite{listings}.
% The package does not define own options and passes given options to
% package \xpackage{listings}.
%
% The syntax of package \xpackage{listings}' key \xoption{inputencoding}
% is extended:
% \begin{quote}
%   |inputencoding=utf8/|\meta{one-byte-encoding}\\
%   Example: |inputencoding=utf8/latin1|
% \end{quote}
% That means the file is encoded in UTF-8 and can
% be converted to the given \meta{one-byte-encoding}.
% The available encodings for \meta{one-byte-encoding} are
% listed in section ``1.2 Supported encodings'' of
% package \xpackage{stringenc}'s documentation \cite{stringenc}.
% Of course, the encoding must encode its characters with
% one byte exactly. This excludes the unicode encodings
% (\xoption{utf8}, \xoption{utf16}, \dots).
%
% Only \cs{lstinputlisting} is supported by the syntax extension
% of key \xoption{inputencoding}.
%
% Internally package \xpackage{listingsutf8} reads the file as binary file
% via primitives of \pdfTeX\ (\cs{pdffiledump}). Then the file
% contents is converted as string using package \xpackage{stringenc} and
% finally the string is read as virtual file by \eTeX's \cs{scantokens}.
%
% \subsection{Future}
%
% Workarounds are not provided for
% \begin{itemize}
% \item \cs{lstinline}
% \item Environment |lstlisting|.
% \item Environments defined by \cs{lstnewenvironment}.
% \end{itemize}
% Perhaps someone will find time to extend package \xpackage{listings}
% with full native support for UTF-8. Then this package would become obsolete.
%
% \StopEventually{
% }
%
% \section{Implementation}
%
%    \begin{macrocode}
%<*package>
%    \end{macrocode}
%
% \subsection{Catcodes and identification}
%
%    \begin{macrocode}
\begingroup\catcode61\catcode48\catcode32=10\relax%
  \catcode13=5 % ^^M
  \endlinechar=13 %
  \catcode123=1 % {
  \catcode125=2 % }
  \catcode64=11 % @
  \def\x{\endgroup
    \expandafter\edef\csname lstU@AtEnd\endcsname{%
      \endlinechar=\the\endlinechar\relax
      \catcode13=\the\catcode13\relax
      \catcode32=\the\catcode32\relax
      \catcode35=\the\catcode35\relax
      \catcode61=\the\catcode61\relax
      \catcode64=\the\catcode64\relax
      \catcode123=\the\catcode123\relax
      \catcode125=\the\catcode125\relax
    }%
  }%
\x\catcode61\catcode48\catcode32=10\relax%
\catcode13=5 % ^^M
\endlinechar=13 %
\catcode35=6 % #
\catcode64=11 % @
\catcode123=1 % {
\catcode125=2 % }
\def\TMP@EnsureCode#1#2{%
  \edef\lstU@AtEnd{%
    \lstU@AtEnd
    \catcode#1=\the\catcode#1\relax
  }%
  \catcode#1=#2\relax
}
\TMP@EnsureCode{10}{12}% ^^J
\TMP@EnsureCode{33}{12}% !
\TMP@EnsureCode{36}{3}% $
\TMP@EnsureCode{38}{4}% &
\TMP@EnsureCode{39}{12}% '
\TMP@EnsureCode{40}{12}% (
\TMP@EnsureCode{41}{12}% )
\TMP@EnsureCode{42}{12}% *
\TMP@EnsureCode{43}{12}% +
\TMP@EnsureCode{44}{12}% ,
\TMP@EnsureCode{45}{12}% -
\TMP@EnsureCode{46}{12}% .
\TMP@EnsureCode{47}{12}% /
\TMP@EnsureCode{58}{12}% :
\TMP@EnsureCode{60}{12}% <
\TMP@EnsureCode{62}{12}% >
\TMP@EnsureCode{91}{12}% [
\TMP@EnsureCode{93}{12}% ]
\TMP@EnsureCode{94}{7}% ^ (superscript)
\TMP@EnsureCode{95}{8}% _ (subscript)
\TMP@EnsureCode{96}{12}% `
\TMP@EnsureCode{124}{12}% |
\TMP@EnsureCode{126}{13}% ~ (active)
\edef\lstU@AtEnd{\lstU@AtEnd\noexpand\endinput}
%    \end{macrocode}
%
%    Package identification.
%    \begin{macrocode}
\NeedsTeXFormat{LaTeX2e}
\ProvidesPackage{listingsutf8}%
  [2011/11/10 v1.2 Allow UTF-8 in listings input (HO)]
%    \end{macrocode}
%
% \subsection{Package options}
%
% Just pass options to package listings.
%
%    \begin{macrocode}
\DeclareOption*{%
  \PassOptionsToPackage\CurrentOption{listings}%
}
\ProcessOptions*
%    \end{macrocode}
%    Key \xoption{inputencoding} was introduced in version
%    2002/04/01 v1.0 of package \xpackage{listings}.
%    \begin{macrocode}
\RequirePackage{listings}[2002/04/01]
%    \end{macrocode}
%    Ensure that \cs{inputencoding} is provided.
%    \begin{macrocode}
\AtBeginDocument{%
  \@ifundefined{inputencoding}{%
    \RequirePackage{inputenc}%
  }{}%
}
%    \end{macrocode}
%
% \subsection{Check prerequisites}
%
%    \begin{macrocode}
\RequirePackage{pdftexcmds}[2011/04/22]
%    \end{macrocode}
%
%    \begin{macrocode}
\def\lstU@temp#1#2{%
  \begingroup\expandafter\expandafter\expandafter\endgroup
  \expandafter\ifx\csname #1\endcsname\relax
    \PackageWarningNoLine{listingsutf8}{%
      Package loading is aborted because of missing %
      \@backslashchar#1.\MessageBreak
      #2%
    }%
    \expandafter\lstU@AtEnd
  \fi
}
\lstU@temp{scantokens}{It is provided by e-TeX}%
\lstU@temp{pdf@unescapehex}{It is provided by pdfTeX >= 1.30}%
\lstU@temp{pdf@filedump}{It is provided by pdfTeX >= 1.30}%
\lstU@temp{pdf@filesize}{It is provided by pdfTeX >= 1.30}%
%    \end{macrocode}
%
%    \begin{macrocode}
\RequirePackage{stringenc}[2010/03/01]
%    \end{macrocode}
%
% \subsection{Add support for UTF-8}
%
%    \begin{macro}{\iflstU@utfviii}
%    \begin{macrocode}
\newif\iflstU@utfviii
%    \end{macrocode}
%    \end{macro}
%
%    \begin{macro}{\lstU@inputenc}
%    \begin{macrocode}
\def\lstU@inputenc#1{%
  \expandafter\lstU@@inputenc#1utf8/utf8/\@nil
}
%    \end{macrocode}
%    \end{macro}
%    \begin{macro}{\lstU@@inputenc}
\def\lstU@@inputenc#1utf8/#2utf8/#3\@nil{%
  \ifx\\#1\\%
    \lstU@utfviiitrue
    \def\lst@inputenc{#2}%
  \else
    \lstU@utfviiifalse
  \fi
}
%    \begin{macrocode}
%    \end{macrocode}
%    \end{macro}
%
%    \begin{macrocode}
\lst@Key{inputencoding}\relax{%
  \def\lst@inputenc{#1}%
  \lstU@inputenc{#1}%
}
%    \end{macrocode}
%
% \subsubsection{Conversion}
%
%    \begin{macro}{\lstU@input}
%    \begin{macrocode}
\def\lstU@input#1{%
  \iflstU@utfviii
    \edef\lstU@text{%
      \pdf@unescapehex{%
        \pdf@filedump{0}{\pdf@filesize{#1}}{#1}%
      }%
    }%
    \lstU@CRLFtoLF\lstU@text
    \StringEncodingConvert\lstU@text\lstU@text{utf8}\lst@inputenc
    \def\lstU@temp{%
      \scantokens\expandafter{\lstU@text}%
    }%
  \else
    \def\lstU@temp{%
      \input{#1}%
    }%
  \fi
  \lstU@temp
}
%    \end{macrocode}
%    \end{macro}
%
% \subsubsection{Convert CR/LF pairs to LF}
%
%    \begin{macro}{\lstU@CRLFtoLF}
%    \begin{macrocode}
\begingroup
  \endlinechar=-1 %
  \@makeother\^^J %
  \@makeother\^^M %
  \gdef\lstU@CRLFtoLF#1{%
    \edef#1{%
      \expandafter\lstU@CRLFtoLF@aux#1^^M^^J\@nil
    }%
  }%
  \gdef\lstU@CRLFtoLF@aux#1^^M^^J#2\@nil{%
    #1%
    \ifx\relax#2\relax
      \@car
    \fi
    ^^J%
    \lstU@CRLFtoLF@aux#2\@nil
  }%
\endgroup %
%    \end{macrocode}
%    \end{macro}
%
% \subsubsection{Patch \cs{lst@InputListing}}
%
%    \begin{macrocode}
\def\lstU@temp#1\def\lst@next#2#3\@nil{%
  \def\lst@InputListing##1{%
    #1%
    \def\lst@next{\lstU@input{##1}}%
    #3%
  }%
}
\expandafter\lstU@temp\lst@InputListing{#1}\@nil
%    \end{macrocode}
%
%    \begin{macrocode}
\lstU@AtEnd%
%</package>
%    \end{macrocode}
%
% \section{Test}
%
% \subsection{Catcode checks for loading}
%
%    \begin{macrocode}
%<*test1>
%    \end{macrocode}
%    \begin{macrocode}
\NeedsTeXFormat{LaTeX2e}
\documentclass{minimal}
\makeatletter
\def\RestoreCatcodes{}
\count@=0 %
\loop
  \edef\RestoreCatcodes{%
    \RestoreCatcodes
    \catcode\the\count@=\the\catcode\count@\relax
  }%
\ifnum\count@<255 %
  \advance\count@\@ne
\repeat

\def\RangeCatcodeInvalid#1#2{%
  \count@=#1\relax
  \loop
    \catcode\count@=15 %
  \ifnum\count@<#2\relax
    \advance\count@\@ne
  \repeat
}
\def\Test{%
  \RangeCatcodeInvalid{0}{47}%
  \RangeCatcodeInvalid{58}{64}%
  \RangeCatcodeInvalid{91}{96}%
  \RangeCatcodeInvalid{123}{127}%
  \catcode`\@=12 %
  \catcode`\\=0 %
  \catcode`\{=1 %
  \catcode`\}=2 %
  \catcode`\#=6 %
  \catcode`\[=12 %
  \catcode`\]=12 %
  \catcode`\%=14 %
  \catcode`\ =10 %
  \catcode13=5 %
  \RequirePackage{listingsutf8}[2011/11/10]\relax
  \RestoreCatcodes
}
\Test
\csname @@end\endcsname
\end
%    \end{macrocode}
%    \begin{macrocode}
%</test1>
%    \end{macrocode}
%
% \subsection{Test example for latin1}
%
%    \begin{macrocode}
%<*test2>
%    \end{macrocode}
%    \begin{macrocode}
\NeedsTeXFormat{LaTeX2e}
\documentclass{minimal}
\usepackage{filecontents}
\def\do#1{%
  \ifx#1\^%
  \else
    \noexpand\do\noexpand#1%
  \fi
}
\expandafter\let\expandafter\dospecials\expandafter\empty
\expandafter\edef\expandafter\dospecials\expandafter{\dospecials}
\begin{filecontents*}{ExampleUTF8.java}
public class ExampleUTF8 {
    public static String testString =
        "Umlauts: " +
        "^^c3^^84^^c3^^96^^c3^^9c^^c3^^a4^^c3^^b6^^c3^^bc^^c3^^9f";
    public static void main(String[] args) {
        System.out.println(testString);
    }
}
\end{filecontents*}
\usepackage{listingsutf8}[2011/11/10]
\def\Text{%
  Umlauts: %
  ^^c3^^84^^c3^^96^^c3^^9c^^c3^^a4^^c3^^b6^^c3^^bc^^c3^^9f%
}
\begin{document}
\lstinputlisting[%
  language=Java,%
  inputencoding=utf8/latin1,%
]{ExampleUTF8.java}
\end{document}
%</test2>
%    \end{macrocode}
%
% \section{Installation}
%
% \subsection{Download}
%
% \paragraph{Package.} This package is available on
% CTAN\footnote{\url{ftp://ftp.ctan.org/tex-archive/}}:
% \begin{description}
% \item[\CTAN{macros/latex/contrib/oberdiek/listingsutf8.dtx}] The source file.
% \item[\CTAN{macros/latex/contrib/oberdiek/listingsutf8.pdf}] Documentation.
% \end{description}
%
%
% \paragraph{Bundle.} All the packages of the bundle `oberdiek'
% are also available in a TDS compliant ZIP archive. There
% the packages are already unpacked and the documentation files
% are generated. The files and directories obey the TDS standard.
% \begin{description}
% \item[\CTAN{install/macros/latex/contrib/oberdiek.tds.zip}]
% \end{description}
% \emph{TDS} refers to the standard ``A Directory Structure
% for \TeX\ Files'' (\CTAN{tds/tds.pdf}). Directories
% with \xfile{texmf} in their name are usually organized this way.
%
% \subsection{Bundle installation}
%
% \paragraph{Unpacking.} Unpack the \xfile{oberdiek.tds.zip} in the
% TDS tree (also known as \xfile{texmf} tree) of your choice.
% Example (linux):
% \begin{quote}
%   |unzip oberdiek.tds.zip -d ~/texmf|
% \end{quote}
%
% \paragraph{Script installation.}
% Check the directory \xfile{TDS:scripts/oberdiek/} for
% scripts that need further installation steps.
% Package \xpackage{attachfile2} comes with the Perl script
% \xfile{pdfatfi.pl} that should be installed in such a way
% that it can be called as \texttt{pdfatfi}.
% Example (linux):
% \begin{quote}
%   |chmod +x scripts/oberdiek/pdfatfi.pl|\\
%   |cp scripts/oberdiek/pdfatfi.pl /usr/local/bin/|
% \end{quote}
%
% \subsection{Package installation}
%
% \paragraph{Unpacking.} The \xfile{.dtx} file is a self-extracting
% \docstrip\ archive. The files are extracted by running the
% \xfile{.dtx} through \plainTeX:
% \begin{quote}
%   \verb|tex listingsutf8.dtx|
% \end{quote}
%
% \paragraph{TDS.} Now the different files must be moved into
% the different directories in your installation TDS tree
% (also known as \xfile{texmf} tree):
% \begin{quote}
% \def\t{^^A
% \begin{tabular}{@{}>{\ttfamily}l@{ $\rightarrow$ }>{\ttfamily}l@{}}
%   listingsutf8.sty & tex/latex/oberdiek/listingsutf8.sty\\
%   listingsutf8.pdf & doc/latex/oberdiek/listingsutf8.pdf\\
%   test/listingsutf8-test1.tex & doc/latex/oberdiek/test/listingsutf8-test1.tex\\
%   test/listingsutf8-test2.tex & doc/latex/oberdiek/test/listingsutf8-test2.tex\\
%   test/listingsutf8-test3.tex & doc/latex/oberdiek/test/listingsutf8-test3.tex\\
%   test/listingsutf8-test4.tex & doc/latex/oberdiek/test/listingsutf8-test4.tex\\
%   test/listingsutf8-test5.tex & doc/latex/oberdiek/test/listingsutf8-test5.tex\\
%   listingsutf8.dtx & source/latex/oberdiek/listingsutf8.dtx\\
% \end{tabular}^^A
% }^^A
% \sbox0{\t}^^A
% \ifdim\wd0>\linewidth
%   \begingroup
%     \advance\linewidth by\leftmargin
%     \advance\linewidth by\rightmargin
%   \edef\x{\endgroup
%     \def\noexpand\lw{\the\linewidth}^^A
%   }\x
%   \def\lwbox{^^A
%     \leavevmode
%     \hbox to \linewidth{^^A
%       \kern-\leftmargin\relax
%       \hss
%       \usebox0
%       \hss
%       \kern-\rightmargin\relax
%     }^^A
%   }^^A
%   \ifdim\wd0>\lw
%     \sbox0{\small\t}^^A
%     \ifdim\wd0>\linewidth
%       \ifdim\wd0>\lw
%         \sbox0{\footnotesize\t}^^A
%         \ifdim\wd0>\linewidth
%           \ifdim\wd0>\lw
%             \sbox0{\scriptsize\t}^^A
%             \ifdim\wd0>\linewidth
%               \ifdim\wd0>\lw
%                 \sbox0{\tiny\t}^^A
%                 \ifdim\wd0>\linewidth
%                   \lwbox
%                 \else
%                   \usebox0
%                 \fi
%               \else
%                 \lwbox
%               \fi
%             \else
%               \usebox0
%             \fi
%           \else
%             \lwbox
%           \fi
%         \else
%           \usebox0
%         \fi
%       \else
%         \lwbox
%       \fi
%     \else
%       \usebox0
%     \fi
%   \else
%     \lwbox
%   \fi
% \else
%   \usebox0
% \fi
% \end{quote}
% If you have a \xfile{docstrip.cfg} that configures and enables \docstrip's
% TDS installing feature, then some files can already be in the right
% place, see the documentation of \docstrip.
%
% \subsection{Refresh file name databases}
%
% If your \TeX~distribution
% (\teTeX, \mikTeX, \dots) relies on file name databases, you must refresh
% these. For example, \teTeX\ users run \verb|texhash| or
% \verb|mktexlsr|.
%
% \subsection{Some details for the interested}
%
% \paragraph{Attached source.}
%
% The PDF documentation on CTAN also includes the
% \xfile{.dtx} source file. It can be extracted by
% AcrobatReader 6 or higher. Another option is \textsf{pdftk},
% e.g. unpack the file into the current directory:
% \begin{quote}
%   \verb|pdftk listingsutf8.pdf unpack_files output .|
% \end{quote}
%
% \paragraph{Unpacking with \LaTeX.}
% The \xfile{.dtx} chooses its action depending on the format:
% \begin{description}
% \item[\plainTeX:] Run \docstrip\ and extract the files.
% \item[\LaTeX:] Generate the documentation.
% \end{description}
% If you insist on using \LaTeX\ for \docstrip\ (really,
% \docstrip\ does not need \LaTeX), then inform the autodetect routine
% about your intention:
% \begin{quote}
%   \verb|latex \let\install=y% \iffalse meta-comment
%
% File: listingsutf8.dtx
% Version: 2011/11/10 v1.2
% Info: Allow UTF-8 in listings input
%
% Copyright (C) 2007, 2011 by
%    Heiko Oberdiek <heiko.oberdiek at googlemail.com>
%
% This work may be distributed and/or modified under the
% conditions of the LaTeX Project Public License, either
% version 1.3c of this license or (at your option) any later
% version. This version of this license is in
%    http://www.latex-project.org/lppl/lppl-1-3c.txt
% and the latest version of this license is in
%    http://www.latex-project.org/lppl.txt
% and version 1.3 or later is part of all distributions of
% LaTeX version 2005/12/01 or later.
%
% This work has the LPPL maintenance status "maintained".
%
% This Current Maintainer of this work is Heiko Oberdiek.
%
% This work consists of the main source file listingsutf8.dtx
% and the derived files
%    listingsutf8.sty, listingsutf8.pdf, listingsutf8.ins, listingsutf8.drv,
%    listingsutf8-test1.tex, listingsutf8-test2.tex,
%    listingsutf8-test3.tex, listingsutf8-test4.tex,
%    listingsutf8-test5.tex.
%
% Distribution:
%    CTAN:macros/latex/contrib/oberdiek/listingsutf8.dtx
%    CTAN:macros/latex/contrib/oberdiek/listingsutf8.pdf
%
% Unpacking:
%    (a) If listingsutf8.ins is present:
%           tex listingsutf8.ins
%    (b) Without listingsutf8.ins:
%           tex listingsutf8.dtx
%    (c) If you insist on using LaTeX
%           latex \let\install=y% \iffalse meta-comment
%
% File: listingsutf8.dtx
% Version: 2011/11/10 v1.2
% Info: Allow UTF-8 in listings input
%
% Copyright (C) 2007, 2011 by
%    Heiko Oberdiek <heiko.oberdiek at googlemail.com>
%
% This work may be distributed and/or modified under the
% conditions of the LaTeX Project Public License, either
% version 1.3c of this license or (at your option) any later
% version. This version of this license is in
%    http://www.latex-project.org/lppl/lppl-1-3c.txt
% and the latest version of this license is in
%    http://www.latex-project.org/lppl.txt
% and version 1.3 or later is part of all distributions of
% LaTeX version 2005/12/01 or later.
%
% This work has the LPPL maintenance status "maintained".
%
% This Current Maintainer of this work is Heiko Oberdiek.
%
% This work consists of the main source file listingsutf8.dtx
% and the derived files
%    listingsutf8.sty, listingsutf8.pdf, listingsutf8.ins, listingsutf8.drv,
%    listingsutf8-test1.tex, listingsutf8-test2.tex,
%    listingsutf8-test3.tex, listingsutf8-test4.tex,
%    listingsutf8-test5.tex.
%
% Distribution:
%    CTAN:macros/latex/contrib/oberdiek/listingsutf8.dtx
%    CTAN:macros/latex/contrib/oberdiek/listingsutf8.pdf
%
% Unpacking:
%    (a) If listingsutf8.ins is present:
%           tex listingsutf8.ins
%    (b) Without listingsutf8.ins:
%           tex listingsutf8.dtx
%    (c) If you insist on using LaTeX
%           latex \let\install=y\input{listingsutf8.dtx}
%        (quote the arguments according to the demands of your shell)
%
% Documentation:
%    (a) If listingsutf8.drv is present:
%           latex listingsutf8.drv
%    (b) Without listingsutf8.drv:
%           latex listingsutf8.dtx; ...
%    The class ltxdoc loads the configuration file ltxdoc.cfg
%    if available. Here you can specify further options, e.g.
%    use A4 as paper format:
%       \PassOptionsToClass{a4paper}{article}
%
%    Programm calls to get the documentation (example):
%       pdflatex listingsutf8.dtx
%       makeindex -s gind.ist listingsutf8.idx
%       pdflatex listingsutf8.dtx
%       makeindex -s gind.ist listingsutf8.idx
%       pdflatex listingsutf8.dtx
%
% Installation:
%    TDS:tex/latex/oberdiek/listingsutf8.sty
%    TDS:doc/latex/oberdiek/listingsutf8.pdf
%    TDS:doc/latex/oberdiek/test/listingsutf8-test1.tex
%    TDS:doc/latex/oberdiek/test/listingsutf8-test2.tex
%    TDS:doc/latex/oberdiek/test/listingsutf8-test3.tex
%    TDS:doc/latex/oberdiek/test/listingsutf8-test4.tex
%    TDS:doc/latex/oberdiek/test/listingsutf8-test5.tex
%    TDS:source/latex/oberdiek/listingsutf8.dtx
%
%<*ignore>
\begingroup
  \catcode123=1 %
  \catcode125=2 %
  \def\x{LaTeX2e}%
\expandafter\endgroup
\ifcase 0\ifx\install y1\fi\expandafter
         \ifx\csname processbatchFile\endcsname\relax\else1\fi
         \ifx\fmtname\x\else 1\fi\relax
\else\csname fi\endcsname
%</ignore>
%<*install>
\input docstrip.tex
\Msg{************************************************************************}
\Msg{* Installation}
\Msg{* Package: listingsutf8 2011/11/10 v1.2 Allow UTF-8 in listings input (HO)}
\Msg{************************************************************************}

\keepsilent
\askforoverwritefalse

\let\MetaPrefix\relax
\preamble

This is a generated file.

Project: listingsutf8
Version: 2011/11/10 v1.2

Copyright (C) 2007, 2011 by
   Heiko Oberdiek <heiko.oberdiek at googlemail.com>

This work may be distributed and/or modified under the
conditions of the LaTeX Project Public License, either
version 1.3c of this license or (at your option) any later
version. This version of this license is in
   http://www.latex-project.org/lppl/lppl-1-3c.txt
and the latest version of this license is in
   http://www.latex-project.org/lppl.txt
and version 1.3 or later is part of all distributions of
LaTeX version 2005/12/01 or later.

This work has the LPPL maintenance status "maintained".

This Current Maintainer of this work is Heiko Oberdiek.

This work consists of the main source file listingsutf8.dtx
and the derived files
   listingsutf8.sty, listingsutf8.pdf, listingsutf8.ins, listingsutf8.drv,
   listingsutf8-test1.tex, listingsutf8-test2.tex,
   listingsutf8-test3.tex, listingsutf8-test4.tex,
   listingsutf8-test5.tex.

\endpreamble
\let\MetaPrefix\DoubleperCent

\generate{%
  \file{listingsutf8.ins}{\from{listingsutf8.dtx}{install}}%
  \file{listingsutf8.drv}{\from{listingsutf8.dtx}{driver}}%
  \usedir{tex/latex/oberdiek}%
  \file{listingsutf8.sty}{\from{listingsutf8.dtx}{package}}%
  \usedir{doc/latex/oberdiek/test}%
  \file{listingsutf8-test1.tex}{\from{listingsutf8.dtx}{test1}}%
  \file{listingsutf8-test2.tex}{\from{listingsutf8.dtx}{test2,utf8}}%
  \file{listingsutf8-test3.tex}{\from{listingsutf8.dtx}{test3,utf8x}}%
  \file{listingsutf8-test4.tex}{\from{listingsutf8.dtx}{test4,utf8,noetex}}%
  \file{listingsutf8-test5.tex}{\from{listingsutf8.dtx}{test5,utf8x,noetex}}%
  \nopreamble
  \nopostamble
  \usedir{source/latex/oberdiek/catalogue}%
  \file{listingsutf8.xml}{\from{listingsutf8.dtx}{catalogue}}%
}

\catcode32=13\relax% active space
\let =\space%
\Msg{************************************************************************}
\Msg{*}
\Msg{* To finish the installation you have to move the following}
\Msg{* file into a directory searched by TeX:}
\Msg{*}
\Msg{*     listingsutf8.sty}
\Msg{*}
\Msg{* To produce the documentation run the file `listingsutf8.drv'}
\Msg{* through LaTeX.}
\Msg{*}
\Msg{* Happy TeXing!}
\Msg{*}
\Msg{************************************************************************}

\endbatchfile
%</install>
%<*ignore>
\fi
%</ignore>
%<*driver>
\NeedsTeXFormat{LaTeX2e}
\ProvidesFile{listingsutf8.drv}%
  [2011/11/10 v1.2 Allow UTF-8 in listings input (HO)]%
\documentclass{ltxdoc}
\usepackage{holtxdoc}[2011/11/22]
\begin{document}
  \DocInput{listingsutf8.dtx}%
\end{document}
%</driver>
% \fi
%
% \CheckSum{311}
%
% \CharacterTable
%  {Upper-case    \A\B\C\D\E\F\G\H\I\J\K\L\M\N\O\P\Q\R\S\T\U\V\W\X\Y\Z
%   Lower-case    \a\b\c\d\e\f\g\h\i\j\k\l\m\n\o\p\q\r\s\t\u\v\w\x\y\z
%   Digits        \0\1\2\3\4\5\6\7\8\9
%   Exclamation   \!     Double quote  \"     Hash (number) \#
%   Dollar        \$     Percent       \%     Ampersand     \&
%   Acute accent  \'     Left paren    \(     Right paren   \)
%   Asterisk      \*     Plus          \+     Comma         \,
%   Minus         \-     Point         \.     Solidus       \/
%   Colon         \:     Semicolon     \;     Less than     \<
%   Equals        \=     Greater than  \>     Question mark \?
%   Commercial at \@     Left bracket  \[     Backslash     \\
%   Right bracket \]     Circumflex    \^     Underscore    \_
%   Grave accent  \`     Left brace    \{     Vertical bar  \|
%   Right brace   \}     Tilde         \~}
%
% \GetFileInfo{listingsutf8.drv}
%
% \title{The \xpackage{listingsutf8} package}
% \date{2011/11/10 v1.2}
% \author{Heiko Oberdiek\\\xemail{heiko.oberdiek at googlemail.com}}
%
% \maketitle
%
% \begin{abstract}
% Package \xpackage{listings} does not support files with multi-byte
% encodings such as UTF-8. In case of \cs{lstinputlisting} a simple
% workaround is possible if an one-byte encoding exists that the file
% can be converted to. Also \eTeX\ and \pdfTeX\ regardless of its mode
% are required.
% \end{abstract}
%
% \tableofcontents
%
% \section{Documentation}
%
% \subsection{User interface}
%
% Load this package after or instead of package \xpackage{listings}
% \cite{listings}.
% The package does not define own options and passes given options to
% package \xpackage{listings}.
%
% The syntax of package \xpackage{listings}' key \xoption{inputencoding}
% is extended:
% \begin{quote}
%   |inputencoding=utf8/|\meta{one-byte-encoding}\\
%   Example: |inputencoding=utf8/latin1|
% \end{quote}
% That means the file is encoded in UTF-8 and can
% be converted to the given \meta{one-byte-encoding}.
% The available encodings for \meta{one-byte-encoding} are
% listed in section ``1.2 Supported encodings'' of
% package \xpackage{stringenc}'s documentation \cite{stringenc}.
% Of course, the encoding must encode its characters with
% one byte exactly. This excludes the unicode encodings
% (\xoption{utf8}, \xoption{utf16}, \dots).
%
% Only \cs{lstinputlisting} is supported by the syntax extension
% of key \xoption{inputencoding}.
%
% Internally package \xpackage{listingsutf8} reads the file as binary file
% via primitives of \pdfTeX\ (\cs{pdffiledump}). Then the file
% contents is converted as string using package \xpackage{stringenc} and
% finally the string is read as virtual file by \eTeX's \cs{scantokens}.
%
% \subsection{Future}
%
% Workarounds are not provided for
% \begin{itemize}
% \item \cs{lstinline}
% \item Environment |lstlisting|.
% \item Environments defined by \cs{lstnewenvironment}.
% \end{itemize}
% Perhaps someone will find time to extend package \xpackage{listings}
% with full native support for UTF-8. Then this package would become obsolete.
%
% \StopEventually{
% }
%
% \section{Implementation}
%
%    \begin{macrocode}
%<*package>
%    \end{macrocode}
%
% \subsection{Catcodes and identification}
%
%    \begin{macrocode}
\begingroup\catcode61\catcode48\catcode32=10\relax%
  \catcode13=5 % ^^M
  \endlinechar=13 %
  \catcode123=1 % {
  \catcode125=2 % }
  \catcode64=11 % @
  \def\x{\endgroup
    \expandafter\edef\csname lstU@AtEnd\endcsname{%
      \endlinechar=\the\endlinechar\relax
      \catcode13=\the\catcode13\relax
      \catcode32=\the\catcode32\relax
      \catcode35=\the\catcode35\relax
      \catcode61=\the\catcode61\relax
      \catcode64=\the\catcode64\relax
      \catcode123=\the\catcode123\relax
      \catcode125=\the\catcode125\relax
    }%
  }%
\x\catcode61\catcode48\catcode32=10\relax%
\catcode13=5 % ^^M
\endlinechar=13 %
\catcode35=6 % #
\catcode64=11 % @
\catcode123=1 % {
\catcode125=2 % }
\def\TMP@EnsureCode#1#2{%
  \edef\lstU@AtEnd{%
    \lstU@AtEnd
    \catcode#1=\the\catcode#1\relax
  }%
  \catcode#1=#2\relax
}
\TMP@EnsureCode{10}{12}% ^^J
\TMP@EnsureCode{33}{12}% !
\TMP@EnsureCode{36}{3}% $
\TMP@EnsureCode{38}{4}% &
\TMP@EnsureCode{39}{12}% '
\TMP@EnsureCode{40}{12}% (
\TMP@EnsureCode{41}{12}% )
\TMP@EnsureCode{42}{12}% *
\TMP@EnsureCode{43}{12}% +
\TMP@EnsureCode{44}{12}% ,
\TMP@EnsureCode{45}{12}% -
\TMP@EnsureCode{46}{12}% .
\TMP@EnsureCode{47}{12}% /
\TMP@EnsureCode{58}{12}% :
\TMP@EnsureCode{60}{12}% <
\TMP@EnsureCode{62}{12}% >
\TMP@EnsureCode{91}{12}% [
\TMP@EnsureCode{93}{12}% ]
\TMP@EnsureCode{94}{7}% ^ (superscript)
\TMP@EnsureCode{95}{8}% _ (subscript)
\TMP@EnsureCode{96}{12}% `
\TMP@EnsureCode{124}{12}% |
\TMP@EnsureCode{126}{13}% ~ (active)
\edef\lstU@AtEnd{\lstU@AtEnd\noexpand\endinput}
%    \end{macrocode}
%
%    Package identification.
%    \begin{macrocode}
\NeedsTeXFormat{LaTeX2e}
\ProvidesPackage{listingsutf8}%
  [2011/11/10 v1.2 Allow UTF-8 in listings input (HO)]
%    \end{macrocode}
%
% \subsection{Package options}
%
% Just pass options to package listings.
%
%    \begin{macrocode}
\DeclareOption*{%
  \PassOptionsToPackage\CurrentOption{listings}%
}
\ProcessOptions*
%    \end{macrocode}
%    Key \xoption{inputencoding} was introduced in version
%    2002/04/01 v1.0 of package \xpackage{listings}.
%    \begin{macrocode}
\RequirePackage{listings}[2002/04/01]
%    \end{macrocode}
%    Ensure that \cs{inputencoding} is provided.
%    \begin{macrocode}
\AtBeginDocument{%
  \@ifundefined{inputencoding}{%
    \RequirePackage{inputenc}%
  }{}%
}
%    \end{macrocode}
%
% \subsection{Check prerequisites}
%
%    \begin{macrocode}
\RequirePackage{pdftexcmds}[2011/04/22]
%    \end{macrocode}
%
%    \begin{macrocode}
\def\lstU@temp#1#2{%
  \begingroup\expandafter\expandafter\expandafter\endgroup
  \expandafter\ifx\csname #1\endcsname\relax
    \PackageWarningNoLine{listingsutf8}{%
      Package loading is aborted because of missing %
      \@backslashchar#1.\MessageBreak
      #2%
    }%
    \expandafter\lstU@AtEnd
  \fi
}
\lstU@temp{scantokens}{It is provided by e-TeX}%
\lstU@temp{pdf@unescapehex}{It is provided by pdfTeX >= 1.30}%
\lstU@temp{pdf@filedump}{It is provided by pdfTeX >= 1.30}%
\lstU@temp{pdf@filesize}{It is provided by pdfTeX >= 1.30}%
%    \end{macrocode}
%
%    \begin{macrocode}
\RequirePackage{stringenc}[2010/03/01]
%    \end{macrocode}
%
% \subsection{Add support for UTF-8}
%
%    \begin{macro}{\iflstU@utfviii}
%    \begin{macrocode}
\newif\iflstU@utfviii
%    \end{macrocode}
%    \end{macro}
%
%    \begin{macro}{\lstU@inputenc}
%    \begin{macrocode}
\def\lstU@inputenc#1{%
  \expandafter\lstU@@inputenc#1utf8/utf8/\@nil
}
%    \end{macrocode}
%    \end{macro}
%    \begin{macro}{\lstU@@inputenc}
\def\lstU@@inputenc#1utf8/#2utf8/#3\@nil{%
  \ifx\\#1\\%
    \lstU@utfviiitrue
    \def\lst@inputenc{#2}%
  \else
    \lstU@utfviiifalse
  \fi
}
%    \begin{macrocode}
%    \end{macrocode}
%    \end{macro}
%
%    \begin{macrocode}
\lst@Key{inputencoding}\relax{%
  \def\lst@inputenc{#1}%
  \lstU@inputenc{#1}%
}
%    \end{macrocode}
%
% \subsubsection{Conversion}
%
%    \begin{macro}{\lstU@input}
%    \begin{macrocode}
\def\lstU@input#1{%
  \iflstU@utfviii
    \edef\lstU@text{%
      \pdf@unescapehex{%
        \pdf@filedump{0}{\pdf@filesize{#1}}{#1}%
      }%
    }%
    \lstU@CRLFtoLF\lstU@text
    \StringEncodingConvert\lstU@text\lstU@text{utf8}\lst@inputenc
    \def\lstU@temp{%
      \scantokens\expandafter{\lstU@text}%
    }%
  \else
    \def\lstU@temp{%
      \input{#1}%
    }%
  \fi
  \lstU@temp
}
%    \end{macrocode}
%    \end{macro}
%
% \subsubsection{Convert CR/LF pairs to LF}
%
%    \begin{macro}{\lstU@CRLFtoLF}
%    \begin{macrocode}
\begingroup
  \endlinechar=-1 %
  \@makeother\^^J %
  \@makeother\^^M %
  \gdef\lstU@CRLFtoLF#1{%
    \edef#1{%
      \expandafter\lstU@CRLFtoLF@aux#1^^M^^J\@nil
    }%
  }%
  \gdef\lstU@CRLFtoLF@aux#1^^M^^J#2\@nil{%
    #1%
    \ifx\relax#2\relax
      \@car
    \fi
    ^^J%
    \lstU@CRLFtoLF@aux#2\@nil
  }%
\endgroup %
%    \end{macrocode}
%    \end{macro}
%
% \subsubsection{Patch \cs{lst@InputListing}}
%
%    \begin{macrocode}
\def\lstU@temp#1\def\lst@next#2#3\@nil{%
  \def\lst@InputListing##1{%
    #1%
    \def\lst@next{\lstU@input{##1}}%
    #3%
  }%
}
\expandafter\lstU@temp\lst@InputListing{#1}\@nil
%    \end{macrocode}
%
%    \begin{macrocode}
\lstU@AtEnd%
%</package>
%    \end{macrocode}
%
% \section{Test}
%
% \subsection{Catcode checks for loading}
%
%    \begin{macrocode}
%<*test1>
%    \end{macrocode}
%    \begin{macrocode}
\NeedsTeXFormat{LaTeX2e}
\documentclass{minimal}
\makeatletter
\def\RestoreCatcodes{}
\count@=0 %
\loop
  \edef\RestoreCatcodes{%
    \RestoreCatcodes
    \catcode\the\count@=\the\catcode\count@\relax
  }%
\ifnum\count@<255 %
  \advance\count@\@ne
\repeat

\def\RangeCatcodeInvalid#1#2{%
  \count@=#1\relax
  \loop
    \catcode\count@=15 %
  \ifnum\count@<#2\relax
    \advance\count@\@ne
  \repeat
}
\def\Test{%
  \RangeCatcodeInvalid{0}{47}%
  \RangeCatcodeInvalid{58}{64}%
  \RangeCatcodeInvalid{91}{96}%
  \RangeCatcodeInvalid{123}{127}%
  \catcode`\@=12 %
  \catcode`\\=0 %
  \catcode`\{=1 %
  \catcode`\}=2 %
  \catcode`\#=6 %
  \catcode`\[=12 %
  \catcode`\]=12 %
  \catcode`\%=14 %
  \catcode`\ =10 %
  \catcode13=5 %
  \RequirePackage{listingsutf8}[2011/11/10]\relax
  \RestoreCatcodes
}
\Test
\csname @@end\endcsname
\end
%    \end{macrocode}
%    \begin{macrocode}
%</test1>
%    \end{macrocode}
%
% \subsection{Test example for latin1}
%
%    \begin{macrocode}
%<*test2>
%    \end{macrocode}
%    \begin{macrocode}
\NeedsTeXFormat{LaTeX2e}
\documentclass{minimal}
\usepackage{filecontents}
\def\do#1{%
  \ifx#1\^%
  \else
    \noexpand\do\noexpand#1%
  \fi
}
\expandafter\let\expandafter\dospecials\expandafter\empty
\expandafter\edef\expandafter\dospecials\expandafter{\dospecials}
\begin{filecontents*}{ExampleUTF8.java}
public class ExampleUTF8 {
    public static String testString =
        "Umlauts: " +
        "^^c3^^84^^c3^^96^^c3^^9c^^c3^^a4^^c3^^b6^^c3^^bc^^c3^^9f";
    public static void main(String[] args) {
        System.out.println(testString);
    }
}
\end{filecontents*}
\usepackage{listingsutf8}[2011/11/10]
\def\Text{%
  Umlauts: %
  ^^c3^^84^^c3^^96^^c3^^9c^^c3^^a4^^c3^^b6^^c3^^bc^^c3^^9f%
}
\begin{document}
\lstinputlisting[%
  language=Java,%
  inputencoding=utf8/latin1,%
]{ExampleUTF8.java}
\end{document}
%</test2>
%    \end{macrocode}
%
% \section{Installation}
%
% \subsection{Download}
%
% \paragraph{Package.} This package is available on
% CTAN\footnote{\url{ftp://ftp.ctan.org/tex-archive/}}:
% \begin{description}
% \item[\CTAN{macros/latex/contrib/oberdiek/listingsutf8.dtx}] The source file.
% \item[\CTAN{macros/latex/contrib/oberdiek/listingsutf8.pdf}] Documentation.
% \end{description}
%
%
% \paragraph{Bundle.} All the packages of the bundle `oberdiek'
% are also available in a TDS compliant ZIP archive. There
% the packages are already unpacked and the documentation files
% are generated. The files and directories obey the TDS standard.
% \begin{description}
% \item[\CTAN{install/macros/latex/contrib/oberdiek.tds.zip}]
% \end{description}
% \emph{TDS} refers to the standard ``A Directory Structure
% for \TeX\ Files'' (\CTAN{tds/tds.pdf}). Directories
% with \xfile{texmf} in their name are usually organized this way.
%
% \subsection{Bundle installation}
%
% \paragraph{Unpacking.} Unpack the \xfile{oberdiek.tds.zip} in the
% TDS tree (also known as \xfile{texmf} tree) of your choice.
% Example (linux):
% \begin{quote}
%   |unzip oberdiek.tds.zip -d ~/texmf|
% \end{quote}
%
% \paragraph{Script installation.}
% Check the directory \xfile{TDS:scripts/oberdiek/} for
% scripts that need further installation steps.
% Package \xpackage{attachfile2} comes with the Perl script
% \xfile{pdfatfi.pl} that should be installed in such a way
% that it can be called as \texttt{pdfatfi}.
% Example (linux):
% \begin{quote}
%   |chmod +x scripts/oberdiek/pdfatfi.pl|\\
%   |cp scripts/oberdiek/pdfatfi.pl /usr/local/bin/|
% \end{quote}
%
% \subsection{Package installation}
%
% \paragraph{Unpacking.} The \xfile{.dtx} file is a self-extracting
% \docstrip\ archive. The files are extracted by running the
% \xfile{.dtx} through \plainTeX:
% \begin{quote}
%   \verb|tex listingsutf8.dtx|
% \end{quote}
%
% \paragraph{TDS.} Now the different files must be moved into
% the different directories in your installation TDS tree
% (also known as \xfile{texmf} tree):
% \begin{quote}
% \def\t{^^A
% \begin{tabular}{@{}>{\ttfamily}l@{ $\rightarrow$ }>{\ttfamily}l@{}}
%   listingsutf8.sty & tex/latex/oberdiek/listingsutf8.sty\\
%   listingsutf8.pdf & doc/latex/oberdiek/listingsutf8.pdf\\
%   test/listingsutf8-test1.tex & doc/latex/oberdiek/test/listingsutf8-test1.tex\\
%   test/listingsutf8-test2.tex & doc/latex/oberdiek/test/listingsutf8-test2.tex\\
%   test/listingsutf8-test3.tex & doc/latex/oberdiek/test/listingsutf8-test3.tex\\
%   test/listingsutf8-test4.tex & doc/latex/oberdiek/test/listingsutf8-test4.tex\\
%   test/listingsutf8-test5.tex & doc/latex/oberdiek/test/listingsutf8-test5.tex\\
%   listingsutf8.dtx & source/latex/oberdiek/listingsutf8.dtx\\
% \end{tabular}^^A
% }^^A
% \sbox0{\t}^^A
% \ifdim\wd0>\linewidth
%   \begingroup
%     \advance\linewidth by\leftmargin
%     \advance\linewidth by\rightmargin
%   \edef\x{\endgroup
%     \def\noexpand\lw{\the\linewidth}^^A
%   }\x
%   \def\lwbox{^^A
%     \leavevmode
%     \hbox to \linewidth{^^A
%       \kern-\leftmargin\relax
%       \hss
%       \usebox0
%       \hss
%       \kern-\rightmargin\relax
%     }^^A
%   }^^A
%   \ifdim\wd0>\lw
%     \sbox0{\small\t}^^A
%     \ifdim\wd0>\linewidth
%       \ifdim\wd0>\lw
%         \sbox0{\footnotesize\t}^^A
%         \ifdim\wd0>\linewidth
%           \ifdim\wd0>\lw
%             \sbox0{\scriptsize\t}^^A
%             \ifdim\wd0>\linewidth
%               \ifdim\wd0>\lw
%                 \sbox0{\tiny\t}^^A
%                 \ifdim\wd0>\linewidth
%                   \lwbox
%                 \else
%                   \usebox0
%                 \fi
%               \else
%                 \lwbox
%               \fi
%             \else
%               \usebox0
%             \fi
%           \else
%             \lwbox
%           \fi
%         \else
%           \usebox0
%         \fi
%       \else
%         \lwbox
%       \fi
%     \else
%       \usebox0
%     \fi
%   \else
%     \lwbox
%   \fi
% \else
%   \usebox0
% \fi
% \end{quote}
% If you have a \xfile{docstrip.cfg} that configures and enables \docstrip's
% TDS installing feature, then some files can already be in the right
% place, see the documentation of \docstrip.
%
% \subsection{Refresh file name databases}
%
% If your \TeX~distribution
% (\teTeX, \mikTeX, \dots) relies on file name databases, you must refresh
% these. For example, \teTeX\ users run \verb|texhash| or
% \verb|mktexlsr|.
%
% \subsection{Some details for the interested}
%
% \paragraph{Attached source.}
%
% The PDF documentation on CTAN also includes the
% \xfile{.dtx} source file. It can be extracted by
% AcrobatReader 6 or higher. Another option is \textsf{pdftk},
% e.g. unpack the file into the current directory:
% \begin{quote}
%   \verb|pdftk listingsutf8.pdf unpack_files output .|
% \end{quote}
%
% \paragraph{Unpacking with \LaTeX.}
% The \xfile{.dtx} chooses its action depending on the format:
% \begin{description}
% \item[\plainTeX:] Run \docstrip\ and extract the files.
% \item[\LaTeX:] Generate the documentation.
% \end{description}
% If you insist on using \LaTeX\ for \docstrip\ (really,
% \docstrip\ does not need \LaTeX), then inform the autodetect routine
% about your intention:
% \begin{quote}
%   \verb|latex \let\install=y\input{listingsutf8.dtx}|
% \end{quote}
% Do not forget to quote the argument according to the demands
% of your shell.
%
% \paragraph{Generating the documentation.}
% You can use both the \xfile{.dtx} or the \xfile{.drv} to generate
% the documentation. The process can be configured by the
% configuration file \xfile{ltxdoc.cfg}. For instance, put this
% line into this file, if you want to have A4 as paper format:
% \begin{quote}
%   \verb|\PassOptionsToClass{a4paper}{article}|
% \end{quote}
% An example follows how to generate the
% documentation with pdf\LaTeX:
% \begin{quote}
%\begin{verbatim}
%pdflatex listingsutf8.dtx
%makeindex -s gind.ist listingsutf8.idx
%pdflatex listingsutf8.dtx
%makeindex -s gind.ist listingsutf8.idx
%pdflatex listingsutf8.dtx
%\end{verbatim}
% \end{quote}
%
% \section{Catalogue}
%
% The following XML file can be used as source for the
% \href{http://mirror.ctan.org/help/Catalogue/catalogue.html}{\TeX\ Catalogue}.
% The elements \texttt{caption} and \texttt{description} are imported
% from the original XML file from the Catalogue.
% The name of the XML file in the Catalogue is \xfile{listingsutf8.xml}.
%    \begin{macrocode}
%<*catalogue>
<?xml version='1.0' encoding='us-ascii'?>
<!DOCTYPE entry SYSTEM 'catalogue.dtd'>
<entry datestamp='$Date$' modifier='$Author$' id='listingsutf8'>
  <name>listingsutf8</name>
  <caption>Allow UTF-8 in listings input.</caption>
  <authorref id='auth:oberdiek'/>
  <copyright owner='Heiko Oberdiek' year='2007,2011'/>
  <license type='lppl1.3'/>
  <version number='1.2'/>
  <description>
    Package <xref refid='listings'>listings</xref> does not support files
    with multi-byte encodings such as UTF-8.  In the case of
    <tt>\lstinputlisting</tt>, a simple workaround is possible if a
    one-byte encoding exists that the file can be converted to.  The
    package requires the e-TeX extensions under pdfTeX (in either PDF
    or DVI output mode).
    <p/>
    The package is part of the <xref refid='oberdiek'>oberdiek</xref> bundle.
  </description>
  <documentation details='Package documentation'
      href='ctan:/macros/latex/contrib/oberdiek/listingsutf8.pdf'/>
  <ctan file='true' path='/macros/latex/contrib/oberdiek/listingsutf8.dtx'/>
  <miktex location='oberdiek'/>
  <texlive location='oberdiek'/>
  <install path='/macros/latex/contrib/oberdiek/oberdiek.tds.zip'/>
</entry>
%</catalogue>
%    \end{macrocode}
%
% \begin{thebibliography}{9}
%
% \bibitem{inputenc}
%   Alan Jeffrey, Frank Mittelbach,
%   \textit{inputenc.sty}, 2006/05/05 v1.1b.
%   \CTAN{macros/latex/base/inputenc.dtx}
%
% \bibitem{listings}
%   Carsten Heinz, Brooks Moses:
%  \textit{The \xpackage{listings} package};
%   2007/02/22;\\
%   \CTAN{macros/latex/contrib/listings/}.
%
% \bibitem{stringenc}
%   Heiko Oberdiek:
%   \textit{The \xpackage{stringenc} package};
%   2007/10/22;\\
%   \CTAN{macros/latex/contrib/oberdiek/stringenc.pdf}.
%
% \end{thebibliography}
%
% \begin{History}
%   \begin{Version}{2007/10/22 v1.0}
%   \item
%     First version.
%   \end{Version}
%   \begin{Version}{2007/11/11 v1.1}
%   \item
%     Use of package \xpackage{pdftexcmds}.
%   \end{Version}
%   \begin{Version}{2011/11/10 v1.2}
%   \item
%     DOS line ends CR/LF normalized to LF to avoid empty lines
%     (Bug report of Thomas Benkert in de.comp.text.tex).
%   \end{Version}
% \end{History}
%
% \PrintIndex
%
% \Finale
\endinput

%        (quote the arguments according to the demands of your shell)
%
% Documentation:
%    (a) If listingsutf8.drv is present:
%           latex listingsutf8.drv
%    (b) Without listingsutf8.drv:
%           latex listingsutf8.dtx; ...
%    The class ltxdoc loads the configuration file ltxdoc.cfg
%    if available. Here you can specify further options, e.g.
%    use A4 as paper format:
%       \PassOptionsToClass{a4paper}{article}
%
%    Programm calls to get the documentation (example):
%       pdflatex listingsutf8.dtx
%       makeindex -s gind.ist listingsutf8.idx
%       pdflatex listingsutf8.dtx
%       makeindex -s gind.ist listingsutf8.idx
%       pdflatex listingsutf8.dtx
%
% Installation:
%    TDS:tex/latex/oberdiek/listingsutf8.sty
%    TDS:doc/latex/oberdiek/listingsutf8.pdf
%    TDS:doc/latex/oberdiek/test/listingsutf8-test1.tex
%    TDS:doc/latex/oberdiek/test/listingsutf8-test2.tex
%    TDS:doc/latex/oberdiek/test/listingsutf8-test3.tex
%    TDS:doc/latex/oberdiek/test/listingsutf8-test4.tex
%    TDS:doc/latex/oberdiek/test/listingsutf8-test5.tex
%    TDS:source/latex/oberdiek/listingsutf8.dtx
%
%<*ignore>
\begingroup
  \catcode123=1 %
  \catcode125=2 %
  \def\x{LaTeX2e}%
\expandafter\endgroup
\ifcase 0\ifx\install y1\fi\expandafter
         \ifx\csname processbatchFile\endcsname\relax\else1\fi
         \ifx\fmtname\x\else 1\fi\relax
\else\csname fi\endcsname
%</ignore>
%<*install>
\input docstrip.tex
\Msg{************************************************************************}
\Msg{* Installation}
\Msg{* Package: listingsutf8 2011/11/10 v1.2 Allow UTF-8 in listings input (HO)}
\Msg{************************************************************************}

\keepsilent
\askforoverwritefalse

\let\MetaPrefix\relax
\preamble

This is a generated file.

Project: listingsutf8
Version: 2011/11/10 v1.2

Copyright (C) 2007, 2011 by
   Heiko Oberdiek <heiko.oberdiek at googlemail.com>

This work may be distributed and/or modified under the
conditions of the LaTeX Project Public License, either
version 1.3c of this license or (at your option) any later
version. This version of this license is in
   http://www.latex-project.org/lppl/lppl-1-3c.txt
and the latest version of this license is in
   http://www.latex-project.org/lppl.txt
and version 1.3 or later is part of all distributions of
LaTeX version 2005/12/01 or later.

This work has the LPPL maintenance status "maintained".

This Current Maintainer of this work is Heiko Oberdiek.

This work consists of the main source file listingsutf8.dtx
and the derived files
   listingsutf8.sty, listingsutf8.pdf, listingsutf8.ins, listingsutf8.drv,
   listingsutf8-test1.tex, listingsutf8-test2.tex,
   listingsutf8-test3.tex, listingsutf8-test4.tex,
   listingsutf8-test5.tex.

\endpreamble
\let\MetaPrefix\DoubleperCent

\generate{%
  \file{listingsutf8.ins}{\from{listingsutf8.dtx}{install}}%
  \file{listingsutf8.drv}{\from{listingsutf8.dtx}{driver}}%
  \usedir{tex/latex/oberdiek}%
  \file{listingsutf8.sty}{\from{listingsutf8.dtx}{package}}%
  \usedir{doc/latex/oberdiek/test}%
  \file{listingsutf8-test1.tex}{\from{listingsutf8.dtx}{test1}}%
  \file{listingsutf8-test2.tex}{\from{listingsutf8.dtx}{test2,utf8}}%
  \file{listingsutf8-test3.tex}{\from{listingsutf8.dtx}{test3,utf8x}}%
  \file{listingsutf8-test4.tex}{\from{listingsutf8.dtx}{test4,utf8,noetex}}%
  \file{listingsutf8-test5.tex}{\from{listingsutf8.dtx}{test5,utf8x,noetex}}%
  \nopreamble
  \nopostamble
  \usedir{source/latex/oberdiek/catalogue}%
  \file{listingsutf8.xml}{\from{listingsutf8.dtx}{catalogue}}%
}

\catcode32=13\relax% active space
\let =\space%
\Msg{************************************************************************}
\Msg{*}
\Msg{* To finish the installation you have to move the following}
\Msg{* file into a directory searched by TeX:}
\Msg{*}
\Msg{*     listingsutf8.sty}
\Msg{*}
\Msg{* To produce the documentation run the file `listingsutf8.drv'}
\Msg{* through LaTeX.}
\Msg{*}
\Msg{* Happy TeXing!}
\Msg{*}
\Msg{************************************************************************}

\endbatchfile
%</install>
%<*ignore>
\fi
%</ignore>
%<*driver>
\NeedsTeXFormat{LaTeX2e}
\ProvidesFile{listingsutf8.drv}%
  [2011/11/10 v1.2 Allow UTF-8 in listings input (HO)]%
\documentclass{ltxdoc}
\usepackage{holtxdoc}[2011/11/22]
\begin{document}
  \DocInput{listingsutf8.dtx}%
\end{document}
%</driver>
% \fi
%
% \CheckSum{311}
%
% \CharacterTable
%  {Upper-case    \A\B\C\D\E\F\G\H\I\J\K\L\M\N\O\P\Q\R\S\T\U\V\W\X\Y\Z
%   Lower-case    \a\b\c\d\e\f\g\h\i\j\k\l\m\n\o\p\q\r\s\t\u\v\w\x\y\z
%   Digits        \0\1\2\3\4\5\6\7\8\9
%   Exclamation   \!     Double quote  \"     Hash (number) \#
%   Dollar        \$     Percent       \%     Ampersand     \&
%   Acute accent  \'     Left paren    \(     Right paren   \)
%   Asterisk      \*     Plus          \+     Comma         \,
%   Minus         \-     Point         \.     Solidus       \/
%   Colon         \:     Semicolon     \;     Less than     \<
%   Equals        \=     Greater than  \>     Question mark \?
%   Commercial at \@     Left bracket  \[     Backslash     \\
%   Right bracket \]     Circumflex    \^     Underscore    \_
%   Grave accent  \`     Left brace    \{     Vertical bar  \|
%   Right brace   \}     Tilde         \~}
%
% \GetFileInfo{listingsutf8.drv}
%
% \title{The \xpackage{listingsutf8} package}
% \date{2011/11/10 v1.2}
% \author{Heiko Oberdiek\\\xemail{heiko.oberdiek at googlemail.com}}
%
% \maketitle
%
% \begin{abstract}
% Package \xpackage{listings} does not support files with multi-byte
% encodings such as UTF-8. In case of \cs{lstinputlisting} a simple
% workaround is possible if an one-byte encoding exists that the file
% can be converted to. Also \eTeX\ and \pdfTeX\ regardless of its mode
% are required.
% \end{abstract}
%
% \tableofcontents
%
% \section{Documentation}
%
% \subsection{User interface}
%
% Load this package after or instead of package \xpackage{listings}
% \cite{listings}.
% The package does not define own options and passes given options to
% package \xpackage{listings}.
%
% The syntax of package \xpackage{listings}' key \xoption{inputencoding}
% is extended:
% \begin{quote}
%   |inputencoding=utf8/|\meta{one-byte-encoding}\\
%   Example: |inputencoding=utf8/latin1|
% \end{quote}
% That means the file is encoded in UTF-8 and can
% be converted to the given \meta{one-byte-encoding}.
% The available encodings for \meta{one-byte-encoding} are
% listed in section ``1.2 Supported encodings'' of
% package \xpackage{stringenc}'s documentation \cite{stringenc}.
% Of course, the encoding must encode its characters with
% one byte exactly. This excludes the unicode encodings
% (\xoption{utf8}, \xoption{utf16}, \dots).
%
% Only \cs{lstinputlisting} is supported by the syntax extension
% of key \xoption{inputencoding}.
%
% Internally package \xpackage{listingsutf8} reads the file as binary file
% via primitives of \pdfTeX\ (\cs{pdffiledump}). Then the file
% contents is converted as string using package \xpackage{stringenc} and
% finally the string is read as virtual file by \eTeX's \cs{scantokens}.
%
% \subsection{Future}
%
% Workarounds are not provided for
% \begin{itemize}
% \item \cs{lstinline}
% \item Environment |lstlisting|.
% \item Environments defined by \cs{lstnewenvironment}.
% \end{itemize}
% Perhaps someone will find time to extend package \xpackage{listings}
% with full native support for UTF-8. Then this package would become obsolete.
%
% \StopEventually{
% }
%
% \section{Implementation}
%
%    \begin{macrocode}
%<*package>
%    \end{macrocode}
%
% \subsection{Catcodes and identification}
%
%    \begin{macrocode}
\begingroup\catcode61\catcode48\catcode32=10\relax%
  \catcode13=5 % ^^M
  \endlinechar=13 %
  \catcode123=1 % {
  \catcode125=2 % }
  \catcode64=11 % @
  \def\x{\endgroup
    \expandafter\edef\csname lstU@AtEnd\endcsname{%
      \endlinechar=\the\endlinechar\relax
      \catcode13=\the\catcode13\relax
      \catcode32=\the\catcode32\relax
      \catcode35=\the\catcode35\relax
      \catcode61=\the\catcode61\relax
      \catcode64=\the\catcode64\relax
      \catcode123=\the\catcode123\relax
      \catcode125=\the\catcode125\relax
    }%
  }%
\x\catcode61\catcode48\catcode32=10\relax%
\catcode13=5 % ^^M
\endlinechar=13 %
\catcode35=6 % #
\catcode64=11 % @
\catcode123=1 % {
\catcode125=2 % }
\def\TMP@EnsureCode#1#2{%
  \edef\lstU@AtEnd{%
    \lstU@AtEnd
    \catcode#1=\the\catcode#1\relax
  }%
  \catcode#1=#2\relax
}
\TMP@EnsureCode{10}{12}% ^^J
\TMP@EnsureCode{33}{12}% !
\TMP@EnsureCode{36}{3}% $
\TMP@EnsureCode{38}{4}% &
\TMP@EnsureCode{39}{12}% '
\TMP@EnsureCode{40}{12}% (
\TMP@EnsureCode{41}{12}% )
\TMP@EnsureCode{42}{12}% *
\TMP@EnsureCode{43}{12}% +
\TMP@EnsureCode{44}{12}% ,
\TMP@EnsureCode{45}{12}% -
\TMP@EnsureCode{46}{12}% .
\TMP@EnsureCode{47}{12}% /
\TMP@EnsureCode{58}{12}% :
\TMP@EnsureCode{60}{12}% <
\TMP@EnsureCode{62}{12}% >
\TMP@EnsureCode{91}{12}% [
\TMP@EnsureCode{93}{12}% ]
\TMP@EnsureCode{94}{7}% ^ (superscript)
\TMP@EnsureCode{95}{8}% _ (subscript)
\TMP@EnsureCode{96}{12}% `
\TMP@EnsureCode{124}{12}% |
\TMP@EnsureCode{126}{13}% ~ (active)
\edef\lstU@AtEnd{\lstU@AtEnd\noexpand\endinput}
%    \end{macrocode}
%
%    Package identification.
%    \begin{macrocode}
\NeedsTeXFormat{LaTeX2e}
\ProvidesPackage{listingsutf8}%
  [2011/11/10 v1.2 Allow UTF-8 in listings input (HO)]
%    \end{macrocode}
%
% \subsection{Package options}
%
% Just pass options to package listings.
%
%    \begin{macrocode}
\DeclareOption*{%
  \PassOptionsToPackage\CurrentOption{listings}%
}
\ProcessOptions*
%    \end{macrocode}
%    Key \xoption{inputencoding} was introduced in version
%    2002/04/01 v1.0 of package \xpackage{listings}.
%    \begin{macrocode}
\RequirePackage{listings}[2002/04/01]
%    \end{macrocode}
%    Ensure that \cs{inputencoding} is provided.
%    \begin{macrocode}
\AtBeginDocument{%
  \@ifundefined{inputencoding}{%
    \RequirePackage{inputenc}%
  }{}%
}
%    \end{macrocode}
%
% \subsection{Check prerequisites}
%
%    \begin{macrocode}
\RequirePackage{pdftexcmds}[2011/04/22]
%    \end{macrocode}
%
%    \begin{macrocode}
\def\lstU@temp#1#2{%
  \begingroup\expandafter\expandafter\expandafter\endgroup
  \expandafter\ifx\csname #1\endcsname\relax
    \PackageWarningNoLine{listingsutf8}{%
      Package loading is aborted because of missing %
      \@backslashchar#1.\MessageBreak
      #2%
    }%
    \expandafter\lstU@AtEnd
  \fi
}
\lstU@temp{scantokens}{It is provided by e-TeX}%
\lstU@temp{pdf@unescapehex}{It is provided by pdfTeX >= 1.30}%
\lstU@temp{pdf@filedump}{It is provided by pdfTeX >= 1.30}%
\lstU@temp{pdf@filesize}{It is provided by pdfTeX >= 1.30}%
%    \end{macrocode}
%
%    \begin{macrocode}
\RequirePackage{stringenc}[2010/03/01]
%    \end{macrocode}
%
% \subsection{Add support for UTF-8}
%
%    \begin{macro}{\iflstU@utfviii}
%    \begin{macrocode}
\newif\iflstU@utfviii
%    \end{macrocode}
%    \end{macro}
%
%    \begin{macro}{\lstU@inputenc}
%    \begin{macrocode}
\def\lstU@inputenc#1{%
  \expandafter\lstU@@inputenc#1utf8/utf8/\@nil
}
%    \end{macrocode}
%    \end{macro}
%    \begin{macro}{\lstU@@inputenc}
\def\lstU@@inputenc#1utf8/#2utf8/#3\@nil{%
  \ifx\\#1\\%
    \lstU@utfviiitrue
    \def\lst@inputenc{#2}%
  \else
    \lstU@utfviiifalse
  \fi
}
%    \begin{macrocode}
%    \end{macrocode}
%    \end{macro}
%
%    \begin{macrocode}
\lst@Key{inputencoding}\relax{%
  \def\lst@inputenc{#1}%
  \lstU@inputenc{#1}%
}
%    \end{macrocode}
%
% \subsubsection{Conversion}
%
%    \begin{macro}{\lstU@input}
%    \begin{macrocode}
\def\lstU@input#1{%
  \iflstU@utfviii
    \edef\lstU@text{%
      \pdf@unescapehex{%
        \pdf@filedump{0}{\pdf@filesize{#1}}{#1}%
      }%
    }%
    \lstU@CRLFtoLF\lstU@text
    \StringEncodingConvert\lstU@text\lstU@text{utf8}\lst@inputenc
    \def\lstU@temp{%
      \scantokens\expandafter{\lstU@text}%
    }%
  \else
    \def\lstU@temp{%
      \input{#1}%
    }%
  \fi
  \lstU@temp
}
%    \end{macrocode}
%    \end{macro}
%
% \subsubsection{Convert CR/LF pairs to LF}
%
%    \begin{macro}{\lstU@CRLFtoLF}
%    \begin{macrocode}
\begingroup
  \endlinechar=-1 %
  \@makeother\^^J %
  \@makeother\^^M %
  \gdef\lstU@CRLFtoLF#1{%
    \edef#1{%
      \expandafter\lstU@CRLFtoLF@aux#1^^M^^J\@nil
    }%
  }%
  \gdef\lstU@CRLFtoLF@aux#1^^M^^J#2\@nil{%
    #1%
    \ifx\relax#2\relax
      \@car
    \fi
    ^^J%
    \lstU@CRLFtoLF@aux#2\@nil
  }%
\endgroup %
%    \end{macrocode}
%    \end{macro}
%
% \subsubsection{Patch \cs{lst@InputListing}}
%
%    \begin{macrocode}
\def\lstU@temp#1\def\lst@next#2#3\@nil{%
  \def\lst@InputListing##1{%
    #1%
    \def\lst@next{\lstU@input{##1}}%
    #3%
  }%
}
\expandafter\lstU@temp\lst@InputListing{#1}\@nil
%    \end{macrocode}
%
%    \begin{macrocode}
\lstU@AtEnd%
%</package>
%    \end{macrocode}
%
% \section{Test}
%
% \subsection{Catcode checks for loading}
%
%    \begin{macrocode}
%<*test1>
%    \end{macrocode}
%    \begin{macrocode}
\NeedsTeXFormat{LaTeX2e}
\documentclass{minimal}
\makeatletter
\def\RestoreCatcodes{}
\count@=0 %
\loop
  \edef\RestoreCatcodes{%
    \RestoreCatcodes
    \catcode\the\count@=\the\catcode\count@\relax
  }%
\ifnum\count@<255 %
  \advance\count@\@ne
\repeat

\def\RangeCatcodeInvalid#1#2{%
  \count@=#1\relax
  \loop
    \catcode\count@=15 %
  \ifnum\count@<#2\relax
    \advance\count@\@ne
  \repeat
}
\def\Test{%
  \RangeCatcodeInvalid{0}{47}%
  \RangeCatcodeInvalid{58}{64}%
  \RangeCatcodeInvalid{91}{96}%
  \RangeCatcodeInvalid{123}{127}%
  \catcode`\@=12 %
  \catcode`\\=0 %
  \catcode`\{=1 %
  \catcode`\}=2 %
  \catcode`\#=6 %
  \catcode`\[=12 %
  \catcode`\]=12 %
  \catcode`\%=14 %
  \catcode`\ =10 %
  \catcode13=5 %
  \RequirePackage{listingsutf8}[2011/11/10]\relax
  \RestoreCatcodes
}
\Test
\csname @@end\endcsname
\end
%    \end{macrocode}
%    \begin{macrocode}
%</test1>
%    \end{macrocode}
%
% \subsection{Test example for latin1}
%
%    \begin{macrocode}
%<*test2>
%    \end{macrocode}
%    \begin{macrocode}
\NeedsTeXFormat{LaTeX2e}
\documentclass{minimal}
\usepackage{filecontents}
\def\do#1{%
  \ifx#1\^%
  \else
    \noexpand\do\noexpand#1%
  \fi
}
\expandafter\let\expandafter\dospecials\expandafter\empty
\expandafter\edef\expandafter\dospecials\expandafter{\dospecials}
\begin{filecontents*}{ExampleUTF8.java}
public class ExampleUTF8 {
    public static String testString =
        "Umlauts: " +
        "^^c3^^84^^c3^^96^^c3^^9c^^c3^^a4^^c3^^b6^^c3^^bc^^c3^^9f";
    public static void main(String[] args) {
        System.out.println(testString);
    }
}
\end{filecontents*}
\usepackage{listingsutf8}[2011/11/10]
\def\Text{%
  Umlauts: %
  ^^c3^^84^^c3^^96^^c3^^9c^^c3^^a4^^c3^^b6^^c3^^bc^^c3^^9f%
}
\begin{document}
\lstinputlisting[%
  language=Java,%
  inputencoding=utf8/latin1,%
]{ExampleUTF8.java}
\end{document}
%</test2>
%    \end{macrocode}
%
% \section{Installation}
%
% \subsection{Download}
%
% \paragraph{Package.} This package is available on
% CTAN\footnote{\url{ftp://ftp.ctan.org/tex-archive/}}:
% \begin{description}
% \item[\CTAN{macros/latex/contrib/oberdiek/listingsutf8.dtx}] The source file.
% \item[\CTAN{macros/latex/contrib/oberdiek/listingsutf8.pdf}] Documentation.
% \end{description}
%
%
% \paragraph{Bundle.} All the packages of the bundle `oberdiek'
% are also available in a TDS compliant ZIP archive. There
% the packages are already unpacked and the documentation files
% are generated. The files and directories obey the TDS standard.
% \begin{description}
% \item[\CTAN{install/macros/latex/contrib/oberdiek.tds.zip}]
% \end{description}
% \emph{TDS} refers to the standard ``A Directory Structure
% for \TeX\ Files'' (\CTAN{tds/tds.pdf}). Directories
% with \xfile{texmf} in their name are usually organized this way.
%
% \subsection{Bundle installation}
%
% \paragraph{Unpacking.} Unpack the \xfile{oberdiek.tds.zip} in the
% TDS tree (also known as \xfile{texmf} tree) of your choice.
% Example (linux):
% \begin{quote}
%   |unzip oberdiek.tds.zip -d ~/texmf|
% \end{quote}
%
% \paragraph{Script installation.}
% Check the directory \xfile{TDS:scripts/oberdiek/} for
% scripts that need further installation steps.
% Package \xpackage{attachfile2} comes with the Perl script
% \xfile{pdfatfi.pl} that should be installed in such a way
% that it can be called as \texttt{pdfatfi}.
% Example (linux):
% \begin{quote}
%   |chmod +x scripts/oberdiek/pdfatfi.pl|\\
%   |cp scripts/oberdiek/pdfatfi.pl /usr/local/bin/|
% \end{quote}
%
% \subsection{Package installation}
%
% \paragraph{Unpacking.} The \xfile{.dtx} file is a self-extracting
% \docstrip\ archive. The files are extracted by running the
% \xfile{.dtx} through \plainTeX:
% \begin{quote}
%   \verb|tex listingsutf8.dtx|
% \end{quote}
%
% \paragraph{TDS.} Now the different files must be moved into
% the different directories in your installation TDS tree
% (also known as \xfile{texmf} tree):
% \begin{quote}
% \def\t{^^A
% \begin{tabular}{@{}>{\ttfamily}l@{ $\rightarrow$ }>{\ttfamily}l@{}}
%   listingsutf8.sty & tex/latex/oberdiek/listingsutf8.sty\\
%   listingsutf8.pdf & doc/latex/oberdiek/listingsutf8.pdf\\
%   test/listingsutf8-test1.tex & doc/latex/oberdiek/test/listingsutf8-test1.tex\\
%   test/listingsutf8-test2.tex & doc/latex/oberdiek/test/listingsutf8-test2.tex\\
%   test/listingsutf8-test3.tex & doc/latex/oberdiek/test/listingsutf8-test3.tex\\
%   test/listingsutf8-test4.tex & doc/latex/oberdiek/test/listingsutf8-test4.tex\\
%   test/listingsutf8-test5.tex & doc/latex/oberdiek/test/listingsutf8-test5.tex\\
%   listingsutf8.dtx & source/latex/oberdiek/listingsutf8.dtx\\
% \end{tabular}^^A
% }^^A
% \sbox0{\t}^^A
% \ifdim\wd0>\linewidth
%   \begingroup
%     \advance\linewidth by\leftmargin
%     \advance\linewidth by\rightmargin
%   \edef\x{\endgroup
%     \def\noexpand\lw{\the\linewidth}^^A
%   }\x
%   \def\lwbox{^^A
%     \leavevmode
%     \hbox to \linewidth{^^A
%       \kern-\leftmargin\relax
%       \hss
%       \usebox0
%       \hss
%       \kern-\rightmargin\relax
%     }^^A
%   }^^A
%   \ifdim\wd0>\lw
%     \sbox0{\small\t}^^A
%     \ifdim\wd0>\linewidth
%       \ifdim\wd0>\lw
%         \sbox0{\footnotesize\t}^^A
%         \ifdim\wd0>\linewidth
%           \ifdim\wd0>\lw
%             \sbox0{\scriptsize\t}^^A
%             \ifdim\wd0>\linewidth
%               \ifdim\wd0>\lw
%                 \sbox0{\tiny\t}^^A
%                 \ifdim\wd0>\linewidth
%                   \lwbox
%                 \else
%                   \usebox0
%                 \fi
%               \else
%                 \lwbox
%               \fi
%             \else
%               \usebox0
%             \fi
%           \else
%             \lwbox
%           \fi
%         \else
%           \usebox0
%         \fi
%       \else
%         \lwbox
%       \fi
%     \else
%       \usebox0
%     \fi
%   \else
%     \lwbox
%   \fi
% \else
%   \usebox0
% \fi
% \end{quote}
% If you have a \xfile{docstrip.cfg} that configures and enables \docstrip's
% TDS installing feature, then some files can already be in the right
% place, see the documentation of \docstrip.
%
% \subsection{Refresh file name databases}
%
% If your \TeX~distribution
% (\teTeX, \mikTeX, \dots) relies on file name databases, you must refresh
% these. For example, \teTeX\ users run \verb|texhash| or
% \verb|mktexlsr|.
%
% \subsection{Some details for the interested}
%
% \paragraph{Attached source.}
%
% The PDF documentation on CTAN also includes the
% \xfile{.dtx} source file. It can be extracted by
% AcrobatReader 6 or higher. Another option is \textsf{pdftk},
% e.g. unpack the file into the current directory:
% \begin{quote}
%   \verb|pdftk listingsutf8.pdf unpack_files output .|
% \end{quote}
%
% \paragraph{Unpacking with \LaTeX.}
% The \xfile{.dtx} chooses its action depending on the format:
% \begin{description}
% \item[\plainTeX:] Run \docstrip\ and extract the files.
% \item[\LaTeX:] Generate the documentation.
% \end{description}
% If you insist on using \LaTeX\ for \docstrip\ (really,
% \docstrip\ does not need \LaTeX), then inform the autodetect routine
% about your intention:
% \begin{quote}
%   \verb|latex \let\install=y% \iffalse meta-comment
%
% File: listingsutf8.dtx
% Version: 2011/11/10 v1.2
% Info: Allow UTF-8 in listings input
%
% Copyright (C) 2007, 2011 by
%    Heiko Oberdiek <heiko.oberdiek at googlemail.com>
%
% This work may be distributed and/or modified under the
% conditions of the LaTeX Project Public License, either
% version 1.3c of this license or (at your option) any later
% version. This version of this license is in
%    http://www.latex-project.org/lppl/lppl-1-3c.txt
% and the latest version of this license is in
%    http://www.latex-project.org/lppl.txt
% and version 1.3 or later is part of all distributions of
% LaTeX version 2005/12/01 or later.
%
% This work has the LPPL maintenance status "maintained".
%
% This Current Maintainer of this work is Heiko Oberdiek.
%
% This work consists of the main source file listingsutf8.dtx
% and the derived files
%    listingsutf8.sty, listingsutf8.pdf, listingsutf8.ins, listingsutf8.drv,
%    listingsutf8-test1.tex, listingsutf8-test2.tex,
%    listingsutf8-test3.tex, listingsutf8-test4.tex,
%    listingsutf8-test5.tex.
%
% Distribution:
%    CTAN:macros/latex/contrib/oberdiek/listingsutf8.dtx
%    CTAN:macros/latex/contrib/oberdiek/listingsutf8.pdf
%
% Unpacking:
%    (a) If listingsutf8.ins is present:
%           tex listingsutf8.ins
%    (b) Without listingsutf8.ins:
%           tex listingsutf8.dtx
%    (c) If you insist on using LaTeX
%           latex \let\install=y\input{listingsutf8.dtx}
%        (quote the arguments according to the demands of your shell)
%
% Documentation:
%    (a) If listingsutf8.drv is present:
%           latex listingsutf8.drv
%    (b) Without listingsutf8.drv:
%           latex listingsutf8.dtx; ...
%    The class ltxdoc loads the configuration file ltxdoc.cfg
%    if available. Here you can specify further options, e.g.
%    use A4 as paper format:
%       \PassOptionsToClass{a4paper}{article}
%
%    Programm calls to get the documentation (example):
%       pdflatex listingsutf8.dtx
%       makeindex -s gind.ist listingsutf8.idx
%       pdflatex listingsutf8.dtx
%       makeindex -s gind.ist listingsutf8.idx
%       pdflatex listingsutf8.dtx
%
% Installation:
%    TDS:tex/latex/oberdiek/listingsutf8.sty
%    TDS:doc/latex/oberdiek/listingsutf8.pdf
%    TDS:doc/latex/oberdiek/test/listingsutf8-test1.tex
%    TDS:doc/latex/oberdiek/test/listingsutf8-test2.tex
%    TDS:doc/latex/oberdiek/test/listingsutf8-test3.tex
%    TDS:doc/latex/oberdiek/test/listingsutf8-test4.tex
%    TDS:doc/latex/oberdiek/test/listingsutf8-test5.tex
%    TDS:source/latex/oberdiek/listingsutf8.dtx
%
%<*ignore>
\begingroup
  \catcode123=1 %
  \catcode125=2 %
  \def\x{LaTeX2e}%
\expandafter\endgroup
\ifcase 0\ifx\install y1\fi\expandafter
         \ifx\csname processbatchFile\endcsname\relax\else1\fi
         \ifx\fmtname\x\else 1\fi\relax
\else\csname fi\endcsname
%</ignore>
%<*install>
\input docstrip.tex
\Msg{************************************************************************}
\Msg{* Installation}
\Msg{* Package: listingsutf8 2011/11/10 v1.2 Allow UTF-8 in listings input (HO)}
\Msg{************************************************************************}

\keepsilent
\askforoverwritefalse

\let\MetaPrefix\relax
\preamble

This is a generated file.

Project: listingsutf8
Version: 2011/11/10 v1.2

Copyright (C) 2007, 2011 by
   Heiko Oberdiek <heiko.oberdiek at googlemail.com>

This work may be distributed and/or modified under the
conditions of the LaTeX Project Public License, either
version 1.3c of this license or (at your option) any later
version. This version of this license is in
   http://www.latex-project.org/lppl/lppl-1-3c.txt
and the latest version of this license is in
   http://www.latex-project.org/lppl.txt
and version 1.3 or later is part of all distributions of
LaTeX version 2005/12/01 or later.

This work has the LPPL maintenance status "maintained".

This Current Maintainer of this work is Heiko Oberdiek.

This work consists of the main source file listingsutf8.dtx
and the derived files
   listingsutf8.sty, listingsutf8.pdf, listingsutf8.ins, listingsutf8.drv,
   listingsutf8-test1.tex, listingsutf8-test2.tex,
   listingsutf8-test3.tex, listingsutf8-test4.tex,
   listingsutf8-test5.tex.

\endpreamble
\let\MetaPrefix\DoubleperCent

\generate{%
  \file{listingsutf8.ins}{\from{listingsutf8.dtx}{install}}%
  \file{listingsutf8.drv}{\from{listingsutf8.dtx}{driver}}%
  \usedir{tex/latex/oberdiek}%
  \file{listingsutf8.sty}{\from{listingsutf8.dtx}{package}}%
  \usedir{doc/latex/oberdiek/test}%
  \file{listingsutf8-test1.tex}{\from{listingsutf8.dtx}{test1}}%
  \file{listingsutf8-test2.tex}{\from{listingsutf8.dtx}{test2,utf8}}%
  \file{listingsutf8-test3.tex}{\from{listingsutf8.dtx}{test3,utf8x}}%
  \file{listingsutf8-test4.tex}{\from{listingsutf8.dtx}{test4,utf8,noetex}}%
  \file{listingsutf8-test5.tex}{\from{listingsutf8.dtx}{test5,utf8x,noetex}}%
  \nopreamble
  \nopostamble
  \usedir{source/latex/oberdiek/catalogue}%
  \file{listingsutf8.xml}{\from{listingsutf8.dtx}{catalogue}}%
}

\catcode32=13\relax% active space
\let =\space%
\Msg{************************************************************************}
\Msg{*}
\Msg{* To finish the installation you have to move the following}
\Msg{* file into a directory searched by TeX:}
\Msg{*}
\Msg{*     listingsutf8.sty}
\Msg{*}
\Msg{* To produce the documentation run the file `listingsutf8.drv'}
\Msg{* through LaTeX.}
\Msg{*}
\Msg{* Happy TeXing!}
\Msg{*}
\Msg{************************************************************************}

\endbatchfile
%</install>
%<*ignore>
\fi
%</ignore>
%<*driver>
\NeedsTeXFormat{LaTeX2e}
\ProvidesFile{listingsutf8.drv}%
  [2011/11/10 v1.2 Allow UTF-8 in listings input (HO)]%
\documentclass{ltxdoc}
\usepackage{holtxdoc}[2011/11/22]
\begin{document}
  \DocInput{listingsutf8.dtx}%
\end{document}
%</driver>
% \fi
%
% \CheckSum{311}
%
% \CharacterTable
%  {Upper-case    \A\B\C\D\E\F\G\H\I\J\K\L\M\N\O\P\Q\R\S\T\U\V\W\X\Y\Z
%   Lower-case    \a\b\c\d\e\f\g\h\i\j\k\l\m\n\o\p\q\r\s\t\u\v\w\x\y\z
%   Digits        \0\1\2\3\4\5\6\7\8\9
%   Exclamation   \!     Double quote  \"     Hash (number) \#
%   Dollar        \$     Percent       \%     Ampersand     \&
%   Acute accent  \'     Left paren    \(     Right paren   \)
%   Asterisk      \*     Plus          \+     Comma         \,
%   Minus         \-     Point         \.     Solidus       \/
%   Colon         \:     Semicolon     \;     Less than     \<
%   Equals        \=     Greater than  \>     Question mark \?
%   Commercial at \@     Left bracket  \[     Backslash     \\
%   Right bracket \]     Circumflex    \^     Underscore    \_
%   Grave accent  \`     Left brace    \{     Vertical bar  \|
%   Right brace   \}     Tilde         \~}
%
% \GetFileInfo{listingsutf8.drv}
%
% \title{The \xpackage{listingsutf8} package}
% \date{2011/11/10 v1.2}
% \author{Heiko Oberdiek\\\xemail{heiko.oberdiek at googlemail.com}}
%
% \maketitle
%
% \begin{abstract}
% Package \xpackage{listings} does not support files with multi-byte
% encodings such as UTF-8. In case of \cs{lstinputlisting} a simple
% workaround is possible if an one-byte encoding exists that the file
% can be converted to. Also \eTeX\ and \pdfTeX\ regardless of its mode
% are required.
% \end{abstract}
%
% \tableofcontents
%
% \section{Documentation}
%
% \subsection{User interface}
%
% Load this package after or instead of package \xpackage{listings}
% \cite{listings}.
% The package does not define own options and passes given options to
% package \xpackage{listings}.
%
% The syntax of package \xpackage{listings}' key \xoption{inputencoding}
% is extended:
% \begin{quote}
%   |inputencoding=utf8/|\meta{one-byte-encoding}\\
%   Example: |inputencoding=utf8/latin1|
% \end{quote}
% That means the file is encoded in UTF-8 and can
% be converted to the given \meta{one-byte-encoding}.
% The available encodings for \meta{one-byte-encoding} are
% listed in section ``1.2 Supported encodings'' of
% package \xpackage{stringenc}'s documentation \cite{stringenc}.
% Of course, the encoding must encode its characters with
% one byte exactly. This excludes the unicode encodings
% (\xoption{utf8}, \xoption{utf16}, \dots).
%
% Only \cs{lstinputlisting} is supported by the syntax extension
% of key \xoption{inputencoding}.
%
% Internally package \xpackage{listingsutf8} reads the file as binary file
% via primitives of \pdfTeX\ (\cs{pdffiledump}). Then the file
% contents is converted as string using package \xpackage{stringenc} and
% finally the string is read as virtual file by \eTeX's \cs{scantokens}.
%
% \subsection{Future}
%
% Workarounds are not provided for
% \begin{itemize}
% \item \cs{lstinline}
% \item Environment |lstlisting|.
% \item Environments defined by \cs{lstnewenvironment}.
% \end{itemize}
% Perhaps someone will find time to extend package \xpackage{listings}
% with full native support for UTF-8. Then this package would become obsolete.
%
% \StopEventually{
% }
%
% \section{Implementation}
%
%    \begin{macrocode}
%<*package>
%    \end{macrocode}
%
% \subsection{Catcodes and identification}
%
%    \begin{macrocode}
\begingroup\catcode61\catcode48\catcode32=10\relax%
  \catcode13=5 % ^^M
  \endlinechar=13 %
  \catcode123=1 % {
  \catcode125=2 % }
  \catcode64=11 % @
  \def\x{\endgroup
    \expandafter\edef\csname lstU@AtEnd\endcsname{%
      \endlinechar=\the\endlinechar\relax
      \catcode13=\the\catcode13\relax
      \catcode32=\the\catcode32\relax
      \catcode35=\the\catcode35\relax
      \catcode61=\the\catcode61\relax
      \catcode64=\the\catcode64\relax
      \catcode123=\the\catcode123\relax
      \catcode125=\the\catcode125\relax
    }%
  }%
\x\catcode61\catcode48\catcode32=10\relax%
\catcode13=5 % ^^M
\endlinechar=13 %
\catcode35=6 % #
\catcode64=11 % @
\catcode123=1 % {
\catcode125=2 % }
\def\TMP@EnsureCode#1#2{%
  \edef\lstU@AtEnd{%
    \lstU@AtEnd
    \catcode#1=\the\catcode#1\relax
  }%
  \catcode#1=#2\relax
}
\TMP@EnsureCode{10}{12}% ^^J
\TMP@EnsureCode{33}{12}% !
\TMP@EnsureCode{36}{3}% $
\TMP@EnsureCode{38}{4}% &
\TMP@EnsureCode{39}{12}% '
\TMP@EnsureCode{40}{12}% (
\TMP@EnsureCode{41}{12}% )
\TMP@EnsureCode{42}{12}% *
\TMP@EnsureCode{43}{12}% +
\TMP@EnsureCode{44}{12}% ,
\TMP@EnsureCode{45}{12}% -
\TMP@EnsureCode{46}{12}% .
\TMP@EnsureCode{47}{12}% /
\TMP@EnsureCode{58}{12}% :
\TMP@EnsureCode{60}{12}% <
\TMP@EnsureCode{62}{12}% >
\TMP@EnsureCode{91}{12}% [
\TMP@EnsureCode{93}{12}% ]
\TMP@EnsureCode{94}{7}% ^ (superscript)
\TMP@EnsureCode{95}{8}% _ (subscript)
\TMP@EnsureCode{96}{12}% `
\TMP@EnsureCode{124}{12}% |
\TMP@EnsureCode{126}{13}% ~ (active)
\edef\lstU@AtEnd{\lstU@AtEnd\noexpand\endinput}
%    \end{macrocode}
%
%    Package identification.
%    \begin{macrocode}
\NeedsTeXFormat{LaTeX2e}
\ProvidesPackage{listingsutf8}%
  [2011/11/10 v1.2 Allow UTF-8 in listings input (HO)]
%    \end{macrocode}
%
% \subsection{Package options}
%
% Just pass options to package listings.
%
%    \begin{macrocode}
\DeclareOption*{%
  \PassOptionsToPackage\CurrentOption{listings}%
}
\ProcessOptions*
%    \end{macrocode}
%    Key \xoption{inputencoding} was introduced in version
%    2002/04/01 v1.0 of package \xpackage{listings}.
%    \begin{macrocode}
\RequirePackage{listings}[2002/04/01]
%    \end{macrocode}
%    Ensure that \cs{inputencoding} is provided.
%    \begin{macrocode}
\AtBeginDocument{%
  \@ifundefined{inputencoding}{%
    \RequirePackage{inputenc}%
  }{}%
}
%    \end{macrocode}
%
% \subsection{Check prerequisites}
%
%    \begin{macrocode}
\RequirePackage{pdftexcmds}[2011/04/22]
%    \end{macrocode}
%
%    \begin{macrocode}
\def\lstU@temp#1#2{%
  \begingroup\expandafter\expandafter\expandafter\endgroup
  \expandafter\ifx\csname #1\endcsname\relax
    \PackageWarningNoLine{listingsutf8}{%
      Package loading is aborted because of missing %
      \@backslashchar#1.\MessageBreak
      #2%
    }%
    \expandafter\lstU@AtEnd
  \fi
}
\lstU@temp{scantokens}{It is provided by e-TeX}%
\lstU@temp{pdf@unescapehex}{It is provided by pdfTeX >= 1.30}%
\lstU@temp{pdf@filedump}{It is provided by pdfTeX >= 1.30}%
\lstU@temp{pdf@filesize}{It is provided by pdfTeX >= 1.30}%
%    \end{macrocode}
%
%    \begin{macrocode}
\RequirePackage{stringenc}[2010/03/01]
%    \end{macrocode}
%
% \subsection{Add support for UTF-8}
%
%    \begin{macro}{\iflstU@utfviii}
%    \begin{macrocode}
\newif\iflstU@utfviii
%    \end{macrocode}
%    \end{macro}
%
%    \begin{macro}{\lstU@inputenc}
%    \begin{macrocode}
\def\lstU@inputenc#1{%
  \expandafter\lstU@@inputenc#1utf8/utf8/\@nil
}
%    \end{macrocode}
%    \end{macro}
%    \begin{macro}{\lstU@@inputenc}
\def\lstU@@inputenc#1utf8/#2utf8/#3\@nil{%
  \ifx\\#1\\%
    \lstU@utfviiitrue
    \def\lst@inputenc{#2}%
  \else
    \lstU@utfviiifalse
  \fi
}
%    \begin{macrocode}
%    \end{macrocode}
%    \end{macro}
%
%    \begin{macrocode}
\lst@Key{inputencoding}\relax{%
  \def\lst@inputenc{#1}%
  \lstU@inputenc{#1}%
}
%    \end{macrocode}
%
% \subsubsection{Conversion}
%
%    \begin{macro}{\lstU@input}
%    \begin{macrocode}
\def\lstU@input#1{%
  \iflstU@utfviii
    \edef\lstU@text{%
      \pdf@unescapehex{%
        \pdf@filedump{0}{\pdf@filesize{#1}}{#1}%
      }%
    }%
    \lstU@CRLFtoLF\lstU@text
    \StringEncodingConvert\lstU@text\lstU@text{utf8}\lst@inputenc
    \def\lstU@temp{%
      \scantokens\expandafter{\lstU@text}%
    }%
  \else
    \def\lstU@temp{%
      \input{#1}%
    }%
  \fi
  \lstU@temp
}
%    \end{macrocode}
%    \end{macro}
%
% \subsubsection{Convert CR/LF pairs to LF}
%
%    \begin{macro}{\lstU@CRLFtoLF}
%    \begin{macrocode}
\begingroup
  \endlinechar=-1 %
  \@makeother\^^J %
  \@makeother\^^M %
  \gdef\lstU@CRLFtoLF#1{%
    \edef#1{%
      \expandafter\lstU@CRLFtoLF@aux#1^^M^^J\@nil
    }%
  }%
  \gdef\lstU@CRLFtoLF@aux#1^^M^^J#2\@nil{%
    #1%
    \ifx\relax#2\relax
      \@car
    \fi
    ^^J%
    \lstU@CRLFtoLF@aux#2\@nil
  }%
\endgroup %
%    \end{macrocode}
%    \end{macro}
%
% \subsubsection{Patch \cs{lst@InputListing}}
%
%    \begin{macrocode}
\def\lstU@temp#1\def\lst@next#2#3\@nil{%
  \def\lst@InputListing##1{%
    #1%
    \def\lst@next{\lstU@input{##1}}%
    #3%
  }%
}
\expandafter\lstU@temp\lst@InputListing{#1}\@nil
%    \end{macrocode}
%
%    \begin{macrocode}
\lstU@AtEnd%
%</package>
%    \end{macrocode}
%
% \section{Test}
%
% \subsection{Catcode checks for loading}
%
%    \begin{macrocode}
%<*test1>
%    \end{macrocode}
%    \begin{macrocode}
\NeedsTeXFormat{LaTeX2e}
\documentclass{minimal}
\makeatletter
\def\RestoreCatcodes{}
\count@=0 %
\loop
  \edef\RestoreCatcodes{%
    \RestoreCatcodes
    \catcode\the\count@=\the\catcode\count@\relax
  }%
\ifnum\count@<255 %
  \advance\count@\@ne
\repeat

\def\RangeCatcodeInvalid#1#2{%
  \count@=#1\relax
  \loop
    \catcode\count@=15 %
  \ifnum\count@<#2\relax
    \advance\count@\@ne
  \repeat
}
\def\Test{%
  \RangeCatcodeInvalid{0}{47}%
  \RangeCatcodeInvalid{58}{64}%
  \RangeCatcodeInvalid{91}{96}%
  \RangeCatcodeInvalid{123}{127}%
  \catcode`\@=12 %
  \catcode`\\=0 %
  \catcode`\{=1 %
  \catcode`\}=2 %
  \catcode`\#=6 %
  \catcode`\[=12 %
  \catcode`\]=12 %
  \catcode`\%=14 %
  \catcode`\ =10 %
  \catcode13=5 %
  \RequirePackage{listingsutf8}[2011/11/10]\relax
  \RestoreCatcodes
}
\Test
\csname @@end\endcsname
\end
%    \end{macrocode}
%    \begin{macrocode}
%</test1>
%    \end{macrocode}
%
% \subsection{Test example for latin1}
%
%    \begin{macrocode}
%<*test2>
%    \end{macrocode}
%    \begin{macrocode}
\NeedsTeXFormat{LaTeX2e}
\documentclass{minimal}
\usepackage{filecontents}
\def\do#1{%
  \ifx#1\^%
  \else
    \noexpand\do\noexpand#1%
  \fi
}
\expandafter\let\expandafter\dospecials\expandafter\empty
\expandafter\edef\expandafter\dospecials\expandafter{\dospecials}
\begin{filecontents*}{ExampleUTF8.java}
public class ExampleUTF8 {
    public static String testString =
        "Umlauts: " +
        "^^c3^^84^^c3^^96^^c3^^9c^^c3^^a4^^c3^^b6^^c3^^bc^^c3^^9f";
    public static void main(String[] args) {
        System.out.println(testString);
    }
}
\end{filecontents*}
\usepackage{listingsutf8}[2011/11/10]
\def\Text{%
  Umlauts: %
  ^^c3^^84^^c3^^96^^c3^^9c^^c3^^a4^^c3^^b6^^c3^^bc^^c3^^9f%
}
\begin{document}
\lstinputlisting[%
  language=Java,%
  inputencoding=utf8/latin1,%
]{ExampleUTF8.java}
\end{document}
%</test2>
%    \end{macrocode}
%
% \section{Installation}
%
% \subsection{Download}
%
% \paragraph{Package.} This package is available on
% CTAN\footnote{\url{ftp://ftp.ctan.org/tex-archive/}}:
% \begin{description}
% \item[\CTAN{macros/latex/contrib/oberdiek/listingsutf8.dtx}] The source file.
% \item[\CTAN{macros/latex/contrib/oberdiek/listingsutf8.pdf}] Documentation.
% \end{description}
%
%
% \paragraph{Bundle.} All the packages of the bundle `oberdiek'
% are also available in a TDS compliant ZIP archive. There
% the packages are already unpacked and the documentation files
% are generated. The files and directories obey the TDS standard.
% \begin{description}
% \item[\CTAN{install/macros/latex/contrib/oberdiek.tds.zip}]
% \end{description}
% \emph{TDS} refers to the standard ``A Directory Structure
% for \TeX\ Files'' (\CTAN{tds/tds.pdf}). Directories
% with \xfile{texmf} in their name are usually organized this way.
%
% \subsection{Bundle installation}
%
% \paragraph{Unpacking.} Unpack the \xfile{oberdiek.tds.zip} in the
% TDS tree (also known as \xfile{texmf} tree) of your choice.
% Example (linux):
% \begin{quote}
%   |unzip oberdiek.tds.zip -d ~/texmf|
% \end{quote}
%
% \paragraph{Script installation.}
% Check the directory \xfile{TDS:scripts/oberdiek/} for
% scripts that need further installation steps.
% Package \xpackage{attachfile2} comes with the Perl script
% \xfile{pdfatfi.pl} that should be installed in such a way
% that it can be called as \texttt{pdfatfi}.
% Example (linux):
% \begin{quote}
%   |chmod +x scripts/oberdiek/pdfatfi.pl|\\
%   |cp scripts/oberdiek/pdfatfi.pl /usr/local/bin/|
% \end{quote}
%
% \subsection{Package installation}
%
% \paragraph{Unpacking.} The \xfile{.dtx} file is a self-extracting
% \docstrip\ archive. The files are extracted by running the
% \xfile{.dtx} through \plainTeX:
% \begin{quote}
%   \verb|tex listingsutf8.dtx|
% \end{quote}
%
% \paragraph{TDS.} Now the different files must be moved into
% the different directories in your installation TDS tree
% (also known as \xfile{texmf} tree):
% \begin{quote}
% \def\t{^^A
% \begin{tabular}{@{}>{\ttfamily}l@{ $\rightarrow$ }>{\ttfamily}l@{}}
%   listingsutf8.sty & tex/latex/oberdiek/listingsutf8.sty\\
%   listingsutf8.pdf & doc/latex/oberdiek/listingsutf8.pdf\\
%   test/listingsutf8-test1.tex & doc/latex/oberdiek/test/listingsutf8-test1.tex\\
%   test/listingsutf8-test2.tex & doc/latex/oberdiek/test/listingsutf8-test2.tex\\
%   test/listingsutf8-test3.tex & doc/latex/oberdiek/test/listingsutf8-test3.tex\\
%   test/listingsutf8-test4.tex & doc/latex/oberdiek/test/listingsutf8-test4.tex\\
%   test/listingsutf8-test5.tex & doc/latex/oberdiek/test/listingsutf8-test5.tex\\
%   listingsutf8.dtx & source/latex/oberdiek/listingsutf8.dtx\\
% \end{tabular}^^A
% }^^A
% \sbox0{\t}^^A
% \ifdim\wd0>\linewidth
%   \begingroup
%     \advance\linewidth by\leftmargin
%     \advance\linewidth by\rightmargin
%   \edef\x{\endgroup
%     \def\noexpand\lw{\the\linewidth}^^A
%   }\x
%   \def\lwbox{^^A
%     \leavevmode
%     \hbox to \linewidth{^^A
%       \kern-\leftmargin\relax
%       \hss
%       \usebox0
%       \hss
%       \kern-\rightmargin\relax
%     }^^A
%   }^^A
%   \ifdim\wd0>\lw
%     \sbox0{\small\t}^^A
%     \ifdim\wd0>\linewidth
%       \ifdim\wd0>\lw
%         \sbox0{\footnotesize\t}^^A
%         \ifdim\wd0>\linewidth
%           \ifdim\wd0>\lw
%             \sbox0{\scriptsize\t}^^A
%             \ifdim\wd0>\linewidth
%               \ifdim\wd0>\lw
%                 \sbox0{\tiny\t}^^A
%                 \ifdim\wd0>\linewidth
%                   \lwbox
%                 \else
%                   \usebox0
%                 \fi
%               \else
%                 \lwbox
%               \fi
%             \else
%               \usebox0
%             \fi
%           \else
%             \lwbox
%           \fi
%         \else
%           \usebox0
%         \fi
%       \else
%         \lwbox
%       \fi
%     \else
%       \usebox0
%     \fi
%   \else
%     \lwbox
%   \fi
% \else
%   \usebox0
% \fi
% \end{quote}
% If you have a \xfile{docstrip.cfg} that configures and enables \docstrip's
% TDS installing feature, then some files can already be in the right
% place, see the documentation of \docstrip.
%
% \subsection{Refresh file name databases}
%
% If your \TeX~distribution
% (\teTeX, \mikTeX, \dots) relies on file name databases, you must refresh
% these. For example, \teTeX\ users run \verb|texhash| or
% \verb|mktexlsr|.
%
% \subsection{Some details for the interested}
%
% \paragraph{Attached source.}
%
% The PDF documentation on CTAN also includes the
% \xfile{.dtx} source file. It can be extracted by
% AcrobatReader 6 or higher. Another option is \textsf{pdftk},
% e.g. unpack the file into the current directory:
% \begin{quote}
%   \verb|pdftk listingsutf8.pdf unpack_files output .|
% \end{quote}
%
% \paragraph{Unpacking with \LaTeX.}
% The \xfile{.dtx} chooses its action depending on the format:
% \begin{description}
% \item[\plainTeX:] Run \docstrip\ and extract the files.
% \item[\LaTeX:] Generate the documentation.
% \end{description}
% If you insist on using \LaTeX\ for \docstrip\ (really,
% \docstrip\ does not need \LaTeX), then inform the autodetect routine
% about your intention:
% \begin{quote}
%   \verb|latex \let\install=y\input{listingsutf8.dtx}|
% \end{quote}
% Do not forget to quote the argument according to the demands
% of your shell.
%
% \paragraph{Generating the documentation.}
% You can use both the \xfile{.dtx} or the \xfile{.drv} to generate
% the documentation. The process can be configured by the
% configuration file \xfile{ltxdoc.cfg}. For instance, put this
% line into this file, if you want to have A4 as paper format:
% \begin{quote}
%   \verb|\PassOptionsToClass{a4paper}{article}|
% \end{quote}
% An example follows how to generate the
% documentation with pdf\LaTeX:
% \begin{quote}
%\begin{verbatim}
%pdflatex listingsutf8.dtx
%makeindex -s gind.ist listingsutf8.idx
%pdflatex listingsutf8.dtx
%makeindex -s gind.ist listingsutf8.idx
%pdflatex listingsutf8.dtx
%\end{verbatim}
% \end{quote}
%
% \section{Catalogue}
%
% The following XML file can be used as source for the
% \href{http://mirror.ctan.org/help/Catalogue/catalogue.html}{\TeX\ Catalogue}.
% The elements \texttt{caption} and \texttt{description} are imported
% from the original XML file from the Catalogue.
% The name of the XML file in the Catalogue is \xfile{listingsutf8.xml}.
%    \begin{macrocode}
%<*catalogue>
<?xml version='1.0' encoding='us-ascii'?>
<!DOCTYPE entry SYSTEM 'catalogue.dtd'>
<entry datestamp='$Date$' modifier='$Author$' id='listingsutf8'>
  <name>listingsutf8</name>
  <caption>Allow UTF-8 in listings input.</caption>
  <authorref id='auth:oberdiek'/>
  <copyright owner='Heiko Oberdiek' year='2007,2011'/>
  <license type='lppl1.3'/>
  <version number='1.2'/>
  <description>
    Package <xref refid='listings'>listings</xref> does not support files
    with multi-byte encodings such as UTF-8.  In the case of
    <tt>\lstinputlisting</tt>, a simple workaround is possible if a
    one-byte encoding exists that the file can be converted to.  The
    package requires the e-TeX extensions under pdfTeX (in either PDF
    or DVI output mode).
    <p/>
    The package is part of the <xref refid='oberdiek'>oberdiek</xref> bundle.
  </description>
  <documentation details='Package documentation'
      href='ctan:/macros/latex/contrib/oberdiek/listingsutf8.pdf'/>
  <ctan file='true' path='/macros/latex/contrib/oberdiek/listingsutf8.dtx'/>
  <miktex location='oberdiek'/>
  <texlive location='oberdiek'/>
  <install path='/macros/latex/contrib/oberdiek/oberdiek.tds.zip'/>
</entry>
%</catalogue>
%    \end{macrocode}
%
% \begin{thebibliography}{9}
%
% \bibitem{inputenc}
%   Alan Jeffrey, Frank Mittelbach,
%   \textit{inputenc.sty}, 2006/05/05 v1.1b.
%   \CTAN{macros/latex/base/inputenc.dtx}
%
% \bibitem{listings}
%   Carsten Heinz, Brooks Moses:
%  \textit{The \xpackage{listings} package};
%   2007/02/22;\\
%   \CTAN{macros/latex/contrib/listings/}.
%
% \bibitem{stringenc}
%   Heiko Oberdiek:
%   \textit{The \xpackage{stringenc} package};
%   2007/10/22;\\
%   \CTAN{macros/latex/contrib/oberdiek/stringenc.pdf}.
%
% \end{thebibliography}
%
% \begin{History}
%   \begin{Version}{2007/10/22 v1.0}
%   \item
%     First version.
%   \end{Version}
%   \begin{Version}{2007/11/11 v1.1}
%   \item
%     Use of package \xpackage{pdftexcmds}.
%   \end{Version}
%   \begin{Version}{2011/11/10 v1.2}
%   \item
%     DOS line ends CR/LF normalized to LF to avoid empty lines
%     (Bug report of Thomas Benkert in de.comp.text.tex).
%   \end{Version}
% \end{History}
%
% \PrintIndex
%
% \Finale
\endinput
|
% \end{quote}
% Do not forget to quote the argument according to the demands
% of your shell.
%
% \paragraph{Generating the documentation.}
% You can use both the \xfile{.dtx} or the \xfile{.drv} to generate
% the documentation. The process can be configured by the
% configuration file \xfile{ltxdoc.cfg}. For instance, put this
% line into this file, if you want to have A4 as paper format:
% \begin{quote}
%   \verb|\PassOptionsToClass{a4paper}{article}|
% \end{quote}
% An example follows how to generate the
% documentation with pdf\LaTeX:
% \begin{quote}
%\begin{verbatim}
%pdflatex listingsutf8.dtx
%makeindex -s gind.ist listingsutf8.idx
%pdflatex listingsutf8.dtx
%makeindex -s gind.ist listingsutf8.idx
%pdflatex listingsutf8.dtx
%\end{verbatim}
% \end{quote}
%
% \section{Catalogue}
%
% The following XML file can be used as source for the
% \href{http://mirror.ctan.org/help/Catalogue/catalogue.html}{\TeX\ Catalogue}.
% The elements \texttt{caption} and \texttt{description} are imported
% from the original XML file from the Catalogue.
% The name of the XML file in the Catalogue is \xfile{listingsutf8.xml}.
%    \begin{macrocode}
%<*catalogue>
<?xml version='1.0' encoding='us-ascii'?>
<!DOCTYPE entry SYSTEM 'catalogue.dtd'>
<entry datestamp='$Date$' modifier='$Author$' id='listingsutf8'>
  <name>listingsutf8</name>
  <caption>Allow UTF-8 in listings input.</caption>
  <authorref id='auth:oberdiek'/>
  <copyright owner='Heiko Oberdiek' year='2007,2011'/>
  <license type='lppl1.3'/>
  <version number='1.2'/>
  <description>
    Package <xref refid='listings'>listings</xref> does not support files
    with multi-byte encodings such as UTF-8.  In the case of
    <tt>\lstinputlisting</tt>, a simple workaround is possible if a
    one-byte encoding exists that the file can be converted to.  The
    package requires the e-TeX extensions under pdfTeX (in either PDF
    or DVI output mode).
    <p/>
    The package is part of the <xref refid='oberdiek'>oberdiek</xref> bundle.
  </description>
  <documentation details='Package documentation'
      href='ctan:/macros/latex/contrib/oberdiek/listingsutf8.pdf'/>
  <ctan file='true' path='/macros/latex/contrib/oberdiek/listingsutf8.dtx'/>
  <miktex location='oberdiek'/>
  <texlive location='oberdiek'/>
  <install path='/macros/latex/contrib/oberdiek/oberdiek.tds.zip'/>
</entry>
%</catalogue>
%    \end{macrocode}
%
% \begin{thebibliography}{9}
%
% \bibitem{inputenc}
%   Alan Jeffrey, Frank Mittelbach,
%   \textit{inputenc.sty}, 2006/05/05 v1.1b.
%   \CTAN{macros/latex/base/inputenc.dtx}
%
% \bibitem{listings}
%   Carsten Heinz, Brooks Moses:
%  \textit{The \xpackage{listings} package};
%   2007/02/22;\\
%   \CTAN{macros/latex/contrib/listings/}.
%
% \bibitem{stringenc}
%   Heiko Oberdiek:
%   \textit{The \xpackage{stringenc} package};
%   2007/10/22;\\
%   \CTAN{macros/latex/contrib/oberdiek/stringenc.pdf}.
%
% \end{thebibliography}
%
% \begin{History}
%   \begin{Version}{2007/10/22 v1.0}
%   \item
%     First version.
%   \end{Version}
%   \begin{Version}{2007/11/11 v1.1}
%   \item
%     Use of package \xpackage{pdftexcmds}.
%   \end{Version}
%   \begin{Version}{2011/11/10 v1.2}
%   \item
%     DOS line ends CR/LF normalized to LF to avoid empty lines
%     (Bug report of Thomas Benkert in de.comp.text.tex).
%   \end{Version}
% \end{History}
%
% \PrintIndex
%
% \Finale
\endinput
|
% \end{quote}
% Do not forget to quote the argument according to the demands
% of your shell.
%
% \paragraph{Generating the documentation.}
% You can use both the \xfile{.dtx} or the \xfile{.drv} to generate
% the documentation. The process can be configured by the
% configuration file \xfile{ltxdoc.cfg}. For instance, put this
% line into this file, if you want to have A4 as paper format:
% \begin{quote}
%   \verb|\PassOptionsToClass{a4paper}{article}|
% \end{quote}
% An example follows how to generate the
% documentation with pdf\LaTeX:
% \begin{quote}
%\begin{verbatim}
%pdflatex listingsutf8.dtx
%makeindex -s gind.ist listingsutf8.idx
%pdflatex listingsutf8.dtx
%makeindex -s gind.ist listingsutf8.idx
%pdflatex listingsutf8.dtx
%\end{verbatim}
% \end{quote}
%
% \section{Catalogue}
%
% The following XML file can be used as source for the
% \href{http://mirror.ctan.org/help/Catalogue/catalogue.html}{\TeX\ Catalogue}.
% The elements \texttt{caption} and \texttt{description} are imported
% from the original XML file from the Catalogue.
% The name of the XML file in the Catalogue is \xfile{listingsutf8.xml}.
%    \begin{macrocode}
%<*catalogue>
<?xml version='1.0' encoding='us-ascii'?>
<!DOCTYPE entry SYSTEM 'catalogue.dtd'>
<entry datestamp='$Date$' modifier='$Author$' id='listingsutf8'>
  <name>listingsutf8</name>
  <caption>Allow UTF-8 in listings input.</caption>
  <authorref id='auth:oberdiek'/>
  <copyright owner='Heiko Oberdiek' year='2007,2011'/>
  <license type='lppl1.3'/>
  <version number='1.2'/>
  <description>
    Package <xref refid='listings'>listings</xref> does not support files
    with multi-byte encodings such as UTF-8.  In the case of
    <tt>\lstinputlisting</tt>, a simple workaround is possible if a
    one-byte encoding exists that the file can be converted to.  The
    package requires the e-TeX extensions under pdfTeX (in either PDF
    or DVI output mode).
    <p/>
    The package is part of the <xref refid='oberdiek'>oberdiek</xref> bundle.
  </description>
  <documentation details='Package documentation'
      href='ctan:/macros/latex/contrib/oberdiek/listingsutf8.pdf'/>
  <ctan file='true' path='/macros/latex/contrib/oberdiek/listingsutf8.dtx'/>
  <miktex location='oberdiek'/>
  <texlive location='oberdiek'/>
  <install path='/macros/latex/contrib/oberdiek/oberdiek.tds.zip'/>
</entry>
%</catalogue>
%    \end{macrocode}
%
% \begin{thebibliography}{9}
%
% \bibitem{inputenc}
%   Alan Jeffrey, Frank Mittelbach,
%   \textit{inputenc.sty}, 2006/05/05 v1.1b.
%   \CTAN{macros/latex/base/inputenc.dtx}
%
% \bibitem{listings}
%   Carsten Heinz, Brooks Moses:
%  \textit{The \xpackage{listings} package};
%   2007/02/22;\\
%   \CTAN{macros/latex/contrib/listings/}.
%
% \bibitem{stringenc}
%   Heiko Oberdiek:
%   \textit{The \xpackage{stringenc} package};
%   2007/10/22;\\
%   \CTAN{macros/latex/contrib/oberdiek/stringenc.pdf}.
%
% \end{thebibliography}
%
% \begin{History}
%   \begin{Version}{2007/10/22 v1.0}
%   \item
%     First version.
%   \end{Version}
%   \begin{Version}{2007/11/11 v1.1}
%   \item
%     Use of package \xpackage{pdftexcmds}.
%   \end{Version}
%   \begin{Version}{2011/11/10 v1.2}
%   \item
%     DOS line ends CR/LF normalized to LF to avoid empty lines
%     (Bug report of Thomas Benkert in de.comp.text.tex).
%   \end{Version}
% \end{History}
%
% \PrintIndex
%
% \Finale
\endinput

%        (quote the arguments according to the demands of your shell)
%
% Documentation:
%    (a) If listingsutf8.drv is present:
%           latex listingsutf8.drv
%    (b) Without listingsutf8.drv:
%           latex listingsutf8.dtx; ...
%    The class ltxdoc loads the configuration file ltxdoc.cfg
%    if available. Here you can specify further options, e.g.
%    use A4 as paper format:
%       \PassOptionsToClass{a4paper}{article}
%
%    Programm calls to get the documentation (example):
%       pdflatex listingsutf8.dtx
%       makeindex -s gind.ist listingsutf8.idx
%       pdflatex listingsutf8.dtx
%       makeindex -s gind.ist listingsutf8.idx
%       pdflatex listingsutf8.dtx
%
% Installation:
%    TDS:tex/latex/oberdiek/listingsutf8.sty
%    TDS:doc/latex/oberdiek/listingsutf8.pdf
%    TDS:doc/latex/oberdiek/test/listingsutf8-test1.tex
%    TDS:doc/latex/oberdiek/test/listingsutf8-test2.tex
%    TDS:doc/latex/oberdiek/test/listingsutf8-test3.tex
%    TDS:doc/latex/oberdiek/test/listingsutf8-test4.tex
%    TDS:doc/latex/oberdiek/test/listingsutf8-test5.tex
%    TDS:source/latex/oberdiek/listingsutf8.dtx
%
%<*ignore>
\begingroup
  \catcode123=1 %
  \catcode125=2 %
  \def\x{LaTeX2e}%
\expandafter\endgroup
\ifcase 0\ifx\install y1\fi\expandafter
         \ifx\csname processbatchFile\endcsname\relax\else1\fi
         \ifx\fmtname\x\else 1\fi\relax
\else\csname fi\endcsname
%</ignore>
%<*install>
\input docstrip.tex
\Msg{************************************************************************}
\Msg{* Installation}
\Msg{* Package: listingsutf8 2011/11/10 v1.2 Allow UTF-8 in listings input (HO)}
\Msg{************************************************************************}

\keepsilent
\askforoverwritefalse

\let\MetaPrefix\relax
\preamble

This is a generated file.

Project: listingsutf8
Version: 2011/11/10 v1.2

Copyright (C) 2007, 2011 by
   Heiko Oberdiek <heiko.oberdiek at googlemail.com>

This work may be distributed and/or modified under the
conditions of the LaTeX Project Public License, either
version 1.3c of this license or (at your option) any later
version. This version of this license is in
   http://www.latex-project.org/lppl/lppl-1-3c.txt
and the latest version of this license is in
   http://www.latex-project.org/lppl.txt
and version 1.3 or later is part of all distributions of
LaTeX version 2005/12/01 or later.

This work has the LPPL maintenance status "maintained".

This Current Maintainer of this work is Heiko Oberdiek.

This work consists of the main source file listingsutf8.dtx
and the derived files
   listingsutf8.sty, listingsutf8.pdf, listingsutf8.ins, listingsutf8.drv,
   listingsutf8-test1.tex, listingsutf8-test2.tex,
   listingsutf8-test3.tex, listingsutf8-test4.tex,
   listingsutf8-test5.tex.

\endpreamble
\let\MetaPrefix\DoubleperCent

\generate{%
  \file{listingsutf8.ins}{\from{listingsutf8.dtx}{install}}%
  \file{listingsutf8.drv}{\from{listingsutf8.dtx}{driver}}%
  \usedir{tex/latex/oberdiek}%
  \file{listingsutf8.sty}{\from{listingsutf8.dtx}{package}}%
  \usedir{doc/latex/oberdiek/test}%
  \file{listingsutf8-test1.tex}{\from{listingsutf8.dtx}{test1}}%
  \file{listingsutf8-test2.tex}{\from{listingsutf8.dtx}{test2,utf8}}%
  \file{listingsutf8-test3.tex}{\from{listingsutf8.dtx}{test3,utf8x}}%
  \file{listingsutf8-test4.tex}{\from{listingsutf8.dtx}{test4,utf8,noetex}}%
  \file{listingsutf8-test5.tex}{\from{listingsutf8.dtx}{test5,utf8x,noetex}}%
  \nopreamble
  \nopostamble
  \usedir{source/latex/oberdiek/catalogue}%
  \file{listingsutf8.xml}{\from{listingsutf8.dtx}{catalogue}}%
}

\catcode32=13\relax% active space
\let =\space%
\Msg{************************************************************************}
\Msg{*}
\Msg{* To finish the installation you have to move the following}
\Msg{* file into a directory searched by TeX:}
\Msg{*}
\Msg{*     listingsutf8.sty}
\Msg{*}
\Msg{* To produce the documentation run the file `listingsutf8.drv'}
\Msg{* through LaTeX.}
\Msg{*}
\Msg{* Happy TeXing!}
\Msg{*}
\Msg{************************************************************************}

\endbatchfile
%</install>
%<*ignore>
\fi
%</ignore>
%<*driver>
\NeedsTeXFormat{LaTeX2e}
\ProvidesFile{listingsutf8.drv}%
  [2011/11/10 v1.2 Allow UTF-8 in listings input (HO)]%
\documentclass{ltxdoc}
\usepackage{holtxdoc}[2011/11/22]
\begin{document}
  \DocInput{listingsutf8.dtx}%
\end{document}
%</driver>
% \fi
%
% \CheckSum{311}
%
% \CharacterTable
%  {Upper-case    \A\B\C\D\E\F\G\H\I\J\K\L\M\N\O\P\Q\R\S\T\U\V\W\X\Y\Z
%   Lower-case    \a\b\c\d\e\f\g\h\i\j\k\l\m\n\o\p\q\r\s\t\u\v\w\x\y\z
%   Digits        \0\1\2\3\4\5\6\7\8\9
%   Exclamation   \!     Double quote  \"     Hash (number) \#
%   Dollar        \$     Percent       \%     Ampersand     \&
%   Acute accent  \'     Left paren    \(     Right paren   \)
%   Asterisk      \*     Plus          \+     Comma         \,
%   Minus         \-     Point         \.     Solidus       \/
%   Colon         \:     Semicolon     \;     Less than     \<
%   Equals        \=     Greater than  \>     Question mark \?
%   Commercial at \@     Left bracket  \[     Backslash     \\
%   Right bracket \]     Circumflex    \^     Underscore    \_
%   Grave accent  \`     Left brace    \{     Vertical bar  \|
%   Right brace   \}     Tilde         \~}
%
% \GetFileInfo{listingsutf8.drv}
%
% \title{The \xpackage{listingsutf8} package}
% \date{2011/11/10 v1.2}
% \author{Heiko Oberdiek\\\xemail{heiko.oberdiek at googlemail.com}}
%
% \maketitle
%
% \begin{abstract}
% Package \xpackage{listings} does not support files with multi-byte
% encodings such as UTF-8. In case of \cs{lstinputlisting} a simple
% workaround is possible if an one-byte encoding exists that the file
% can be converted to. Also \eTeX\ and \pdfTeX\ regardless of its mode
% are required.
% \end{abstract}
%
% \tableofcontents
%
% \section{Documentation}
%
% \subsection{User interface}
%
% Load this package after or instead of package \xpackage{listings}
% \cite{listings}.
% The package does not define own options and passes given options to
% package \xpackage{listings}.
%
% The syntax of package \xpackage{listings}' key \xoption{inputencoding}
% is extended:
% \begin{quote}
%   |inputencoding=utf8/|\meta{one-byte-encoding}\\
%   Example: |inputencoding=utf8/latin1|
% \end{quote}
% That means the file is encoded in UTF-8 and can
% be converted to the given \meta{one-byte-encoding}.
% The available encodings for \meta{one-byte-encoding} are
% listed in section ``1.2 Supported encodings'' of
% package \xpackage{stringenc}'s documentation \cite{stringenc}.
% Of course, the encoding must encode its characters with
% one byte exactly. This excludes the unicode encodings
% (\xoption{utf8}, \xoption{utf16}, \dots).
%
% Only \cs{lstinputlisting} is supported by the syntax extension
% of key \xoption{inputencoding}.
%
% Internally package \xpackage{listingsutf8} reads the file as binary file
% via primitives of \pdfTeX\ (\cs{pdffiledump}). Then the file
% contents is converted as string using package \xpackage{stringenc} and
% finally the string is read as virtual file by \eTeX's \cs{scantokens}.
%
% \subsection{Future}
%
% Workarounds are not provided for
% \begin{itemize}
% \item \cs{lstinline}
% \item Environment |lstlisting|.
% \item Environments defined by \cs{lstnewenvironment}.
% \end{itemize}
% Perhaps someone will find time to extend package \xpackage{listings}
% with full native support for UTF-8. Then this package would become obsolete.
%
% \StopEventually{
% }
%
% \section{Implementation}
%
%    \begin{macrocode}
%<*package>
%    \end{macrocode}
%
% \subsection{Catcodes and identification}
%
%    \begin{macrocode}
\begingroup\catcode61\catcode48\catcode32=10\relax%
  \catcode13=5 % ^^M
  \endlinechar=13 %
  \catcode123=1 % {
  \catcode125=2 % }
  \catcode64=11 % @
  \def\x{\endgroup
    \expandafter\edef\csname lstU@AtEnd\endcsname{%
      \endlinechar=\the\endlinechar\relax
      \catcode13=\the\catcode13\relax
      \catcode32=\the\catcode32\relax
      \catcode35=\the\catcode35\relax
      \catcode61=\the\catcode61\relax
      \catcode64=\the\catcode64\relax
      \catcode123=\the\catcode123\relax
      \catcode125=\the\catcode125\relax
    }%
  }%
\x\catcode61\catcode48\catcode32=10\relax%
\catcode13=5 % ^^M
\endlinechar=13 %
\catcode35=6 % #
\catcode64=11 % @
\catcode123=1 % {
\catcode125=2 % }
\def\TMP@EnsureCode#1#2{%
  \edef\lstU@AtEnd{%
    \lstU@AtEnd
    \catcode#1=\the\catcode#1\relax
  }%
  \catcode#1=#2\relax
}
\TMP@EnsureCode{10}{12}% ^^J
\TMP@EnsureCode{33}{12}% !
\TMP@EnsureCode{36}{3}% $
\TMP@EnsureCode{38}{4}% &
\TMP@EnsureCode{39}{12}% '
\TMP@EnsureCode{40}{12}% (
\TMP@EnsureCode{41}{12}% )
\TMP@EnsureCode{42}{12}% *
\TMP@EnsureCode{43}{12}% +
\TMP@EnsureCode{44}{12}% ,
\TMP@EnsureCode{45}{12}% -
\TMP@EnsureCode{46}{12}% .
\TMP@EnsureCode{47}{12}% /
\TMP@EnsureCode{58}{12}% :
\TMP@EnsureCode{60}{12}% <
\TMP@EnsureCode{62}{12}% >
\TMP@EnsureCode{91}{12}% [
\TMP@EnsureCode{93}{12}% ]
\TMP@EnsureCode{94}{7}% ^ (superscript)
\TMP@EnsureCode{95}{8}% _ (subscript)
\TMP@EnsureCode{96}{12}% `
\TMP@EnsureCode{124}{12}% |
\TMP@EnsureCode{126}{13}% ~ (active)
\edef\lstU@AtEnd{\lstU@AtEnd\noexpand\endinput}
%    \end{macrocode}
%
%    Package identification.
%    \begin{macrocode}
\NeedsTeXFormat{LaTeX2e}
\ProvidesPackage{listingsutf8}%
  [2011/11/10 v1.2 Allow UTF-8 in listings input (HO)]
%    \end{macrocode}
%
% \subsection{Package options}
%
% Just pass options to package listings.
%
%    \begin{macrocode}
\DeclareOption*{%
  \PassOptionsToPackage\CurrentOption{listings}%
}
\ProcessOptions*
%    \end{macrocode}
%    Key \xoption{inputencoding} was introduced in version
%    2002/04/01 v1.0 of package \xpackage{listings}.
%    \begin{macrocode}
\RequirePackage{listings}[2002/04/01]
%    \end{macrocode}
%    Ensure that \cs{inputencoding} is provided.
%    \begin{macrocode}
\AtBeginDocument{%
  \@ifundefined{inputencoding}{%
    \RequirePackage{inputenc}%
  }{}%
}
%    \end{macrocode}
%
% \subsection{Check prerequisites}
%
%    \begin{macrocode}
\RequirePackage{pdftexcmds}[2011/04/22]
%    \end{macrocode}
%
%    \begin{macrocode}
\def\lstU@temp#1#2{%
  \begingroup\expandafter\expandafter\expandafter\endgroup
  \expandafter\ifx\csname #1\endcsname\relax
    \PackageWarningNoLine{listingsutf8}{%
      Package loading is aborted because of missing %
      \@backslashchar#1.\MessageBreak
      #2%
    }%
    \expandafter\lstU@AtEnd
  \fi
}
\lstU@temp{scantokens}{It is provided by e-TeX}%
\lstU@temp{pdf@unescapehex}{It is provided by pdfTeX >= 1.30}%
\lstU@temp{pdf@filedump}{It is provided by pdfTeX >= 1.30}%
\lstU@temp{pdf@filesize}{It is provided by pdfTeX >= 1.30}%
%    \end{macrocode}
%
%    \begin{macrocode}
\RequirePackage{stringenc}[2010/03/01]
%    \end{macrocode}
%
% \subsection{Add support for UTF-8}
%
%    \begin{macro}{\iflstU@utfviii}
%    \begin{macrocode}
\newif\iflstU@utfviii
%    \end{macrocode}
%    \end{macro}
%
%    \begin{macro}{\lstU@inputenc}
%    \begin{macrocode}
\def\lstU@inputenc#1{%
  \expandafter\lstU@@inputenc#1utf8/utf8/\@nil
}
%    \end{macrocode}
%    \end{macro}
%    \begin{macro}{\lstU@@inputenc}
\def\lstU@@inputenc#1utf8/#2utf8/#3\@nil{%
  \ifx\\#1\\%
    \lstU@utfviiitrue
    \def\lst@inputenc{#2}%
  \else
    \lstU@utfviiifalse
  \fi
}
%    \begin{macrocode}
%    \end{macrocode}
%    \end{macro}
%
%    \begin{macrocode}
\lst@Key{inputencoding}\relax{%
  \def\lst@inputenc{#1}%
  \lstU@inputenc{#1}%
}
%    \end{macrocode}
%
% \subsubsection{Conversion}
%
%    \begin{macro}{\lstU@input}
%    \begin{macrocode}
\def\lstU@input#1{%
  \iflstU@utfviii
    \edef\lstU@text{%
      \pdf@unescapehex{%
        \pdf@filedump{0}{\pdf@filesize{#1}}{#1}%
      }%
    }%
    \lstU@CRLFtoLF\lstU@text
    \StringEncodingConvert\lstU@text\lstU@text{utf8}\lst@inputenc
    \def\lstU@temp{%
      \scantokens\expandafter{\lstU@text}%
    }%
  \else
    \def\lstU@temp{%
      \input{#1}%
    }%
  \fi
  \lstU@temp
}
%    \end{macrocode}
%    \end{macro}
%
% \subsubsection{Convert CR/LF pairs to LF}
%
%    \begin{macro}{\lstU@CRLFtoLF}
%    \begin{macrocode}
\begingroup
  \endlinechar=-1 %
  \@makeother\^^J %
  \@makeother\^^M %
  \gdef\lstU@CRLFtoLF#1{%
    \edef#1{%
      \expandafter\lstU@CRLFtoLF@aux#1^^M^^J\@nil
    }%
  }%
  \gdef\lstU@CRLFtoLF@aux#1^^M^^J#2\@nil{%
    #1%
    \ifx\relax#2\relax
      \@car
    \fi
    ^^J%
    \lstU@CRLFtoLF@aux#2\@nil
  }%
\endgroup %
%    \end{macrocode}
%    \end{macro}
%
% \subsubsection{Patch \cs{lst@InputListing}}
%
%    \begin{macrocode}
\def\lstU@temp#1\def\lst@next#2#3\@nil{%
  \def\lst@InputListing##1{%
    #1%
    \def\lst@next{\lstU@input{##1}}%
    #3%
  }%
}
\expandafter\lstU@temp\lst@InputListing{#1}\@nil
%    \end{macrocode}
%
%    \begin{macrocode}
\lstU@AtEnd%
%</package>
%    \end{macrocode}
%
% \section{Test}
%
% \subsection{Catcode checks for loading}
%
%    \begin{macrocode}
%<*test1>
%    \end{macrocode}
%    \begin{macrocode}
\NeedsTeXFormat{LaTeX2e}
\documentclass{minimal}
\makeatletter
\def\RestoreCatcodes{}
\count@=0 %
\loop
  \edef\RestoreCatcodes{%
    \RestoreCatcodes
    \catcode\the\count@=\the\catcode\count@\relax
  }%
\ifnum\count@<255 %
  \advance\count@\@ne
\repeat

\def\RangeCatcodeInvalid#1#2{%
  \count@=#1\relax
  \loop
    \catcode\count@=15 %
  \ifnum\count@<#2\relax
    \advance\count@\@ne
  \repeat
}
\def\Test{%
  \RangeCatcodeInvalid{0}{47}%
  \RangeCatcodeInvalid{58}{64}%
  \RangeCatcodeInvalid{91}{96}%
  \RangeCatcodeInvalid{123}{127}%
  \catcode`\@=12 %
  \catcode`\\=0 %
  \catcode`\{=1 %
  \catcode`\}=2 %
  \catcode`\#=6 %
  \catcode`\[=12 %
  \catcode`\]=12 %
  \catcode`\%=14 %
  \catcode`\ =10 %
  \catcode13=5 %
  \RequirePackage{listingsutf8}[2011/11/10]\relax
  \RestoreCatcodes
}
\Test
\csname @@end\endcsname
\end
%    \end{macrocode}
%    \begin{macrocode}
%</test1>
%    \end{macrocode}
%
% \subsection{Test example for latin1}
%
%    \begin{macrocode}
%<*test2>
%    \end{macrocode}
%    \begin{macrocode}
\NeedsTeXFormat{LaTeX2e}
\documentclass{minimal}
\usepackage{filecontents}
\def\do#1{%
  \ifx#1\^%
  \else
    \noexpand\do\noexpand#1%
  \fi
}
\expandafter\let\expandafter\dospecials\expandafter\empty
\expandafter\edef\expandafter\dospecials\expandafter{\dospecials}
\begin{filecontents*}{ExampleUTF8.java}
public class ExampleUTF8 {
    public static String testString =
        "Umlauts: " +
        "^^c3^^84^^c3^^96^^c3^^9c^^c3^^a4^^c3^^b6^^c3^^bc^^c3^^9f";
    public static void main(String[] args) {
        System.out.println(testString);
    }
}
\end{filecontents*}
\usepackage{listingsutf8}[2011/11/10]
\def\Text{%
  Umlauts: %
  ^^c3^^84^^c3^^96^^c3^^9c^^c3^^a4^^c3^^b6^^c3^^bc^^c3^^9f%
}
\begin{document}
\lstinputlisting[%
  language=Java,%
  inputencoding=utf8/latin1,%
]{ExampleUTF8.java}
\end{document}
%</test2>
%    \end{macrocode}
%
% \section{Installation}
%
% \subsection{Download}
%
% \paragraph{Package.} This package is available on
% CTAN\footnote{\url{ftp://ftp.ctan.org/tex-archive/}}:
% \begin{description}
% \item[\CTAN{macros/latex/contrib/oberdiek/listingsutf8.dtx}] The source file.
% \item[\CTAN{macros/latex/contrib/oberdiek/listingsutf8.pdf}] Documentation.
% \end{description}
%
%
% \paragraph{Bundle.} All the packages of the bundle `oberdiek'
% are also available in a TDS compliant ZIP archive. There
% the packages are already unpacked and the documentation files
% are generated. The files and directories obey the TDS standard.
% \begin{description}
% \item[\CTAN{install/macros/latex/contrib/oberdiek.tds.zip}]
% \end{description}
% \emph{TDS} refers to the standard ``A Directory Structure
% for \TeX\ Files'' (\CTAN{tds/tds.pdf}). Directories
% with \xfile{texmf} in their name are usually organized this way.
%
% \subsection{Bundle installation}
%
% \paragraph{Unpacking.} Unpack the \xfile{oberdiek.tds.zip} in the
% TDS tree (also known as \xfile{texmf} tree) of your choice.
% Example (linux):
% \begin{quote}
%   |unzip oberdiek.tds.zip -d ~/texmf|
% \end{quote}
%
% \paragraph{Script installation.}
% Check the directory \xfile{TDS:scripts/oberdiek/} for
% scripts that need further installation steps.
% Package \xpackage{attachfile2} comes with the Perl script
% \xfile{pdfatfi.pl} that should be installed in such a way
% that it can be called as \texttt{pdfatfi}.
% Example (linux):
% \begin{quote}
%   |chmod +x scripts/oberdiek/pdfatfi.pl|\\
%   |cp scripts/oberdiek/pdfatfi.pl /usr/local/bin/|
% \end{quote}
%
% \subsection{Package installation}
%
% \paragraph{Unpacking.} The \xfile{.dtx} file is a self-extracting
% \docstrip\ archive. The files are extracted by running the
% \xfile{.dtx} through \plainTeX:
% \begin{quote}
%   \verb|tex listingsutf8.dtx|
% \end{quote}
%
% \paragraph{TDS.} Now the different files must be moved into
% the different directories in your installation TDS tree
% (also known as \xfile{texmf} tree):
% \begin{quote}
% \def\t{^^A
% \begin{tabular}{@{}>{\ttfamily}l@{ $\rightarrow$ }>{\ttfamily}l@{}}
%   listingsutf8.sty & tex/latex/oberdiek/listingsutf8.sty\\
%   listingsutf8.pdf & doc/latex/oberdiek/listingsutf8.pdf\\
%   test/listingsutf8-test1.tex & doc/latex/oberdiek/test/listingsutf8-test1.tex\\
%   test/listingsutf8-test2.tex & doc/latex/oberdiek/test/listingsutf8-test2.tex\\
%   test/listingsutf8-test3.tex & doc/latex/oberdiek/test/listingsutf8-test3.tex\\
%   test/listingsutf8-test4.tex & doc/latex/oberdiek/test/listingsutf8-test4.tex\\
%   test/listingsutf8-test5.tex & doc/latex/oberdiek/test/listingsutf8-test5.tex\\
%   listingsutf8.dtx & source/latex/oberdiek/listingsutf8.dtx\\
% \end{tabular}^^A
% }^^A
% \sbox0{\t}^^A
% \ifdim\wd0>\linewidth
%   \begingroup
%     \advance\linewidth by\leftmargin
%     \advance\linewidth by\rightmargin
%   \edef\x{\endgroup
%     \def\noexpand\lw{\the\linewidth}^^A
%   }\x
%   \def\lwbox{^^A
%     \leavevmode
%     \hbox to \linewidth{^^A
%       \kern-\leftmargin\relax
%       \hss
%       \usebox0
%       \hss
%       \kern-\rightmargin\relax
%     }^^A
%   }^^A
%   \ifdim\wd0>\lw
%     \sbox0{\small\t}^^A
%     \ifdim\wd0>\linewidth
%       \ifdim\wd0>\lw
%         \sbox0{\footnotesize\t}^^A
%         \ifdim\wd0>\linewidth
%           \ifdim\wd0>\lw
%             \sbox0{\scriptsize\t}^^A
%             \ifdim\wd0>\linewidth
%               \ifdim\wd0>\lw
%                 \sbox0{\tiny\t}^^A
%                 \ifdim\wd0>\linewidth
%                   \lwbox
%                 \else
%                   \usebox0
%                 \fi
%               \else
%                 \lwbox
%               \fi
%             \else
%               \usebox0
%             \fi
%           \else
%             \lwbox
%           \fi
%         \else
%           \usebox0
%         \fi
%       \else
%         \lwbox
%       \fi
%     \else
%       \usebox0
%     \fi
%   \else
%     \lwbox
%   \fi
% \else
%   \usebox0
% \fi
% \end{quote}
% If you have a \xfile{docstrip.cfg} that configures and enables \docstrip's
% TDS installing feature, then some files can already be in the right
% place, see the documentation of \docstrip.
%
% \subsection{Refresh file name databases}
%
% If your \TeX~distribution
% (\teTeX, \mikTeX, \dots) relies on file name databases, you must refresh
% these. For example, \teTeX\ users run \verb|texhash| or
% \verb|mktexlsr|.
%
% \subsection{Some details for the interested}
%
% \paragraph{Attached source.}
%
% The PDF documentation on CTAN also includes the
% \xfile{.dtx} source file. It can be extracted by
% AcrobatReader 6 or higher. Another option is \textsf{pdftk},
% e.g. unpack the file into the current directory:
% \begin{quote}
%   \verb|pdftk listingsutf8.pdf unpack_files output .|
% \end{quote}
%
% \paragraph{Unpacking with \LaTeX.}
% The \xfile{.dtx} chooses its action depending on the format:
% \begin{description}
% \item[\plainTeX:] Run \docstrip\ and extract the files.
% \item[\LaTeX:] Generate the documentation.
% \end{description}
% If you insist on using \LaTeX\ for \docstrip\ (really,
% \docstrip\ does not need \LaTeX), then inform the autodetect routine
% about your intention:
% \begin{quote}
%   \verb|latex \let\install=y% \iffalse meta-comment
%
% File: listingsutf8.dtx
% Version: 2011/11/10 v1.2
% Info: Allow UTF-8 in listings input
%
% Copyright (C) 2007, 2011 by
%    Heiko Oberdiek <heiko.oberdiek at googlemail.com>
%
% This work may be distributed and/or modified under the
% conditions of the LaTeX Project Public License, either
% version 1.3c of this license or (at your option) any later
% version. This version of this license is in
%    http://www.latex-project.org/lppl/lppl-1-3c.txt
% and the latest version of this license is in
%    http://www.latex-project.org/lppl.txt
% and version 1.3 or later is part of all distributions of
% LaTeX version 2005/12/01 or later.
%
% This work has the LPPL maintenance status "maintained".
%
% This Current Maintainer of this work is Heiko Oberdiek.
%
% This work consists of the main source file listingsutf8.dtx
% and the derived files
%    listingsutf8.sty, listingsutf8.pdf, listingsutf8.ins, listingsutf8.drv,
%    listingsutf8-test1.tex, listingsutf8-test2.tex,
%    listingsutf8-test3.tex, listingsutf8-test4.tex,
%    listingsutf8-test5.tex.
%
% Distribution:
%    CTAN:macros/latex/contrib/oberdiek/listingsutf8.dtx
%    CTAN:macros/latex/contrib/oberdiek/listingsutf8.pdf
%
% Unpacking:
%    (a) If listingsutf8.ins is present:
%           tex listingsutf8.ins
%    (b) Without listingsutf8.ins:
%           tex listingsutf8.dtx
%    (c) If you insist on using LaTeX
%           latex \let\install=y% \iffalse meta-comment
%
% File: listingsutf8.dtx
% Version: 2011/11/10 v1.2
% Info: Allow UTF-8 in listings input
%
% Copyright (C) 2007, 2011 by
%    Heiko Oberdiek <heiko.oberdiek at googlemail.com>
%
% This work may be distributed and/or modified under the
% conditions of the LaTeX Project Public License, either
% version 1.3c of this license or (at your option) any later
% version. This version of this license is in
%    http://www.latex-project.org/lppl/lppl-1-3c.txt
% and the latest version of this license is in
%    http://www.latex-project.org/lppl.txt
% and version 1.3 or later is part of all distributions of
% LaTeX version 2005/12/01 or later.
%
% This work has the LPPL maintenance status "maintained".
%
% This Current Maintainer of this work is Heiko Oberdiek.
%
% This work consists of the main source file listingsutf8.dtx
% and the derived files
%    listingsutf8.sty, listingsutf8.pdf, listingsutf8.ins, listingsutf8.drv,
%    listingsutf8-test1.tex, listingsutf8-test2.tex,
%    listingsutf8-test3.tex, listingsutf8-test4.tex,
%    listingsutf8-test5.tex.
%
% Distribution:
%    CTAN:macros/latex/contrib/oberdiek/listingsutf8.dtx
%    CTAN:macros/latex/contrib/oberdiek/listingsutf8.pdf
%
% Unpacking:
%    (a) If listingsutf8.ins is present:
%           tex listingsutf8.ins
%    (b) Without listingsutf8.ins:
%           tex listingsutf8.dtx
%    (c) If you insist on using LaTeX
%           latex \let\install=y% \iffalse meta-comment
%
% File: listingsutf8.dtx
% Version: 2011/11/10 v1.2
% Info: Allow UTF-8 in listings input
%
% Copyright (C) 2007, 2011 by
%    Heiko Oberdiek <heiko.oberdiek at googlemail.com>
%
% This work may be distributed and/or modified under the
% conditions of the LaTeX Project Public License, either
% version 1.3c of this license or (at your option) any later
% version. This version of this license is in
%    http://www.latex-project.org/lppl/lppl-1-3c.txt
% and the latest version of this license is in
%    http://www.latex-project.org/lppl.txt
% and version 1.3 or later is part of all distributions of
% LaTeX version 2005/12/01 or later.
%
% This work has the LPPL maintenance status "maintained".
%
% This Current Maintainer of this work is Heiko Oberdiek.
%
% This work consists of the main source file listingsutf8.dtx
% and the derived files
%    listingsutf8.sty, listingsutf8.pdf, listingsutf8.ins, listingsutf8.drv,
%    listingsutf8-test1.tex, listingsutf8-test2.tex,
%    listingsutf8-test3.tex, listingsutf8-test4.tex,
%    listingsutf8-test5.tex.
%
% Distribution:
%    CTAN:macros/latex/contrib/oberdiek/listingsutf8.dtx
%    CTAN:macros/latex/contrib/oberdiek/listingsutf8.pdf
%
% Unpacking:
%    (a) If listingsutf8.ins is present:
%           tex listingsutf8.ins
%    (b) Without listingsutf8.ins:
%           tex listingsutf8.dtx
%    (c) If you insist on using LaTeX
%           latex \let\install=y\input{listingsutf8.dtx}
%        (quote the arguments according to the demands of your shell)
%
% Documentation:
%    (a) If listingsutf8.drv is present:
%           latex listingsutf8.drv
%    (b) Without listingsutf8.drv:
%           latex listingsutf8.dtx; ...
%    The class ltxdoc loads the configuration file ltxdoc.cfg
%    if available. Here you can specify further options, e.g.
%    use A4 as paper format:
%       \PassOptionsToClass{a4paper}{article}
%
%    Programm calls to get the documentation (example):
%       pdflatex listingsutf8.dtx
%       makeindex -s gind.ist listingsutf8.idx
%       pdflatex listingsutf8.dtx
%       makeindex -s gind.ist listingsutf8.idx
%       pdflatex listingsutf8.dtx
%
% Installation:
%    TDS:tex/latex/oberdiek/listingsutf8.sty
%    TDS:doc/latex/oberdiek/listingsutf8.pdf
%    TDS:doc/latex/oberdiek/test/listingsutf8-test1.tex
%    TDS:doc/latex/oberdiek/test/listingsutf8-test2.tex
%    TDS:doc/latex/oberdiek/test/listingsutf8-test3.tex
%    TDS:doc/latex/oberdiek/test/listingsutf8-test4.tex
%    TDS:doc/latex/oberdiek/test/listingsutf8-test5.tex
%    TDS:source/latex/oberdiek/listingsutf8.dtx
%
%<*ignore>
\begingroup
  \catcode123=1 %
  \catcode125=2 %
  \def\x{LaTeX2e}%
\expandafter\endgroup
\ifcase 0\ifx\install y1\fi\expandafter
         \ifx\csname processbatchFile\endcsname\relax\else1\fi
         \ifx\fmtname\x\else 1\fi\relax
\else\csname fi\endcsname
%</ignore>
%<*install>
\input docstrip.tex
\Msg{************************************************************************}
\Msg{* Installation}
\Msg{* Package: listingsutf8 2011/11/10 v1.2 Allow UTF-8 in listings input (HO)}
\Msg{************************************************************************}

\keepsilent
\askforoverwritefalse

\let\MetaPrefix\relax
\preamble

This is a generated file.

Project: listingsutf8
Version: 2011/11/10 v1.2

Copyright (C) 2007, 2011 by
   Heiko Oberdiek <heiko.oberdiek at googlemail.com>

This work may be distributed and/or modified under the
conditions of the LaTeX Project Public License, either
version 1.3c of this license or (at your option) any later
version. This version of this license is in
   http://www.latex-project.org/lppl/lppl-1-3c.txt
and the latest version of this license is in
   http://www.latex-project.org/lppl.txt
and version 1.3 or later is part of all distributions of
LaTeX version 2005/12/01 or later.

This work has the LPPL maintenance status "maintained".

This Current Maintainer of this work is Heiko Oberdiek.

This work consists of the main source file listingsutf8.dtx
and the derived files
   listingsutf8.sty, listingsutf8.pdf, listingsutf8.ins, listingsutf8.drv,
   listingsutf8-test1.tex, listingsutf8-test2.tex,
   listingsutf8-test3.tex, listingsutf8-test4.tex,
   listingsutf8-test5.tex.

\endpreamble
\let\MetaPrefix\DoubleperCent

\generate{%
  \file{listingsutf8.ins}{\from{listingsutf8.dtx}{install}}%
  \file{listingsutf8.drv}{\from{listingsutf8.dtx}{driver}}%
  \usedir{tex/latex/oberdiek}%
  \file{listingsutf8.sty}{\from{listingsutf8.dtx}{package}}%
  \usedir{doc/latex/oberdiek/test}%
  \file{listingsutf8-test1.tex}{\from{listingsutf8.dtx}{test1}}%
  \file{listingsutf8-test2.tex}{\from{listingsutf8.dtx}{test2,utf8}}%
  \file{listingsutf8-test3.tex}{\from{listingsutf8.dtx}{test3,utf8x}}%
  \file{listingsutf8-test4.tex}{\from{listingsutf8.dtx}{test4,utf8,noetex}}%
  \file{listingsutf8-test5.tex}{\from{listingsutf8.dtx}{test5,utf8x,noetex}}%
  \nopreamble
  \nopostamble
  \usedir{source/latex/oberdiek/catalogue}%
  \file{listingsutf8.xml}{\from{listingsutf8.dtx}{catalogue}}%
}

\catcode32=13\relax% active space
\let =\space%
\Msg{************************************************************************}
\Msg{*}
\Msg{* To finish the installation you have to move the following}
\Msg{* file into a directory searched by TeX:}
\Msg{*}
\Msg{*     listingsutf8.sty}
\Msg{*}
\Msg{* To produce the documentation run the file `listingsutf8.drv'}
\Msg{* through LaTeX.}
\Msg{*}
\Msg{* Happy TeXing!}
\Msg{*}
\Msg{************************************************************************}

\endbatchfile
%</install>
%<*ignore>
\fi
%</ignore>
%<*driver>
\NeedsTeXFormat{LaTeX2e}
\ProvidesFile{listingsutf8.drv}%
  [2011/11/10 v1.2 Allow UTF-8 in listings input (HO)]%
\documentclass{ltxdoc}
\usepackage{holtxdoc}[2011/11/22]
\begin{document}
  \DocInput{listingsutf8.dtx}%
\end{document}
%</driver>
% \fi
%
% \CheckSum{311}
%
% \CharacterTable
%  {Upper-case    \A\B\C\D\E\F\G\H\I\J\K\L\M\N\O\P\Q\R\S\T\U\V\W\X\Y\Z
%   Lower-case    \a\b\c\d\e\f\g\h\i\j\k\l\m\n\o\p\q\r\s\t\u\v\w\x\y\z
%   Digits        \0\1\2\3\4\5\6\7\8\9
%   Exclamation   \!     Double quote  \"     Hash (number) \#
%   Dollar        \$     Percent       \%     Ampersand     \&
%   Acute accent  \'     Left paren    \(     Right paren   \)
%   Asterisk      \*     Plus          \+     Comma         \,
%   Minus         \-     Point         \.     Solidus       \/
%   Colon         \:     Semicolon     \;     Less than     \<
%   Equals        \=     Greater than  \>     Question mark \?
%   Commercial at \@     Left bracket  \[     Backslash     \\
%   Right bracket \]     Circumflex    \^     Underscore    \_
%   Grave accent  \`     Left brace    \{     Vertical bar  \|
%   Right brace   \}     Tilde         \~}
%
% \GetFileInfo{listingsutf8.drv}
%
% \title{The \xpackage{listingsutf8} package}
% \date{2011/11/10 v1.2}
% \author{Heiko Oberdiek\\\xemail{heiko.oberdiek at googlemail.com}}
%
% \maketitle
%
% \begin{abstract}
% Package \xpackage{listings} does not support files with multi-byte
% encodings such as UTF-8. In case of \cs{lstinputlisting} a simple
% workaround is possible if an one-byte encoding exists that the file
% can be converted to. Also \eTeX\ and \pdfTeX\ regardless of its mode
% are required.
% \end{abstract}
%
% \tableofcontents
%
% \section{Documentation}
%
% \subsection{User interface}
%
% Load this package after or instead of package \xpackage{listings}
% \cite{listings}.
% The package does not define own options and passes given options to
% package \xpackage{listings}.
%
% The syntax of package \xpackage{listings}' key \xoption{inputencoding}
% is extended:
% \begin{quote}
%   |inputencoding=utf8/|\meta{one-byte-encoding}\\
%   Example: |inputencoding=utf8/latin1|
% \end{quote}
% That means the file is encoded in UTF-8 and can
% be converted to the given \meta{one-byte-encoding}.
% The available encodings for \meta{one-byte-encoding} are
% listed in section ``1.2 Supported encodings'' of
% package \xpackage{stringenc}'s documentation \cite{stringenc}.
% Of course, the encoding must encode its characters with
% one byte exactly. This excludes the unicode encodings
% (\xoption{utf8}, \xoption{utf16}, \dots).
%
% Only \cs{lstinputlisting} is supported by the syntax extension
% of key \xoption{inputencoding}.
%
% Internally package \xpackage{listingsutf8} reads the file as binary file
% via primitives of \pdfTeX\ (\cs{pdffiledump}). Then the file
% contents is converted as string using package \xpackage{stringenc} and
% finally the string is read as virtual file by \eTeX's \cs{scantokens}.
%
% \subsection{Future}
%
% Workarounds are not provided for
% \begin{itemize}
% \item \cs{lstinline}
% \item Environment |lstlisting|.
% \item Environments defined by \cs{lstnewenvironment}.
% \end{itemize}
% Perhaps someone will find time to extend package \xpackage{listings}
% with full native support for UTF-8. Then this package would become obsolete.
%
% \StopEventually{
% }
%
% \section{Implementation}
%
%    \begin{macrocode}
%<*package>
%    \end{macrocode}
%
% \subsection{Catcodes and identification}
%
%    \begin{macrocode}
\begingroup\catcode61\catcode48\catcode32=10\relax%
  \catcode13=5 % ^^M
  \endlinechar=13 %
  \catcode123=1 % {
  \catcode125=2 % }
  \catcode64=11 % @
  \def\x{\endgroup
    \expandafter\edef\csname lstU@AtEnd\endcsname{%
      \endlinechar=\the\endlinechar\relax
      \catcode13=\the\catcode13\relax
      \catcode32=\the\catcode32\relax
      \catcode35=\the\catcode35\relax
      \catcode61=\the\catcode61\relax
      \catcode64=\the\catcode64\relax
      \catcode123=\the\catcode123\relax
      \catcode125=\the\catcode125\relax
    }%
  }%
\x\catcode61\catcode48\catcode32=10\relax%
\catcode13=5 % ^^M
\endlinechar=13 %
\catcode35=6 % #
\catcode64=11 % @
\catcode123=1 % {
\catcode125=2 % }
\def\TMP@EnsureCode#1#2{%
  \edef\lstU@AtEnd{%
    \lstU@AtEnd
    \catcode#1=\the\catcode#1\relax
  }%
  \catcode#1=#2\relax
}
\TMP@EnsureCode{10}{12}% ^^J
\TMP@EnsureCode{33}{12}% !
\TMP@EnsureCode{36}{3}% $
\TMP@EnsureCode{38}{4}% &
\TMP@EnsureCode{39}{12}% '
\TMP@EnsureCode{40}{12}% (
\TMP@EnsureCode{41}{12}% )
\TMP@EnsureCode{42}{12}% *
\TMP@EnsureCode{43}{12}% +
\TMP@EnsureCode{44}{12}% ,
\TMP@EnsureCode{45}{12}% -
\TMP@EnsureCode{46}{12}% .
\TMP@EnsureCode{47}{12}% /
\TMP@EnsureCode{58}{12}% :
\TMP@EnsureCode{60}{12}% <
\TMP@EnsureCode{62}{12}% >
\TMP@EnsureCode{91}{12}% [
\TMP@EnsureCode{93}{12}% ]
\TMP@EnsureCode{94}{7}% ^ (superscript)
\TMP@EnsureCode{95}{8}% _ (subscript)
\TMP@EnsureCode{96}{12}% `
\TMP@EnsureCode{124}{12}% |
\TMP@EnsureCode{126}{13}% ~ (active)
\edef\lstU@AtEnd{\lstU@AtEnd\noexpand\endinput}
%    \end{macrocode}
%
%    Package identification.
%    \begin{macrocode}
\NeedsTeXFormat{LaTeX2e}
\ProvidesPackage{listingsutf8}%
  [2011/11/10 v1.2 Allow UTF-8 in listings input (HO)]
%    \end{macrocode}
%
% \subsection{Package options}
%
% Just pass options to package listings.
%
%    \begin{macrocode}
\DeclareOption*{%
  \PassOptionsToPackage\CurrentOption{listings}%
}
\ProcessOptions*
%    \end{macrocode}
%    Key \xoption{inputencoding} was introduced in version
%    2002/04/01 v1.0 of package \xpackage{listings}.
%    \begin{macrocode}
\RequirePackage{listings}[2002/04/01]
%    \end{macrocode}
%    Ensure that \cs{inputencoding} is provided.
%    \begin{macrocode}
\AtBeginDocument{%
  \@ifundefined{inputencoding}{%
    \RequirePackage{inputenc}%
  }{}%
}
%    \end{macrocode}
%
% \subsection{Check prerequisites}
%
%    \begin{macrocode}
\RequirePackage{pdftexcmds}[2011/04/22]
%    \end{macrocode}
%
%    \begin{macrocode}
\def\lstU@temp#1#2{%
  \begingroup\expandafter\expandafter\expandafter\endgroup
  \expandafter\ifx\csname #1\endcsname\relax
    \PackageWarningNoLine{listingsutf8}{%
      Package loading is aborted because of missing %
      \@backslashchar#1.\MessageBreak
      #2%
    }%
    \expandafter\lstU@AtEnd
  \fi
}
\lstU@temp{scantokens}{It is provided by e-TeX}%
\lstU@temp{pdf@unescapehex}{It is provided by pdfTeX >= 1.30}%
\lstU@temp{pdf@filedump}{It is provided by pdfTeX >= 1.30}%
\lstU@temp{pdf@filesize}{It is provided by pdfTeX >= 1.30}%
%    \end{macrocode}
%
%    \begin{macrocode}
\RequirePackage{stringenc}[2010/03/01]
%    \end{macrocode}
%
% \subsection{Add support for UTF-8}
%
%    \begin{macro}{\iflstU@utfviii}
%    \begin{macrocode}
\newif\iflstU@utfviii
%    \end{macrocode}
%    \end{macro}
%
%    \begin{macro}{\lstU@inputenc}
%    \begin{macrocode}
\def\lstU@inputenc#1{%
  \expandafter\lstU@@inputenc#1utf8/utf8/\@nil
}
%    \end{macrocode}
%    \end{macro}
%    \begin{macro}{\lstU@@inputenc}
\def\lstU@@inputenc#1utf8/#2utf8/#3\@nil{%
  \ifx\\#1\\%
    \lstU@utfviiitrue
    \def\lst@inputenc{#2}%
  \else
    \lstU@utfviiifalse
  \fi
}
%    \begin{macrocode}
%    \end{macrocode}
%    \end{macro}
%
%    \begin{macrocode}
\lst@Key{inputencoding}\relax{%
  \def\lst@inputenc{#1}%
  \lstU@inputenc{#1}%
}
%    \end{macrocode}
%
% \subsubsection{Conversion}
%
%    \begin{macro}{\lstU@input}
%    \begin{macrocode}
\def\lstU@input#1{%
  \iflstU@utfviii
    \edef\lstU@text{%
      \pdf@unescapehex{%
        \pdf@filedump{0}{\pdf@filesize{#1}}{#1}%
      }%
    }%
    \lstU@CRLFtoLF\lstU@text
    \StringEncodingConvert\lstU@text\lstU@text{utf8}\lst@inputenc
    \def\lstU@temp{%
      \scantokens\expandafter{\lstU@text}%
    }%
  \else
    \def\lstU@temp{%
      \input{#1}%
    }%
  \fi
  \lstU@temp
}
%    \end{macrocode}
%    \end{macro}
%
% \subsubsection{Convert CR/LF pairs to LF}
%
%    \begin{macro}{\lstU@CRLFtoLF}
%    \begin{macrocode}
\begingroup
  \endlinechar=-1 %
  \@makeother\^^J %
  \@makeother\^^M %
  \gdef\lstU@CRLFtoLF#1{%
    \edef#1{%
      \expandafter\lstU@CRLFtoLF@aux#1^^M^^J\@nil
    }%
  }%
  \gdef\lstU@CRLFtoLF@aux#1^^M^^J#2\@nil{%
    #1%
    \ifx\relax#2\relax
      \@car
    \fi
    ^^J%
    \lstU@CRLFtoLF@aux#2\@nil
  }%
\endgroup %
%    \end{macrocode}
%    \end{macro}
%
% \subsubsection{Patch \cs{lst@InputListing}}
%
%    \begin{macrocode}
\def\lstU@temp#1\def\lst@next#2#3\@nil{%
  \def\lst@InputListing##1{%
    #1%
    \def\lst@next{\lstU@input{##1}}%
    #3%
  }%
}
\expandafter\lstU@temp\lst@InputListing{#1}\@nil
%    \end{macrocode}
%
%    \begin{macrocode}
\lstU@AtEnd%
%</package>
%    \end{macrocode}
%
% \section{Test}
%
% \subsection{Catcode checks for loading}
%
%    \begin{macrocode}
%<*test1>
%    \end{macrocode}
%    \begin{macrocode}
\NeedsTeXFormat{LaTeX2e}
\documentclass{minimal}
\makeatletter
\def\RestoreCatcodes{}
\count@=0 %
\loop
  \edef\RestoreCatcodes{%
    \RestoreCatcodes
    \catcode\the\count@=\the\catcode\count@\relax
  }%
\ifnum\count@<255 %
  \advance\count@\@ne
\repeat

\def\RangeCatcodeInvalid#1#2{%
  \count@=#1\relax
  \loop
    \catcode\count@=15 %
  \ifnum\count@<#2\relax
    \advance\count@\@ne
  \repeat
}
\def\Test{%
  \RangeCatcodeInvalid{0}{47}%
  \RangeCatcodeInvalid{58}{64}%
  \RangeCatcodeInvalid{91}{96}%
  \RangeCatcodeInvalid{123}{127}%
  \catcode`\@=12 %
  \catcode`\\=0 %
  \catcode`\{=1 %
  \catcode`\}=2 %
  \catcode`\#=6 %
  \catcode`\[=12 %
  \catcode`\]=12 %
  \catcode`\%=14 %
  \catcode`\ =10 %
  \catcode13=5 %
  \RequirePackage{listingsutf8}[2011/11/10]\relax
  \RestoreCatcodes
}
\Test
\csname @@end\endcsname
\end
%    \end{macrocode}
%    \begin{macrocode}
%</test1>
%    \end{macrocode}
%
% \subsection{Test example for latin1}
%
%    \begin{macrocode}
%<*test2>
%    \end{macrocode}
%    \begin{macrocode}
\NeedsTeXFormat{LaTeX2e}
\documentclass{minimal}
\usepackage{filecontents}
\def\do#1{%
  \ifx#1\^%
  \else
    \noexpand\do\noexpand#1%
  \fi
}
\expandafter\let\expandafter\dospecials\expandafter\empty
\expandafter\edef\expandafter\dospecials\expandafter{\dospecials}
\begin{filecontents*}{ExampleUTF8.java}
public class ExampleUTF8 {
    public static String testString =
        "Umlauts: " +
        "^^c3^^84^^c3^^96^^c3^^9c^^c3^^a4^^c3^^b6^^c3^^bc^^c3^^9f";
    public static void main(String[] args) {
        System.out.println(testString);
    }
}
\end{filecontents*}
\usepackage{listingsutf8}[2011/11/10]
\def\Text{%
  Umlauts: %
  ^^c3^^84^^c3^^96^^c3^^9c^^c3^^a4^^c3^^b6^^c3^^bc^^c3^^9f%
}
\begin{document}
\lstinputlisting[%
  language=Java,%
  inputencoding=utf8/latin1,%
]{ExampleUTF8.java}
\end{document}
%</test2>
%    \end{macrocode}
%
% \section{Installation}
%
% \subsection{Download}
%
% \paragraph{Package.} This package is available on
% CTAN\footnote{\url{ftp://ftp.ctan.org/tex-archive/}}:
% \begin{description}
% \item[\CTAN{macros/latex/contrib/oberdiek/listingsutf8.dtx}] The source file.
% \item[\CTAN{macros/latex/contrib/oberdiek/listingsutf8.pdf}] Documentation.
% \end{description}
%
%
% \paragraph{Bundle.} All the packages of the bundle `oberdiek'
% are also available in a TDS compliant ZIP archive. There
% the packages are already unpacked and the documentation files
% are generated. The files and directories obey the TDS standard.
% \begin{description}
% \item[\CTAN{install/macros/latex/contrib/oberdiek.tds.zip}]
% \end{description}
% \emph{TDS} refers to the standard ``A Directory Structure
% for \TeX\ Files'' (\CTAN{tds/tds.pdf}). Directories
% with \xfile{texmf} in their name are usually organized this way.
%
% \subsection{Bundle installation}
%
% \paragraph{Unpacking.} Unpack the \xfile{oberdiek.tds.zip} in the
% TDS tree (also known as \xfile{texmf} tree) of your choice.
% Example (linux):
% \begin{quote}
%   |unzip oberdiek.tds.zip -d ~/texmf|
% \end{quote}
%
% \paragraph{Script installation.}
% Check the directory \xfile{TDS:scripts/oberdiek/} for
% scripts that need further installation steps.
% Package \xpackage{attachfile2} comes with the Perl script
% \xfile{pdfatfi.pl} that should be installed in such a way
% that it can be called as \texttt{pdfatfi}.
% Example (linux):
% \begin{quote}
%   |chmod +x scripts/oberdiek/pdfatfi.pl|\\
%   |cp scripts/oberdiek/pdfatfi.pl /usr/local/bin/|
% \end{quote}
%
% \subsection{Package installation}
%
% \paragraph{Unpacking.} The \xfile{.dtx} file is a self-extracting
% \docstrip\ archive. The files are extracted by running the
% \xfile{.dtx} through \plainTeX:
% \begin{quote}
%   \verb|tex listingsutf8.dtx|
% \end{quote}
%
% \paragraph{TDS.} Now the different files must be moved into
% the different directories in your installation TDS tree
% (also known as \xfile{texmf} tree):
% \begin{quote}
% \def\t{^^A
% \begin{tabular}{@{}>{\ttfamily}l@{ $\rightarrow$ }>{\ttfamily}l@{}}
%   listingsutf8.sty & tex/latex/oberdiek/listingsutf8.sty\\
%   listingsutf8.pdf & doc/latex/oberdiek/listingsutf8.pdf\\
%   test/listingsutf8-test1.tex & doc/latex/oberdiek/test/listingsutf8-test1.tex\\
%   test/listingsutf8-test2.tex & doc/latex/oberdiek/test/listingsutf8-test2.tex\\
%   test/listingsutf8-test3.tex & doc/latex/oberdiek/test/listingsutf8-test3.tex\\
%   test/listingsutf8-test4.tex & doc/latex/oberdiek/test/listingsutf8-test4.tex\\
%   test/listingsutf8-test5.tex & doc/latex/oberdiek/test/listingsutf8-test5.tex\\
%   listingsutf8.dtx & source/latex/oberdiek/listingsutf8.dtx\\
% \end{tabular}^^A
% }^^A
% \sbox0{\t}^^A
% \ifdim\wd0>\linewidth
%   \begingroup
%     \advance\linewidth by\leftmargin
%     \advance\linewidth by\rightmargin
%   \edef\x{\endgroup
%     \def\noexpand\lw{\the\linewidth}^^A
%   }\x
%   \def\lwbox{^^A
%     \leavevmode
%     \hbox to \linewidth{^^A
%       \kern-\leftmargin\relax
%       \hss
%       \usebox0
%       \hss
%       \kern-\rightmargin\relax
%     }^^A
%   }^^A
%   \ifdim\wd0>\lw
%     \sbox0{\small\t}^^A
%     \ifdim\wd0>\linewidth
%       \ifdim\wd0>\lw
%         \sbox0{\footnotesize\t}^^A
%         \ifdim\wd0>\linewidth
%           \ifdim\wd0>\lw
%             \sbox0{\scriptsize\t}^^A
%             \ifdim\wd0>\linewidth
%               \ifdim\wd0>\lw
%                 \sbox0{\tiny\t}^^A
%                 \ifdim\wd0>\linewidth
%                   \lwbox
%                 \else
%                   \usebox0
%                 \fi
%               \else
%                 \lwbox
%               \fi
%             \else
%               \usebox0
%             \fi
%           \else
%             \lwbox
%           \fi
%         \else
%           \usebox0
%         \fi
%       \else
%         \lwbox
%       \fi
%     \else
%       \usebox0
%     \fi
%   \else
%     \lwbox
%   \fi
% \else
%   \usebox0
% \fi
% \end{quote}
% If you have a \xfile{docstrip.cfg} that configures and enables \docstrip's
% TDS installing feature, then some files can already be in the right
% place, see the documentation of \docstrip.
%
% \subsection{Refresh file name databases}
%
% If your \TeX~distribution
% (\teTeX, \mikTeX, \dots) relies on file name databases, you must refresh
% these. For example, \teTeX\ users run \verb|texhash| or
% \verb|mktexlsr|.
%
% \subsection{Some details for the interested}
%
% \paragraph{Attached source.}
%
% The PDF documentation on CTAN also includes the
% \xfile{.dtx} source file. It can be extracted by
% AcrobatReader 6 or higher. Another option is \textsf{pdftk},
% e.g. unpack the file into the current directory:
% \begin{quote}
%   \verb|pdftk listingsutf8.pdf unpack_files output .|
% \end{quote}
%
% \paragraph{Unpacking with \LaTeX.}
% The \xfile{.dtx} chooses its action depending on the format:
% \begin{description}
% \item[\plainTeX:] Run \docstrip\ and extract the files.
% \item[\LaTeX:] Generate the documentation.
% \end{description}
% If you insist on using \LaTeX\ for \docstrip\ (really,
% \docstrip\ does not need \LaTeX), then inform the autodetect routine
% about your intention:
% \begin{quote}
%   \verb|latex \let\install=y\input{listingsutf8.dtx}|
% \end{quote}
% Do not forget to quote the argument according to the demands
% of your shell.
%
% \paragraph{Generating the documentation.}
% You can use both the \xfile{.dtx} or the \xfile{.drv} to generate
% the documentation. The process can be configured by the
% configuration file \xfile{ltxdoc.cfg}. For instance, put this
% line into this file, if you want to have A4 as paper format:
% \begin{quote}
%   \verb|\PassOptionsToClass{a4paper}{article}|
% \end{quote}
% An example follows how to generate the
% documentation with pdf\LaTeX:
% \begin{quote}
%\begin{verbatim}
%pdflatex listingsutf8.dtx
%makeindex -s gind.ist listingsutf8.idx
%pdflatex listingsutf8.dtx
%makeindex -s gind.ist listingsutf8.idx
%pdflatex listingsutf8.dtx
%\end{verbatim}
% \end{quote}
%
% \section{Catalogue}
%
% The following XML file can be used as source for the
% \href{http://mirror.ctan.org/help/Catalogue/catalogue.html}{\TeX\ Catalogue}.
% The elements \texttt{caption} and \texttt{description} are imported
% from the original XML file from the Catalogue.
% The name of the XML file in the Catalogue is \xfile{listingsutf8.xml}.
%    \begin{macrocode}
%<*catalogue>
<?xml version='1.0' encoding='us-ascii'?>
<!DOCTYPE entry SYSTEM 'catalogue.dtd'>
<entry datestamp='$Date$' modifier='$Author$' id='listingsutf8'>
  <name>listingsutf8</name>
  <caption>Allow UTF-8 in listings input.</caption>
  <authorref id='auth:oberdiek'/>
  <copyright owner='Heiko Oberdiek' year='2007,2011'/>
  <license type='lppl1.3'/>
  <version number='1.2'/>
  <description>
    Package <xref refid='listings'>listings</xref> does not support files
    with multi-byte encodings such as UTF-8.  In the case of
    <tt>\lstinputlisting</tt>, a simple workaround is possible if a
    one-byte encoding exists that the file can be converted to.  The
    package requires the e-TeX extensions under pdfTeX (in either PDF
    or DVI output mode).
    <p/>
    The package is part of the <xref refid='oberdiek'>oberdiek</xref> bundle.
  </description>
  <documentation details='Package documentation'
      href='ctan:/macros/latex/contrib/oberdiek/listingsutf8.pdf'/>
  <ctan file='true' path='/macros/latex/contrib/oberdiek/listingsutf8.dtx'/>
  <miktex location='oberdiek'/>
  <texlive location='oberdiek'/>
  <install path='/macros/latex/contrib/oberdiek/oberdiek.tds.zip'/>
</entry>
%</catalogue>
%    \end{macrocode}
%
% \begin{thebibliography}{9}
%
% \bibitem{inputenc}
%   Alan Jeffrey, Frank Mittelbach,
%   \textit{inputenc.sty}, 2006/05/05 v1.1b.
%   \CTAN{macros/latex/base/inputenc.dtx}
%
% \bibitem{listings}
%   Carsten Heinz, Brooks Moses:
%  \textit{The \xpackage{listings} package};
%   2007/02/22;\\
%   \CTAN{macros/latex/contrib/listings/}.
%
% \bibitem{stringenc}
%   Heiko Oberdiek:
%   \textit{The \xpackage{stringenc} package};
%   2007/10/22;\\
%   \CTAN{macros/latex/contrib/oberdiek/stringenc.pdf}.
%
% \end{thebibliography}
%
% \begin{History}
%   \begin{Version}{2007/10/22 v1.0}
%   \item
%     First version.
%   \end{Version}
%   \begin{Version}{2007/11/11 v1.1}
%   \item
%     Use of package \xpackage{pdftexcmds}.
%   \end{Version}
%   \begin{Version}{2011/11/10 v1.2}
%   \item
%     DOS line ends CR/LF normalized to LF to avoid empty lines
%     (Bug report of Thomas Benkert in de.comp.text.tex).
%   \end{Version}
% \end{History}
%
% \PrintIndex
%
% \Finale
\endinput

%        (quote the arguments according to the demands of your shell)
%
% Documentation:
%    (a) If listingsutf8.drv is present:
%           latex listingsutf8.drv
%    (b) Without listingsutf8.drv:
%           latex listingsutf8.dtx; ...
%    The class ltxdoc loads the configuration file ltxdoc.cfg
%    if available. Here you can specify further options, e.g.
%    use A4 as paper format:
%       \PassOptionsToClass{a4paper}{article}
%
%    Programm calls to get the documentation (example):
%       pdflatex listingsutf8.dtx
%       makeindex -s gind.ist listingsutf8.idx
%       pdflatex listingsutf8.dtx
%       makeindex -s gind.ist listingsutf8.idx
%       pdflatex listingsutf8.dtx
%
% Installation:
%    TDS:tex/latex/oberdiek/listingsutf8.sty
%    TDS:doc/latex/oberdiek/listingsutf8.pdf
%    TDS:doc/latex/oberdiek/test/listingsutf8-test1.tex
%    TDS:doc/latex/oberdiek/test/listingsutf8-test2.tex
%    TDS:doc/latex/oberdiek/test/listingsutf8-test3.tex
%    TDS:doc/latex/oberdiek/test/listingsutf8-test4.tex
%    TDS:doc/latex/oberdiek/test/listingsutf8-test5.tex
%    TDS:source/latex/oberdiek/listingsutf8.dtx
%
%<*ignore>
\begingroup
  \catcode123=1 %
  \catcode125=2 %
  \def\x{LaTeX2e}%
\expandafter\endgroup
\ifcase 0\ifx\install y1\fi\expandafter
         \ifx\csname processbatchFile\endcsname\relax\else1\fi
         \ifx\fmtname\x\else 1\fi\relax
\else\csname fi\endcsname
%</ignore>
%<*install>
\input docstrip.tex
\Msg{************************************************************************}
\Msg{* Installation}
\Msg{* Package: listingsutf8 2011/11/10 v1.2 Allow UTF-8 in listings input (HO)}
\Msg{************************************************************************}

\keepsilent
\askforoverwritefalse

\let\MetaPrefix\relax
\preamble

This is a generated file.

Project: listingsutf8
Version: 2011/11/10 v1.2

Copyright (C) 2007, 2011 by
   Heiko Oberdiek <heiko.oberdiek at googlemail.com>

This work may be distributed and/or modified under the
conditions of the LaTeX Project Public License, either
version 1.3c of this license or (at your option) any later
version. This version of this license is in
   http://www.latex-project.org/lppl/lppl-1-3c.txt
and the latest version of this license is in
   http://www.latex-project.org/lppl.txt
and version 1.3 or later is part of all distributions of
LaTeX version 2005/12/01 or later.

This work has the LPPL maintenance status "maintained".

This Current Maintainer of this work is Heiko Oberdiek.

This work consists of the main source file listingsutf8.dtx
and the derived files
   listingsutf8.sty, listingsutf8.pdf, listingsutf8.ins, listingsutf8.drv,
   listingsutf8-test1.tex, listingsutf8-test2.tex,
   listingsutf8-test3.tex, listingsutf8-test4.tex,
   listingsutf8-test5.tex.

\endpreamble
\let\MetaPrefix\DoubleperCent

\generate{%
  \file{listingsutf8.ins}{\from{listingsutf8.dtx}{install}}%
  \file{listingsutf8.drv}{\from{listingsutf8.dtx}{driver}}%
  \usedir{tex/latex/oberdiek}%
  \file{listingsutf8.sty}{\from{listingsutf8.dtx}{package}}%
  \usedir{doc/latex/oberdiek/test}%
  \file{listingsutf8-test1.tex}{\from{listingsutf8.dtx}{test1}}%
  \file{listingsutf8-test2.tex}{\from{listingsutf8.dtx}{test2,utf8}}%
  \file{listingsutf8-test3.tex}{\from{listingsutf8.dtx}{test3,utf8x}}%
  \file{listingsutf8-test4.tex}{\from{listingsutf8.dtx}{test4,utf8,noetex}}%
  \file{listingsutf8-test5.tex}{\from{listingsutf8.dtx}{test5,utf8x,noetex}}%
  \nopreamble
  \nopostamble
  \usedir{source/latex/oberdiek/catalogue}%
  \file{listingsutf8.xml}{\from{listingsutf8.dtx}{catalogue}}%
}

\catcode32=13\relax% active space
\let =\space%
\Msg{************************************************************************}
\Msg{*}
\Msg{* To finish the installation you have to move the following}
\Msg{* file into a directory searched by TeX:}
\Msg{*}
\Msg{*     listingsutf8.sty}
\Msg{*}
\Msg{* To produce the documentation run the file `listingsutf8.drv'}
\Msg{* through LaTeX.}
\Msg{*}
\Msg{* Happy TeXing!}
\Msg{*}
\Msg{************************************************************************}

\endbatchfile
%</install>
%<*ignore>
\fi
%</ignore>
%<*driver>
\NeedsTeXFormat{LaTeX2e}
\ProvidesFile{listingsutf8.drv}%
  [2011/11/10 v1.2 Allow UTF-8 in listings input (HO)]%
\documentclass{ltxdoc}
\usepackage{holtxdoc}[2011/11/22]
\begin{document}
  \DocInput{listingsutf8.dtx}%
\end{document}
%</driver>
% \fi
%
% \CheckSum{311}
%
% \CharacterTable
%  {Upper-case    \A\B\C\D\E\F\G\H\I\J\K\L\M\N\O\P\Q\R\S\T\U\V\W\X\Y\Z
%   Lower-case    \a\b\c\d\e\f\g\h\i\j\k\l\m\n\o\p\q\r\s\t\u\v\w\x\y\z
%   Digits        \0\1\2\3\4\5\6\7\8\9
%   Exclamation   \!     Double quote  \"     Hash (number) \#
%   Dollar        \$     Percent       \%     Ampersand     \&
%   Acute accent  \'     Left paren    \(     Right paren   \)
%   Asterisk      \*     Plus          \+     Comma         \,
%   Minus         \-     Point         \.     Solidus       \/
%   Colon         \:     Semicolon     \;     Less than     \<
%   Equals        \=     Greater than  \>     Question mark \?
%   Commercial at \@     Left bracket  \[     Backslash     \\
%   Right bracket \]     Circumflex    \^     Underscore    \_
%   Grave accent  \`     Left brace    \{     Vertical bar  \|
%   Right brace   \}     Tilde         \~}
%
% \GetFileInfo{listingsutf8.drv}
%
% \title{The \xpackage{listingsutf8} package}
% \date{2011/11/10 v1.2}
% \author{Heiko Oberdiek\\\xemail{heiko.oberdiek at googlemail.com}}
%
% \maketitle
%
% \begin{abstract}
% Package \xpackage{listings} does not support files with multi-byte
% encodings such as UTF-8. In case of \cs{lstinputlisting} a simple
% workaround is possible if an one-byte encoding exists that the file
% can be converted to. Also \eTeX\ and \pdfTeX\ regardless of its mode
% are required.
% \end{abstract}
%
% \tableofcontents
%
% \section{Documentation}
%
% \subsection{User interface}
%
% Load this package after or instead of package \xpackage{listings}
% \cite{listings}.
% The package does not define own options and passes given options to
% package \xpackage{listings}.
%
% The syntax of package \xpackage{listings}' key \xoption{inputencoding}
% is extended:
% \begin{quote}
%   |inputencoding=utf8/|\meta{one-byte-encoding}\\
%   Example: |inputencoding=utf8/latin1|
% \end{quote}
% That means the file is encoded in UTF-8 and can
% be converted to the given \meta{one-byte-encoding}.
% The available encodings for \meta{one-byte-encoding} are
% listed in section ``1.2 Supported encodings'' of
% package \xpackage{stringenc}'s documentation \cite{stringenc}.
% Of course, the encoding must encode its characters with
% one byte exactly. This excludes the unicode encodings
% (\xoption{utf8}, \xoption{utf16}, \dots).
%
% Only \cs{lstinputlisting} is supported by the syntax extension
% of key \xoption{inputencoding}.
%
% Internally package \xpackage{listingsutf8} reads the file as binary file
% via primitives of \pdfTeX\ (\cs{pdffiledump}). Then the file
% contents is converted as string using package \xpackage{stringenc} and
% finally the string is read as virtual file by \eTeX's \cs{scantokens}.
%
% \subsection{Future}
%
% Workarounds are not provided for
% \begin{itemize}
% \item \cs{lstinline}
% \item Environment |lstlisting|.
% \item Environments defined by \cs{lstnewenvironment}.
% \end{itemize}
% Perhaps someone will find time to extend package \xpackage{listings}
% with full native support for UTF-8. Then this package would become obsolete.
%
% \StopEventually{
% }
%
% \section{Implementation}
%
%    \begin{macrocode}
%<*package>
%    \end{macrocode}
%
% \subsection{Catcodes and identification}
%
%    \begin{macrocode}
\begingroup\catcode61\catcode48\catcode32=10\relax%
  \catcode13=5 % ^^M
  \endlinechar=13 %
  \catcode123=1 % {
  \catcode125=2 % }
  \catcode64=11 % @
  \def\x{\endgroup
    \expandafter\edef\csname lstU@AtEnd\endcsname{%
      \endlinechar=\the\endlinechar\relax
      \catcode13=\the\catcode13\relax
      \catcode32=\the\catcode32\relax
      \catcode35=\the\catcode35\relax
      \catcode61=\the\catcode61\relax
      \catcode64=\the\catcode64\relax
      \catcode123=\the\catcode123\relax
      \catcode125=\the\catcode125\relax
    }%
  }%
\x\catcode61\catcode48\catcode32=10\relax%
\catcode13=5 % ^^M
\endlinechar=13 %
\catcode35=6 % #
\catcode64=11 % @
\catcode123=1 % {
\catcode125=2 % }
\def\TMP@EnsureCode#1#2{%
  \edef\lstU@AtEnd{%
    \lstU@AtEnd
    \catcode#1=\the\catcode#1\relax
  }%
  \catcode#1=#2\relax
}
\TMP@EnsureCode{10}{12}% ^^J
\TMP@EnsureCode{33}{12}% !
\TMP@EnsureCode{36}{3}% $
\TMP@EnsureCode{38}{4}% &
\TMP@EnsureCode{39}{12}% '
\TMP@EnsureCode{40}{12}% (
\TMP@EnsureCode{41}{12}% )
\TMP@EnsureCode{42}{12}% *
\TMP@EnsureCode{43}{12}% +
\TMP@EnsureCode{44}{12}% ,
\TMP@EnsureCode{45}{12}% -
\TMP@EnsureCode{46}{12}% .
\TMP@EnsureCode{47}{12}% /
\TMP@EnsureCode{58}{12}% :
\TMP@EnsureCode{60}{12}% <
\TMP@EnsureCode{62}{12}% >
\TMP@EnsureCode{91}{12}% [
\TMP@EnsureCode{93}{12}% ]
\TMP@EnsureCode{94}{7}% ^ (superscript)
\TMP@EnsureCode{95}{8}% _ (subscript)
\TMP@EnsureCode{96}{12}% `
\TMP@EnsureCode{124}{12}% |
\TMP@EnsureCode{126}{13}% ~ (active)
\edef\lstU@AtEnd{\lstU@AtEnd\noexpand\endinput}
%    \end{macrocode}
%
%    Package identification.
%    \begin{macrocode}
\NeedsTeXFormat{LaTeX2e}
\ProvidesPackage{listingsutf8}%
  [2011/11/10 v1.2 Allow UTF-8 in listings input (HO)]
%    \end{macrocode}
%
% \subsection{Package options}
%
% Just pass options to package listings.
%
%    \begin{macrocode}
\DeclareOption*{%
  \PassOptionsToPackage\CurrentOption{listings}%
}
\ProcessOptions*
%    \end{macrocode}
%    Key \xoption{inputencoding} was introduced in version
%    2002/04/01 v1.0 of package \xpackage{listings}.
%    \begin{macrocode}
\RequirePackage{listings}[2002/04/01]
%    \end{macrocode}
%    Ensure that \cs{inputencoding} is provided.
%    \begin{macrocode}
\AtBeginDocument{%
  \@ifundefined{inputencoding}{%
    \RequirePackage{inputenc}%
  }{}%
}
%    \end{macrocode}
%
% \subsection{Check prerequisites}
%
%    \begin{macrocode}
\RequirePackage{pdftexcmds}[2011/04/22]
%    \end{macrocode}
%
%    \begin{macrocode}
\def\lstU@temp#1#2{%
  \begingroup\expandafter\expandafter\expandafter\endgroup
  \expandafter\ifx\csname #1\endcsname\relax
    \PackageWarningNoLine{listingsutf8}{%
      Package loading is aborted because of missing %
      \@backslashchar#1.\MessageBreak
      #2%
    }%
    \expandafter\lstU@AtEnd
  \fi
}
\lstU@temp{scantokens}{It is provided by e-TeX}%
\lstU@temp{pdf@unescapehex}{It is provided by pdfTeX >= 1.30}%
\lstU@temp{pdf@filedump}{It is provided by pdfTeX >= 1.30}%
\lstU@temp{pdf@filesize}{It is provided by pdfTeX >= 1.30}%
%    \end{macrocode}
%
%    \begin{macrocode}
\RequirePackage{stringenc}[2010/03/01]
%    \end{macrocode}
%
% \subsection{Add support for UTF-8}
%
%    \begin{macro}{\iflstU@utfviii}
%    \begin{macrocode}
\newif\iflstU@utfviii
%    \end{macrocode}
%    \end{macro}
%
%    \begin{macro}{\lstU@inputenc}
%    \begin{macrocode}
\def\lstU@inputenc#1{%
  \expandafter\lstU@@inputenc#1utf8/utf8/\@nil
}
%    \end{macrocode}
%    \end{macro}
%    \begin{macro}{\lstU@@inputenc}
\def\lstU@@inputenc#1utf8/#2utf8/#3\@nil{%
  \ifx\\#1\\%
    \lstU@utfviiitrue
    \def\lst@inputenc{#2}%
  \else
    \lstU@utfviiifalse
  \fi
}
%    \begin{macrocode}
%    \end{macrocode}
%    \end{macro}
%
%    \begin{macrocode}
\lst@Key{inputencoding}\relax{%
  \def\lst@inputenc{#1}%
  \lstU@inputenc{#1}%
}
%    \end{macrocode}
%
% \subsubsection{Conversion}
%
%    \begin{macro}{\lstU@input}
%    \begin{macrocode}
\def\lstU@input#1{%
  \iflstU@utfviii
    \edef\lstU@text{%
      \pdf@unescapehex{%
        \pdf@filedump{0}{\pdf@filesize{#1}}{#1}%
      }%
    }%
    \lstU@CRLFtoLF\lstU@text
    \StringEncodingConvert\lstU@text\lstU@text{utf8}\lst@inputenc
    \def\lstU@temp{%
      \scantokens\expandafter{\lstU@text}%
    }%
  \else
    \def\lstU@temp{%
      \input{#1}%
    }%
  \fi
  \lstU@temp
}
%    \end{macrocode}
%    \end{macro}
%
% \subsubsection{Convert CR/LF pairs to LF}
%
%    \begin{macro}{\lstU@CRLFtoLF}
%    \begin{macrocode}
\begingroup
  \endlinechar=-1 %
  \@makeother\^^J %
  \@makeother\^^M %
  \gdef\lstU@CRLFtoLF#1{%
    \edef#1{%
      \expandafter\lstU@CRLFtoLF@aux#1^^M^^J\@nil
    }%
  }%
  \gdef\lstU@CRLFtoLF@aux#1^^M^^J#2\@nil{%
    #1%
    \ifx\relax#2\relax
      \@car
    \fi
    ^^J%
    \lstU@CRLFtoLF@aux#2\@nil
  }%
\endgroup %
%    \end{macrocode}
%    \end{macro}
%
% \subsubsection{Patch \cs{lst@InputListing}}
%
%    \begin{macrocode}
\def\lstU@temp#1\def\lst@next#2#3\@nil{%
  \def\lst@InputListing##1{%
    #1%
    \def\lst@next{\lstU@input{##1}}%
    #3%
  }%
}
\expandafter\lstU@temp\lst@InputListing{#1}\@nil
%    \end{macrocode}
%
%    \begin{macrocode}
\lstU@AtEnd%
%</package>
%    \end{macrocode}
%
% \section{Test}
%
% \subsection{Catcode checks for loading}
%
%    \begin{macrocode}
%<*test1>
%    \end{macrocode}
%    \begin{macrocode}
\NeedsTeXFormat{LaTeX2e}
\documentclass{minimal}
\makeatletter
\def\RestoreCatcodes{}
\count@=0 %
\loop
  \edef\RestoreCatcodes{%
    \RestoreCatcodes
    \catcode\the\count@=\the\catcode\count@\relax
  }%
\ifnum\count@<255 %
  \advance\count@\@ne
\repeat

\def\RangeCatcodeInvalid#1#2{%
  \count@=#1\relax
  \loop
    \catcode\count@=15 %
  \ifnum\count@<#2\relax
    \advance\count@\@ne
  \repeat
}
\def\Test{%
  \RangeCatcodeInvalid{0}{47}%
  \RangeCatcodeInvalid{58}{64}%
  \RangeCatcodeInvalid{91}{96}%
  \RangeCatcodeInvalid{123}{127}%
  \catcode`\@=12 %
  \catcode`\\=0 %
  \catcode`\{=1 %
  \catcode`\}=2 %
  \catcode`\#=6 %
  \catcode`\[=12 %
  \catcode`\]=12 %
  \catcode`\%=14 %
  \catcode`\ =10 %
  \catcode13=5 %
  \RequirePackage{listingsutf8}[2011/11/10]\relax
  \RestoreCatcodes
}
\Test
\csname @@end\endcsname
\end
%    \end{macrocode}
%    \begin{macrocode}
%</test1>
%    \end{macrocode}
%
% \subsection{Test example for latin1}
%
%    \begin{macrocode}
%<*test2>
%    \end{macrocode}
%    \begin{macrocode}
\NeedsTeXFormat{LaTeX2e}
\documentclass{minimal}
\usepackage{filecontents}
\def\do#1{%
  \ifx#1\^%
  \else
    \noexpand\do\noexpand#1%
  \fi
}
\expandafter\let\expandafter\dospecials\expandafter\empty
\expandafter\edef\expandafter\dospecials\expandafter{\dospecials}
\begin{filecontents*}{ExampleUTF8.java}
public class ExampleUTF8 {
    public static String testString =
        "Umlauts: " +
        "^^c3^^84^^c3^^96^^c3^^9c^^c3^^a4^^c3^^b6^^c3^^bc^^c3^^9f";
    public static void main(String[] args) {
        System.out.println(testString);
    }
}
\end{filecontents*}
\usepackage{listingsutf8}[2011/11/10]
\def\Text{%
  Umlauts: %
  ^^c3^^84^^c3^^96^^c3^^9c^^c3^^a4^^c3^^b6^^c3^^bc^^c3^^9f%
}
\begin{document}
\lstinputlisting[%
  language=Java,%
  inputencoding=utf8/latin1,%
]{ExampleUTF8.java}
\end{document}
%</test2>
%    \end{macrocode}
%
% \section{Installation}
%
% \subsection{Download}
%
% \paragraph{Package.} This package is available on
% CTAN\footnote{\url{ftp://ftp.ctan.org/tex-archive/}}:
% \begin{description}
% \item[\CTAN{macros/latex/contrib/oberdiek/listingsutf8.dtx}] The source file.
% \item[\CTAN{macros/latex/contrib/oberdiek/listingsutf8.pdf}] Documentation.
% \end{description}
%
%
% \paragraph{Bundle.} All the packages of the bundle `oberdiek'
% are also available in a TDS compliant ZIP archive. There
% the packages are already unpacked and the documentation files
% are generated. The files and directories obey the TDS standard.
% \begin{description}
% \item[\CTAN{install/macros/latex/contrib/oberdiek.tds.zip}]
% \end{description}
% \emph{TDS} refers to the standard ``A Directory Structure
% for \TeX\ Files'' (\CTAN{tds/tds.pdf}). Directories
% with \xfile{texmf} in their name are usually organized this way.
%
% \subsection{Bundle installation}
%
% \paragraph{Unpacking.} Unpack the \xfile{oberdiek.tds.zip} in the
% TDS tree (also known as \xfile{texmf} tree) of your choice.
% Example (linux):
% \begin{quote}
%   |unzip oberdiek.tds.zip -d ~/texmf|
% \end{quote}
%
% \paragraph{Script installation.}
% Check the directory \xfile{TDS:scripts/oberdiek/} for
% scripts that need further installation steps.
% Package \xpackage{attachfile2} comes with the Perl script
% \xfile{pdfatfi.pl} that should be installed in such a way
% that it can be called as \texttt{pdfatfi}.
% Example (linux):
% \begin{quote}
%   |chmod +x scripts/oberdiek/pdfatfi.pl|\\
%   |cp scripts/oberdiek/pdfatfi.pl /usr/local/bin/|
% \end{quote}
%
% \subsection{Package installation}
%
% \paragraph{Unpacking.} The \xfile{.dtx} file is a self-extracting
% \docstrip\ archive. The files are extracted by running the
% \xfile{.dtx} through \plainTeX:
% \begin{quote}
%   \verb|tex listingsutf8.dtx|
% \end{quote}
%
% \paragraph{TDS.} Now the different files must be moved into
% the different directories in your installation TDS tree
% (also known as \xfile{texmf} tree):
% \begin{quote}
% \def\t{^^A
% \begin{tabular}{@{}>{\ttfamily}l@{ $\rightarrow$ }>{\ttfamily}l@{}}
%   listingsutf8.sty & tex/latex/oberdiek/listingsutf8.sty\\
%   listingsutf8.pdf & doc/latex/oberdiek/listingsutf8.pdf\\
%   test/listingsutf8-test1.tex & doc/latex/oberdiek/test/listingsutf8-test1.tex\\
%   test/listingsutf8-test2.tex & doc/latex/oberdiek/test/listingsutf8-test2.tex\\
%   test/listingsutf8-test3.tex & doc/latex/oberdiek/test/listingsutf8-test3.tex\\
%   test/listingsutf8-test4.tex & doc/latex/oberdiek/test/listingsutf8-test4.tex\\
%   test/listingsutf8-test5.tex & doc/latex/oberdiek/test/listingsutf8-test5.tex\\
%   listingsutf8.dtx & source/latex/oberdiek/listingsutf8.dtx\\
% \end{tabular}^^A
% }^^A
% \sbox0{\t}^^A
% \ifdim\wd0>\linewidth
%   \begingroup
%     \advance\linewidth by\leftmargin
%     \advance\linewidth by\rightmargin
%   \edef\x{\endgroup
%     \def\noexpand\lw{\the\linewidth}^^A
%   }\x
%   \def\lwbox{^^A
%     \leavevmode
%     \hbox to \linewidth{^^A
%       \kern-\leftmargin\relax
%       \hss
%       \usebox0
%       \hss
%       \kern-\rightmargin\relax
%     }^^A
%   }^^A
%   \ifdim\wd0>\lw
%     \sbox0{\small\t}^^A
%     \ifdim\wd0>\linewidth
%       \ifdim\wd0>\lw
%         \sbox0{\footnotesize\t}^^A
%         \ifdim\wd0>\linewidth
%           \ifdim\wd0>\lw
%             \sbox0{\scriptsize\t}^^A
%             \ifdim\wd0>\linewidth
%               \ifdim\wd0>\lw
%                 \sbox0{\tiny\t}^^A
%                 \ifdim\wd0>\linewidth
%                   \lwbox
%                 \else
%                   \usebox0
%                 \fi
%               \else
%                 \lwbox
%               \fi
%             \else
%               \usebox0
%             \fi
%           \else
%             \lwbox
%           \fi
%         \else
%           \usebox0
%         \fi
%       \else
%         \lwbox
%       \fi
%     \else
%       \usebox0
%     \fi
%   \else
%     \lwbox
%   \fi
% \else
%   \usebox0
% \fi
% \end{quote}
% If you have a \xfile{docstrip.cfg} that configures and enables \docstrip's
% TDS installing feature, then some files can already be in the right
% place, see the documentation of \docstrip.
%
% \subsection{Refresh file name databases}
%
% If your \TeX~distribution
% (\teTeX, \mikTeX, \dots) relies on file name databases, you must refresh
% these. For example, \teTeX\ users run \verb|texhash| or
% \verb|mktexlsr|.
%
% \subsection{Some details for the interested}
%
% \paragraph{Attached source.}
%
% The PDF documentation on CTAN also includes the
% \xfile{.dtx} source file. It can be extracted by
% AcrobatReader 6 or higher. Another option is \textsf{pdftk},
% e.g. unpack the file into the current directory:
% \begin{quote}
%   \verb|pdftk listingsutf8.pdf unpack_files output .|
% \end{quote}
%
% \paragraph{Unpacking with \LaTeX.}
% The \xfile{.dtx} chooses its action depending on the format:
% \begin{description}
% \item[\plainTeX:] Run \docstrip\ and extract the files.
% \item[\LaTeX:] Generate the documentation.
% \end{description}
% If you insist on using \LaTeX\ for \docstrip\ (really,
% \docstrip\ does not need \LaTeX), then inform the autodetect routine
% about your intention:
% \begin{quote}
%   \verb|latex \let\install=y% \iffalse meta-comment
%
% File: listingsutf8.dtx
% Version: 2011/11/10 v1.2
% Info: Allow UTF-8 in listings input
%
% Copyright (C) 2007, 2011 by
%    Heiko Oberdiek <heiko.oberdiek at googlemail.com>
%
% This work may be distributed and/or modified under the
% conditions of the LaTeX Project Public License, either
% version 1.3c of this license or (at your option) any later
% version. This version of this license is in
%    http://www.latex-project.org/lppl/lppl-1-3c.txt
% and the latest version of this license is in
%    http://www.latex-project.org/lppl.txt
% and version 1.3 or later is part of all distributions of
% LaTeX version 2005/12/01 or later.
%
% This work has the LPPL maintenance status "maintained".
%
% This Current Maintainer of this work is Heiko Oberdiek.
%
% This work consists of the main source file listingsutf8.dtx
% and the derived files
%    listingsutf8.sty, listingsutf8.pdf, listingsutf8.ins, listingsutf8.drv,
%    listingsutf8-test1.tex, listingsutf8-test2.tex,
%    listingsutf8-test3.tex, listingsutf8-test4.tex,
%    listingsutf8-test5.tex.
%
% Distribution:
%    CTAN:macros/latex/contrib/oberdiek/listingsutf8.dtx
%    CTAN:macros/latex/contrib/oberdiek/listingsutf8.pdf
%
% Unpacking:
%    (a) If listingsutf8.ins is present:
%           tex listingsutf8.ins
%    (b) Without listingsutf8.ins:
%           tex listingsutf8.dtx
%    (c) If you insist on using LaTeX
%           latex \let\install=y\input{listingsutf8.dtx}
%        (quote the arguments according to the demands of your shell)
%
% Documentation:
%    (a) If listingsutf8.drv is present:
%           latex listingsutf8.drv
%    (b) Without listingsutf8.drv:
%           latex listingsutf8.dtx; ...
%    The class ltxdoc loads the configuration file ltxdoc.cfg
%    if available. Here you can specify further options, e.g.
%    use A4 as paper format:
%       \PassOptionsToClass{a4paper}{article}
%
%    Programm calls to get the documentation (example):
%       pdflatex listingsutf8.dtx
%       makeindex -s gind.ist listingsutf8.idx
%       pdflatex listingsutf8.dtx
%       makeindex -s gind.ist listingsutf8.idx
%       pdflatex listingsutf8.dtx
%
% Installation:
%    TDS:tex/latex/oberdiek/listingsutf8.sty
%    TDS:doc/latex/oberdiek/listingsutf8.pdf
%    TDS:doc/latex/oberdiek/test/listingsutf8-test1.tex
%    TDS:doc/latex/oberdiek/test/listingsutf8-test2.tex
%    TDS:doc/latex/oberdiek/test/listingsutf8-test3.tex
%    TDS:doc/latex/oberdiek/test/listingsutf8-test4.tex
%    TDS:doc/latex/oberdiek/test/listingsutf8-test5.tex
%    TDS:source/latex/oberdiek/listingsutf8.dtx
%
%<*ignore>
\begingroup
  \catcode123=1 %
  \catcode125=2 %
  \def\x{LaTeX2e}%
\expandafter\endgroup
\ifcase 0\ifx\install y1\fi\expandafter
         \ifx\csname processbatchFile\endcsname\relax\else1\fi
         \ifx\fmtname\x\else 1\fi\relax
\else\csname fi\endcsname
%</ignore>
%<*install>
\input docstrip.tex
\Msg{************************************************************************}
\Msg{* Installation}
\Msg{* Package: listingsutf8 2011/11/10 v1.2 Allow UTF-8 in listings input (HO)}
\Msg{************************************************************************}

\keepsilent
\askforoverwritefalse

\let\MetaPrefix\relax
\preamble

This is a generated file.

Project: listingsutf8
Version: 2011/11/10 v1.2

Copyright (C) 2007, 2011 by
   Heiko Oberdiek <heiko.oberdiek at googlemail.com>

This work may be distributed and/or modified under the
conditions of the LaTeX Project Public License, either
version 1.3c of this license or (at your option) any later
version. This version of this license is in
   http://www.latex-project.org/lppl/lppl-1-3c.txt
and the latest version of this license is in
   http://www.latex-project.org/lppl.txt
and version 1.3 or later is part of all distributions of
LaTeX version 2005/12/01 or later.

This work has the LPPL maintenance status "maintained".

This Current Maintainer of this work is Heiko Oberdiek.

This work consists of the main source file listingsutf8.dtx
and the derived files
   listingsutf8.sty, listingsutf8.pdf, listingsutf8.ins, listingsutf8.drv,
   listingsutf8-test1.tex, listingsutf8-test2.tex,
   listingsutf8-test3.tex, listingsutf8-test4.tex,
   listingsutf8-test5.tex.

\endpreamble
\let\MetaPrefix\DoubleperCent

\generate{%
  \file{listingsutf8.ins}{\from{listingsutf8.dtx}{install}}%
  \file{listingsutf8.drv}{\from{listingsutf8.dtx}{driver}}%
  \usedir{tex/latex/oberdiek}%
  \file{listingsutf8.sty}{\from{listingsutf8.dtx}{package}}%
  \usedir{doc/latex/oberdiek/test}%
  \file{listingsutf8-test1.tex}{\from{listingsutf8.dtx}{test1}}%
  \file{listingsutf8-test2.tex}{\from{listingsutf8.dtx}{test2,utf8}}%
  \file{listingsutf8-test3.tex}{\from{listingsutf8.dtx}{test3,utf8x}}%
  \file{listingsutf8-test4.tex}{\from{listingsutf8.dtx}{test4,utf8,noetex}}%
  \file{listingsutf8-test5.tex}{\from{listingsutf8.dtx}{test5,utf8x,noetex}}%
  \nopreamble
  \nopostamble
  \usedir{source/latex/oberdiek/catalogue}%
  \file{listingsutf8.xml}{\from{listingsutf8.dtx}{catalogue}}%
}

\catcode32=13\relax% active space
\let =\space%
\Msg{************************************************************************}
\Msg{*}
\Msg{* To finish the installation you have to move the following}
\Msg{* file into a directory searched by TeX:}
\Msg{*}
\Msg{*     listingsutf8.sty}
\Msg{*}
\Msg{* To produce the documentation run the file `listingsutf8.drv'}
\Msg{* through LaTeX.}
\Msg{*}
\Msg{* Happy TeXing!}
\Msg{*}
\Msg{************************************************************************}

\endbatchfile
%</install>
%<*ignore>
\fi
%</ignore>
%<*driver>
\NeedsTeXFormat{LaTeX2e}
\ProvidesFile{listingsutf8.drv}%
  [2011/11/10 v1.2 Allow UTF-8 in listings input (HO)]%
\documentclass{ltxdoc}
\usepackage{holtxdoc}[2011/11/22]
\begin{document}
  \DocInput{listingsutf8.dtx}%
\end{document}
%</driver>
% \fi
%
% \CheckSum{311}
%
% \CharacterTable
%  {Upper-case    \A\B\C\D\E\F\G\H\I\J\K\L\M\N\O\P\Q\R\S\T\U\V\W\X\Y\Z
%   Lower-case    \a\b\c\d\e\f\g\h\i\j\k\l\m\n\o\p\q\r\s\t\u\v\w\x\y\z
%   Digits        \0\1\2\3\4\5\6\7\8\9
%   Exclamation   \!     Double quote  \"     Hash (number) \#
%   Dollar        \$     Percent       \%     Ampersand     \&
%   Acute accent  \'     Left paren    \(     Right paren   \)
%   Asterisk      \*     Plus          \+     Comma         \,
%   Minus         \-     Point         \.     Solidus       \/
%   Colon         \:     Semicolon     \;     Less than     \<
%   Equals        \=     Greater than  \>     Question mark \?
%   Commercial at \@     Left bracket  \[     Backslash     \\
%   Right bracket \]     Circumflex    \^     Underscore    \_
%   Grave accent  \`     Left brace    \{     Vertical bar  \|
%   Right brace   \}     Tilde         \~}
%
% \GetFileInfo{listingsutf8.drv}
%
% \title{The \xpackage{listingsutf8} package}
% \date{2011/11/10 v1.2}
% \author{Heiko Oberdiek\\\xemail{heiko.oberdiek at googlemail.com}}
%
% \maketitle
%
% \begin{abstract}
% Package \xpackage{listings} does not support files with multi-byte
% encodings such as UTF-8. In case of \cs{lstinputlisting} a simple
% workaround is possible if an one-byte encoding exists that the file
% can be converted to. Also \eTeX\ and \pdfTeX\ regardless of its mode
% are required.
% \end{abstract}
%
% \tableofcontents
%
% \section{Documentation}
%
% \subsection{User interface}
%
% Load this package after or instead of package \xpackage{listings}
% \cite{listings}.
% The package does not define own options and passes given options to
% package \xpackage{listings}.
%
% The syntax of package \xpackage{listings}' key \xoption{inputencoding}
% is extended:
% \begin{quote}
%   |inputencoding=utf8/|\meta{one-byte-encoding}\\
%   Example: |inputencoding=utf8/latin1|
% \end{quote}
% That means the file is encoded in UTF-8 and can
% be converted to the given \meta{one-byte-encoding}.
% The available encodings for \meta{one-byte-encoding} are
% listed in section ``1.2 Supported encodings'' of
% package \xpackage{stringenc}'s documentation \cite{stringenc}.
% Of course, the encoding must encode its characters with
% one byte exactly. This excludes the unicode encodings
% (\xoption{utf8}, \xoption{utf16}, \dots).
%
% Only \cs{lstinputlisting} is supported by the syntax extension
% of key \xoption{inputencoding}.
%
% Internally package \xpackage{listingsutf8} reads the file as binary file
% via primitives of \pdfTeX\ (\cs{pdffiledump}). Then the file
% contents is converted as string using package \xpackage{stringenc} and
% finally the string is read as virtual file by \eTeX's \cs{scantokens}.
%
% \subsection{Future}
%
% Workarounds are not provided for
% \begin{itemize}
% \item \cs{lstinline}
% \item Environment |lstlisting|.
% \item Environments defined by \cs{lstnewenvironment}.
% \end{itemize}
% Perhaps someone will find time to extend package \xpackage{listings}
% with full native support for UTF-8. Then this package would become obsolete.
%
% \StopEventually{
% }
%
% \section{Implementation}
%
%    \begin{macrocode}
%<*package>
%    \end{macrocode}
%
% \subsection{Catcodes and identification}
%
%    \begin{macrocode}
\begingroup\catcode61\catcode48\catcode32=10\relax%
  \catcode13=5 % ^^M
  \endlinechar=13 %
  \catcode123=1 % {
  \catcode125=2 % }
  \catcode64=11 % @
  \def\x{\endgroup
    \expandafter\edef\csname lstU@AtEnd\endcsname{%
      \endlinechar=\the\endlinechar\relax
      \catcode13=\the\catcode13\relax
      \catcode32=\the\catcode32\relax
      \catcode35=\the\catcode35\relax
      \catcode61=\the\catcode61\relax
      \catcode64=\the\catcode64\relax
      \catcode123=\the\catcode123\relax
      \catcode125=\the\catcode125\relax
    }%
  }%
\x\catcode61\catcode48\catcode32=10\relax%
\catcode13=5 % ^^M
\endlinechar=13 %
\catcode35=6 % #
\catcode64=11 % @
\catcode123=1 % {
\catcode125=2 % }
\def\TMP@EnsureCode#1#2{%
  \edef\lstU@AtEnd{%
    \lstU@AtEnd
    \catcode#1=\the\catcode#1\relax
  }%
  \catcode#1=#2\relax
}
\TMP@EnsureCode{10}{12}% ^^J
\TMP@EnsureCode{33}{12}% !
\TMP@EnsureCode{36}{3}% $
\TMP@EnsureCode{38}{4}% &
\TMP@EnsureCode{39}{12}% '
\TMP@EnsureCode{40}{12}% (
\TMP@EnsureCode{41}{12}% )
\TMP@EnsureCode{42}{12}% *
\TMP@EnsureCode{43}{12}% +
\TMP@EnsureCode{44}{12}% ,
\TMP@EnsureCode{45}{12}% -
\TMP@EnsureCode{46}{12}% .
\TMP@EnsureCode{47}{12}% /
\TMP@EnsureCode{58}{12}% :
\TMP@EnsureCode{60}{12}% <
\TMP@EnsureCode{62}{12}% >
\TMP@EnsureCode{91}{12}% [
\TMP@EnsureCode{93}{12}% ]
\TMP@EnsureCode{94}{7}% ^ (superscript)
\TMP@EnsureCode{95}{8}% _ (subscript)
\TMP@EnsureCode{96}{12}% `
\TMP@EnsureCode{124}{12}% |
\TMP@EnsureCode{126}{13}% ~ (active)
\edef\lstU@AtEnd{\lstU@AtEnd\noexpand\endinput}
%    \end{macrocode}
%
%    Package identification.
%    \begin{macrocode}
\NeedsTeXFormat{LaTeX2e}
\ProvidesPackage{listingsutf8}%
  [2011/11/10 v1.2 Allow UTF-8 in listings input (HO)]
%    \end{macrocode}
%
% \subsection{Package options}
%
% Just pass options to package listings.
%
%    \begin{macrocode}
\DeclareOption*{%
  \PassOptionsToPackage\CurrentOption{listings}%
}
\ProcessOptions*
%    \end{macrocode}
%    Key \xoption{inputencoding} was introduced in version
%    2002/04/01 v1.0 of package \xpackage{listings}.
%    \begin{macrocode}
\RequirePackage{listings}[2002/04/01]
%    \end{macrocode}
%    Ensure that \cs{inputencoding} is provided.
%    \begin{macrocode}
\AtBeginDocument{%
  \@ifundefined{inputencoding}{%
    \RequirePackage{inputenc}%
  }{}%
}
%    \end{macrocode}
%
% \subsection{Check prerequisites}
%
%    \begin{macrocode}
\RequirePackage{pdftexcmds}[2011/04/22]
%    \end{macrocode}
%
%    \begin{macrocode}
\def\lstU@temp#1#2{%
  \begingroup\expandafter\expandafter\expandafter\endgroup
  \expandafter\ifx\csname #1\endcsname\relax
    \PackageWarningNoLine{listingsutf8}{%
      Package loading is aborted because of missing %
      \@backslashchar#1.\MessageBreak
      #2%
    }%
    \expandafter\lstU@AtEnd
  \fi
}
\lstU@temp{scantokens}{It is provided by e-TeX}%
\lstU@temp{pdf@unescapehex}{It is provided by pdfTeX >= 1.30}%
\lstU@temp{pdf@filedump}{It is provided by pdfTeX >= 1.30}%
\lstU@temp{pdf@filesize}{It is provided by pdfTeX >= 1.30}%
%    \end{macrocode}
%
%    \begin{macrocode}
\RequirePackage{stringenc}[2010/03/01]
%    \end{macrocode}
%
% \subsection{Add support for UTF-8}
%
%    \begin{macro}{\iflstU@utfviii}
%    \begin{macrocode}
\newif\iflstU@utfviii
%    \end{macrocode}
%    \end{macro}
%
%    \begin{macro}{\lstU@inputenc}
%    \begin{macrocode}
\def\lstU@inputenc#1{%
  \expandafter\lstU@@inputenc#1utf8/utf8/\@nil
}
%    \end{macrocode}
%    \end{macro}
%    \begin{macro}{\lstU@@inputenc}
\def\lstU@@inputenc#1utf8/#2utf8/#3\@nil{%
  \ifx\\#1\\%
    \lstU@utfviiitrue
    \def\lst@inputenc{#2}%
  \else
    \lstU@utfviiifalse
  \fi
}
%    \begin{macrocode}
%    \end{macrocode}
%    \end{macro}
%
%    \begin{macrocode}
\lst@Key{inputencoding}\relax{%
  \def\lst@inputenc{#1}%
  \lstU@inputenc{#1}%
}
%    \end{macrocode}
%
% \subsubsection{Conversion}
%
%    \begin{macro}{\lstU@input}
%    \begin{macrocode}
\def\lstU@input#1{%
  \iflstU@utfviii
    \edef\lstU@text{%
      \pdf@unescapehex{%
        \pdf@filedump{0}{\pdf@filesize{#1}}{#1}%
      }%
    }%
    \lstU@CRLFtoLF\lstU@text
    \StringEncodingConvert\lstU@text\lstU@text{utf8}\lst@inputenc
    \def\lstU@temp{%
      \scantokens\expandafter{\lstU@text}%
    }%
  \else
    \def\lstU@temp{%
      \input{#1}%
    }%
  \fi
  \lstU@temp
}
%    \end{macrocode}
%    \end{macro}
%
% \subsubsection{Convert CR/LF pairs to LF}
%
%    \begin{macro}{\lstU@CRLFtoLF}
%    \begin{macrocode}
\begingroup
  \endlinechar=-1 %
  \@makeother\^^J %
  \@makeother\^^M %
  \gdef\lstU@CRLFtoLF#1{%
    \edef#1{%
      \expandafter\lstU@CRLFtoLF@aux#1^^M^^J\@nil
    }%
  }%
  \gdef\lstU@CRLFtoLF@aux#1^^M^^J#2\@nil{%
    #1%
    \ifx\relax#2\relax
      \@car
    \fi
    ^^J%
    \lstU@CRLFtoLF@aux#2\@nil
  }%
\endgroup %
%    \end{macrocode}
%    \end{macro}
%
% \subsubsection{Patch \cs{lst@InputListing}}
%
%    \begin{macrocode}
\def\lstU@temp#1\def\lst@next#2#3\@nil{%
  \def\lst@InputListing##1{%
    #1%
    \def\lst@next{\lstU@input{##1}}%
    #3%
  }%
}
\expandafter\lstU@temp\lst@InputListing{#1}\@nil
%    \end{macrocode}
%
%    \begin{macrocode}
\lstU@AtEnd%
%</package>
%    \end{macrocode}
%
% \section{Test}
%
% \subsection{Catcode checks for loading}
%
%    \begin{macrocode}
%<*test1>
%    \end{macrocode}
%    \begin{macrocode}
\NeedsTeXFormat{LaTeX2e}
\documentclass{minimal}
\makeatletter
\def\RestoreCatcodes{}
\count@=0 %
\loop
  \edef\RestoreCatcodes{%
    \RestoreCatcodes
    \catcode\the\count@=\the\catcode\count@\relax
  }%
\ifnum\count@<255 %
  \advance\count@\@ne
\repeat

\def\RangeCatcodeInvalid#1#2{%
  \count@=#1\relax
  \loop
    \catcode\count@=15 %
  \ifnum\count@<#2\relax
    \advance\count@\@ne
  \repeat
}
\def\Test{%
  \RangeCatcodeInvalid{0}{47}%
  \RangeCatcodeInvalid{58}{64}%
  \RangeCatcodeInvalid{91}{96}%
  \RangeCatcodeInvalid{123}{127}%
  \catcode`\@=12 %
  \catcode`\\=0 %
  \catcode`\{=1 %
  \catcode`\}=2 %
  \catcode`\#=6 %
  \catcode`\[=12 %
  \catcode`\]=12 %
  \catcode`\%=14 %
  \catcode`\ =10 %
  \catcode13=5 %
  \RequirePackage{listingsutf8}[2011/11/10]\relax
  \RestoreCatcodes
}
\Test
\csname @@end\endcsname
\end
%    \end{macrocode}
%    \begin{macrocode}
%</test1>
%    \end{macrocode}
%
% \subsection{Test example for latin1}
%
%    \begin{macrocode}
%<*test2>
%    \end{macrocode}
%    \begin{macrocode}
\NeedsTeXFormat{LaTeX2e}
\documentclass{minimal}
\usepackage{filecontents}
\def\do#1{%
  \ifx#1\^%
  \else
    \noexpand\do\noexpand#1%
  \fi
}
\expandafter\let\expandafter\dospecials\expandafter\empty
\expandafter\edef\expandafter\dospecials\expandafter{\dospecials}
\begin{filecontents*}{ExampleUTF8.java}
public class ExampleUTF8 {
    public static String testString =
        "Umlauts: " +
        "^^c3^^84^^c3^^96^^c3^^9c^^c3^^a4^^c3^^b6^^c3^^bc^^c3^^9f";
    public static void main(String[] args) {
        System.out.println(testString);
    }
}
\end{filecontents*}
\usepackage{listingsutf8}[2011/11/10]
\def\Text{%
  Umlauts: %
  ^^c3^^84^^c3^^96^^c3^^9c^^c3^^a4^^c3^^b6^^c3^^bc^^c3^^9f%
}
\begin{document}
\lstinputlisting[%
  language=Java,%
  inputencoding=utf8/latin1,%
]{ExampleUTF8.java}
\end{document}
%</test2>
%    \end{macrocode}
%
% \section{Installation}
%
% \subsection{Download}
%
% \paragraph{Package.} This package is available on
% CTAN\footnote{\url{ftp://ftp.ctan.org/tex-archive/}}:
% \begin{description}
% \item[\CTAN{macros/latex/contrib/oberdiek/listingsutf8.dtx}] The source file.
% \item[\CTAN{macros/latex/contrib/oberdiek/listingsutf8.pdf}] Documentation.
% \end{description}
%
%
% \paragraph{Bundle.} All the packages of the bundle `oberdiek'
% are also available in a TDS compliant ZIP archive. There
% the packages are already unpacked and the documentation files
% are generated. The files and directories obey the TDS standard.
% \begin{description}
% \item[\CTAN{install/macros/latex/contrib/oberdiek.tds.zip}]
% \end{description}
% \emph{TDS} refers to the standard ``A Directory Structure
% for \TeX\ Files'' (\CTAN{tds/tds.pdf}). Directories
% with \xfile{texmf} in their name are usually organized this way.
%
% \subsection{Bundle installation}
%
% \paragraph{Unpacking.} Unpack the \xfile{oberdiek.tds.zip} in the
% TDS tree (also known as \xfile{texmf} tree) of your choice.
% Example (linux):
% \begin{quote}
%   |unzip oberdiek.tds.zip -d ~/texmf|
% \end{quote}
%
% \paragraph{Script installation.}
% Check the directory \xfile{TDS:scripts/oberdiek/} for
% scripts that need further installation steps.
% Package \xpackage{attachfile2} comes with the Perl script
% \xfile{pdfatfi.pl} that should be installed in such a way
% that it can be called as \texttt{pdfatfi}.
% Example (linux):
% \begin{quote}
%   |chmod +x scripts/oberdiek/pdfatfi.pl|\\
%   |cp scripts/oberdiek/pdfatfi.pl /usr/local/bin/|
% \end{quote}
%
% \subsection{Package installation}
%
% \paragraph{Unpacking.} The \xfile{.dtx} file is a self-extracting
% \docstrip\ archive. The files are extracted by running the
% \xfile{.dtx} through \plainTeX:
% \begin{quote}
%   \verb|tex listingsutf8.dtx|
% \end{quote}
%
% \paragraph{TDS.} Now the different files must be moved into
% the different directories in your installation TDS tree
% (also known as \xfile{texmf} tree):
% \begin{quote}
% \def\t{^^A
% \begin{tabular}{@{}>{\ttfamily}l@{ $\rightarrow$ }>{\ttfamily}l@{}}
%   listingsutf8.sty & tex/latex/oberdiek/listingsutf8.sty\\
%   listingsutf8.pdf & doc/latex/oberdiek/listingsutf8.pdf\\
%   test/listingsutf8-test1.tex & doc/latex/oberdiek/test/listingsutf8-test1.tex\\
%   test/listingsutf8-test2.tex & doc/latex/oberdiek/test/listingsutf8-test2.tex\\
%   test/listingsutf8-test3.tex & doc/latex/oberdiek/test/listingsutf8-test3.tex\\
%   test/listingsutf8-test4.tex & doc/latex/oberdiek/test/listingsutf8-test4.tex\\
%   test/listingsutf8-test5.tex & doc/latex/oberdiek/test/listingsutf8-test5.tex\\
%   listingsutf8.dtx & source/latex/oberdiek/listingsutf8.dtx\\
% \end{tabular}^^A
% }^^A
% \sbox0{\t}^^A
% \ifdim\wd0>\linewidth
%   \begingroup
%     \advance\linewidth by\leftmargin
%     \advance\linewidth by\rightmargin
%   \edef\x{\endgroup
%     \def\noexpand\lw{\the\linewidth}^^A
%   }\x
%   \def\lwbox{^^A
%     \leavevmode
%     \hbox to \linewidth{^^A
%       \kern-\leftmargin\relax
%       \hss
%       \usebox0
%       \hss
%       \kern-\rightmargin\relax
%     }^^A
%   }^^A
%   \ifdim\wd0>\lw
%     \sbox0{\small\t}^^A
%     \ifdim\wd0>\linewidth
%       \ifdim\wd0>\lw
%         \sbox0{\footnotesize\t}^^A
%         \ifdim\wd0>\linewidth
%           \ifdim\wd0>\lw
%             \sbox0{\scriptsize\t}^^A
%             \ifdim\wd0>\linewidth
%               \ifdim\wd0>\lw
%                 \sbox0{\tiny\t}^^A
%                 \ifdim\wd0>\linewidth
%                   \lwbox
%                 \else
%                   \usebox0
%                 \fi
%               \else
%                 \lwbox
%               \fi
%             \else
%               \usebox0
%             \fi
%           \else
%             \lwbox
%           \fi
%         \else
%           \usebox0
%         \fi
%       \else
%         \lwbox
%       \fi
%     \else
%       \usebox0
%     \fi
%   \else
%     \lwbox
%   \fi
% \else
%   \usebox0
% \fi
% \end{quote}
% If you have a \xfile{docstrip.cfg} that configures and enables \docstrip's
% TDS installing feature, then some files can already be in the right
% place, see the documentation of \docstrip.
%
% \subsection{Refresh file name databases}
%
% If your \TeX~distribution
% (\teTeX, \mikTeX, \dots) relies on file name databases, you must refresh
% these. For example, \teTeX\ users run \verb|texhash| or
% \verb|mktexlsr|.
%
% \subsection{Some details for the interested}
%
% \paragraph{Attached source.}
%
% The PDF documentation on CTAN also includes the
% \xfile{.dtx} source file. It can be extracted by
% AcrobatReader 6 or higher. Another option is \textsf{pdftk},
% e.g. unpack the file into the current directory:
% \begin{quote}
%   \verb|pdftk listingsutf8.pdf unpack_files output .|
% \end{quote}
%
% \paragraph{Unpacking with \LaTeX.}
% The \xfile{.dtx} chooses its action depending on the format:
% \begin{description}
% \item[\plainTeX:] Run \docstrip\ and extract the files.
% \item[\LaTeX:] Generate the documentation.
% \end{description}
% If you insist on using \LaTeX\ for \docstrip\ (really,
% \docstrip\ does not need \LaTeX), then inform the autodetect routine
% about your intention:
% \begin{quote}
%   \verb|latex \let\install=y\input{listingsutf8.dtx}|
% \end{quote}
% Do not forget to quote the argument according to the demands
% of your shell.
%
% \paragraph{Generating the documentation.}
% You can use both the \xfile{.dtx} or the \xfile{.drv} to generate
% the documentation. The process can be configured by the
% configuration file \xfile{ltxdoc.cfg}. For instance, put this
% line into this file, if you want to have A4 as paper format:
% \begin{quote}
%   \verb|\PassOptionsToClass{a4paper}{article}|
% \end{quote}
% An example follows how to generate the
% documentation with pdf\LaTeX:
% \begin{quote}
%\begin{verbatim}
%pdflatex listingsutf8.dtx
%makeindex -s gind.ist listingsutf8.idx
%pdflatex listingsutf8.dtx
%makeindex -s gind.ist listingsutf8.idx
%pdflatex listingsutf8.dtx
%\end{verbatim}
% \end{quote}
%
% \section{Catalogue}
%
% The following XML file can be used as source for the
% \href{http://mirror.ctan.org/help/Catalogue/catalogue.html}{\TeX\ Catalogue}.
% The elements \texttt{caption} and \texttt{description} are imported
% from the original XML file from the Catalogue.
% The name of the XML file in the Catalogue is \xfile{listingsutf8.xml}.
%    \begin{macrocode}
%<*catalogue>
<?xml version='1.0' encoding='us-ascii'?>
<!DOCTYPE entry SYSTEM 'catalogue.dtd'>
<entry datestamp='$Date$' modifier='$Author$' id='listingsutf8'>
  <name>listingsutf8</name>
  <caption>Allow UTF-8 in listings input.</caption>
  <authorref id='auth:oberdiek'/>
  <copyright owner='Heiko Oberdiek' year='2007,2011'/>
  <license type='lppl1.3'/>
  <version number='1.2'/>
  <description>
    Package <xref refid='listings'>listings</xref> does not support files
    with multi-byte encodings such as UTF-8.  In the case of
    <tt>\lstinputlisting</tt>, a simple workaround is possible if a
    one-byte encoding exists that the file can be converted to.  The
    package requires the e-TeX extensions under pdfTeX (in either PDF
    or DVI output mode).
    <p/>
    The package is part of the <xref refid='oberdiek'>oberdiek</xref> bundle.
  </description>
  <documentation details='Package documentation'
      href='ctan:/macros/latex/contrib/oberdiek/listingsutf8.pdf'/>
  <ctan file='true' path='/macros/latex/contrib/oberdiek/listingsutf8.dtx'/>
  <miktex location='oberdiek'/>
  <texlive location='oberdiek'/>
  <install path='/macros/latex/contrib/oberdiek/oberdiek.tds.zip'/>
</entry>
%</catalogue>
%    \end{macrocode}
%
% \begin{thebibliography}{9}
%
% \bibitem{inputenc}
%   Alan Jeffrey, Frank Mittelbach,
%   \textit{inputenc.sty}, 2006/05/05 v1.1b.
%   \CTAN{macros/latex/base/inputenc.dtx}
%
% \bibitem{listings}
%   Carsten Heinz, Brooks Moses:
%  \textit{The \xpackage{listings} package};
%   2007/02/22;\\
%   \CTAN{macros/latex/contrib/listings/}.
%
% \bibitem{stringenc}
%   Heiko Oberdiek:
%   \textit{The \xpackage{stringenc} package};
%   2007/10/22;\\
%   \CTAN{macros/latex/contrib/oberdiek/stringenc.pdf}.
%
% \end{thebibliography}
%
% \begin{History}
%   \begin{Version}{2007/10/22 v1.0}
%   \item
%     First version.
%   \end{Version}
%   \begin{Version}{2007/11/11 v1.1}
%   \item
%     Use of package \xpackage{pdftexcmds}.
%   \end{Version}
%   \begin{Version}{2011/11/10 v1.2}
%   \item
%     DOS line ends CR/LF normalized to LF to avoid empty lines
%     (Bug report of Thomas Benkert in de.comp.text.tex).
%   \end{Version}
% \end{History}
%
% \PrintIndex
%
% \Finale
\endinput
|
% \end{quote}
% Do not forget to quote the argument according to the demands
% of your shell.
%
% \paragraph{Generating the documentation.}
% You can use both the \xfile{.dtx} or the \xfile{.drv} to generate
% the documentation. The process can be configured by the
% configuration file \xfile{ltxdoc.cfg}. For instance, put this
% line into this file, if you want to have A4 as paper format:
% \begin{quote}
%   \verb|\PassOptionsToClass{a4paper}{article}|
% \end{quote}
% An example follows how to generate the
% documentation with pdf\LaTeX:
% \begin{quote}
%\begin{verbatim}
%pdflatex listingsutf8.dtx
%makeindex -s gind.ist listingsutf8.idx
%pdflatex listingsutf8.dtx
%makeindex -s gind.ist listingsutf8.idx
%pdflatex listingsutf8.dtx
%\end{verbatim}
% \end{quote}
%
% \section{Catalogue}
%
% The following XML file can be used as source for the
% \href{http://mirror.ctan.org/help/Catalogue/catalogue.html}{\TeX\ Catalogue}.
% The elements \texttt{caption} and \texttt{description} are imported
% from the original XML file from the Catalogue.
% The name of the XML file in the Catalogue is \xfile{listingsutf8.xml}.
%    \begin{macrocode}
%<*catalogue>
<?xml version='1.0' encoding='us-ascii'?>
<!DOCTYPE entry SYSTEM 'catalogue.dtd'>
<entry datestamp='$Date$' modifier='$Author$' id='listingsutf8'>
  <name>listingsutf8</name>
  <caption>Allow UTF-8 in listings input.</caption>
  <authorref id='auth:oberdiek'/>
  <copyright owner='Heiko Oberdiek' year='2007,2011'/>
  <license type='lppl1.3'/>
  <version number='1.2'/>
  <description>
    Package <xref refid='listings'>listings</xref> does not support files
    with multi-byte encodings such as UTF-8.  In the case of
    <tt>\lstinputlisting</tt>, a simple workaround is possible if a
    one-byte encoding exists that the file can be converted to.  The
    package requires the e-TeX extensions under pdfTeX (in either PDF
    or DVI output mode).
    <p/>
    The package is part of the <xref refid='oberdiek'>oberdiek</xref> bundle.
  </description>
  <documentation details='Package documentation'
      href='ctan:/macros/latex/contrib/oberdiek/listingsutf8.pdf'/>
  <ctan file='true' path='/macros/latex/contrib/oberdiek/listingsutf8.dtx'/>
  <miktex location='oberdiek'/>
  <texlive location='oberdiek'/>
  <install path='/macros/latex/contrib/oberdiek/oberdiek.tds.zip'/>
</entry>
%</catalogue>
%    \end{macrocode}
%
% \begin{thebibliography}{9}
%
% \bibitem{inputenc}
%   Alan Jeffrey, Frank Mittelbach,
%   \textit{inputenc.sty}, 2006/05/05 v1.1b.
%   \CTAN{macros/latex/base/inputenc.dtx}
%
% \bibitem{listings}
%   Carsten Heinz, Brooks Moses:
%  \textit{The \xpackage{listings} package};
%   2007/02/22;\\
%   \CTAN{macros/latex/contrib/listings/}.
%
% \bibitem{stringenc}
%   Heiko Oberdiek:
%   \textit{The \xpackage{stringenc} package};
%   2007/10/22;\\
%   \CTAN{macros/latex/contrib/oberdiek/stringenc.pdf}.
%
% \end{thebibliography}
%
% \begin{History}
%   \begin{Version}{2007/10/22 v1.0}
%   \item
%     First version.
%   \end{Version}
%   \begin{Version}{2007/11/11 v1.1}
%   \item
%     Use of package \xpackage{pdftexcmds}.
%   \end{Version}
%   \begin{Version}{2011/11/10 v1.2}
%   \item
%     DOS line ends CR/LF normalized to LF to avoid empty lines
%     (Bug report of Thomas Benkert in de.comp.text.tex).
%   \end{Version}
% \end{History}
%
% \PrintIndex
%
% \Finale
\endinput

%        (quote the arguments according to the demands of your shell)
%
% Documentation:
%    (a) If listingsutf8.drv is present:
%           latex listingsutf8.drv
%    (b) Without listingsutf8.drv:
%           latex listingsutf8.dtx; ...
%    The class ltxdoc loads the configuration file ltxdoc.cfg
%    if available. Here you can specify further options, e.g.
%    use A4 as paper format:
%       \PassOptionsToClass{a4paper}{article}
%
%    Programm calls to get the documentation (example):
%       pdflatex listingsutf8.dtx
%       makeindex -s gind.ist listingsutf8.idx
%       pdflatex listingsutf8.dtx
%       makeindex -s gind.ist listingsutf8.idx
%       pdflatex listingsutf8.dtx
%
% Installation:
%    TDS:tex/latex/oberdiek/listingsutf8.sty
%    TDS:doc/latex/oberdiek/listingsutf8.pdf
%    TDS:doc/latex/oberdiek/test/listingsutf8-test1.tex
%    TDS:doc/latex/oberdiek/test/listingsutf8-test2.tex
%    TDS:doc/latex/oberdiek/test/listingsutf8-test3.tex
%    TDS:doc/latex/oberdiek/test/listingsutf8-test4.tex
%    TDS:doc/latex/oberdiek/test/listingsutf8-test5.tex
%    TDS:source/latex/oberdiek/listingsutf8.dtx
%
%<*ignore>
\begingroup
  \catcode123=1 %
  \catcode125=2 %
  \def\x{LaTeX2e}%
\expandafter\endgroup
\ifcase 0\ifx\install y1\fi\expandafter
         \ifx\csname processbatchFile\endcsname\relax\else1\fi
         \ifx\fmtname\x\else 1\fi\relax
\else\csname fi\endcsname
%</ignore>
%<*install>
\input docstrip.tex
\Msg{************************************************************************}
\Msg{* Installation}
\Msg{* Package: listingsutf8 2011/11/10 v1.2 Allow UTF-8 in listings input (HO)}
\Msg{************************************************************************}

\keepsilent
\askforoverwritefalse

\let\MetaPrefix\relax
\preamble

This is a generated file.

Project: listingsutf8
Version: 2011/11/10 v1.2

Copyright (C) 2007, 2011 by
   Heiko Oberdiek <heiko.oberdiek at googlemail.com>

This work may be distributed and/or modified under the
conditions of the LaTeX Project Public License, either
version 1.3c of this license or (at your option) any later
version. This version of this license is in
   http://www.latex-project.org/lppl/lppl-1-3c.txt
and the latest version of this license is in
   http://www.latex-project.org/lppl.txt
and version 1.3 or later is part of all distributions of
LaTeX version 2005/12/01 or later.

This work has the LPPL maintenance status "maintained".

This Current Maintainer of this work is Heiko Oberdiek.

This work consists of the main source file listingsutf8.dtx
and the derived files
   listingsutf8.sty, listingsutf8.pdf, listingsutf8.ins, listingsutf8.drv,
   listingsutf8-test1.tex, listingsutf8-test2.tex,
   listingsutf8-test3.tex, listingsutf8-test4.tex,
   listingsutf8-test5.tex.

\endpreamble
\let\MetaPrefix\DoubleperCent

\generate{%
  \file{listingsutf8.ins}{\from{listingsutf8.dtx}{install}}%
  \file{listingsutf8.drv}{\from{listingsutf8.dtx}{driver}}%
  \usedir{tex/latex/oberdiek}%
  \file{listingsutf8.sty}{\from{listingsutf8.dtx}{package}}%
  \usedir{doc/latex/oberdiek/test}%
  \file{listingsutf8-test1.tex}{\from{listingsutf8.dtx}{test1}}%
  \file{listingsutf8-test2.tex}{\from{listingsutf8.dtx}{test2,utf8}}%
  \file{listingsutf8-test3.tex}{\from{listingsutf8.dtx}{test3,utf8x}}%
  \file{listingsutf8-test4.tex}{\from{listingsutf8.dtx}{test4,utf8,noetex}}%
  \file{listingsutf8-test5.tex}{\from{listingsutf8.dtx}{test5,utf8x,noetex}}%
  \nopreamble
  \nopostamble
  \usedir{source/latex/oberdiek/catalogue}%
  \file{listingsutf8.xml}{\from{listingsutf8.dtx}{catalogue}}%
}

\catcode32=13\relax% active space
\let =\space%
\Msg{************************************************************************}
\Msg{*}
\Msg{* To finish the installation you have to move the following}
\Msg{* file into a directory searched by TeX:}
\Msg{*}
\Msg{*     listingsutf8.sty}
\Msg{*}
\Msg{* To produce the documentation run the file `listingsutf8.drv'}
\Msg{* through LaTeX.}
\Msg{*}
\Msg{* Happy TeXing!}
\Msg{*}
\Msg{************************************************************************}

\endbatchfile
%</install>
%<*ignore>
\fi
%</ignore>
%<*driver>
\NeedsTeXFormat{LaTeX2e}
\ProvidesFile{listingsutf8.drv}%
  [2011/11/10 v1.2 Allow UTF-8 in listings input (HO)]%
\documentclass{ltxdoc}
\usepackage{holtxdoc}[2011/11/22]
\begin{document}
  \DocInput{listingsutf8.dtx}%
\end{document}
%</driver>
% \fi
%
% \CheckSum{311}
%
% \CharacterTable
%  {Upper-case    \A\B\C\D\E\F\G\H\I\J\K\L\M\N\O\P\Q\R\S\T\U\V\W\X\Y\Z
%   Lower-case    \a\b\c\d\e\f\g\h\i\j\k\l\m\n\o\p\q\r\s\t\u\v\w\x\y\z
%   Digits        \0\1\2\3\4\5\6\7\8\9
%   Exclamation   \!     Double quote  \"     Hash (number) \#
%   Dollar        \$     Percent       \%     Ampersand     \&
%   Acute accent  \'     Left paren    \(     Right paren   \)
%   Asterisk      \*     Plus          \+     Comma         \,
%   Minus         \-     Point         \.     Solidus       \/
%   Colon         \:     Semicolon     \;     Less than     \<
%   Equals        \=     Greater than  \>     Question mark \?
%   Commercial at \@     Left bracket  \[     Backslash     \\
%   Right bracket \]     Circumflex    \^     Underscore    \_
%   Grave accent  \`     Left brace    \{     Vertical bar  \|
%   Right brace   \}     Tilde         \~}
%
% \GetFileInfo{listingsutf8.drv}
%
% \title{The \xpackage{listingsutf8} package}
% \date{2011/11/10 v1.2}
% \author{Heiko Oberdiek\\\xemail{heiko.oberdiek at googlemail.com}}
%
% \maketitle
%
% \begin{abstract}
% Package \xpackage{listings} does not support files with multi-byte
% encodings such as UTF-8. In case of \cs{lstinputlisting} a simple
% workaround is possible if an one-byte encoding exists that the file
% can be converted to. Also \eTeX\ and \pdfTeX\ regardless of its mode
% are required.
% \end{abstract}
%
% \tableofcontents
%
% \section{Documentation}
%
% \subsection{User interface}
%
% Load this package after or instead of package \xpackage{listings}
% \cite{listings}.
% The package does not define own options and passes given options to
% package \xpackage{listings}.
%
% The syntax of package \xpackage{listings}' key \xoption{inputencoding}
% is extended:
% \begin{quote}
%   |inputencoding=utf8/|\meta{one-byte-encoding}\\
%   Example: |inputencoding=utf8/latin1|
% \end{quote}
% That means the file is encoded in UTF-8 and can
% be converted to the given \meta{one-byte-encoding}.
% The available encodings for \meta{one-byte-encoding} are
% listed in section ``1.2 Supported encodings'' of
% package \xpackage{stringenc}'s documentation \cite{stringenc}.
% Of course, the encoding must encode its characters with
% one byte exactly. This excludes the unicode encodings
% (\xoption{utf8}, \xoption{utf16}, \dots).
%
% Only \cs{lstinputlisting} is supported by the syntax extension
% of key \xoption{inputencoding}.
%
% Internally package \xpackage{listingsutf8} reads the file as binary file
% via primitives of \pdfTeX\ (\cs{pdffiledump}). Then the file
% contents is converted as string using package \xpackage{stringenc} and
% finally the string is read as virtual file by \eTeX's \cs{scantokens}.
%
% \subsection{Future}
%
% Workarounds are not provided for
% \begin{itemize}
% \item \cs{lstinline}
% \item Environment |lstlisting|.
% \item Environments defined by \cs{lstnewenvironment}.
% \end{itemize}
% Perhaps someone will find time to extend package \xpackage{listings}
% with full native support for UTF-8. Then this package would become obsolete.
%
% \StopEventually{
% }
%
% \section{Implementation}
%
%    \begin{macrocode}
%<*package>
%    \end{macrocode}
%
% \subsection{Catcodes and identification}
%
%    \begin{macrocode}
\begingroup\catcode61\catcode48\catcode32=10\relax%
  \catcode13=5 % ^^M
  \endlinechar=13 %
  \catcode123=1 % {
  \catcode125=2 % }
  \catcode64=11 % @
  \def\x{\endgroup
    \expandafter\edef\csname lstU@AtEnd\endcsname{%
      \endlinechar=\the\endlinechar\relax
      \catcode13=\the\catcode13\relax
      \catcode32=\the\catcode32\relax
      \catcode35=\the\catcode35\relax
      \catcode61=\the\catcode61\relax
      \catcode64=\the\catcode64\relax
      \catcode123=\the\catcode123\relax
      \catcode125=\the\catcode125\relax
    }%
  }%
\x\catcode61\catcode48\catcode32=10\relax%
\catcode13=5 % ^^M
\endlinechar=13 %
\catcode35=6 % #
\catcode64=11 % @
\catcode123=1 % {
\catcode125=2 % }
\def\TMP@EnsureCode#1#2{%
  \edef\lstU@AtEnd{%
    \lstU@AtEnd
    \catcode#1=\the\catcode#1\relax
  }%
  \catcode#1=#2\relax
}
\TMP@EnsureCode{10}{12}% ^^J
\TMP@EnsureCode{33}{12}% !
\TMP@EnsureCode{36}{3}% $
\TMP@EnsureCode{38}{4}% &
\TMP@EnsureCode{39}{12}% '
\TMP@EnsureCode{40}{12}% (
\TMP@EnsureCode{41}{12}% )
\TMP@EnsureCode{42}{12}% *
\TMP@EnsureCode{43}{12}% +
\TMP@EnsureCode{44}{12}% ,
\TMP@EnsureCode{45}{12}% -
\TMP@EnsureCode{46}{12}% .
\TMP@EnsureCode{47}{12}% /
\TMP@EnsureCode{58}{12}% :
\TMP@EnsureCode{60}{12}% <
\TMP@EnsureCode{62}{12}% >
\TMP@EnsureCode{91}{12}% [
\TMP@EnsureCode{93}{12}% ]
\TMP@EnsureCode{94}{7}% ^ (superscript)
\TMP@EnsureCode{95}{8}% _ (subscript)
\TMP@EnsureCode{96}{12}% `
\TMP@EnsureCode{124}{12}% |
\TMP@EnsureCode{126}{13}% ~ (active)
\edef\lstU@AtEnd{\lstU@AtEnd\noexpand\endinput}
%    \end{macrocode}
%
%    Package identification.
%    \begin{macrocode}
\NeedsTeXFormat{LaTeX2e}
\ProvidesPackage{listingsutf8}%
  [2011/11/10 v1.2 Allow UTF-8 in listings input (HO)]
%    \end{macrocode}
%
% \subsection{Package options}
%
% Just pass options to package listings.
%
%    \begin{macrocode}
\DeclareOption*{%
  \PassOptionsToPackage\CurrentOption{listings}%
}
\ProcessOptions*
%    \end{macrocode}
%    Key \xoption{inputencoding} was introduced in version
%    2002/04/01 v1.0 of package \xpackage{listings}.
%    \begin{macrocode}
\RequirePackage{listings}[2002/04/01]
%    \end{macrocode}
%    Ensure that \cs{inputencoding} is provided.
%    \begin{macrocode}
\AtBeginDocument{%
  \@ifundefined{inputencoding}{%
    \RequirePackage{inputenc}%
  }{}%
}
%    \end{macrocode}
%
% \subsection{Check prerequisites}
%
%    \begin{macrocode}
\RequirePackage{pdftexcmds}[2011/04/22]
%    \end{macrocode}
%
%    \begin{macrocode}
\def\lstU@temp#1#2{%
  \begingroup\expandafter\expandafter\expandafter\endgroup
  \expandafter\ifx\csname #1\endcsname\relax
    \PackageWarningNoLine{listingsutf8}{%
      Package loading is aborted because of missing %
      \@backslashchar#1.\MessageBreak
      #2%
    }%
    \expandafter\lstU@AtEnd
  \fi
}
\lstU@temp{scantokens}{It is provided by e-TeX}%
\lstU@temp{pdf@unescapehex}{It is provided by pdfTeX >= 1.30}%
\lstU@temp{pdf@filedump}{It is provided by pdfTeX >= 1.30}%
\lstU@temp{pdf@filesize}{It is provided by pdfTeX >= 1.30}%
%    \end{macrocode}
%
%    \begin{macrocode}
\RequirePackage{stringenc}[2010/03/01]
%    \end{macrocode}
%
% \subsection{Add support for UTF-8}
%
%    \begin{macro}{\iflstU@utfviii}
%    \begin{macrocode}
\newif\iflstU@utfviii
%    \end{macrocode}
%    \end{macro}
%
%    \begin{macro}{\lstU@inputenc}
%    \begin{macrocode}
\def\lstU@inputenc#1{%
  \expandafter\lstU@@inputenc#1utf8/utf8/\@nil
}
%    \end{macrocode}
%    \end{macro}
%    \begin{macro}{\lstU@@inputenc}
\def\lstU@@inputenc#1utf8/#2utf8/#3\@nil{%
  \ifx\\#1\\%
    \lstU@utfviiitrue
    \def\lst@inputenc{#2}%
  \else
    \lstU@utfviiifalse
  \fi
}
%    \begin{macrocode}
%    \end{macrocode}
%    \end{macro}
%
%    \begin{macrocode}
\lst@Key{inputencoding}\relax{%
  \def\lst@inputenc{#1}%
  \lstU@inputenc{#1}%
}
%    \end{macrocode}
%
% \subsubsection{Conversion}
%
%    \begin{macro}{\lstU@input}
%    \begin{macrocode}
\def\lstU@input#1{%
  \iflstU@utfviii
    \edef\lstU@text{%
      \pdf@unescapehex{%
        \pdf@filedump{0}{\pdf@filesize{#1}}{#1}%
      }%
    }%
    \lstU@CRLFtoLF\lstU@text
    \StringEncodingConvert\lstU@text\lstU@text{utf8}\lst@inputenc
    \def\lstU@temp{%
      \scantokens\expandafter{\lstU@text}%
    }%
  \else
    \def\lstU@temp{%
      \input{#1}%
    }%
  \fi
  \lstU@temp
}
%    \end{macrocode}
%    \end{macro}
%
% \subsubsection{Convert CR/LF pairs to LF}
%
%    \begin{macro}{\lstU@CRLFtoLF}
%    \begin{macrocode}
\begingroup
  \endlinechar=-1 %
  \@makeother\^^J %
  \@makeother\^^M %
  \gdef\lstU@CRLFtoLF#1{%
    \edef#1{%
      \expandafter\lstU@CRLFtoLF@aux#1^^M^^J\@nil
    }%
  }%
  \gdef\lstU@CRLFtoLF@aux#1^^M^^J#2\@nil{%
    #1%
    \ifx\relax#2\relax
      \@car
    \fi
    ^^J%
    \lstU@CRLFtoLF@aux#2\@nil
  }%
\endgroup %
%    \end{macrocode}
%    \end{macro}
%
% \subsubsection{Patch \cs{lst@InputListing}}
%
%    \begin{macrocode}
\def\lstU@temp#1\def\lst@next#2#3\@nil{%
  \def\lst@InputListing##1{%
    #1%
    \def\lst@next{\lstU@input{##1}}%
    #3%
  }%
}
\expandafter\lstU@temp\lst@InputListing{#1}\@nil
%    \end{macrocode}
%
%    \begin{macrocode}
\lstU@AtEnd%
%</package>
%    \end{macrocode}
%
% \section{Test}
%
% \subsection{Catcode checks for loading}
%
%    \begin{macrocode}
%<*test1>
%    \end{macrocode}
%    \begin{macrocode}
\NeedsTeXFormat{LaTeX2e}
\documentclass{minimal}
\makeatletter
\def\RestoreCatcodes{}
\count@=0 %
\loop
  \edef\RestoreCatcodes{%
    \RestoreCatcodes
    \catcode\the\count@=\the\catcode\count@\relax
  }%
\ifnum\count@<255 %
  \advance\count@\@ne
\repeat

\def\RangeCatcodeInvalid#1#2{%
  \count@=#1\relax
  \loop
    \catcode\count@=15 %
  \ifnum\count@<#2\relax
    \advance\count@\@ne
  \repeat
}
\def\Test{%
  \RangeCatcodeInvalid{0}{47}%
  \RangeCatcodeInvalid{58}{64}%
  \RangeCatcodeInvalid{91}{96}%
  \RangeCatcodeInvalid{123}{127}%
  \catcode`\@=12 %
  \catcode`\\=0 %
  \catcode`\{=1 %
  \catcode`\}=2 %
  \catcode`\#=6 %
  \catcode`\[=12 %
  \catcode`\]=12 %
  \catcode`\%=14 %
  \catcode`\ =10 %
  \catcode13=5 %
  \RequirePackage{listingsutf8}[2011/11/10]\relax
  \RestoreCatcodes
}
\Test
\csname @@end\endcsname
\end
%    \end{macrocode}
%    \begin{macrocode}
%</test1>
%    \end{macrocode}
%
% \subsection{Test example for latin1}
%
%    \begin{macrocode}
%<*test2>
%    \end{macrocode}
%    \begin{macrocode}
\NeedsTeXFormat{LaTeX2e}
\documentclass{minimal}
\usepackage{filecontents}
\def\do#1{%
  \ifx#1\^%
  \else
    \noexpand\do\noexpand#1%
  \fi
}
\expandafter\let\expandafter\dospecials\expandafter\empty
\expandafter\edef\expandafter\dospecials\expandafter{\dospecials}
\begin{filecontents*}{ExampleUTF8.java}
public class ExampleUTF8 {
    public static String testString =
        "Umlauts: " +
        "^^c3^^84^^c3^^96^^c3^^9c^^c3^^a4^^c3^^b6^^c3^^bc^^c3^^9f";
    public static void main(String[] args) {
        System.out.println(testString);
    }
}
\end{filecontents*}
\usepackage{listingsutf8}[2011/11/10]
\def\Text{%
  Umlauts: %
  ^^c3^^84^^c3^^96^^c3^^9c^^c3^^a4^^c3^^b6^^c3^^bc^^c3^^9f%
}
\begin{document}
\lstinputlisting[%
  language=Java,%
  inputencoding=utf8/latin1,%
]{ExampleUTF8.java}
\end{document}
%</test2>
%    \end{macrocode}
%
% \section{Installation}
%
% \subsection{Download}
%
% \paragraph{Package.} This package is available on
% CTAN\footnote{\url{ftp://ftp.ctan.org/tex-archive/}}:
% \begin{description}
% \item[\CTAN{macros/latex/contrib/oberdiek/listingsutf8.dtx}] The source file.
% \item[\CTAN{macros/latex/contrib/oberdiek/listingsutf8.pdf}] Documentation.
% \end{description}
%
%
% \paragraph{Bundle.} All the packages of the bundle `oberdiek'
% are also available in a TDS compliant ZIP archive. There
% the packages are already unpacked and the documentation files
% are generated. The files and directories obey the TDS standard.
% \begin{description}
% \item[\CTAN{install/macros/latex/contrib/oberdiek.tds.zip}]
% \end{description}
% \emph{TDS} refers to the standard ``A Directory Structure
% for \TeX\ Files'' (\CTAN{tds/tds.pdf}). Directories
% with \xfile{texmf} in their name are usually organized this way.
%
% \subsection{Bundle installation}
%
% \paragraph{Unpacking.} Unpack the \xfile{oberdiek.tds.zip} in the
% TDS tree (also known as \xfile{texmf} tree) of your choice.
% Example (linux):
% \begin{quote}
%   |unzip oberdiek.tds.zip -d ~/texmf|
% \end{quote}
%
% \paragraph{Script installation.}
% Check the directory \xfile{TDS:scripts/oberdiek/} for
% scripts that need further installation steps.
% Package \xpackage{attachfile2} comes with the Perl script
% \xfile{pdfatfi.pl} that should be installed in such a way
% that it can be called as \texttt{pdfatfi}.
% Example (linux):
% \begin{quote}
%   |chmod +x scripts/oberdiek/pdfatfi.pl|\\
%   |cp scripts/oberdiek/pdfatfi.pl /usr/local/bin/|
% \end{quote}
%
% \subsection{Package installation}
%
% \paragraph{Unpacking.} The \xfile{.dtx} file is a self-extracting
% \docstrip\ archive. The files are extracted by running the
% \xfile{.dtx} through \plainTeX:
% \begin{quote}
%   \verb|tex listingsutf8.dtx|
% \end{quote}
%
% \paragraph{TDS.} Now the different files must be moved into
% the different directories in your installation TDS tree
% (also known as \xfile{texmf} tree):
% \begin{quote}
% \def\t{^^A
% \begin{tabular}{@{}>{\ttfamily}l@{ $\rightarrow$ }>{\ttfamily}l@{}}
%   listingsutf8.sty & tex/latex/oberdiek/listingsutf8.sty\\
%   listingsutf8.pdf & doc/latex/oberdiek/listingsutf8.pdf\\
%   test/listingsutf8-test1.tex & doc/latex/oberdiek/test/listingsutf8-test1.tex\\
%   test/listingsutf8-test2.tex & doc/latex/oberdiek/test/listingsutf8-test2.tex\\
%   test/listingsutf8-test3.tex & doc/latex/oberdiek/test/listingsutf8-test3.tex\\
%   test/listingsutf8-test4.tex & doc/latex/oberdiek/test/listingsutf8-test4.tex\\
%   test/listingsutf8-test5.tex & doc/latex/oberdiek/test/listingsutf8-test5.tex\\
%   listingsutf8.dtx & source/latex/oberdiek/listingsutf8.dtx\\
% \end{tabular}^^A
% }^^A
% \sbox0{\t}^^A
% \ifdim\wd0>\linewidth
%   \begingroup
%     \advance\linewidth by\leftmargin
%     \advance\linewidth by\rightmargin
%   \edef\x{\endgroup
%     \def\noexpand\lw{\the\linewidth}^^A
%   }\x
%   \def\lwbox{^^A
%     \leavevmode
%     \hbox to \linewidth{^^A
%       \kern-\leftmargin\relax
%       \hss
%       \usebox0
%       \hss
%       \kern-\rightmargin\relax
%     }^^A
%   }^^A
%   \ifdim\wd0>\lw
%     \sbox0{\small\t}^^A
%     \ifdim\wd0>\linewidth
%       \ifdim\wd0>\lw
%         \sbox0{\footnotesize\t}^^A
%         \ifdim\wd0>\linewidth
%           \ifdim\wd0>\lw
%             \sbox0{\scriptsize\t}^^A
%             \ifdim\wd0>\linewidth
%               \ifdim\wd0>\lw
%                 \sbox0{\tiny\t}^^A
%                 \ifdim\wd0>\linewidth
%                   \lwbox
%                 \else
%                   \usebox0
%                 \fi
%               \else
%                 \lwbox
%               \fi
%             \else
%               \usebox0
%             \fi
%           \else
%             \lwbox
%           \fi
%         \else
%           \usebox0
%         \fi
%       \else
%         \lwbox
%       \fi
%     \else
%       \usebox0
%     \fi
%   \else
%     \lwbox
%   \fi
% \else
%   \usebox0
% \fi
% \end{quote}
% If you have a \xfile{docstrip.cfg} that configures and enables \docstrip's
% TDS installing feature, then some files can already be in the right
% place, see the documentation of \docstrip.
%
% \subsection{Refresh file name databases}
%
% If your \TeX~distribution
% (\teTeX, \mikTeX, \dots) relies on file name databases, you must refresh
% these. For example, \teTeX\ users run \verb|texhash| or
% \verb|mktexlsr|.
%
% \subsection{Some details for the interested}
%
% \paragraph{Attached source.}
%
% The PDF documentation on CTAN also includes the
% \xfile{.dtx} source file. It can be extracted by
% AcrobatReader 6 or higher. Another option is \textsf{pdftk},
% e.g. unpack the file into the current directory:
% \begin{quote}
%   \verb|pdftk listingsutf8.pdf unpack_files output .|
% \end{quote}
%
% \paragraph{Unpacking with \LaTeX.}
% The \xfile{.dtx} chooses its action depending on the format:
% \begin{description}
% \item[\plainTeX:] Run \docstrip\ and extract the files.
% \item[\LaTeX:] Generate the documentation.
% \end{description}
% If you insist on using \LaTeX\ for \docstrip\ (really,
% \docstrip\ does not need \LaTeX), then inform the autodetect routine
% about your intention:
% \begin{quote}
%   \verb|latex \let\install=y% \iffalse meta-comment
%
% File: listingsutf8.dtx
% Version: 2011/11/10 v1.2
% Info: Allow UTF-8 in listings input
%
% Copyright (C) 2007, 2011 by
%    Heiko Oberdiek <heiko.oberdiek at googlemail.com>
%
% This work may be distributed and/or modified under the
% conditions of the LaTeX Project Public License, either
% version 1.3c of this license or (at your option) any later
% version. This version of this license is in
%    http://www.latex-project.org/lppl/lppl-1-3c.txt
% and the latest version of this license is in
%    http://www.latex-project.org/lppl.txt
% and version 1.3 or later is part of all distributions of
% LaTeX version 2005/12/01 or later.
%
% This work has the LPPL maintenance status "maintained".
%
% This Current Maintainer of this work is Heiko Oberdiek.
%
% This work consists of the main source file listingsutf8.dtx
% and the derived files
%    listingsutf8.sty, listingsutf8.pdf, listingsutf8.ins, listingsutf8.drv,
%    listingsutf8-test1.tex, listingsutf8-test2.tex,
%    listingsutf8-test3.tex, listingsutf8-test4.tex,
%    listingsutf8-test5.tex.
%
% Distribution:
%    CTAN:macros/latex/contrib/oberdiek/listingsutf8.dtx
%    CTAN:macros/latex/contrib/oberdiek/listingsutf8.pdf
%
% Unpacking:
%    (a) If listingsutf8.ins is present:
%           tex listingsutf8.ins
%    (b) Without listingsutf8.ins:
%           tex listingsutf8.dtx
%    (c) If you insist on using LaTeX
%           latex \let\install=y% \iffalse meta-comment
%
% File: listingsutf8.dtx
% Version: 2011/11/10 v1.2
% Info: Allow UTF-8 in listings input
%
% Copyright (C) 2007, 2011 by
%    Heiko Oberdiek <heiko.oberdiek at googlemail.com>
%
% This work may be distributed and/or modified under the
% conditions of the LaTeX Project Public License, either
% version 1.3c of this license or (at your option) any later
% version. This version of this license is in
%    http://www.latex-project.org/lppl/lppl-1-3c.txt
% and the latest version of this license is in
%    http://www.latex-project.org/lppl.txt
% and version 1.3 or later is part of all distributions of
% LaTeX version 2005/12/01 or later.
%
% This work has the LPPL maintenance status "maintained".
%
% This Current Maintainer of this work is Heiko Oberdiek.
%
% This work consists of the main source file listingsutf8.dtx
% and the derived files
%    listingsutf8.sty, listingsutf8.pdf, listingsutf8.ins, listingsutf8.drv,
%    listingsutf8-test1.tex, listingsutf8-test2.tex,
%    listingsutf8-test3.tex, listingsutf8-test4.tex,
%    listingsutf8-test5.tex.
%
% Distribution:
%    CTAN:macros/latex/contrib/oberdiek/listingsutf8.dtx
%    CTAN:macros/latex/contrib/oberdiek/listingsutf8.pdf
%
% Unpacking:
%    (a) If listingsutf8.ins is present:
%           tex listingsutf8.ins
%    (b) Without listingsutf8.ins:
%           tex listingsutf8.dtx
%    (c) If you insist on using LaTeX
%           latex \let\install=y\input{listingsutf8.dtx}
%        (quote the arguments according to the demands of your shell)
%
% Documentation:
%    (a) If listingsutf8.drv is present:
%           latex listingsutf8.drv
%    (b) Without listingsutf8.drv:
%           latex listingsutf8.dtx; ...
%    The class ltxdoc loads the configuration file ltxdoc.cfg
%    if available. Here you can specify further options, e.g.
%    use A4 as paper format:
%       \PassOptionsToClass{a4paper}{article}
%
%    Programm calls to get the documentation (example):
%       pdflatex listingsutf8.dtx
%       makeindex -s gind.ist listingsutf8.idx
%       pdflatex listingsutf8.dtx
%       makeindex -s gind.ist listingsutf8.idx
%       pdflatex listingsutf8.dtx
%
% Installation:
%    TDS:tex/latex/oberdiek/listingsutf8.sty
%    TDS:doc/latex/oberdiek/listingsutf8.pdf
%    TDS:doc/latex/oberdiek/test/listingsutf8-test1.tex
%    TDS:doc/latex/oberdiek/test/listingsutf8-test2.tex
%    TDS:doc/latex/oberdiek/test/listingsutf8-test3.tex
%    TDS:doc/latex/oberdiek/test/listingsutf8-test4.tex
%    TDS:doc/latex/oberdiek/test/listingsutf8-test5.tex
%    TDS:source/latex/oberdiek/listingsutf8.dtx
%
%<*ignore>
\begingroup
  \catcode123=1 %
  \catcode125=2 %
  \def\x{LaTeX2e}%
\expandafter\endgroup
\ifcase 0\ifx\install y1\fi\expandafter
         \ifx\csname processbatchFile\endcsname\relax\else1\fi
         \ifx\fmtname\x\else 1\fi\relax
\else\csname fi\endcsname
%</ignore>
%<*install>
\input docstrip.tex
\Msg{************************************************************************}
\Msg{* Installation}
\Msg{* Package: listingsutf8 2011/11/10 v1.2 Allow UTF-8 in listings input (HO)}
\Msg{************************************************************************}

\keepsilent
\askforoverwritefalse

\let\MetaPrefix\relax
\preamble

This is a generated file.

Project: listingsutf8
Version: 2011/11/10 v1.2

Copyright (C) 2007, 2011 by
   Heiko Oberdiek <heiko.oberdiek at googlemail.com>

This work may be distributed and/or modified under the
conditions of the LaTeX Project Public License, either
version 1.3c of this license or (at your option) any later
version. This version of this license is in
   http://www.latex-project.org/lppl/lppl-1-3c.txt
and the latest version of this license is in
   http://www.latex-project.org/lppl.txt
and version 1.3 or later is part of all distributions of
LaTeX version 2005/12/01 or later.

This work has the LPPL maintenance status "maintained".

This Current Maintainer of this work is Heiko Oberdiek.

This work consists of the main source file listingsutf8.dtx
and the derived files
   listingsutf8.sty, listingsutf8.pdf, listingsutf8.ins, listingsutf8.drv,
   listingsutf8-test1.tex, listingsutf8-test2.tex,
   listingsutf8-test3.tex, listingsutf8-test4.tex,
   listingsutf8-test5.tex.

\endpreamble
\let\MetaPrefix\DoubleperCent

\generate{%
  \file{listingsutf8.ins}{\from{listingsutf8.dtx}{install}}%
  \file{listingsutf8.drv}{\from{listingsutf8.dtx}{driver}}%
  \usedir{tex/latex/oberdiek}%
  \file{listingsutf8.sty}{\from{listingsutf8.dtx}{package}}%
  \usedir{doc/latex/oberdiek/test}%
  \file{listingsutf8-test1.tex}{\from{listingsutf8.dtx}{test1}}%
  \file{listingsutf8-test2.tex}{\from{listingsutf8.dtx}{test2,utf8}}%
  \file{listingsutf8-test3.tex}{\from{listingsutf8.dtx}{test3,utf8x}}%
  \file{listingsutf8-test4.tex}{\from{listingsutf8.dtx}{test4,utf8,noetex}}%
  \file{listingsutf8-test5.tex}{\from{listingsutf8.dtx}{test5,utf8x,noetex}}%
  \nopreamble
  \nopostamble
  \usedir{source/latex/oberdiek/catalogue}%
  \file{listingsutf8.xml}{\from{listingsutf8.dtx}{catalogue}}%
}

\catcode32=13\relax% active space
\let =\space%
\Msg{************************************************************************}
\Msg{*}
\Msg{* To finish the installation you have to move the following}
\Msg{* file into a directory searched by TeX:}
\Msg{*}
\Msg{*     listingsutf8.sty}
\Msg{*}
\Msg{* To produce the documentation run the file `listingsutf8.drv'}
\Msg{* through LaTeX.}
\Msg{*}
\Msg{* Happy TeXing!}
\Msg{*}
\Msg{************************************************************************}

\endbatchfile
%</install>
%<*ignore>
\fi
%</ignore>
%<*driver>
\NeedsTeXFormat{LaTeX2e}
\ProvidesFile{listingsutf8.drv}%
  [2011/11/10 v1.2 Allow UTF-8 in listings input (HO)]%
\documentclass{ltxdoc}
\usepackage{holtxdoc}[2011/11/22]
\begin{document}
  \DocInput{listingsutf8.dtx}%
\end{document}
%</driver>
% \fi
%
% \CheckSum{311}
%
% \CharacterTable
%  {Upper-case    \A\B\C\D\E\F\G\H\I\J\K\L\M\N\O\P\Q\R\S\T\U\V\W\X\Y\Z
%   Lower-case    \a\b\c\d\e\f\g\h\i\j\k\l\m\n\o\p\q\r\s\t\u\v\w\x\y\z
%   Digits        \0\1\2\3\4\5\6\7\8\9
%   Exclamation   \!     Double quote  \"     Hash (number) \#
%   Dollar        \$     Percent       \%     Ampersand     \&
%   Acute accent  \'     Left paren    \(     Right paren   \)
%   Asterisk      \*     Plus          \+     Comma         \,
%   Minus         \-     Point         \.     Solidus       \/
%   Colon         \:     Semicolon     \;     Less than     \<
%   Equals        \=     Greater than  \>     Question mark \?
%   Commercial at \@     Left bracket  \[     Backslash     \\
%   Right bracket \]     Circumflex    \^     Underscore    \_
%   Grave accent  \`     Left brace    \{     Vertical bar  \|
%   Right brace   \}     Tilde         \~}
%
% \GetFileInfo{listingsutf8.drv}
%
% \title{The \xpackage{listingsutf8} package}
% \date{2011/11/10 v1.2}
% \author{Heiko Oberdiek\\\xemail{heiko.oberdiek at googlemail.com}}
%
% \maketitle
%
% \begin{abstract}
% Package \xpackage{listings} does not support files with multi-byte
% encodings such as UTF-8. In case of \cs{lstinputlisting} a simple
% workaround is possible if an one-byte encoding exists that the file
% can be converted to. Also \eTeX\ and \pdfTeX\ regardless of its mode
% are required.
% \end{abstract}
%
% \tableofcontents
%
% \section{Documentation}
%
% \subsection{User interface}
%
% Load this package after or instead of package \xpackage{listings}
% \cite{listings}.
% The package does not define own options and passes given options to
% package \xpackage{listings}.
%
% The syntax of package \xpackage{listings}' key \xoption{inputencoding}
% is extended:
% \begin{quote}
%   |inputencoding=utf8/|\meta{one-byte-encoding}\\
%   Example: |inputencoding=utf8/latin1|
% \end{quote}
% That means the file is encoded in UTF-8 and can
% be converted to the given \meta{one-byte-encoding}.
% The available encodings for \meta{one-byte-encoding} are
% listed in section ``1.2 Supported encodings'' of
% package \xpackage{stringenc}'s documentation \cite{stringenc}.
% Of course, the encoding must encode its characters with
% one byte exactly. This excludes the unicode encodings
% (\xoption{utf8}, \xoption{utf16}, \dots).
%
% Only \cs{lstinputlisting} is supported by the syntax extension
% of key \xoption{inputencoding}.
%
% Internally package \xpackage{listingsutf8} reads the file as binary file
% via primitives of \pdfTeX\ (\cs{pdffiledump}). Then the file
% contents is converted as string using package \xpackage{stringenc} and
% finally the string is read as virtual file by \eTeX's \cs{scantokens}.
%
% \subsection{Future}
%
% Workarounds are not provided for
% \begin{itemize}
% \item \cs{lstinline}
% \item Environment |lstlisting|.
% \item Environments defined by \cs{lstnewenvironment}.
% \end{itemize}
% Perhaps someone will find time to extend package \xpackage{listings}
% with full native support for UTF-8. Then this package would become obsolete.
%
% \StopEventually{
% }
%
% \section{Implementation}
%
%    \begin{macrocode}
%<*package>
%    \end{macrocode}
%
% \subsection{Catcodes and identification}
%
%    \begin{macrocode}
\begingroup\catcode61\catcode48\catcode32=10\relax%
  \catcode13=5 % ^^M
  \endlinechar=13 %
  \catcode123=1 % {
  \catcode125=2 % }
  \catcode64=11 % @
  \def\x{\endgroup
    \expandafter\edef\csname lstU@AtEnd\endcsname{%
      \endlinechar=\the\endlinechar\relax
      \catcode13=\the\catcode13\relax
      \catcode32=\the\catcode32\relax
      \catcode35=\the\catcode35\relax
      \catcode61=\the\catcode61\relax
      \catcode64=\the\catcode64\relax
      \catcode123=\the\catcode123\relax
      \catcode125=\the\catcode125\relax
    }%
  }%
\x\catcode61\catcode48\catcode32=10\relax%
\catcode13=5 % ^^M
\endlinechar=13 %
\catcode35=6 % #
\catcode64=11 % @
\catcode123=1 % {
\catcode125=2 % }
\def\TMP@EnsureCode#1#2{%
  \edef\lstU@AtEnd{%
    \lstU@AtEnd
    \catcode#1=\the\catcode#1\relax
  }%
  \catcode#1=#2\relax
}
\TMP@EnsureCode{10}{12}% ^^J
\TMP@EnsureCode{33}{12}% !
\TMP@EnsureCode{36}{3}% $
\TMP@EnsureCode{38}{4}% &
\TMP@EnsureCode{39}{12}% '
\TMP@EnsureCode{40}{12}% (
\TMP@EnsureCode{41}{12}% )
\TMP@EnsureCode{42}{12}% *
\TMP@EnsureCode{43}{12}% +
\TMP@EnsureCode{44}{12}% ,
\TMP@EnsureCode{45}{12}% -
\TMP@EnsureCode{46}{12}% .
\TMP@EnsureCode{47}{12}% /
\TMP@EnsureCode{58}{12}% :
\TMP@EnsureCode{60}{12}% <
\TMP@EnsureCode{62}{12}% >
\TMP@EnsureCode{91}{12}% [
\TMP@EnsureCode{93}{12}% ]
\TMP@EnsureCode{94}{7}% ^ (superscript)
\TMP@EnsureCode{95}{8}% _ (subscript)
\TMP@EnsureCode{96}{12}% `
\TMP@EnsureCode{124}{12}% |
\TMP@EnsureCode{126}{13}% ~ (active)
\edef\lstU@AtEnd{\lstU@AtEnd\noexpand\endinput}
%    \end{macrocode}
%
%    Package identification.
%    \begin{macrocode}
\NeedsTeXFormat{LaTeX2e}
\ProvidesPackage{listingsutf8}%
  [2011/11/10 v1.2 Allow UTF-8 in listings input (HO)]
%    \end{macrocode}
%
% \subsection{Package options}
%
% Just pass options to package listings.
%
%    \begin{macrocode}
\DeclareOption*{%
  \PassOptionsToPackage\CurrentOption{listings}%
}
\ProcessOptions*
%    \end{macrocode}
%    Key \xoption{inputencoding} was introduced in version
%    2002/04/01 v1.0 of package \xpackage{listings}.
%    \begin{macrocode}
\RequirePackage{listings}[2002/04/01]
%    \end{macrocode}
%    Ensure that \cs{inputencoding} is provided.
%    \begin{macrocode}
\AtBeginDocument{%
  \@ifundefined{inputencoding}{%
    \RequirePackage{inputenc}%
  }{}%
}
%    \end{macrocode}
%
% \subsection{Check prerequisites}
%
%    \begin{macrocode}
\RequirePackage{pdftexcmds}[2011/04/22]
%    \end{macrocode}
%
%    \begin{macrocode}
\def\lstU@temp#1#2{%
  \begingroup\expandafter\expandafter\expandafter\endgroup
  \expandafter\ifx\csname #1\endcsname\relax
    \PackageWarningNoLine{listingsutf8}{%
      Package loading is aborted because of missing %
      \@backslashchar#1.\MessageBreak
      #2%
    }%
    \expandafter\lstU@AtEnd
  \fi
}
\lstU@temp{scantokens}{It is provided by e-TeX}%
\lstU@temp{pdf@unescapehex}{It is provided by pdfTeX >= 1.30}%
\lstU@temp{pdf@filedump}{It is provided by pdfTeX >= 1.30}%
\lstU@temp{pdf@filesize}{It is provided by pdfTeX >= 1.30}%
%    \end{macrocode}
%
%    \begin{macrocode}
\RequirePackage{stringenc}[2010/03/01]
%    \end{macrocode}
%
% \subsection{Add support for UTF-8}
%
%    \begin{macro}{\iflstU@utfviii}
%    \begin{macrocode}
\newif\iflstU@utfviii
%    \end{macrocode}
%    \end{macro}
%
%    \begin{macro}{\lstU@inputenc}
%    \begin{macrocode}
\def\lstU@inputenc#1{%
  \expandafter\lstU@@inputenc#1utf8/utf8/\@nil
}
%    \end{macrocode}
%    \end{macro}
%    \begin{macro}{\lstU@@inputenc}
\def\lstU@@inputenc#1utf8/#2utf8/#3\@nil{%
  \ifx\\#1\\%
    \lstU@utfviiitrue
    \def\lst@inputenc{#2}%
  \else
    \lstU@utfviiifalse
  \fi
}
%    \begin{macrocode}
%    \end{macrocode}
%    \end{macro}
%
%    \begin{macrocode}
\lst@Key{inputencoding}\relax{%
  \def\lst@inputenc{#1}%
  \lstU@inputenc{#1}%
}
%    \end{macrocode}
%
% \subsubsection{Conversion}
%
%    \begin{macro}{\lstU@input}
%    \begin{macrocode}
\def\lstU@input#1{%
  \iflstU@utfviii
    \edef\lstU@text{%
      \pdf@unescapehex{%
        \pdf@filedump{0}{\pdf@filesize{#1}}{#1}%
      }%
    }%
    \lstU@CRLFtoLF\lstU@text
    \StringEncodingConvert\lstU@text\lstU@text{utf8}\lst@inputenc
    \def\lstU@temp{%
      \scantokens\expandafter{\lstU@text}%
    }%
  \else
    \def\lstU@temp{%
      \input{#1}%
    }%
  \fi
  \lstU@temp
}
%    \end{macrocode}
%    \end{macro}
%
% \subsubsection{Convert CR/LF pairs to LF}
%
%    \begin{macro}{\lstU@CRLFtoLF}
%    \begin{macrocode}
\begingroup
  \endlinechar=-1 %
  \@makeother\^^J %
  \@makeother\^^M %
  \gdef\lstU@CRLFtoLF#1{%
    \edef#1{%
      \expandafter\lstU@CRLFtoLF@aux#1^^M^^J\@nil
    }%
  }%
  \gdef\lstU@CRLFtoLF@aux#1^^M^^J#2\@nil{%
    #1%
    \ifx\relax#2\relax
      \@car
    \fi
    ^^J%
    \lstU@CRLFtoLF@aux#2\@nil
  }%
\endgroup %
%    \end{macrocode}
%    \end{macro}
%
% \subsubsection{Patch \cs{lst@InputListing}}
%
%    \begin{macrocode}
\def\lstU@temp#1\def\lst@next#2#3\@nil{%
  \def\lst@InputListing##1{%
    #1%
    \def\lst@next{\lstU@input{##1}}%
    #3%
  }%
}
\expandafter\lstU@temp\lst@InputListing{#1}\@nil
%    \end{macrocode}
%
%    \begin{macrocode}
\lstU@AtEnd%
%</package>
%    \end{macrocode}
%
% \section{Test}
%
% \subsection{Catcode checks for loading}
%
%    \begin{macrocode}
%<*test1>
%    \end{macrocode}
%    \begin{macrocode}
\NeedsTeXFormat{LaTeX2e}
\documentclass{minimal}
\makeatletter
\def\RestoreCatcodes{}
\count@=0 %
\loop
  \edef\RestoreCatcodes{%
    \RestoreCatcodes
    \catcode\the\count@=\the\catcode\count@\relax
  }%
\ifnum\count@<255 %
  \advance\count@\@ne
\repeat

\def\RangeCatcodeInvalid#1#2{%
  \count@=#1\relax
  \loop
    \catcode\count@=15 %
  \ifnum\count@<#2\relax
    \advance\count@\@ne
  \repeat
}
\def\Test{%
  \RangeCatcodeInvalid{0}{47}%
  \RangeCatcodeInvalid{58}{64}%
  \RangeCatcodeInvalid{91}{96}%
  \RangeCatcodeInvalid{123}{127}%
  \catcode`\@=12 %
  \catcode`\\=0 %
  \catcode`\{=1 %
  \catcode`\}=2 %
  \catcode`\#=6 %
  \catcode`\[=12 %
  \catcode`\]=12 %
  \catcode`\%=14 %
  \catcode`\ =10 %
  \catcode13=5 %
  \RequirePackage{listingsutf8}[2011/11/10]\relax
  \RestoreCatcodes
}
\Test
\csname @@end\endcsname
\end
%    \end{macrocode}
%    \begin{macrocode}
%</test1>
%    \end{macrocode}
%
% \subsection{Test example for latin1}
%
%    \begin{macrocode}
%<*test2>
%    \end{macrocode}
%    \begin{macrocode}
\NeedsTeXFormat{LaTeX2e}
\documentclass{minimal}
\usepackage{filecontents}
\def\do#1{%
  \ifx#1\^%
  \else
    \noexpand\do\noexpand#1%
  \fi
}
\expandafter\let\expandafter\dospecials\expandafter\empty
\expandafter\edef\expandafter\dospecials\expandafter{\dospecials}
\begin{filecontents*}{ExampleUTF8.java}
public class ExampleUTF8 {
    public static String testString =
        "Umlauts: " +
        "^^c3^^84^^c3^^96^^c3^^9c^^c3^^a4^^c3^^b6^^c3^^bc^^c3^^9f";
    public static void main(String[] args) {
        System.out.println(testString);
    }
}
\end{filecontents*}
\usepackage{listingsutf8}[2011/11/10]
\def\Text{%
  Umlauts: %
  ^^c3^^84^^c3^^96^^c3^^9c^^c3^^a4^^c3^^b6^^c3^^bc^^c3^^9f%
}
\begin{document}
\lstinputlisting[%
  language=Java,%
  inputencoding=utf8/latin1,%
]{ExampleUTF8.java}
\end{document}
%</test2>
%    \end{macrocode}
%
% \section{Installation}
%
% \subsection{Download}
%
% \paragraph{Package.} This package is available on
% CTAN\footnote{\url{ftp://ftp.ctan.org/tex-archive/}}:
% \begin{description}
% \item[\CTAN{macros/latex/contrib/oberdiek/listingsutf8.dtx}] The source file.
% \item[\CTAN{macros/latex/contrib/oberdiek/listingsutf8.pdf}] Documentation.
% \end{description}
%
%
% \paragraph{Bundle.} All the packages of the bundle `oberdiek'
% are also available in a TDS compliant ZIP archive. There
% the packages are already unpacked and the documentation files
% are generated. The files and directories obey the TDS standard.
% \begin{description}
% \item[\CTAN{install/macros/latex/contrib/oberdiek.tds.zip}]
% \end{description}
% \emph{TDS} refers to the standard ``A Directory Structure
% for \TeX\ Files'' (\CTAN{tds/tds.pdf}). Directories
% with \xfile{texmf} in their name are usually organized this way.
%
% \subsection{Bundle installation}
%
% \paragraph{Unpacking.} Unpack the \xfile{oberdiek.tds.zip} in the
% TDS tree (also known as \xfile{texmf} tree) of your choice.
% Example (linux):
% \begin{quote}
%   |unzip oberdiek.tds.zip -d ~/texmf|
% \end{quote}
%
% \paragraph{Script installation.}
% Check the directory \xfile{TDS:scripts/oberdiek/} for
% scripts that need further installation steps.
% Package \xpackage{attachfile2} comes with the Perl script
% \xfile{pdfatfi.pl} that should be installed in such a way
% that it can be called as \texttt{pdfatfi}.
% Example (linux):
% \begin{quote}
%   |chmod +x scripts/oberdiek/pdfatfi.pl|\\
%   |cp scripts/oberdiek/pdfatfi.pl /usr/local/bin/|
% \end{quote}
%
% \subsection{Package installation}
%
% \paragraph{Unpacking.} The \xfile{.dtx} file is a self-extracting
% \docstrip\ archive. The files are extracted by running the
% \xfile{.dtx} through \plainTeX:
% \begin{quote}
%   \verb|tex listingsutf8.dtx|
% \end{quote}
%
% \paragraph{TDS.} Now the different files must be moved into
% the different directories in your installation TDS tree
% (also known as \xfile{texmf} tree):
% \begin{quote}
% \def\t{^^A
% \begin{tabular}{@{}>{\ttfamily}l@{ $\rightarrow$ }>{\ttfamily}l@{}}
%   listingsutf8.sty & tex/latex/oberdiek/listingsutf8.sty\\
%   listingsutf8.pdf & doc/latex/oberdiek/listingsutf8.pdf\\
%   test/listingsutf8-test1.tex & doc/latex/oberdiek/test/listingsutf8-test1.tex\\
%   test/listingsutf8-test2.tex & doc/latex/oberdiek/test/listingsutf8-test2.tex\\
%   test/listingsutf8-test3.tex & doc/latex/oberdiek/test/listingsutf8-test3.tex\\
%   test/listingsutf8-test4.tex & doc/latex/oberdiek/test/listingsutf8-test4.tex\\
%   test/listingsutf8-test5.tex & doc/latex/oberdiek/test/listingsutf8-test5.tex\\
%   listingsutf8.dtx & source/latex/oberdiek/listingsutf8.dtx\\
% \end{tabular}^^A
% }^^A
% \sbox0{\t}^^A
% \ifdim\wd0>\linewidth
%   \begingroup
%     \advance\linewidth by\leftmargin
%     \advance\linewidth by\rightmargin
%   \edef\x{\endgroup
%     \def\noexpand\lw{\the\linewidth}^^A
%   }\x
%   \def\lwbox{^^A
%     \leavevmode
%     \hbox to \linewidth{^^A
%       \kern-\leftmargin\relax
%       \hss
%       \usebox0
%       \hss
%       \kern-\rightmargin\relax
%     }^^A
%   }^^A
%   \ifdim\wd0>\lw
%     \sbox0{\small\t}^^A
%     \ifdim\wd0>\linewidth
%       \ifdim\wd0>\lw
%         \sbox0{\footnotesize\t}^^A
%         \ifdim\wd0>\linewidth
%           \ifdim\wd0>\lw
%             \sbox0{\scriptsize\t}^^A
%             \ifdim\wd0>\linewidth
%               \ifdim\wd0>\lw
%                 \sbox0{\tiny\t}^^A
%                 \ifdim\wd0>\linewidth
%                   \lwbox
%                 \else
%                   \usebox0
%                 \fi
%               \else
%                 \lwbox
%               \fi
%             \else
%               \usebox0
%             \fi
%           \else
%             \lwbox
%           \fi
%         \else
%           \usebox0
%         \fi
%       \else
%         \lwbox
%       \fi
%     \else
%       \usebox0
%     \fi
%   \else
%     \lwbox
%   \fi
% \else
%   \usebox0
% \fi
% \end{quote}
% If you have a \xfile{docstrip.cfg} that configures and enables \docstrip's
% TDS installing feature, then some files can already be in the right
% place, see the documentation of \docstrip.
%
% \subsection{Refresh file name databases}
%
% If your \TeX~distribution
% (\teTeX, \mikTeX, \dots) relies on file name databases, you must refresh
% these. For example, \teTeX\ users run \verb|texhash| or
% \verb|mktexlsr|.
%
% \subsection{Some details for the interested}
%
% \paragraph{Attached source.}
%
% The PDF documentation on CTAN also includes the
% \xfile{.dtx} source file. It can be extracted by
% AcrobatReader 6 or higher. Another option is \textsf{pdftk},
% e.g. unpack the file into the current directory:
% \begin{quote}
%   \verb|pdftk listingsutf8.pdf unpack_files output .|
% \end{quote}
%
% \paragraph{Unpacking with \LaTeX.}
% The \xfile{.dtx} chooses its action depending on the format:
% \begin{description}
% \item[\plainTeX:] Run \docstrip\ and extract the files.
% \item[\LaTeX:] Generate the documentation.
% \end{description}
% If you insist on using \LaTeX\ for \docstrip\ (really,
% \docstrip\ does not need \LaTeX), then inform the autodetect routine
% about your intention:
% \begin{quote}
%   \verb|latex \let\install=y\input{listingsutf8.dtx}|
% \end{quote}
% Do not forget to quote the argument according to the demands
% of your shell.
%
% \paragraph{Generating the documentation.}
% You can use both the \xfile{.dtx} or the \xfile{.drv} to generate
% the documentation. The process can be configured by the
% configuration file \xfile{ltxdoc.cfg}. For instance, put this
% line into this file, if you want to have A4 as paper format:
% \begin{quote}
%   \verb|\PassOptionsToClass{a4paper}{article}|
% \end{quote}
% An example follows how to generate the
% documentation with pdf\LaTeX:
% \begin{quote}
%\begin{verbatim}
%pdflatex listingsutf8.dtx
%makeindex -s gind.ist listingsutf8.idx
%pdflatex listingsutf8.dtx
%makeindex -s gind.ist listingsutf8.idx
%pdflatex listingsutf8.dtx
%\end{verbatim}
% \end{quote}
%
% \section{Catalogue}
%
% The following XML file can be used as source for the
% \href{http://mirror.ctan.org/help/Catalogue/catalogue.html}{\TeX\ Catalogue}.
% The elements \texttt{caption} and \texttt{description} are imported
% from the original XML file from the Catalogue.
% The name of the XML file in the Catalogue is \xfile{listingsutf8.xml}.
%    \begin{macrocode}
%<*catalogue>
<?xml version='1.0' encoding='us-ascii'?>
<!DOCTYPE entry SYSTEM 'catalogue.dtd'>
<entry datestamp='$Date$' modifier='$Author$' id='listingsutf8'>
  <name>listingsutf8</name>
  <caption>Allow UTF-8 in listings input.</caption>
  <authorref id='auth:oberdiek'/>
  <copyright owner='Heiko Oberdiek' year='2007,2011'/>
  <license type='lppl1.3'/>
  <version number='1.2'/>
  <description>
    Package <xref refid='listings'>listings</xref> does not support files
    with multi-byte encodings such as UTF-8.  In the case of
    <tt>\lstinputlisting</tt>, a simple workaround is possible if a
    one-byte encoding exists that the file can be converted to.  The
    package requires the e-TeX extensions under pdfTeX (in either PDF
    or DVI output mode).
    <p/>
    The package is part of the <xref refid='oberdiek'>oberdiek</xref> bundle.
  </description>
  <documentation details='Package documentation'
      href='ctan:/macros/latex/contrib/oberdiek/listingsutf8.pdf'/>
  <ctan file='true' path='/macros/latex/contrib/oberdiek/listingsutf8.dtx'/>
  <miktex location='oberdiek'/>
  <texlive location='oberdiek'/>
  <install path='/macros/latex/contrib/oberdiek/oberdiek.tds.zip'/>
</entry>
%</catalogue>
%    \end{macrocode}
%
% \begin{thebibliography}{9}
%
% \bibitem{inputenc}
%   Alan Jeffrey, Frank Mittelbach,
%   \textit{inputenc.sty}, 2006/05/05 v1.1b.
%   \CTAN{macros/latex/base/inputenc.dtx}
%
% \bibitem{listings}
%   Carsten Heinz, Brooks Moses:
%  \textit{The \xpackage{listings} package};
%   2007/02/22;\\
%   \CTAN{macros/latex/contrib/listings/}.
%
% \bibitem{stringenc}
%   Heiko Oberdiek:
%   \textit{The \xpackage{stringenc} package};
%   2007/10/22;\\
%   \CTAN{macros/latex/contrib/oberdiek/stringenc.pdf}.
%
% \end{thebibliography}
%
% \begin{History}
%   \begin{Version}{2007/10/22 v1.0}
%   \item
%     First version.
%   \end{Version}
%   \begin{Version}{2007/11/11 v1.1}
%   \item
%     Use of package \xpackage{pdftexcmds}.
%   \end{Version}
%   \begin{Version}{2011/11/10 v1.2}
%   \item
%     DOS line ends CR/LF normalized to LF to avoid empty lines
%     (Bug report of Thomas Benkert in de.comp.text.tex).
%   \end{Version}
% \end{History}
%
% \PrintIndex
%
% \Finale
\endinput

%        (quote the arguments according to the demands of your shell)
%
% Documentation:
%    (a) If listingsutf8.drv is present:
%           latex listingsutf8.drv
%    (b) Without listingsutf8.drv:
%           latex listingsutf8.dtx; ...
%    The class ltxdoc loads the configuration file ltxdoc.cfg
%    if available. Here you can specify further options, e.g.
%    use A4 as paper format:
%       \PassOptionsToClass{a4paper}{article}
%
%    Programm calls to get the documentation (example):
%       pdflatex listingsutf8.dtx
%       makeindex -s gind.ist listingsutf8.idx
%       pdflatex listingsutf8.dtx
%       makeindex -s gind.ist listingsutf8.idx
%       pdflatex listingsutf8.dtx
%
% Installation:
%    TDS:tex/latex/oberdiek/listingsutf8.sty
%    TDS:doc/latex/oberdiek/listingsutf8.pdf
%    TDS:doc/latex/oberdiek/test/listingsutf8-test1.tex
%    TDS:doc/latex/oberdiek/test/listingsutf8-test2.tex
%    TDS:doc/latex/oberdiek/test/listingsutf8-test3.tex
%    TDS:doc/latex/oberdiek/test/listingsutf8-test4.tex
%    TDS:doc/latex/oberdiek/test/listingsutf8-test5.tex
%    TDS:source/latex/oberdiek/listingsutf8.dtx
%
%<*ignore>
\begingroup
  \catcode123=1 %
  \catcode125=2 %
  \def\x{LaTeX2e}%
\expandafter\endgroup
\ifcase 0\ifx\install y1\fi\expandafter
         \ifx\csname processbatchFile\endcsname\relax\else1\fi
         \ifx\fmtname\x\else 1\fi\relax
\else\csname fi\endcsname
%</ignore>
%<*install>
\input docstrip.tex
\Msg{************************************************************************}
\Msg{* Installation}
\Msg{* Package: listingsutf8 2011/11/10 v1.2 Allow UTF-8 in listings input (HO)}
\Msg{************************************************************************}

\keepsilent
\askforoverwritefalse

\let\MetaPrefix\relax
\preamble

This is a generated file.

Project: listingsutf8
Version: 2011/11/10 v1.2

Copyright (C) 2007, 2011 by
   Heiko Oberdiek <heiko.oberdiek at googlemail.com>

This work may be distributed and/or modified under the
conditions of the LaTeX Project Public License, either
version 1.3c of this license or (at your option) any later
version. This version of this license is in
   http://www.latex-project.org/lppl/lppl-1-3c.txt
and the latest version of this license is in
   http://www.latex-project.org/lppl.txt
and version 1.3 or later is part of all distributions of
LaTeX version 2005/12/01 or later.

This work has the LPPL maintenance status "maintained".

This Current Maintainer of this work is Heiko Oberdiek.

This work consists of the main source file listingsutf8.dtx
and the derived files
   listingsutf8.sty, listingsutf8.pdf, listingsutf8.ins, listingsutf8.drv,
   listingsutf8-test1.tex, listingsutf8-test2.tex,
   listingsutf8-test3.tex, listingsutf8-test4.tex,
   listingsutf8-test5.tex.

\endpreamble
\let\MetaPrefix\DoubleperCent

\generate{%
  \file{listingsutf8.ins}{\from{listingsutf8.dtx}{install}}%
  \file{listingsutf8.drv}{\from{listingsutf8.dtx}{driver}}%
  \usedir{tex/latex/oberdiek}%
  \file{listingsutf8.sty}{\from{listingsutf8.dtx}{package}}%
  \usedir{doc/latex/oberdiek/test}%
  \file{listingsutf8-test1.tex}{\from{listingsutf8.dtx}{test1}}%
  \file{listingsutf8-test2.tex}{\from{listingsutf8.dtx}{test2,utf8}}%
  \file{listingsutf8-test3.tex}{\from{listingsutf8.dtx}{test3,utf8x}}%
  \file{listingsutf8-test4.tex}{\from{listingsutf8.dtx}{test4,utf8,noetex}}%
  \file{listingsutf8-test5.tex}{\from{listingsutf8.dtx}{test5,utf8x,noetex}}%
  \nopreamble
  \nopostamble
  \usedir{source/latex/oberdiek/catalogue}%
  \file{listingsutf8.xml}{\from{listingsutf8.dtx}{catalogue}}%
}

\catcode32=13\relax% active space
\let =\space%
\Msg{************************************************************************}
\Msg{*}
\Msg{* To finish the installation you have to move the following}
\Msg{* file into a directory searched by TeX:}
\Msg{*}
\Msg{*     listingsutf8.sty}
\Msg{*}
\Msg{* To produce the documentation run the file `listingsutf8.drv'}
\Msg{* through LaTeX.}
\Msg{*}
\Msg{* Happy TeXing!}
\Msg{*}
\Msg{************************************************************************}

\endbatchfile
%</install>
%<*ignore>
\fi
%</ignore>
%<*driver>
\NeedsTeXFormat{LaTeX2e}
\ProvidesFile{listingsutf8.drv}%
  [2011/11/10 v1.2 Allow UTF-8 in listings input (HO)]%
\documentclass{ltxdoc}
\usepackage{holtxdoc}[2011/11/22]
\begin{document}
  \DocInput{listingsutf8.dtx}%
\end{document}
%</driver>
% \fi
%
% \CheckSum{311}
%
% \CharacterTable
%  {Upper-case    \A\B\C\D\E\F\G\H\I\J\K\L\M\N\O\P\Q\R\S\T\U\V\W\X\Y\Z
%   Lower-case    \a\b\c\d\e\f\g\h\i\j\k\l\m\n\o\p\q\r\s\t\u\v\w\x\y\z
%   Digits        \0\1\2\3\4\5\6\7\8\9
%   Exclamation   \!     Double quote  \"     Hash (number) \#
%   Dollar        \$     Percent       \%     Ampersand     \&
%   Acute accent  \'     Left paren    \(     Right paren   \)
%   Asterisk      \*     Plus          \+     Comma         \,
%   Minus         \-     Point         \.     Solidus       \/
%   Colon         \:     Semicolon     \;     Less than     \<
%   Equals        \=     Greater than  \>     Question mark \?
%   Commercial at \@     Left bracket  \[     Backslash     \\
%   Right bracket \]     Circumflex    \^     Underscore    \_
%   Grave accent  \`     Left brace    \{     Vertical bar  \|
%   Right brace   \}     Tilde         \~}
%
% \GetFileInfo{listingsutf8.drv}
%
% \title{The \xpackage{listingsutf8} package}
% \date{2011/11/10 v1.2}
% \author{Heiko Oberdiek\\\xemail{heiko.oberdiek at googlemail.com}}
%
% \maketitle
%
% \begin{abstract}
% Package \xpackage{listings} does not support files with multi-byte
% encodings such as UTF-8. In case of \cs{lstinputlisting} a simple
% workaround is possible if an one-byte encoding exists that the file
% can be converted to. Also \eTeX\ and \pdfTeX\ regardless of its mode
% are required.
% \end{abstract}
%
% \tableofcontents
%
% \section{Documentation}
%
% \subsection{User interface}
%
% Load this package after or instead of package \xpackage{listings}
% \cite{listings}.
% The package does not define own options and passes given options to
% package \xpackage{listings}.
%
% The syntax of package \xpackage{listings}' key \xoption{inputencoding}
% is extended:
% \begin{quote}
%   |inputencoding=utf8/|\meta{one-byte-encoding}\\
%   Example: |inputencoding=utf8/latin1|
% \end{quote}
% That means the file is encoded in UTF-8 and can
% be converted to the given \meta{one-byte-encoding}.
% The available encodings for \meta{one-byte-encoding} are
% listed in section ``1.2 Supported encodings'' of
% package \xpackage{stringenc}'s documentation \cite{stringenc}.
% Of course, the encoding must encode its characters with
% one byte exactly. This excludes the unicode encodings
% (\xoption{utf8}, \xoption{utf16}, \dots).
%
% Only \cs{lstinputlisting} is supported by the syntax extension
% of key \xoption{inputencoding}.
%
% Internally package \xpackage{listingsutf8} reads the file as binary file
% via primitives of \pdfTeX\ (\cs{pdffiledump}). Then the file
% contents is converted as string using package \xpackage{stringenc} and
% finally the string is read as virtual file by \eTeX's \cs{scantokens}.
%
% \subsection{Future}
%
% Workarounds are not provided for
% \begin{itemize}
% \item \cs{lstinline}
% \item Environment |lstlisting|.
% \item Environments defined by \cs{lstnewenvironment}.
% \end{itemize}
% Perhaps someone will find time to extend package \xpackage{listings}
% with full native support for UTF-8. Then this package would become obsolete.
%
% \StopEventually{
% }
%
% \section{Implementation}
%
%    \begin{macrocode}
%<*package>
%    \end{macrocode}
%
% \subsection{Catcodes and identification}
%
%    \begin{macrocode}
\begingroup\catcode61\catcode48\catcode32=10\relax%
  \catcode13=5 % ^^M
  \endlinechar=13 %
  \catcode123=1 % {
  \catcode125=2 % }
  \catcode64=11 % @
  \def\x{\endgroup
    \expandafter\edef\csname lstU@AtEnd\endcsname{%
      \endlinechar=\the\endlinechar\relax
      \catcode13=\the\catcode13\relax
      \catcode32=\the\catcode32\relax
      \catcode35=\the\catcode35\relax
      \catcode61=\the\catcode61\relax
      \catcode64=\the\catcode64\relax
      \catcode123=\the\catcode123\relax
      \catcode125=\the\catcode125\relax
    }%
  }%
\x\catcode61\catcode48\catcode32=10\relax%
\catcode13=5 % ^^M
\endlinechar=13 %
\catcode35=6 % #
\catcode64=11 % @
\catcode123=1 % {
\catcode125=2 % }
\def\TMP@EnsureCode#1#2{%
  \edef\lstU@AtEnd{%
    \lstU@AtEnd
    \catcode#1=\the\catcode#1\relax
  }%
  \catcode#1=#2\relax
}
\TMP@EnsureCode{10}{12}% ^^J
\TMP@EnsureCode{33}{12}% !
\TMP@EnsureCode{36}{3}% $
\TMP@EnsureCode{38}{4}% &
\TMP@EnsureCode{39}{12}% '
\TMP@EnsureCode{40}{12}% (
\TMP@EnsureCode{41}{12}% )
\TMP@EnsureCode{42}{12}% *
\TMP@EnsureCode{43}{12}% +
\TMP@EnsureCode{44}{12}% ,
\TMP@EnsureCode{45}{12}% -
\TMP@EnsureCode{46}{12}% .
\TMP@EnsureCode{47}{12}% /
\TMP@EnsureCode{58}{12}% :
\TMP@EnsureCode{60}{12}% <
\TMP@EnsureCode{62}{12}% >
\TMP@EnsureCode{91}{12}% [
\TMP@EnsureCode{93}{12}% ]
\TMP@EnsureCode{94}{7}% ^ (superscript)
\TMP@EnsureCode{95}{8}% _ (subscript)
\TMP@EnsureCode{96}{12}% `
\TMP@EnsureCode{124}{12}% |
\TMP@EnsureCode{126}{13}% ~ (active)
\edef\lstU@AtEnd{\lstU@AtEnd\noexpand\endinput}
%    \end{macrocode}
%
%    Package identification.
%    \begin{macrocode}
\NeedsTeXFormat{LaTeX2e}
\ProvidesPackage{listingsutf8}%
  [2011/11/10 v1.2 Allow UTF-8 in listings input (HO)]
%    \end{macrocode}
%
% \subsection{Package options}
%
% Just pass options to package listings.
%
%    \begin{macrocode}
\DeclareOption*{%
  \PassOptionsToPackage\CurrentOption{listings}%
}
\ProcessOptions*
%    \end{macrocode}
%    Key \xoption{inputencoding} was introduced in version
%    2002/04/01 v1.0 of package \xpackage{listings}.
%    \begin{macrocode}
\RequirePackage{listings}[2002/04/01]
%    \end{macrocode}
%    Ensure that \cs{inputencoding} is provided.
%    \begin{macrocode}
\AtBeginDocument{%
  \@ifundefined{inputencoding}{%
    \RequirePackage{inputenc}%
  }{}%
}
%    \end{macrocode}
%
% \subsection{Check prerequisites}
%
%    \begin{macrocode}
\RequirePackage{pdftexcmds}[2011/04/22]
%    \end{macrocode}
%
%    \begin{macrocode}
\def\lstU@temp#1#2{%
  \begingroup\expandafter\expandafter\expandafter\endgroup
  \expandafter\ifx\csname #1\endcsname\relax
    \PackageWarningNoLine{listingsutf8}{%
      Package loading is aborted because of missing %
      \@backslashchar#1.\MessageBreak
      #2%
    }%
    \expandafter\lstU@AtEnd
  \fi
}
\lstU@temp{scantokens}{It is provided by e-TeX}%
\lstU@temp{pdf@unescapehex}{It is provided by pdfTeX >= 1.30}%
\lstU@temp{pdf@filedump}{It is provided by pdfTeX >= 1.30}%
\lstU@temp{pdf@filesize}{It is provided by pdfTeX >= 1.30}%
%    \end{macrocode}
%
%    \begin{macrocode}
\RequirePackage{stringenc}[2010/03/01]
%    \end{macrocode}
%
% \subsection{Add support for UTF-8}
%
%    \begin{macro}{\iflstU@utfviii}
%    \begin{macrocode}
\newif\iflstU@utfviii
%    \end{macrocode}
%    \end{macro}
%
%    \begin{macro}{\lstU@inputenc}
%    \begin{macrocode}
\def\lstU@inputenc#1{%
  \expandafter\lstU@@inputenc#1utf8/utf8/\@nil
}
%    \end{macrocode}
%    \end{macro}
%    \begin{macro}{\lstU@@inputenc}
\def\lstU@@inputenc#1utf8/#2utf8/#3\@nil{%
  \ifx\\#1\\%
    \lstU@utfviiitrue
    \def\lst@inputenc{#2}%
  \else
    \lstU@utfviiifalse
  \fi
}
%    \begin{macrocode}
%    \end{macrocode}
%    \end{macro}
%
%    \begin{macrocode}
\lst@Key{inputencoding}\relax{%
  \def\lst@inputenc{#1}%
  \lstU@inputenc{#1}%
}
%    \end{macrocode}
%
% \subsubsection{Conversion}
%
%    \begin{macro}{\lstU@input}
%    \begin{macrocode}
\def\lstU@input#1{%
  \iflstU@utfviii
    \edef\lstU@text{%
      \pdf@unescapehex{%
        \pdf@filedump{0}{\pdf@filesize{#1}}{#1}%
      }%
    }%
    \lstU@CRLFtoLF\lstU@text
    \StringEncodingConvert\lstU@text\lstU@text{utf8}\lst@inputenc
    \def\lstU@temp{%
      \scantokens\expandafter{\lstU@text}%
    }%
  \else
    \def\lstU@temp{%
      \input{#1}%
    }%
  \fi
  \lstU@temp
}
%    \end{macrocode}
%    \end{macro}
%
% \subsubsection{Convert CR/LF pairs to LF}
%
%    \begin{macro}{\lstU@CRLFtoLF}
%    \begin{macrocode}
\begingroup
  \endlinechar=-1 %
  \@makeother\^^J %
  \@makeother\^^M %
  \gdef\lstU@CRLFtoLF#1{%
    \edef#1{%
      \expandafter\lstU@CRLFtoLF@aux#1^^M^^J\@nil
    }%
  }%
  \gdef\lstU@CRLFtoLF@aux#1^^M^^J#2\@nil{%
    #1%
    \ifx\relax#2\relax
      \@car
    \fi
    ^^J%
    \lstU@CRLFtoLF@aux#2\@nil
  }%
\endgroup %
%    \end{macrocode}
%    \end{macro}
%
% \subsubsection{Patch \cs{lst@InputListing}}
%
%    \begin{macrocode}
\def\lstU@temp#1\def\lst@next#2#3\@nil{%
  \def\lst@InputListing##1{%
    #1%
    \def\lst@next{\lstU@input{##1}}%
    #3%
  }%
}
\expandafter\lstU@temp\lst@InputListing{#1}\@nil
%    \end{macrocode}
%
%    \begin{macrocode}
\lstU@AtEnd%
%</package>
%    \end{macrocode}
%
% \section{Test}
%
% \subsection{Catcode checks for loading}
%
%    \begin{macrocode}
%<*test1>
%    \end{macrocode}
%    \begin{macrocode}
\NeedsTeXFormat{LaTeX2e}
\documentclass{minimal}
\makeatletter
\def\RestoreCatcodes{}
\count@=0 %
\loop
  \edef\RestoreCatcodes{%
    \RestoreCatcodes
    \catcode\the\count@=\the\catcode\count@\relax
  }%
\ifnum\count@<255 %
  \advance\count@\@ne
\repeat

\def\RangeCatcodeInvalid#1#2{%
  \count@=#1\relax
  \loop
    \catcode\count@=15 %
  \ifnum\count@<#2\relax
    \advance\count@\@ne
  \repeat
}
\def\Test{%
  \RangeCatcodeInvalid{0}{47}%
  \RangeCatcodeInvalid{58}{64}%
  \RangeCatcodeInvalid{91}{96}%
  \RangeCatcodeInvalid{123}{127}%
  \catcode`\@=12 %
  \catcode`\\=0 %
  \catcode`\{=1 %
  \catcode`\}=2 %
  \catcode`\#=6 %
  \catcode`\[=12 %
  \catcode`\]=12 %
  \catcode`\%=14 %
  \catcode`\ =10 %
  \catcode13=5 %
  \RequirePackage{listingsutf8}[2011/11/10]\relax
  \RestoreCatcodes
}
\Test
\csname @@end\endcsname
\end
%    \end{macrocode}
%    \begin{macrocode}
%</test1>
%    \end{macrocode}
%
% \subsection{Test example for latin1}
%
%    \begin{macrocode}
%<*test2>
%    \end{macrocode}
%    \begin{macrocode}
\NeedsTeXFormat{LaTeX2e}
\documentclass{minimal}
\usepackage{filecontents}
\def\do#1{%
  \ifx#1\^%
  \else
    \noexpand\do\noexpand#1%
  \fi
}
\expandafter\let\expandafter\dospecials\expandafter\empty
\expandafter\edef\expandafter\dospecials\expandafter{\dospecials}
\begin{filecontents*}{ExampleUTF8.java}
public class ExampleUTF8 {
    public static String testString =
        "Umlauts: " +
        "^^c3^^84^^c3^^96^^c3^^9c^^c3^^a4^^c3^^b6^^c3^^bc^^c3^^9f";
    public static void main(String[] args) {
        System.out.println(testString);
    }
}
\end{filecontents*}
\usepackage{listingsutf8}[2011/11/10]
\def\Text{%
  Umlauts: %
  ^^c3^^84^^c3^^96^^c3^^9c^^c3^^a4^^c3^^b6^^c3^^bc^^c3^^9f%
}
\begin{document}
\lstinputlisting[%
  language=Java,%
  inputencoding=utf8/latin1,%
]{ExampleUTF8.java}
\end{document}
%</test2>
%    \end{macrocode}
%
% \section{Installation}
%
% \subsection{Download}
%
% \paragraph{Package.} This package is available on
% CTAN\footnote{\url{ftp://ftp.ctan.org/tex-archive/}}:
% \begin{description}
% \item[\CTAN{macros/latex/contrib/oberdiek/listingsutf8.dtx}] The source file.
% \item[\CTAN{macros/latex/contrib/oberdiek/listingsutf8.pdf}] Documentation.
% \end{description}
%
%
% \paragraph{Bundle.} All the packages of the bundle `oberdiek'
% are also available in a TDS compliant ZIP archive. There
% the packages are already unpacked and the documentation files
% are generated. The files and directories obey the TDS standard.
% \begin{description}
% \item[\CTAN{install/macros/latex/contrib/oberdiek.tds.zip}]
% \end{description}
% \emph{TDS} refers to the standard ``A Directory Structure
% for \TeX\ Files'' (\CTAN{tds/tds.pdf}). Directories
% with \xfile{texmf} in their name are usually organized this way.
%
% \subsection{Bundle installation}
%
% \paragraph{Unpacking.} Unpack the \xfile{oberdiek.tds.zip} in the
% TDS tree (also known as \xfile{texmf} tree) of your choice.
% Example (linux):
% \begin{quote}
%   |unzip oberdiek.tds.zip -d ~/texmf|
% \end{quote}
%
% \paragraph{Script installation.}
% Check the directory \xfile{TDS:scripts/oberdiek/} for
% scripts that need further installation steps.
% Package \xpackage{attachfile2} comes with the Perl script
% \xfile{pdfatfi.pl} that should be installed in such a way
% that it can be called as \texttt{pdfatfi}.
% Example (linux):
% \begin{quote}
%   |chmod +x scripts/oberdiek/pdfatfi.pl|\\
%   |cp scripts/oberdiek/pdfatfi.pl /usr/local/bin/|
% \end{quote}
%
% \subsection{Package installation}
%
% \paragraph{Unpacking.} The \xfile{.dtx} file is a self-extracting
% \docstrip\ archive. The files are extracted by running the
% \xfile{.dtx} through \plainTeX:
% \begin{quote}
%   \verb|tex listingsutf8.dtx|
% \end{quote}
%
% \paragraph{TDS.} Now the different files must be moved into
% the different directories in your installation TDS tree
% (also known as \xfile{texmf} tree):
% \begin{quote}
% \def\t{^^A
% \begin{tabular}{@{}>{\ttfamily}l@{ $\rightarrow$ }>{\ttfamily}l@{}}
%   listingsutf8.sty & tex/latex/oberdiek/listingsutf8.sty\\
%   listingsutf8.pdf & doc/latex/oberdiek/listingsutf8.pdf\\
%   test/listingsutf8-test1.tex & doc/latex/oberdiek/test/listingsutf8-test1.tex\\
%   test/listingsutf8-test2.tex & doc/latex/oberdiek/test/listingsutf8-test2.tex\\
%   test/listingsutf8-test3.tex & doc/latex/oberdiek/test/listingsutf8-test3.tex\\
%   test/listingsutf8-test4.tex & doc/latex/oberdiek/test/listingsutf8-test4.tex\\
%   test/listingsutf8-test5.tex & doc/latex/oberdiek/test/listingsutf8-test5.tex\\
%   listingsutf8.dtx & source/latex/oberdiek/listingsutf8.dtx\\
% \end{tabular}^^A
% }^^A
% \sbox0{\t}^^A
% \ifdim\wd0>\linewidth
%   \begingroup
%     \advance\linewidth by\leftmargin
%     \advance\linewidth by\rightmargin
%   \edef\x{\endgroup
%     \def\noexpand\lw{\the\linewidth}^^A
%   }\x
%   \def\lwbox{^^A
%     \leavevmode
%     \hbox to \linewidth{^^A
%       \kern-\leftmargin\relax
%       \hss
%       \usebox0
%       \hss
%       \kern-\rightmargin\relax
%     }^^A
%   }^^A
%   \ifdim\wd0>\lw
%     \sbox0{\small\t}^^A
%     \ifdim\wd0>\linewidth
%       \ifdim\wd0>\lw
%         \sbox0{\footnotesize\t}^^A
%         \ifdim\wd0>\linewidth
%           \ifdim\wd0>\lw
%             \sbox0{\scriptsize\t}^^A
%             \ifdim\wd0>\linewidth
%               \ifdim\wd0>\lw
%                 \sbox0{\tiny\t}^^A
%                 \ifdim\wd0>\linewidth
%                   \lwbox
%                 \else
%                   \usebox0
%                 \fi
%               \else
%                 \lwbox
%               \fi
%             \else
%               \usebox0
%             \fi
%           \else
%             \lwbox
%           \fi
%         \else
%           \usebox0
%         \fi
%       \else
%         \lwbox
%       \fi
%     \else
%       \usebox0
%     \fi
%   \else
%     \lwbox
%   \fi
% \else
%   \usebox0
% \fi
% \end{quote}
% If you have a \xfile{docstrip.cfg} that configures and enables \docstrip's
% TDS installing feature, then some files can already be in the right
% place, see the documentation of \docstrip.
%
% \subsection{Refresh file name databases}
%
% If your \TeX~distribution
% (\teTeX, \mikTeX, \dots) relies on file name databases, you must refresh
% these. For example, \teTeX\ users run \verb|texhash| or
% \verb|mktexlsr|.
%
% \subsection{Some details for the interested}
%
% \paragraph{Attached source.}
%
% The PDF documentation on CTAN also includes the
% \xfile{.dtx} source file. It can be extracted by
% AcrobatReader 6 or higher. Another option is \textsf{pdftk},
% e.g. unpack the file into the current directory:
% \begin{quote}
%   \verb|pdftk listingsutf8.pdf unpack_files output .|
% \end{quote}
%
% \paragraph{Unpacking with \LaTeX.}
% The \xfile{.dtx} chooses its action depending on the format:
% \begin{description}
% \item[\plainTeX:] Run \docstrip\ and extract the files.
% \item[\LaTeX:] Generate the documentation.
% \end{description}
% If you insist on using \LaTeX\ for \docstrip\ (really,
% \docstrip\ does not need \LaTeX), then inform the autodetect routine
% about your intention:
% \begin{quote}
%   \verb|latex \let\install=y% \iffalse meta-comment
%
% File: listingsutf8.dtx
% Version: 2011/11/10 v1.2
% Info: Allow UTF-8 in listings input
%
% Copyright (C) 2007, 2011 by
%    Heiko Oberdiek <heiko.oberdiek at googlemail.com>
%
% This work may be distributed and/or modified under the
% conditions of the LaTeX Project Public License, either
% version 1.3c of this license or (at your option) any later
% version. This version of this license is in
%    http://www.latex-project.org/lppl/lppl-1-3c.txt
% and the latest version of this license is in
%    http://www.latex-project.org/lppl.txt
% and version 1.3 or later is part of all distributions of
% LaTeX version 2005/12/01 or later.
%
% This work has the LPPL maintenance status "maintained".
%
% This Current Maintainer of this work is Heiko Oberdiek.
%
% This work consists of the main source file listingsutf8.dtx
% and the derived files
%    listingsutf8.sty, listingsutf8.pdf, listingsutf8.ins, listingsutf8.drv,
%    listingsutf8-test1.tex, listingsutf8-test2.tex,
%    listingsutf8-test3.tex, listingsutf8-test4.tex,
%    listingsutf8-test5.tex.
%
% Distribution:
%    CTAN:macros/latex/contrib/oberdiek/listingsutf8.dtx
%    CTAN:macros/latex/contrib/oberdiek/listingsutf8.pdf
%
% Unpacking:
%    (a) If listingsutf8.ins is present:
%           tex listingsutf8.ins
%    (b) Without listingsutf8.ins:
%           tex listingsutf8.dtx
%    (c) If you insist on using LaTeX
%           latex \let\install=y\input{listingsutf8.dtx}
%        (quote the arguments according to the demands of your shell)
%
% Documentation:
%    (a) If listingsutf8.drv is present:
%           latex listingsutf8.drv
%    (b) Without listingsutf8.drv:
%           latex listingsutf8.dtx; ...
%    The class ltxdoc loads the configuration file ltxdoc.cfg
%    if available. Here you can specify further options, e.g.
%    use A4 as paper format:
%       \PassOptionsToClass{a4paper}{article}
%
%    Programm calls to get the documentation (example):
%       pdflatex listingsutf8.dtx
%       makeindex -s gind.ist listingsutf8.idx
%       pdflatex listingsutf8.dtx
%       makeindex -s gind.ist listingsutf8.idx
%       pdflatex listingsutf8.dtx
%
% Installation:
%    TDS:tex/latex/oberdiek/listingsutf8.sty
%    TDS:doc/latex/oberdiek/listingsutf8.pdf
%    TDS:doc/latex/oberdiek/test/listingsutf8-test1.tex
%    TDS:doc/latex/oberdiek/test/listingsutf8-test2.tex
%    TDS:doc/latex/oberdiek/test/listingsutf8-test3.tex
%    TDS:doc/latex/oberdiek/test/listingsutf8-test4.tex
%    TDS:doc/latex/oberdiek/test/listingsutf8-test5.tex
%    TDS:source/latex/oberdiek/listingsutf8.dtx
%
%<*ignore>
\begingroup
  \catcode123=1 %
  \catcode125=2 %
  \def\x{LaTeX2e}%
\expandafter\endgroup
\ifcase 0\ifx\install y1\fi\expandafter
         \ifx\csname processbatchFile\endcsname\relax\else1\fi
         \ifx\fmtname\x\else 1\fi\relax
\else\csname fi\endcsname
%</ignore>
%<*install>
\input docstrip.tex
\Msg{************************************************************************}
\Msg{* Installation}
\Msg{* Package: listingsutf8 2011/11/10 v1.2 Allow UTF-8 in listings input (HO)}
\Msg{************************************************************************}

\keepsilent
\askforoverwritefalse

\let\MetaPrefix\relax
\preamble

This is a generated file.

Project: listingsutf8
Version: 2011/11/10 v1.2

Copyright (C) 2007, 2011 by
   Heiko Oberdiek <heiko.oberdiek at googlemail.com>

This work may be distributed and/or modified under the
conditions of the LaTeX Project Public License, either
version 1.3c of this license or (at your option) any later
version. This version of this license is in
   http://www.latex-project.org/lppl/lppl-1-3c.txt
and the latest version of this license is in
   http://www.latex-project.org/lppl.txt
and version 1.3 or later is part of all distributions of
LaTeX version 2005/12/01 or later.

This work has the LPPL maintenance status "maintained".

This Current Maintainer of this work is Heiko Oberdiek.

This work consists of the main source file listingsutf8.dtx
and the derived files
   listingsutf8.sty, listingsutf8.pdf, listingsutf8.ins, listingsutf8.drv,
   listingsutf8-test1.tex, listingsutf8-test2.tex,
   listingsutf8-test3.tex, listingsutf8-test4.tex,
   listingsutf8-test5.tex.

\endpreamble
\let\MetaPrefix\DoubleperCent

\generate{%
  \file{listingsutf8.ins}{\from{listingsutf8.dtx}{install}}%
  \file{listingsutf8.drv}{\from{listingsutf8.dtx}{driver}}%
  \usedir{tex/latex/oberdiek}%
  \file{listingsutf8.sty}{\from{listingsutf8.dtx}{package}}%
  \usedir{doc/latex/oberdiek/test}%
  \file{listingsutf8-test1.tex}{\from{listingsutf8.dtx}{test1}}%
  \file{listingsutf8-test2.tex}{\from{listingsutf8.dtx}{test2,utf8}}%
  \file{listingsutf8-test3.tex}{\from{listingsutf8.dtx}{test3,utf8x}}%
  \file{listingsutf8-test4.tex}{\from{listingsutf8.dtx}{test4,utf8,noetex}}%
  \file{listingsutf8-test5.tex}{\from{listingsutf8.dtx}{test5,utf8x,noetex}}%
  \nopreamble
  \nopostamble
  \usedir{source/latex/oberdiek/catalogue}%
  \file{listingsutf8.xml}{\from{listingsutf8.dtx}{catalogue}}%
}

\catcode32=13\relax% active space
\let =\space%
\Msg{************************************************************************}
\Msg{*}
\Msg{* To finish the installation you have to move the following}
\Msg{* file into a directory searched by TeX:}
\Msg{*}
\Msg{*     listingsutf8.sty}
\Msg{*}
\Msg{* To produce the documentation run the file `listingsutf8.drv'}
\Msg{* through LaTeX.}
\Msg{*}
\Msg{* Happy TeXing!}
\Msg{*}
\Msg{************************************************************************}

\endbatchfile
%</install>
%<*ignore>
\fi
%</ignore>
%<*driver>
\NeedsTeXFormat{LaTeX2e}
\ProvidesFile{listingsutf8.drv}%
  [2011/11/10 v1.2 Allow UTF-8 in listings input (HO)]%
\documentclass{ltxdoc}
\usepackage{holtxdoc}[2011/11/22]
\begin{document}
  \DocInput{listingsutf8.dtx}%
\end{document}
%</driver>
% \fi
%
% \CheckSum{311}
%
% \CharacterTable
%  {Upper-case    \A\B\C\D\E\F\G\H\I\J\K\L\M\N\O\P\Q\R\S\T\U\V\W\X\Y\Z
%   Lower-case    \a\b\c\d\e\f\g\h\i\j\k\l\m\n\o\p\q\r\s\t\u\v\w\x\y\z
%   Digits        \0\1\2\3\4\5\6\7\8\9
%   Exclamation   \!     Double quote  \"     Hash (number) \#
%   Dollar        \$     Percent       \%     Ampersand     \&
%   Acute accent  \'     Left paren    \(     Right paren   \)
%   Asterisk      \*     Plus          \+     Comma         \,
%   Minus         \-     Point         \.     Solidus       \/
%   Colon         \:     Semicolon     \;     Less than     \<
%   Equals        \=     Greater than  \>     Question mark \?
%   Commercial at \@     Left bracket  \[     Backslash     \\
%   Right bracket \]     Circumflex    \^     Underscore    \_
%   Grave accent  \`     Left brace    \{     Vertical bar  \|
%   Right brace   \}     Tilde         \~}
%
% \GetFileInfo{listingsutf8.drv}
%
% \title{The \xpackage{listingsutf8} package}
% \date{2011/11/10 v1.2}
% \author{Heiko Oberdiek\\\xemail{heiko.oberdiek at googlemail.com}}
%
% \maketitle
%
% \begin{abstract}
% Package \xpackage{listings} does not support files with multi-byte
% encodings such as UTF-8. In case of \cs{lstinputlisting} a simple
% workaround is possible if an one-byte encoding exists that the file
% can be converted to. Also \eTeX\ and \pdfTeX\ regardless of its mode
% are required.
% \end{abstract}
%
% \tableofcontents
%
% \section{Documentation}
%
% \subsection{User interface}
%
% Load this package after or instead of package \xpackage{listings}
% \cite{listings}.
% The package does not define own options and passes given options to
% package \xpackage{listings}.
%
% The syntax of package \xpackage{listings}' key \xoption{inputencoding}
% is extended:
% \begin{quote}
%   |inputencoding=utf8/|\meta{one-byte-encoding}\\
%   Example: |inputencoding=utf8/latin1|
% \end{quote}
% That means the file is encoded in UTF-8 and can
% be converted to the given \meta{one-byte-encoding}.
% The available encodings for \meta{one-byte-encoding} are
% listed in section ``1.2 Supported encodings'' of
% package \xpackage{stringenc}'s documentation \cite{stringenc}.
% Of course, the encoding must encode its characters with
% one byte exactly. This excludes the unicode encodings
% (\xoption{utf8}, \xoption{utf16}, \dots).
%
% Only \cs{lstinputlisting} is supported by the syntax extension
% of key \xoption{inputencoding}.
%
% Internally package \xpackage{listingsutf8} reads the file as binary file
% via primitives of \pdfTeX\ (\cs{pdffiledump}). Then the file
% contents is converted as string using package \xpackage{stringenc} and
% finally the string is read as virtual file by \eTeX's \cs{scantokens}.
%
% \subsection{Future}
%
% Workarounds are not provided for
% \begin{itemize}
% \item \cs{lstinline}
% \item Environment |lstlisting|.
% \item Environments defined by \cs{lstnewenvironment}.
% \end{itemize}
% Perhaps someone will find time to extend package \xpackage{listings}
% with full native support for UTF-8. Then this package would become obsolete.
%
% \StopEventually{
% }
%
% \section{Implementation}
%
%    \begin{macrocode}
%<*package>
%    \end{macrocode}
%
% \subsection{Catcodes and identification}
%
%    \begin{macrocode}
\begingroup\catcode61\catcode48\catcode32=10\relax%
  \catcode13=5 % ^^M
  \endlinechar=13 %
  \catcode123=1 % {
  \catcode125=2 % }
  \catcode64=11 % @
  \def\x{\endgroup
    \expandafter\edef\csname lstU@AtEnd\endcsname{%
      \endlinechar=\the\endlinechar\relax
      \catcode13=\the\catcode13\relax
      \catcode32=\the\catcode32\relax
      \catcode35=\the\catcode35\relax
      \catcode61=\the\catcode61\relax
      \catcode64=\the\catcode64\relax
      \catcode123=\the\catcode123\relax
      \catcode125=\the\catcode125\relax
    }%
  }%
\x\catcode61\catcode48\catcode32=10\relax%
\catcode13=5 % ^^M
\endlinechar=13 %
\catcode35=6 % #
\catcode64=11 % @
\catcode123=1 % {
\catcode125=2 % }
\def\TMP@EnsureCode#1#2{%
  \edef\lstU@AtEnd{%
    \lstU@AtEnd
    \catcode#1=\the\catcode#1\relax
  }%
  \catcode#1=#2\relax
}
\TMP@EnsureCode{10}{12}% ^^J
\TMP@EnsureCode{33}{12}% !
\TMP@EnsureCode{36}{3}% $
\TMP@EnsureCode{38}{4}% &
\TMP@EnsureCode{39}{12}% '
\TMP@EnsureCode{40}{12}% (
\TMP@EnsureCode{41}{12}% )
\TMP@EnsureCode{42}{12}% *
\TMP@EnsureCode{43}{12}% +
\TMP@EnsureCode{44}{12}% ,
\TMP@EnsureCode{45}{12}% -
\TMP@EnsureCode{46}{12}% .
\TMP@EnsureCode{47}{12}% /
\TMP@EnsureCode{58}{12}% :
\TMP@EnsureCode{60}{12}% <
\TMP@EnsureCode{62}{12}% >
\TMP@EnsureCode{91}{12}% [
\TMP@EnsureCode{93}{12}% ]
\TMP@EnsureCode{94}{7}% ^ (superscript)
\TMP@EnsureCode{95}{8}% _ (subscript)
\TMP@EnsureCode{96}{12}% `
\TMP@EnsureCode{124}{12}% |
\TMP@EnsureCode{126}{13}% ~ (active)
\edef\lstU@AtEnd{\lstU@AtEnd\noexpand\endinput}
%    \end{macrocode}
%
%    Package identification.
%    \begin{macrocode}
\NeedsTeXFormat{LaTeX2e}
\ProvidesPackage{listingsutf8}%
  [2011/11/10 v1.2 Allow UTF-8 in listings input (HO)]
%    \end{macrocode}
%
% \subsection{Package options}
%
% Just pass options to package listings.
%
%    \begin{macrocode}
\DeclareOption*{%
  \PassOptionsToPackage\CurrentOption{listings}%
}
\ProcessOptions*
%    \end{macrocode}
%    Key \xoption{inputencoding} was introduced in version
%    2002/04/01 v1.0 of package \xpackage{listings}.
%    \begin{macrocode}
\RequirePackage{listings}[2002/04/01]
%    \end{macrocode}
%    Ensure that \cs{inputencoding} is provided.
%    \begin{macrocode}
\AtBeginDocument{%
  \@ifundefined{inputencoding}{%
    \RequirePackage{inputenc}%
  }{}%
}
%    \end{macrocode}
%
% \subsection{Check prerequisites}
%
%    \begin{macrocode}
\RequirePackage{pdftexcmds}[2011/04/22]
%    \end{macrocode}
%
%    \begin{macrocode}
\def\lstU@temp#1#2{%
  \begingroup\expandafter\expandafter\expandafter\endgroup
  \expandafter\ifx\csname #1\endcsname\relax
    \PackageWarningNoLine{listingsutf8}{%
      Package loading is aborted because of missing %
      \@backslashchar#1.\MessageBreak
      #2%
    }%
    \expandafter\lstU@AtEnd
  \fi
}
\lstU@temp{scantokens}{It is provided by e-TeX}%
\lstU@temp{pdf@unescapehex}{It is provided by pdfTeX >= 1.30}%
\lstU@temp{pdf@filedump}{It is provided by pdfTeX >= 1.30}%
\lstU@temp{pdf@filesize}{It is provided by pdfTeX >= 1.30}%
%    \end{macrocode}
%
%    \begin{macrocode}
\RequirePackage{stringenc}[2010/03/01]
%    \end{macrocode}
%
% \subsection{Add support for UTF-8}
%
%    \begin{macro}{\iflstU@utfviii}
%    \begin{macrocode}
\newif\iflstU@utfviii
%    \end{macrocode}
%    \end{macro}
%
%    \begin{macro}{\lstU@inputenc}
%    \begin{macrocode}
\def\lstU@inputenc#1{%
  \expandafter\lstU@@inputenc#1utf8/utf8/\@nil
}
%    \end{macrocode}
%    \end{macro}
%    \begin{macro}{\lstU@@inputenc}
\def\lstU@@inputenc#1utf8/#2utf8/#3\@nil{%
  \ifx\\#1\\%
    \lstU@utfviiitrue
    \def\lst@inputenc{#2}%
  \else
    \lstU@utfviiifalse
  \fi
}
%    \begin{macrocode}
%    \end{macrocode}
%    \end{macro}
%
%    \begin{macrocode}
\lst@Key{inputencoding}\relax{%
  \def\lst@inputenc{#1}%
  \lstU@inputenc{#1}%
}
%    \end{macrocode}
%
% \subsubsection{Conversion}
%
%    \begin{macro}{\lstU@input}
%    \begin{macrocode}
\def\lstU@input#1{%
  \iflstU@utfviii
    \edef\lstU@text{%
      \pdf@unescapehex{%
        \pdf@filedump{0}{\pdf@filesize{#1}}{#1}%
      }%
    }%
    \lstU@CRLFtoLF\lstU@text
    \StringEncodingConvert\lstU@text\lstU@text{utf8}\lst@inputenc
    \def\lstU@temp{%
      \scantokens\expandafter{\lstU@text}%
    }%
  \else
    \def\lstU@temp{%
      \input{#1}%
    }%
  \fi
  \lstU@temp
}
%    \end{macrocode}
%    \end{macro}
%
% \subsubsection{Convert CR/LF pairs to LF}
%
%    \begin{macro}{\lstU@CRLFtoLF}
%    \begin{macrocode}
\begingroup
  \endlinechar=-1 %
  \@makeother\^^J %
  \@makeother\^^M %
  \gdef\lstU@CRLFtoLF#1{%
    \edef#1{%
      \expandafter\lstU@CRLFtoLF@aux#1^^M^^J\@nil
    }%
  }%
  \gdef\lstU@CRLFtoLF@aux#1^^M^^J#2\@nil{%
    #1%
    \ifx\relax#2\relax
      \@car
    \fi
    ^^J%
    \lstU@CRLFtoLF@aux#2\@nil
  }%
\endgroup %
%    \end{macrocode}
%    \end{macro}
%
% \subsubsection{Patch \cs{lst@InputListing}}
%
%    \begin{macrocode}
\def\lstU@temp#1\def\lst@next#2#3\@nil{%
  \def\lst@InputListing##1{%
    #1%
    \def\lst@next{\lstU@input{##1}}%
    #3%
  }%
}
\expandafter\lstU@temp\lst@InputListing{#1}\@nil
%    \end{macrocode}
%
%    \begin{macrocode}
\lstU@AtEnd%
%</package>
%    \end{macrocode}
%
% \section{Test}
%
% \subsection{Catcode checks for loading}
%
%    \begin{macrocode}
%<*test1>
%    \end{macrocode}
%    \begin{macrocode}
\NeedsTeXFormat{LaTeX2e}
\documentclass{minimal}
\makeatletter
\def\RestoreCatcodes{}
\count@=0 %
\loop
  \edef\RestoreCatcodes{%
    \RestoreCatcodes
    \catcode\the\count@=\the\catcode\count@\relax
  }%
\ifnum\count@<255 %
  \advance\count@\@ne
\repeat

\def\RangeCatcodeInvalid#1#2{%
  \count@=#1\relax
  \loop
    \catcode\count@=15 %
  \ifnum\count@<#2\relax
    \advance\count@\@ne
  \repeat
}
\def\Test{%
  \RangeCatcodeInvalid{0}{47}%
  \RangeCatcodeInvalid{58}{64}%
  \RangeCatcodeInvalid{91}{96}%
  \RangeCatcodeInvalid{123}{127}%
  \catcode`\@=12 %
  \catcode`\\=0 %
  \catcode`\{=1 %
  \catcode`\}=2 %
  \catcode`\#=6 %
  \catcode`\[=12 %
  \catcode`\]=12 %
  \catcode`\%=14 %
  \catcode`\ =10 %
  \catcode13=5 %
  \RequirePackage{listingsutf8}[2011/11/10]\relax
  \RestoreCatcodes
}
\Test
\csname @@end\endcsname
\end
%    \end{macrocode}
%    \begin{macrocode}
%</test1>
%    \end{macrocode}
%
% \subsection{Test example for latin1}
%
%    \begin{macrocode}
%<*test2>
%    \end{macrocode}
%    \begin{macrocode}
\NeedsTeXFormat{LaTeX2e}
\documentclass{minimal}
\usepackage{filecontents}
\def\do#1{%
  \ifx#1\^%
  \else
    \noexpand\do\noexpand#1%
  \fi
}
\expandafter\let\expandafter\dospecials\expandafter\empty
\expandafter\edef\expandafter\dospecials\expandafter{\dospecials}
\begin{filecontents*}{ExampleUTF8.java}
public class ExampleUTF8 {
    public static String testString =
        "Umlauts: " +
        "^^c3^^84^^c3^^96^^c3^^9c^^c3^^a4^^c3^^b6^^c3^^bc^^c3^^9f";
    public static void main(String[] args) {
        System.out.println(testString);
    }
}
\end{filecontents*}
\usepackage{listingsutf8}[2011/11/10]
\def\Text{%
  Umlauts: %
  ^^c3^^84^^c3^^96^^c3^^9c^^c3^^a4^^c3^^b6^^c3^^bc^^c3^^9f%
}
\begin{document}
\lstinputlisting[%
  language=Java,%
  inputencoding=utf8/latin1,%
]{ExampleUTF8.java}
\end{document}
%</test2>
%    \end{macrocode}
%
% \section{Installation}
%
% \subsection{Download}
%
% \paragraph{Package.} This package is available on
% CTAN\footnote{\url{ftp://ftp.ctan.org/tex-archive/}}:
% \begin{description}
% \item[\CTAN{macros/latex/contrib/oberdiek/listingsutf8.dtx}] The source file.
% \item[\CTAN{macros/latex/contrib/oberdiek/listingsutf8.pdf}] Documentation.
% \end{description}
%
%
% \paragraph{Bundle.} All the packages of the bundle `oberdiek'
% are also available in a TDS compliant ZIP archive. There
% the packages are already unpacked and the documentation files
% are generated. The files and directories obey the TDS standard.
% \begin{description}
% \item[\CTAN{install/macros/latex/contrib/oberdiek.tds.zip}]
% \end{description}
% \emph{TDS} refers to the standard ``A Directory Structure
% for \TeX\ Files'' (\CTAN{tds/tds.pdf}). Directories
% with \xfile{texmf} in their name are usually organized this way.
%
% \subsection{Bundle installation}
%
% \paragraph{Unpacking.} Unpack the \xfile{oberdiek.tds.zip} in the
% TDS tree (also known as \xfile{texmf} tree) of your choice.
% Example (linux):
% \begin{quote}
%   |unzip oberdiek.tds.zip -d ~/texmf|
% \end{quote}
%
% \paragraph{Script installation.}
% Check the directory \xfile{TDS:scripts/oberdiek/} for
% scripts that need further installation steps.
% Package \xpackage{attachfile2} comes with the Perl script
% \xfile{pdfatfi.pl} that should be installed in such a way
% that it can be called as \texttt{pdfatfi}.
% Example (linux):
% \begin{quote}
%   |chmod +x scripts/oberdiek/pdfatfi.pl|\\
%   |cp scripts/oberdiek/pdfatfi.pl /usr/local/bin/|
% \end{quote}
%
% \subsection{Package installation}
%
% \paragraph{Unpacking.} The \xfile{.dtx} file is a self-extracting
% \docstrip\ archive. The files are extracted by running the
% \xfile{.dtx} through \plainTeX:
% \begin{quote}
%   \verb|tex listingsutf8.dtx|
% \end{quote}
%
% \paragraph{TDS.} Now the different files must be moved into
% the different directories in your installation TDS tree
% (also known as \xfile{texmf} tree):
% \begin{quote}
% \def\t{^^A
% \begin{tabular}{@{}>{\ttfamily}l@{ $\rightarrow$ }>{\ttfamily}l@{}}
%   listingsutf8.sty & tex/latex/oberdiek/listingsutf8.sty\\
%   listingsutf8.pdf & doc/latex/oberdiek/listingsutf8.pdf\\
%   test/listingsutf8-test1.tex & doc/latex/oberdiek/test/listingsutf8-test1.tex\\
%   test/listingsutf8-test2.tex & doc/latex/oberdiek/test/listingsutf8-test2.tex\\
%   test/listingsutf8-test3.tex & doc/latex/oberdiek/test/listingsutf8-test3.tex\\
%   test/listingsutf8-test4.tex & doc/latex/oberdiek/test/listingsutf8-test4.tex\\
%   test/listingsutf8-test5.tex & doc/latex/oberdiek/test/listingsutf8-test5.tex\\
%   listingsutf8.dtx & source/latex/oberdiek/listingsutf8.dtx\\
% \end{tabular}^^A
% }^^A
% \sbox0{\t}^^A
% \ifdim\wd0>\linewidth
%   \begingroup
%     \advance\linewidth by\leftmargin
%     \advance\linewidth by\rightmargin
%   \edef\x{\endgroup
%     \def\noexpand\lw{\the\linewidth}^^A
%   }\x
%   \def\lwbox{^^A
%     \leavevmode
%     \hbox to \linewidth{^^A
%       \kern-\leftmargin\relax
%       \hss
%       \usebox0
%       \hss
%       \kern-\rightmargin\relax
%     }^^A
%   }^^A
%   \ifdim\wd0>\lw
%     \sbox0{\small\t}^^A
%     \ifdim\wd0>\linewidth
%       \ifdim\wd0>\lw
%         \sbox0{\footnotesize\t}^^A
%         \ifdim\wd0>\linewidth
%           \ifdim\wd0>\lw
%             \sbox0{\scriptsize\t}^^A
%             \ifdim\wd0>\linewidth
%               \ifdim\wd0>\lw
%                 \sbox0{\tiny\t}^^A
%                 \ifdim\wd0>\linewidth
%                   \lwbox
%                 \else
%                   \usebox0
%                 \fi
%               \else
%                 \lwbox
%               \fi
%             \else
%               \usebox0
%             \fi
%           \else
%             \lwbox
%           \fi
%         \else
%           \usebox0
%         \fi
%       \else
%         \lwbox
%       \fi
%     \else
%       \usebox0
%     \fi
%   \else
%     \lwbox
%   \fi
% \else
%   \usebox0
% \fi
% \end{quote}
% If you have a \xfile{docstrip.cfg} that configures and enables \docstrip's
% TDS installing feature, then some files can already be in the right
% place, see the documentation of \docstrip.
%
% \subsection{Refresh file name databases}
%
% If your \TeX~distribution
% (\teTeX, \mikTeX, \dots) relies on file name databases, you must refresh
% these. For example, \teTeX\ users run \verb|texhash| or
% \verb|mktexlsr|.
%
% \subsection{Some details for the interested}
%
% \paragraph{Attached source.}
%
% The PDF documentation on CTAN also includes the
% \xfile{.dtx} source file. It can be extracted by
% AcrobatReader 6 or higher. Another option is \textsf{pdftk},
% e.g. unpack the file into the current directory:
% \begin{quote}
%   \verb|pdftk listingsutf8.pdf unpack_files output .|
% \end{quote}
%
% \paragraph{Unpacking with \LaTeX.}
% The \xfile{.dtx} chooses its action depending on the format:
% \begin{description}
% \item[\plainTeX:] Run \docstrip\ and extract the files.
% \item[\LaTeX:] Generate the documentation.
% \end{description}
% If you insist on using \LaTeX\ for \docstrip\ (really,
% \docstrip\ does not need \LaTeX), then inform the autodetect routine
% about your intention:
% \begin{quote}
%   \verb|latex \let\install=y\input{listingsutf8.dtx}|
% \end{quote}
% Do not forget to quote the argument according to the demands
% of your shell.
%
% \paragraph{Generating the documentation.}
% You can use both the \xfile{.dtx} or the \xfile{.drv} to generate
% the documentation. The process can be configured by the
% configuration file \xfile{ltxdoc.cfg}. For instance, put this
% line into this file, if you want to have A4 as paper format:
% \begin{quote}
%   \verb|\PassOptionsToClass{a4paper}{article}|
% \end{quote}
% An example follows how to generate the
% documentation with pdf\LaTeX:
% \begin{quote}
%\begin{verbatim}
%pdflatex listingsutf8.dtx
%makeindex -s gind.ist listingsutf8.idx
%pdflatex listingsutf8.dtx
%makeindex -s gind.ist listingsutf8.idx
%pdflatex listingsutf8.dtx
%\end{verbatim}
% \end{quote}
%
% \section{Catalogue}
%
% The following XML file can be used as source for the
% \href{http://mirror.ctan.org/help/Catalogue/catalogue.html}{\TeX\ Catalogue}.
% The elements \texttt{caption} and \texttt{description} are imported
% from the original XML file from the Catalogue.
% The name of the XML file in the Catalogue is \xfile{listingsutf8.xml}.
%    \begin{macrocode}
%<*catalogue>
<?xml version='1.0' encoding='us-ascii'?>
<!DOCTYPE entry SYSTEM 'catalogue.dtd'>
<entry datestamp='$Date$' modifier='$Author$' id='listingsutf8'>
  <name>listingsutf8</name>
  <caption>Allow UTF-8 in listings input.</caption>
  <authorref id='auth:oberdiek'/>
  <copyright owner='Heiko Oberdiek' year='2007,2011'/>
  <license type='lppl1.3'/>
  <version number='1.2'/>
  <description>
    Package <xref refid='listings'>listings</xref> does not support files
    with multi-byte encodings such as UTF-8.  In the case of
    <tt>\lstinputlisting</tt>, a simple workaround is possible if a
    one-byte encoding exists that the file can be converted to.  The
    package requires the e-TeX extensions under pdfTeX (in either PDF
    or DVI output mode).
    <p/>
    The package is part of the <xref refid='oberdiek'>oberdiek</xref> bundle.
  </description>
  <documentation details='Package documentation'
      href='ctan:/macros/latex/contrib/oberdiek/listingsutf8.pdf'/>
  <ctan file='true' path='/macros/latex/contrib/oberdiek/listingsutf8.dtx'/>
  <miktex location='oberdiek'/>
  <texlive location='oberdiek'/>
  <install path='/macros/latex/contrib/oberdiek/oberdiek.tds.zip'/>
</entry>
%</catalogue>
%    \end{macrocode}
%
% \begin{thebibliography}{9}
%
% \bibitem{inputenc}
%   Alan Jeffrey, Frank Mittelbach,
%   \textit{inputenc.sty}, 2006/05/05 v1.1b.
%   \CTAN{macros/latex/base/inputenc.dtx}
%
% \bibitem{listings}
%   Carsten Heinz, Brooks Moses:
%  \textit{The \xpackage{listings} package};
%   2007/02/22;\\
%   \CTAN{macros/latex/contrib/listings/}.
%
% \bibitem{stringenc}
%   Heiko Oberdiek:
%   \textit{The \xpackage{stringenc} package};
%   2007/10/22;\\
%   \CTAN{macros/latex/contrib/oberdiek/stringenc.pdf}.
%
% \end{thebibliography}
%
% \begin{History}
%   \begin{Version}{2007/10/22 v1.0}
%   \item
%     First version.
%   \end{Version}
%   \begin{Version}{2007/11/11 v1.1}
%   \item
%     Use of package \xpackage{pdftexcmds}.
%   \end{Version}
%   \begin{Version}{2011/11/10 v1.2}
%   \item
%     DOS line ends CR/LF normalized to LF to avoid empty lines
%     (Bug report of Thomas Benkert in de.comp.text.tex).
%   \end{Version}
% \end{History}
%
% \PrintIndex
%
% \Finale
\endinput
|
% \end{quote}
% Do not forget to quote the argument according to the demands
% of your shell.
%
% \paragraph{Generating the documentation.}
% You can use both the \xfile{.dtx} or the \xfile{.drv} to generate
% the documentation. The process can be configured by the
% configuration file \xfile{ltxdoc.cfg}. For instance, put this
% line into this file, if you want to have A4 as paper format:
% \begin{quote}
%   \verb|\PassOptionsToClass{a4paper}{article}|
% \end{quote}
% An example follows how to generate the
% documentation with pdf\LaTeX:
% \begin{quote}
%\begin{verbatim}
%pdflatex listingsutf8.dtx
%makeindex -s gind.ist listingsutf8.idx
%pdflatex listingsutf8.dtx
%makeindex -s gind.ist listingsutf8.idx
%pdflatex listingsutf8.dtx
%\end{verbatim}
% \end{quote}
%
% \section{Catalogue}
%
% The following XML file can be used as source for the
% \href{http://mirror.ctan.org/help/Catalogue/catalogue.html}{\TeX\ Catalogue}.
% The elements \texttt{caption} and \texttt{description} are imported
% from the original XML file from the Catalogue.
% The name of the XML file in the Catalogue is \xfile{listingsutf8.xml}.
%    \begin{macrocode}
%<*catalogue>
<?xml version='1.0' encoding='us-ascii'?>
<!DOCTYPE entry SYSTEM 'catalogue.dtd'>
<entry datestamp='$Date$' modifier='$Author$' id='listingsutf8'>
  <name>listingsutf8</name>
  <caption>Allow UTF-8 in listings input.</caption>
  <authorref id='auth:oberdiek'/>
  <copyright owner='Heiko Oberdiek' year='2007,2011'/>
  <license type='lppl1.3'/>
  <version number='1.2'/>
  <description>
    Package <xref refid='listings'>listings</xref> does not support files
    with multi-byte encodings such as UTF-8.  In the case of
    <tt>\lstinputlisting</tt>, a simple workaround is possible if a
    one-byte encoding exists that the file can be converted to.  The
    package requires the e-TeX extensions under pdfTeX (in either PDF
    or DVI output mode).
    <p/>
    The package is part of the <xref refid='oberdiek'>oberdiek</xref> bundle.
  </description>
  <documentation details='Package documentation'
      href='ctan:/macros/latex/contrib/oberdiek/listingsutf8.pdf'/>
  <ctan file='true' path='/macros/latex/contrib/oberdiek/listingsutf8.dtx'/>
  <miktex location='oberdiek'/>
  <texlive location='oberdiek'/>
  <install path='/macros/latex/contrib/oberdiek/oberdiek.tds.zip'/>
</entry>
%</catalogue>
%    \end{macrocode}
%
% \begin{thebibliography}{9}
%
% \bibitem{inputenc}
%   Alan Jeffrey, Frank Mittelbach,
%   \textit{inputenc.sty}, 2006/05/05 v1.1b.
%   \CTAN{macros/latex/base/inputenc.dtx}
%
% \bibitem{listings}
%   Carsten Heinz, Brooks Moses:
%  \textit{The \xpackage{listings} package};
%   2007/02/22;\\
%   \CTAN{macros/latex/contrib/listings/}.
%
% \bibitem{stringenc}
%   Heiko Oberdiek:
%   \textit{The \xpackage{stringenc} package};
%   2007/10/22;\\
%   \CTAN{macros/latex/contrib/oberdiek/stringenc.pdf}.
%
% \end{thebibliography}
%
% \begin{History}
%   \begin{Version}{2007/10/22 v1.0}
%   \item
%     First version.
%   \end{Version}
%   \begin{Version}{2007/11/11 v1.1}
%   \item
%     Use of package \xpackage{pdftexcmds}.
%   \end{Version}
%   \begin{Version}{2011/11/10 v1.2}
%   \item
%     DOS line ends CR/LF normalized to LF to avoid empty lines
%     (Bug report of Thomas Benkert in de.comp.text.tex).
%   \end{Version}
% \end{History}
%
% \PrintIndex
%
% \Finale
\endinput
|
% \end{quote}
% Do not forget to quote the argument according to the demands
% of your shell.
%
% \paragraph{Generating the documentation.}
% You can use both the \xfile{.dtx} or the \xfile{.drv} to generate
% the documentation. The process can be configured by the
% configuration file \xfile{ltxdoc.cfg}. For instance, put this
% line into this file, if you want to have A4 as paper format:
% \begin{quote}
%   \verb|\PassOptionsToClass{a4paper}{article}|
% \end{quote}
% An example follows how to generate the
% documentation with pdf\LaTeX:
% \begin{quote}
%\begin{verbatim}
%pdflatex listingsutf8.dtx
%makeindex -s gind.ist listingsutf8.idx
%pdflatex listingsutf8.dtx
%makeindex -s gind.ist listingsutf8.idx
%pdflatex listingsutf8.dtx
%\end{verbatim}
% \end{quote}
%
% \section{Catalogue}
%
% The following XML file can be used as source for the
% \href{http://mirror.ctan.org/help/Catalogue/catalogue.html}{\TeX\ Catalogue}.
% The elements \texttt{caption} and \texttt{description} are imported
% from the original XML file from the Catalogue.
% The name of the XML file in the Catalogue is \xfile{listingsutf8.xml}.
%    \begin{macrocode}
%<*catalogue>
<?xml version='1.0' encoding='us-ascii'?>
<!DOCTYPE entry SYSTEM 'catalogue.dtd'>
<entry datestamp='$Date$' modifier='$Author$' id='listingsutf8'>
  <name>listingsutf8</name>
  <caption>Allow UTF-8 in listings input.</caption>
  <authorref id='auth:oberdiek'/>
  <copyright owner='Heiko Oberdiek' year='2007,2011'/>
  <license type='lppl1.3'/>
  <version number='1.2'/>
  <description>
    Package <xref refid='listings'>listings</xref> does not support files
    with multi-byte encodings such as UTF-8.  In the case of
    <tt>\lstinputlisting</tt>, a simple workaround is possible if a
    one-byte encoding exists that the file can be converted to.  The
    package requires the e-TeX extensions under pdfTeX (in either PDF
    or DVI output mode).
    <p/>
    The package is part of the <xref refid='oberdiek'>oberdiek</xref> bundle.
  </description>
  <documentation details='Package documentation'
      href='ctan:/macros/latex/contrib/oberdiek/listingsutf8.pdf'/>
  <ctan file='true' path='/macros/latex/contrib/oberdiek/listingsutf8.dtx'/>
  <miktex location='oberdiek'/>
  <texlive location='oberdiek'/>
  <install path='/macros/latex/contrib/oberdiek/oberdiek.tds.zip'/>
</entry>
%</catalogue>
%    \end{macrocode}
%
% \begin{thebibliography}{9}
%
% \bibitem{inputenc}
%   Alan Jeffrey, Frank Mittelbach,
%   \textit{inputenc.sty}, 2006/05/05 v1.1b.
%   \CTAN{macros/latex/base/inputenc.dtx}
%
% \bibitem{listings}
%   Carsten Heinz, Brooks Moses:
%  \textit{The \xpackage{listings} package};
%   2007/02/22;\\
%   \CTAN{macros/latex/contrib/listings/}.
%
% \bibitem{stringenc}
%   Heiko Oberdiek:
%   \textit{The \xpackage{stringenc} package};
%   2007/10/22;\\
%   \CTAN{macros/latex/contrib/oberdiek/stringenc.pdf}.
%
% \end{thebibliography}
%
% \begin{History}
%   \begin{Version}{2007/10/22 v1.0}
%   \item
%     First version.
%   \end{Version}
%   \begin{Version}{2007/11/11 v1.1}
%   \item
%     Use of package \xpackage{pdftexcmds}.
%   \end{Version}
%   \begin{Version}{2011/11/10 v1.2}
%   \item
%     DOS line ends CR/LF normalized to LF to avoid empty lines
%     (Bug report of Thomas Benkert in de.comp.text.tex).
%   \end{Version}
% \end{History}
%
% \PrintIndex
%
% \Finale
\endinput
|
% \end{quote}
% Do not forget to quote the argument according to the demands
% of your shell.
%
% \paragraph{Generating the documentation.}
% You can use both the \xfile{.dtx} or the \xfile{.drv} to generate
% the documentation. The process can be configured by the
% configuration file \xfile{ltxdoc.cfg}. For instance, put this
% line into this file, if you want to have A4 as paper format:
% \begin{quote}
%   \verb|\PassOptionsToClass{a4paper}{article}|
% \end{quote}
% An example follows how to generate the
% documentation with pdf\LaTeX:
% \begin{quote}
%\begin{verbatim}
%pdflatex listingsutf8.dtx
%makeindex -s gind.ist listingsutf8.idx
%pdflatex listingsutf8.dtx
%makeindex -s gind.ist listingsutf8.idx
%pdflatex listingsutf8.dtx
%\end{verbatim}
% \end{quote}
%
% \section{Catalogue}
%
% The following XML file can be used as source for the
% \href{http://mirror.ctan.org/help/Catalogue/catalogue.html}{\TeX\ Catalogue}.
% The elements \texttt{caption} and \texttt{description} are imported
% from the original XML file from the Catalogue.
% The name of the XML file in the Catalogue is \xfile{listingsutf8.xml}.
%    \begin{macrocode}
%<*catalogue>
<?xml version='1.0' encoding='us-ascii'?>
<!DOCTYPE entry SYSTEM 'catalogue.dtd'>
<entry datestamp='$Date$' modifier='$Author$' id='listingsutf8'>
  <name>listingsutf8</name>
  <caption>Allow UTF-8 in listings input.</caption>
  <authorref id='auth:oberdiek'/>
  <copyright owner='Heiko Oberdiek' year='2007,2011'/>
  <license type='lppl1.3'/>
  <version number='1.2'/>
  <description>
    Package <xref refid='listings'>listings</xref> does not support files
    with multi-byte encodings such as UTF-8.  In the case of
    <tt>\lstinputlisting</tt>, a simple workaround is possible if a
    one-byte encoding exists that the file can be converted to.  The
    package requires the e-TeX extensions under pdfTeX (in either PDF
    or DVI output mode).
    <p/>
    The package is part of the <xref refid='oberdiek'>oberdiek</xref> bundle.
  </description>
  <documentation details='Package documentation'
      href='ctan:/macros/latex/contrib/oberdiek/listingsutf8.pdf'/>
  <ctan file='true' path='/macros/latex/contrib/oberdiek/listingsutf8.dtx'/>
  <miktex location='oberdiek'/>
  <texlive location='oberdiek'/>
  <install path='/macros/latex/contrib/oberdiek/oberdiek.tds.zip'/>
</entry>
%</catalogue>
%    \end{macrocode}
%
% \begin{thebibliography}{9}
%
% \bibitem{inputenc}
%   Alan Jeffrey, Frank Mittelbach,
%   \textit{inputenc.sty}, 2006/05/05 v1.1b.
%   \CTAN{macros/latex/base/inputenc.dtx}
%
% \bibitem{listings}
%   Carsten Heinz, Brooks Moses:
%  \textit{The \xpackage{listings} package};
%   2007/02/22;\\
%   \CTAN{macros/latex/contrib/listings/}.
%
% \bibitem{stringenc}
%   Heiko Oberdiek:
%   \textit{The \xpackage{stringenc} package};
%   2007/10/22;\\
%   \CTAN{macros/latex/contrib/oberdiek/stringenc.pdf}.
%
% \end{thebibliography}
%
% \begin{History}
%   \begin{Version}{2007/10/22 v1.0}
%   \item
%     First version.
%   \end{Version}
%   \begin{Version}{2007/11/11 v1.1}
%   \item
%     Use of package \xpackage{pdftexcmds}.
%   \end{Version}
%   \begin{Version}{2011/11/10 v1.2}
%   \item
%     DOS line ends CR/LF normalized to LF to avoid empty lines
%     (Bug report of Thomas Benkert in de.comp.text.tex).
%   \end{Version}
% \end{History}
%
% \PrintIndex
%
% \Finale
\endinput
