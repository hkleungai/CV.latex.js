% \iffalse meta-comment
%
% Copyright 1993 1994 1995 1996 1997 1998 1999 2000 2001 2002 2003 2004 2005 2006 2007 2008 2009
% The LaTeX3 Project and any individual authors listed elsewhere
% in this file. 
% 
% This file is part of the LaTeX base system.
% -------------------------------------------
% 
% It may be distributed and/or modified under the
% conditions of the LaTeX Project Public License, either version 1.3c
% of this license or (at your option) any later version.
% The latest version of this license is in
%    http://www.latex-project.org/lppl.txt
% and version 1.3c or later is part of all distributions of LaTeX 
% version 2005/12/01 or later.
% 
% This file has the LPPL maintenance status "maintained".
% 
% The list of all files belonging to the LaTeX base distribution is
% given in the file `manifest.txt'. See also `legal.txt' for additional
% information.
% 
% The list of derived (unpacked) files belonging to the distribution 
% and covered by LPPL is defined by the unpacking scripts (with 
% extension .ins) which are part of the distribution.
% 
% \fi
% Filename: ltnews01.tex
 
% This is issue 1 of LaTeX News.

\documentclass
%   [type1fonts]
   {ltnews}
 
\publicationmonth{June}
\publicationyear{1994}
\publicationissue{1}
 
\begin{document}
 
\maketitle
 
\section{Welcome to \LaTeXNews}
 
An issue of \emph{\LaTeXNews} will accompany every future release of 
\LaTeX.  It will tell you about important events, such as major bug 
fixes, newly available packages, or any other \LaTeX{} news.
 
\section{\LaTeXe---the new \LaTeX{} release}
 
The most important news is the release of \LaTeXe, the new version of
the \LaTeX{} software.  This version has better support for fonts,
graphics and colour, and will be actively maintained by the \LaTeX3
project team.  Upgrades will be issued every six months, in June and 
December.
 
\section{Why a new \LaTeX?}
 
Over the years many extensions have been developed for \LaTeX. This
is, of course, a sure sign of its continuing popularity but it has had
one unfortunate result: incompatible \LaTeX{} formats came into use at
different sites. Thus, to process documents from various places, a
site maintainer was forced to keep \LaTeX{} (with and without \NFSS),
\SLiTeX, \AmSLaTeX, and so on.  In addition, when looking at a source
file it was not always clear for which format the document was written.
 
To put an end to this unsatisfactory
situation a new release of \LaTeX{} was produced.
It brings all such extensions back under a single format and thus
prevents the proliferation of mutually incompatible dialects of
\LaTeX~2.09. The new release was available for several months as a
test version, and the final release of 1~June officially
replaces the old version.
 
\section{Processing documents with \LaTeXe}
 
Documents written for \LaTeX~2.09 will
still be read by \LaTeXe.  Any such document is run in
\emph{\LaTeX~2.09 compatibility mode}.
 
Unfortunately, compatibility mode comes with a price: it can run 
up to 50\% slower than \LaTeX~2.09 did.  If you want to run your document 
in the faster \emph{native mode}, you should try replacing the line:
\begin{verbatim}
   \documentstyle[<options>,<packages>]{<class>}
\end{verbatim}
with:
\begin{verbatim}
   \documentclass[<options>]{<class>}
   \usepackage{latexsym,<packages>}
\end{verbatim}
Unfortunately, this will not always work, because some \LaTeX~2.09 
packages will only work in \LaTeXe{} compatibility mode.  You should find 
out if there is a \LaTeXe{} version of the package available.
 
\LaTeXe{} native mode also gives access to the new features of \LaTeXe, 
described in \emph{\LaTeXe{} for authors}.
 
\section{New packages}
 
\LaTeXe{} has much better support for graphics, colour, fonts, and 
multi-lingual typesetting.  The following software should be available 
from the distributor who brought you \LaTeXe:

\begin{citations}
\item[babel] for typesetting in many languages.
\item[color] for colour support.
\item[graphics] for including images.
\item[mfnfss] for using bitmap fonts.
\item[psnfss] for using Type~1 fonts.
\item[tools]  other packages by the \LaTeX3 team.
\end{citations}
The packages come with full documentation, and are also described in 
\emph{\LaTeX: A Document Processing System} or 
\emph{The \LaTeX{} Companion}.
 
\section{Further information}
 
More information about \LaTeXe{} is to be found in:

\begin{citations}
\item[\LaTeX: A Document Preparation System]
   Leslie Lamport, \AW, 2nd ed, 1994.
\item[The \LaTeX{} Companion]
   Goossens, Mittelbach and Samarin, \AW, 1994.
\end{citations}
The \LaTeX{} distribution comes with documentation on the new features of 
\LaTeX:

\begin{citations}
\item[\LaTeXe{} for authors] 
   describes the new features of \LaTeX{} documents,
   in the file \verb|usrguide.tex|.
\item[\LaTeXe{} for class and package writers]
   describes the new features of \LaTeX{} classes and packages,
   in the file \verb|clsguide.tex|.
\item[\LaTeXe{} font selection]
   describes the new features of \LaTeX{} fonts for
   class and package writers,
   in the file \verb|fntguide.tex|.
\end{citations}
For more information on \TeX{} and \LaTeX, get in touch with your local 
\TeX{} Users Group, or the international \TeX{} Users Group, 
P.~O.~Box~869, Santa~Barbara, CA~93102-0869, USA, Fax:~+1~805~963~8358, 
EMail:~tug@tug.org.
 
\end{document}
 


