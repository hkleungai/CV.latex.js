% \iffalse meta-comment
%
% File: letltxmacro.dtx
% Version: 2010/09/02 v1.4
% Info: Let assignment for LaTeX macros
%
% Copyright (C) 2008, 2010 by
%    Heiko Oberdiek <heiko.oberdiek at googlemail.com>
%
% This work may be distributed and/or modified under the
% conditions of the LaTeX Project Public License, either
% version 1.3c of this license or (at your option) any later
% version. This version of this license is in
%    http://www.latex-project.org/lppl/lppl-1-3c.txt
% and the latest version of this license is in
%    http://www.latex-project.org/lppl.txt
% and version 1.3 or later is part of all distributions of
% LaTeX version 2005/12/01 or later.
%
% This work has the LPPL maintenance status "maintained".
%
% This Current Maintainer of this work is Heiko Oberdiek.
%
% This work consists of the main source file letltxmacro.dtx
% and the derived files
%    letltxmacro.sty, letltxmacro.pdf, letltxmacro.ins, letltxmacro.drv,
%    letltxmacro-showcases.tex, letltxmacro-test1.tex,
%    letltxmacro-test2.tex.
%
% Distribution:
%    CTAN:macros/latex/contrib/oberdiek/letltxmacro.dtx
%    CTAN:macros/latex/contrib/oberdiek/letltxmacro.pdf
%
% Unpacking:
%    (a) If letltxmacro.ins is present:
%           tex letltxmacro.ins
%    (b) Without letltxmacro.ins:
%           tex letltxmacro.dtx
%    (c) If you insist on using LaTeX
%           latex \let\install=y% \iffalse meta-comment
%
% File: letltxmacro.dtx
% Version: 2010/09/02 v1.4
% Info: Let assignment for LaTeX macros
%
% Copyright (C) 2008, 2010 by
%    Heiko Oberdiek <heiko.oberdiek at googlemail.com>
%
% This work may be distributed and/or modified under the
% conditions of the LaTeX Project Public License, either
% version 1.3c of this license or (at your option) any later
% version. This version of this license is in
%    http://www.latex-project.org/lppl/lppl-1-3c.txt
% and the latest version of this license is in
%    http://www.latex-project.org/lppl.txt
% and version 1.3 or later is part of all distributions of
% LaTeX version 2005/12/01 or later.
%
% This work has the LPPL maintenance status "maintained".
%
% This Current Maintainer of this work is Heiko Oberdiek.
%
% This work consists of the main source file letltxmacro.dtx
% and the derived files
%    letltxmacro.sty, letltxmacro.pdf, letltxmacro.ins, letltxmacro.drv,
%    letltxmacro-showcases.tex, letltxmacro-test1.tex,
%    letltxmacro-test2.tex.
%
% Distribution:
%    CTAN:macros/latex/contrib/oberdiek/letltxmacro.dtx
%    CTAN:macros/latex/contrib/oberdiek/letltxmacro.pdf
%
% Unpacking:
%    (a) If letltxmacro.ins is present:
%           tex letltxmacro.ins
%    (b) Without letltxmacro.ins:
%           tex letltxmacro.dtx
%    (c) If you insist on using LaTeX
%           latex \let\install=y% \iffalse meta-comment
%
% File: letltxmacro.dtx
% Version: 2010/09/02 v1.4
% Info: Let assignment for LaTeX macros
%
% Copyright (C) 2008, 2010 by
%    Heiko Oberdiek <heiko.oberdiek at googlemail.com>
%
% This work may be distributed and/or modified under the
% conditions of the LaTeX Project Public License, either
% version 1.3c of this license or (at your option) any later
% version. This version of this license is in
%    http://www.latex-project.org/lppl/lppl-1-3c.txt
% and the latest version of this license is in
%    http://www.latex-project.org/lppl.txt
% and version 1.3 or later is part of all distributions of
% LaTeX version 2005/12/01 or later.
%
% This work has the LPPL maintenance status "maintained".
%
% This Current Maintainer of this work is Heiko Oberdiek.
%
% This work consists of the main source file letltxmacro.dtx
% and the derived files
%    letltxmacro.sty, letltxmacro.pdf, letltxmacro.ins, letltxmacro.drv,
%    letltxmacro-showcases.tex, letltxmacro-test1.tex,
%    letltxmacro-test2.tex.
%
% Distribution:
%    CTAN:macros/latex/contrib/oberdiek/letltxmacro.dtx
%    CTAN:macros/latex/contrib/oberdiek/letltxmacro.pdf
%
% Unpacking:
%    (a) If letltxmacro.ins is present:
%           tex letltxmacro.ins
%    (b) Without letltxmacro.ins:
%           tex letltxmacro.dtx
%    (c) If you insist on using LaTeX
%           latex \let\install=y% \iffalse meta-comment
%
% File: letltxmacro.dtx
% Version: 2010/09/02 v1.4
% Info: Let assignment for LaTeX macros
%
% Copyright (C) 2008, 2010 by
%    Heiko Oberdiek <heiko.oberdiek at googlemail.com>
%
% This work may be distributed and/or modified under the
% conditions of the LaTeX Project Public License, either
% version 1.3c of this license or (at your option) any later
% version. This version of this license is in
%    http://www.latex-project.org/lppl/lppl-1-3c.txt
% and the latest version of this license is in
%    http://www.latex-project.org/lppl.txt
% and version 1.3 or later is part of all distributions of
% LaTeX version 2005/12/01 or later.
%
% This work has the LPPL maintenance status "maintained".
%
% This Current Maintainer of this work is Heiko Oberdiek.
%
% This work consists of the main source file letltxmacro.dtx
% and the derived files
%    letltxmacro.sty, letltxmacro.pdf, letltxmacro.ins, letltxmacro.drv,
%    letltxmacro-showcases.tex, letltxmacro-test1.tex,
%    letltxmacro-test2.tex.
%
% Distribution:
%    CTAN:macros/latex/contrib/oberdiek/letltxmacro.dtx
%    CTAN:macros/latex/contrib/oberdiek/letltxmacro.pdf
%
% Unpacking:
%    (a) If letltxmacro.ins is present:
%           tex letltxmacro.ins
%    (b) Without letltxmacro.ins:
%           tex letltxmacro.dtx
%    (c) If you insist on using LaTeX
%           latex \let\install=y\input{letltxmacro.dtx}
%        (quote the arguments according to the demands of your shell)
%
% Documentation:
%    (a) If letltxmacro.drv is present:
%           latex letltxmacro.drv
%    (b) Without letltxmacro.drv:
%           latex letltxmacro.dtx; ...
%    The class ltxdoc loads the configuration file ltxdoc.cfg
%    if available. Here you can specify further options, e.g.
%    use A4 as paper format:
%       \PassOptionsToClass{a4paper}{article}
%
%    Programm calls to get the documentation (example):
%       pdflatex letltxmacro.dtx
%       makeindex -s gind.ist letltxmacro.idx
%       pdflatex letltxmacro.dtx
%       makeindex -s gind.ist letltxmacro.idx
%       pdflatex letltxmacro.dtx
%
% Installation:
%    TDS:tex/latex/oberdiek/letltxmacro.sty
%    TDS:doc/latex/oberdiek/letltxmacro.pdf
%    TDS:doc/latex/oberdiek/letltxmacro-showcases.tex
%    TDS:doc/latex/oberdiek/test/letltxmacro-test1.tex
%    TDS:doc/latex/oberdiek/test/letltxmacro-test2.tex
%    TDS:source/latex/oberdiek/letltxmacro.dtx
%
%<*ignore>
\begingroup
  \catcode123=1 %
  \catcode125=2 %
  \def\x{LaTeX2e}%
\expandafter\endgroup
\ifcase 0\ifx\install y1\fi\expandafter
         \ifx\csname processbatchFile\endcsname\relax\else1\fi
         \ifx\fmtname\x\else 1\fi\relax
\else\csname fi\endcsname
%</ignore>
%<*install>
\input docstrip.tex
\Msg{************************************************************************}
\Msg{* Installation}
\Msg{* Package: letltxmacro 2010/09/02 v1.4 Let assignment for LaTeX macros (HO)}
\Msg{************************************************************************}

\keepsilent
\askforoverwritefalse

\let\MetaPrefix\relax
\preamble

This is a generated file.

Project: letltxmacro
Version: 2010/09/02 v1.4

Copyright (C) 2008, 2010 by
   Heiko Oberdiek <heiko.oberdiek at googlemail.com>

This work may be distributed and/or modified under the
conditions of the LaTeX Project Public License, either
version 1.3c of this license or (at your option) any later
version. This version of this license is in
   http://www.latex-project.org/lppl/lppl-1-3c.txt
and the latest version of this license is in
   http://www.latex-project.org/lppl.txt
and version 1.3 or later is part of all distributions of
LaTeX version 2005/12/01 or later.

This work has the LPPL maintenance status "maintained".

This Current Maintainer of this work is Heiko Oberdiek.

This work consists of the main source file letltxmacro.dtx
and the derived files
   letltxmacro.sty, letltxmacro.pdf, letltxmacro.ins, letltxmacro.drv,
   letltxmacro-showcases.tex, letltxmacro-test1.tex,
   letltxmacro-test2.tex.

\endpreamble
\let\MetaPrefix\DoubleperCent

\generate{%
  \file{letltxmacro.ins}{\from{letltxmacro.dtx}{install}}%
  \file{letltxmacro.drv}{\from{letltxmacro.dtx}{driver}}%
  \usedir{tex/latex/oberdiek}%
  \file{letltxmacro.sty}{\from{letltxmacro.dtx}{package}}%
  \usedir{doc/latex/oberdiek}%
  \file{letltxmacro-showcases.tex}{\from{letltxmacro.dtx}{showcases}}%
  \usedir{doc/latex/oberdiek/test}%
  \file{letltxmacro-test1.tex}{\from{letltxmacro.dtx}{test1}}%
  \file{letltxmacro-test2.tex}{\from{letltxmacro.dtx}{test2}}%
  \nopreamble
  \nopostamble
  \usedir{source/latex/oberdiek/catalogue}%
  \file{letltxmacro.xml}{\from{letltxmacro.dtx}{catalogue}}%
}

\catcode32=13\relax% active space
\let =\space%
\Msg{************************************************************************}
\Msg{*}
\Msg{* To finish the installation you have to move the following}
\Msg{* file into a directory searched by TeX:}
\Msg{*}
\Msg{*     letltxmacro.sty}
\Msg{*}
\Msg{* To produce the documentation run the file `letltxmacro.drv'}
\Msg{* through LaTeX.}
\Msg{*}
\Msg{* Happy TeXing!}
\Msg{*}
\Msg{************************************************************************}

\endbatchfile
%</install>
%<*ignore>
\fi
%</ignore>
%<*driver>
\NeedsTeXFormat{LaTeX2e}
\ProvidesFile{letltxmacro.drv}%
  [2010/09/02 v1.4 Let assignment for LaTeX macros (HO)]%
\documentclass{ltxdoc}
\usepackage{holtxdoc}[2011/11/22]
\begin{document}
  \DocInput{letltxmacro.dtx}%
\end{document}
%</driver>
% \fi
%
% \CheckSum{990}
%
% \CharacterTable
%  {Upper-case    \A\B\C\D\E\F\G\H\I\J\K\L\M\N\O\P\Q\R\S\T\U\V\W\X\Y\Z
%   Lower-case    \a\b\c\d\e\f\g\h\i\j\k\l\m\n\o\p\q\r\s\t\u\v\w\x\y\z
%   Digits        \0\1\2\3\4\5\6\7\8\9
%   Exclamation   \!     Double quote  \"     Hash (number) \#
%   Dollar        \$     Percent       \%     Ampersand     \&
%   Acute accent  \'     Left paren    \(     Right paren   \)
%   Asterisk      \*     Plus          \+     Comma         \,
%   Minus         \-     Point         \.     Solidus       \/
%   Colon         \:     Semicolon     \;     Less than     \<
%   Equals        \=     Greater than  \>     Question mark \?
%   Commercial at \@     Left bracket  \[     Backslash     \\
%   Right bracket \]     Circumflex    \^     Underscore    \_
%   Grave accent  \`     Left brace    \{     Vertical bar  \|
%   Right brace   \}     Tilde         \~}
%
% \GetFileInfo{letltxmacro.drv}
%
% \title{The \xpackage{letltxmacro} package}
% \date{2010/09/02 v1.4}
% \author{Heiko Oberdiek\\\xemail{heiko.oberdiek at googlemail.com}}
%
% \maketitle
%
% \begin{abstract}
% \TeX's \cs{let} assignment does not work for \LaTeX\ macros
% with optional arguments or for macros that are defined
% as robust macros by \cs{DeclareRobustCommand}. This package
% defines \cs{LetLtxMacro} that also takes care of the involved
% internal macros.
% \end{abstract}
%
% \tableofcontents
%
% \section{Documentation}
%
% If someone wants to redefine a macro with using the old
% meaning, then one method is \TeX's command \cs{let}:
%\begin{quote}
%\begin{verbatim}
%\newcommand{\Macro}{\typeout{Test Macro}}
%\let\SavedMacro=\Macro
%\renewcommand{\Macro}{%
%  \typeout{Begin}%
%  \SavedMacro
%  \typeout{End}%
%}
%\end{verbatim}
%\end{quote}
% However, this method fails, if \cs{Macro} is defined
% by \cs{DeclareRobustCommand} and/or has an optional argument.
% In both cases \LaTeX\ defines an additional internal macro
% that is forgotten in the simple \cs{let} assignment of
% the example above.
%
% \begin{declcs}{LetLtxMacro} \M{new macro} \M{old macro}
% \end{declcs}
% Macro \cs{LetLtxMacro} behaves similar to \TeX's \cs{let}
% assignment, but it takes care of macros that are
% defined by \cs{DeclareRobustCommand} and/or have optional
% arguments. Example:
%\begin{quote}
%\begin{verbatim}
%\DeclareRobustCommand{\Macro}[1][default]{...}
%\LetLtxMacro{\SavedMacro}{\Macro}
%\end{verbatim}
%\end{quote}
% Then macro \cs{SavedMacro} only uses internal macro names
% that are derived from \cs{SavedMacro}'s macro name. Macro \cs{Macro}
% can now be redefined without affecting \cs{SavedMacro}.
%
% \begin{declcs}{GlobalLetLtxMacro} \M{new macro} \M{old macro}
% \end{declcs}
% Like \cs{LetLtxMacro}, but the \meta{new macro} is defined globally.
% Since version 2010/09/02~v1.4.
%
% \subsection{Supported macro definition commands}
%
% \begin{quote}
%   \begin{tabular}{@{}ll@{}}
%     \cs{newcommand}, \cs{renewcommand} & latex/base\\
%     \cs{newenvironment}, \cs{renewenvironment} & latex/base\\
%     \cs{DeclareRobustCommand}& latex/base\\
%     \cs{newrobustcmd}, \cs{renewrobustcmd} & etoolbox\\
%     \cs{robustify} & etoolbox 2008/06/22 v1.6\\
%   \end{tabular}
% \end{quote}
%
% \StopEventually{
% }
%
% \section{Implementation}
%
% \subsection{Show cases}
%
% \subsubsection{\xfile{letltxmacro-showcases.tex}}
%
%    \begin{macrocode}
%<*showcases>
\NeedsTeXFormat{LaTeX2e}
\makeatletter
%    \end{macrocode}
%    \begin{macro}{\Line}
%    The result is displayed by macro \cs{Line}. The percent symbol
%    at line start allows easy grepping and inserting into the DTX
%    file.
%    \begin{macrocode}
\newcommand*{\Line}[1]{%
  \typeout{\@percentchar#1}%
}
%    \end{macrocode}
%    \end{macro}
%    \begin{macrocode}
\newcommand*{\ShowCmdName}[1]{%
  \@ifundefined{#1}{}{%
    \Line{%
      \space\space(\expandafter\string\csname#1\endcsname) = %
      (\expandafter\meaning\csname#1\endcsname)%
    }%
  }%
}
\newcommand*{\ShowCmds}[1]{%
  \ShowCmdName{#1}%
  \ShowCmdName{#1 }%
  \ShowCmdName{\\#1}%
  \ShowCmdName{\\#1 }%
}
\let\\\@backslashchar
%    \end{macrocode}
%    \begin{macro}{\ShowDef}
%    \begin{macrocode}
\newcommand*{\ShowDef}[2]{%
  \begingroup
    \Line{}%
    \newcommand*{\DefString}{#2}%
    \@onelevel@sanitize\DefString
    \Line{\DefString}%
    #2%
    \ShowCmds{#1}%
  \endgroup
}
%    \end{macrocode}
%    \end{macro}
%    \begin{macrocode}
\typeout{}
\Line{* LaTeX definitions:}
\ShowDef{cmd}{%
  \newcommand{\cmd}[2][default]{}%
}
\ShowDef{cmd}{%
  \DeclareRobustCommand{\cmd}{}%
}
\ShowDef{cmd}{%
  \DeclareRobustCommand{\cmd}[2][default]{}%
}
\typeout{}
%    \end{macrocode}
% The minimal version of package \xpackage{etoolbox} is 2008/06/12 v1.6a
% because it fixes \cs{robustify}.
%    \begin{macrocode}
\RequirePackage{etoolbox}[2008/06/12]%
\Line{}
\Line{* etoolbox's robust definitions:}
\ShowDef{cmd}{%
  \newrobustcmd{\cmd}{}%
}
\ShowDef{cmd}{%
  \newrobustcmd{\cmd}[2][default]{}%
}
\Line{}
\Line{* etoolbox's \string\robustify:}
\ShowDef{cmd}{%
  \newcommand{\cmd}[2][default]{} %
  \robustify{\cmd}%
}
\ShowDef{cmd}{%
  \DeclareRobustCommand{\cmd}{} %
  \robustify{\cmd}%
}
\ShowDef{cmd}{%
  \DeclareRobustCommand{\cmd}[2][default]{} %
  \robustify{\cmd}%
}
\typeout{}
\@@end
%</showcases>
%    \end{macrocode}
%
% \subsubsection{Result}
%
% \begingroup
%   \makeatletter
%   \let\org@verbatim\@verbatim
%   \def\@verbatim{^^A
%     \org@verbatim
%     \catcode`\~=\active
%   }^^A
%   \let~\textvisiblespace
%\begin{verbatim}
%* LaTeX definitions:
%
%\newcommand {\cmd }[2][default]{}
%  (\cmd) = (macro:->\@protected@testopt \cmd \\cmd {default})
%  (\\cmd) = (\long macro:[#1]#2->)
%
%\DeclareRobustCommand {\cmd }{}
%  (\cmd) = (macro:->\protect \cmd~ )
%  (\cmd~) = (\long macro:->)
%
%\DeclareRobustCommand {\cmd }[2][default]{}
%  (\cmd) = (macro:->\protect \cmd~ )
%  (\cmd~) = (macro:->\@protected@testopt \cmd~ \\cmd~ {default})
%  (\\cmd~) = (\long macro:[#1]#2->)
%
%* etoolbox's robust definitions:
%
%\newrobustcmd {\cmd }{}
%  (\cmd) = (\protected\long macro:->)
%
%\newrobustcmd {\cmd }[2][default]{}
%  (\cmd) = (\protected macro:->\@testopt \\cmd {default})
%  (\\cmd) = (\long macro:[#1]#2->)
%
%* etoolbox's \robustify:
%
%\newcommand {\cmd }[2][default]{} \robustify {\cmd }
%  (\cmd) = (\protected macro:->\@protected@testopt \cmd \\cmd {default})
%  (\\cmd) = (\long macro:[#1]#2->)
%
%\DeclareRobustCommand {\cmd }{} \robustify {\cmd }
%  (\cmd) = (\protected macro:->)
%
%\DeclareRobustCommand {\cmd }[2][default]{} \robustify {\cmd }
%  (\cmd) = (\protected macro:->\@protected@testopt \cmd~ \\cmd~ {default})
%  (\cmd~) = (macro:->\@protected@testopt \cmd~ \\cmd~ {default})
%  (\\cmd~) = (\long macro:[#1]#2->)
%\end{verbatim}
% \endgroup
%
% \subsection{Package}
%
%    \begin{macrocode}
%<*package>
%    \end{macrocode}
%
% \subsubsection{Catcodes and identification}
%
%    \begin{macrocode}
\begingroup\catcode61\catcode48\catcode32=10\relax%
  \catcode13=5 % ^^M
  \endlinechar=13 %
  \catcode123=1 % {
  \catcode125=2 % }
  \catcode64=11 % @
  \def\x{\endgroup
    \expandafter\edef\csname llm@AtEnd\endcsname{%
      \endlinechar=\the\endlinechar\relax
      \catcode13=\the\catcode13\relax
      \catcode32=\the\catcode32\relax
      \catcode35=\the\catcode35\relax
      \catcode61=\the\catcode61\relax
      \catcode64=\the\catcode64\relax
      \catcode123=\the\catcode123\relax
      \catcode125=\the\catcode125\relax
    }%
  }%
\x\catcode61\catcode48\catcode32=10\relax%
\catcode13=5 % ^^M
\endlinechar=13 %
\catcode35=6 % #
\catcode64=11 % @
\catcode123=1 % {
\catcode125=2 % }
\def\TMP@EnsureCode#1#2{%
  \edef\llm@AtEnd{%
    \llm@AtEnd
    \catcode#1=\the\catcode#1\relax
  }%
  \catcode#1=#2\relax
}
\TMP@EnsureCode{40}{12}% (
\TMP@EnsureCode{41}{12}% )
\TMP@EnsureCode{42}{12}% *
\TMP@EnsureCode{45}{12}% -
\TMP@EnsureCode{46}{12}% .
\TMP@EnsureCode{47}{12}% /
\TMP@EnsureCode{58}{12}% :
\TMP@EnsureCode{62}{12}% >
\TMP@EnsureCode{91}{12}% [
\TMP@EnsureCode{93}{12}% ]
\edef\llm@AtEnd{%
  \llm@AtEnd
  \escapechar\the\escapechar\relax
  \noexpand\endinput
}
\escapechar=92 % `\\
%    \end{macrocode}
%
%    Package identification.
%    \begin{macrocode}
\NeedsTeXFormat{LaTeX2e}
\ProvidesPackage{letltxmacro}%
  [2010/09/02 v1.4 Let assignment for LaTeX macros (HO)]
%    \end{macrocode}
%
% \subsubsection{Main macros}
%
%    \begin{macro}{\LetLtxMacro}
%    \begin{macrocode}
\newcommand*{\LetLtxMacro}{%
  \llm@ModeLetLtxMacro{}%
}
%    \end{macrocode}
%    \end{macro}
%    \begin{macro}{\GlobalLetLtxMacro}
%    \begin{macrocode}
\newcommand*{\GlobalLetLtxMacro}{%
  \llm@ModeLetLtxMacro\global
}
%    \end{macrocode}
%    \end{macro}
%
%    \begin{macro}{\llm@ModeLetLtxMacro}
%    \begin{macrocode}
\newcommand*{\llm@ModeLetLtxMacro}[3]{%
  \edef\llm@escapechar{\the\escapechar}%
  \escapechar=-1 %
  \edef\reserved@a{%
    \noexpand\protect
    \expandafter\noexpand
    \csname\string#3 \endcsname
  }%
  \ifx\reserved@a#3\relax
    #1\edef#2{%
      \noexpand\protect
      \expandafter\noexpand
      \csname\string#2 \endcsname
    }%
    #1\expandafter\let
    \csname\string#2 \expandafter\endcsname
    \csname\string#3 \endcsname
    \expandafter\llm@LetLtxMacro
        \csname\string#2 \expandafter\endcsname
        \csname\string#3 \endcsname{#1}%
  \else
    \llm@LetLtxMacro{#2}{#3}{#1}%
  \fi
  \escapechar=\llm@escapechar\relax
}
%    \end{macrocode}
%    \end{macro}
%    \begin{macro}{\llm@LetLtxMacro}
%    \begin{macrocode}
\def\llm@LetLtxMacro#1#2#3{%
  \escapechar=92 %
  \expandafter\llm@CheckParams\meaning#2:->\@nil{%
    \begingroup
      \def\@protected@testopt{%
        \expandafter\@testopt\@gobble
      }%
      \def\@testopt##1##2{%
        \toks@={##2}%
      }%
      \let\llm@testopt\@empty
      \edef\x{%
        \noexpand\@protected@testopt
        \noexpand#2%
        \expandafter\noexpand\csname\string#2\endcsname
      }%
      \expandafter\expandafter\expandafter\def
      \expandafter\expandafter\expandafter\y
      \expandafter\expandafter\expandafter{%
        \expandafter\llm@CarThree#2{}{}{}\llm@nil
      }%
      \ifx\x\y
        #2%
        \def\llm@testopt{%
          \noexpand\@protected@testopt
          \noexpand#1%
        }%
      \else
        \edef\x{%
          \noexpand\@testopt
          \expandafter\noexpand
          \csname\string#2\endcsname
        }%
        \expandafter\expandafter\expandafter\def
        \expandafter\expandafter\expandafter\y
        \expandafter\expandafter\expandafter{%
          \expandafter\llm@CarTwo#2{}{}\llm@nil
        }%
        \ifx\x\y
          #2%
          \def\llm@testopt{%
            \noexpand\@testopt
          }%
        \fi
      \fi
      \ifx\llm@testopt\@empty
      \else
        \llm@protected\xdef\llm@GlobalTemp{%
          \llm@testopt
          \expandafter\noexpand
          \csname\string#1\endcsname
          {\the\toks@}%
        }%
      \fi
    \expandafter\endgroup\ifx\llm@testopt\@empty
      #3\let#1=#2\relax
    \else
      #3\let#1=\llm@GlobalTemp
      #3\expandafter\let
          \csname\string#1\expandafter\endcsname
          \csname\string#2\endcsname
    \fi
  }{%
    #3\let#1=#2\relax
  }%
}
%    \end{macrocode}
%    \end{macro}
%    \begin{macro}{\llm@CheckParams}
%    \begin{macrocode}
\def\llm@CheckParams#1:->#2\@nil{%
  \begingroup
    \def\x{#1}%
  \ifx\x\llm@macro
    \endgroup
    \def\llm@protected{}%
    \expandafter\@firstoftwo
  \else
    \ifx\x\llm@protectedmacro
      \endgroup
      \def\llm@protected{\protected}%
      \expandafter\expandafter\expandafter\@firstoftwo
    \else
      \endgroup
      \expandafter\expandafter\expandafter\@secondoftwo
    \fi
  \fi
}
%    \end{macrocode}
%    \end{macro}
%    \begin{macro}{\llm@macro}
%    \begin{macrocode}
\def\llm@macro{macro}
\@onelevel@sanitize\llm@macro
%    \end{macrocode}
%    \end{macro}
%    \begin{macro}{\llm@protectedmacro}
%    \begin{macrocode}
\def\llm@protectedmacro{\protected macro}
\@onelevel@sanitize\llm@protectedmacro
%    \end{macrocode}
%    \end{macro}
%    \begin{macro}{\llm@CarThree}
%    \begin{macrocode}
\def\llm@CarThree#1#2#3#4\llm@nil{#1#2#3}%
%    \end{macrocode}
%    \end{macro}
%    \begin{macro}{\llm@CarTwo}
%    \begin{macrocode}
\def\llm@CarTwo#1#2#3\llm@nil{#1#2}%
%    \end{macrocode}
%    \end{macro}
%
%    \begin{macrocode}
\llm@AtEnd%
%</package>
%    \end{macrocode}
%
% \section{Test}
%
% \subsection{Catcode checks for loading}
%
%    \begin{macrocode}
%<*test1>
%    \end{macrocode}
%    \begin{macrocode}
\NeedsTeXFormat{LaTeX2e}
\documentclass{minimal}
\makeatletter
\def\RestoreCatcodes{}
\count@=0 %
\loop
  \edef\RestoreCatcodes{%
    \RestoreCatcodes
    \catcode\the\count@=\the\catcode\count@\relax
  }%
\ifnum\count@<255 %
  \advance\count@\@ne
\repeat

\def\RangeCatcodeInvalid#1#2{%
  \count@=#1\relax
  \loop
    \catcode\count@=15 %
  \ifnum\count@<#2\relax
    \advance\count@\@ne
  \repeat
}
\def\Test{%
  \RangeCatcodeInvalid{0}{47}%
  \RangeCatcodeInvalid{58}{64}%
  \RangeCatcodeInvalid{91}{96}%
  \RangeCatcodeInvalid{123}{127}%
  \catcode`\@=12 %
  \catcode`\\=0 %
  \catcode`\{=1 %
  \catcode`\}=2 %
  \catcode`\#=6 %
  \catcode`\[=12 %
  \catcode`\]=12 %
  \catcode`\%=14 %
  \catcode`\ =10 %
  \catcode13=5 %
  \RequirePackage{letltxmacro}[2010/09/02]\relax
  \RestoreCatcodes
}
\Test
\csname @@end\endcsname
\end
%    \end{macrocode}
%    \begin{macrocode}
%</test1>
%    \end{macrocode}
%
% \subsection{Package tests}
%
%    \begin{macrocode}
%<*test2>
%    \end{macrocode}
%    \begin{macrocode}
\NeedsTeXFormat{LaTeX2e}
\nofiles
\documentclass{minimal}

\usepackage{letltxmacro}[2010/09/02]

\usepackage{qstest}
\IncludeTests{*}
\LogTests{log}{*}{*}

\makeatletter

\def\TestDef#1{%
  \begingroup
    \@makeother\\%
    \@makeother\ %
    \expandafter\@TestDef\csname #1\endcsname
}
\def\@TestDef#1#2{%
    \edef\a{\expandafter\strip@prefix\meaning#1}%
    \edef\b{\detokenize{#2}}%
    \ifx\a\b
    \else
      \typeout{1 [\a]}%
      \typeout{2 [\b]}%
    \fi
  \endgroup
  \Expect*{\expandafter\strip@prefix\meaning#1}*{\detokenize{#2}}%
}
\def\TestEquals#1#2{%
  \Expect*{%
    \expandafter\ifx\csname#1\expandafter\endcsname
                    \csname#2\endcsname
      equals%
    \else
      wrong%
    \fi
  }{equals}%
}
\def\SaveA{%
  \let\SavedA\TestA
  \expandafter\let\csname\string\SavedA\expandafter\endcsname
                  \csname\string\TestA\endcsname
  \expandafter\let\csname SavedA \expandafter\endcsname
                  \csname TestA \endcsname
}
\def\CheckA{%
  \TestEquals{SavedA}{TestA}%
  \TestEquals{\string\SavedA}{\string\TestA}%
  \TestEquals{SavedA }{TestA }%
}

\begin{qstest}{robust}{robust}
  \begingroup
    \DeclareRobustCommand*{\TestA}{Test}%
    \SaveA
    \TestDef{TestA}{\protect \TestA  }%
    \CheckA
    \DeclareRobustCommand{\TestA}{Test}%
    \SaveA
    \TestDef{TestA}{\protect \TestA  }%
    \LetLtxMacro\TestB\TestA
    \TestDef{TestB}{\protect \TestB  }%
    \TestEquals{TestB }{TestA }%
    \CheckA
    \GlobalLetLtxMacro\TestC\TestA
    \CheckA
  \endgroup
  \TestDef{TestC}{\protect \TestC  }%
  \DeclareRobustCommand{\TestA}{Test}%
  \TestEquals{TestC }{TestA }%
\end{qstest}

\begin{qstest}{default}{default}
  \begingroup
    \newcommand{\TestA}[1][\relax default \empty]{TestA #1}%
    \SaveA
    \TestDef
    {TestA}{\@protected@testopt \TestA \\TestA {\relax default \empty }}%
    \LetLtxMacro\TestB\TestA
    \TestDef
    {TestB}{\@protected@testopt \TestB \\TestB {\relax default \empty }}%
    \CheckA
    \TestEquals{\string\TestB}{\string\TestA}%
    \GlobalLetLtxMacro\TestC\TestA
    \CheckA
  \endgroup
  \TestDef
  {TestC}{\@protected@testopt \TestC \\TestC {\relax default \empty }}%
  \newcommand{\TestA}[1][\relax default \empty]{TestA #1}%
  \TestEquals{\string\TestC}{\string\TestA}%
\end{qstest}

\begin{qstest}{robustdefault}{robustdefault}
  \begingroup
    \DeclareRobustCommand{\TestA}[1][\default]{TestA}%
    \SaveA
    \LetLtxMacro\TestB\TestA
    \TestDef{TestB}{\protect \TestB  }%
    \TestDef{TestB }{\@protected@testopt \TestB  \\TestB  {\default }}%
    \CheckA
    \GlobalLetLtxMacro\TestC\TestA
    \CheckA
  \endgroup
  \TestDef{TestC}{\protect \TestC  }%
  \TestDef{TestC }{\@protected@testopt \TestC  \\TestC  {\default }}%
\end{qstest}

\begin{qstest}{plain}{plain}
  \LetLtxMacro\NewRelax\relax
  \Expect*{\meaning\relax}*{\string\relax}%
  \Expect*{\meaning\NewRelax}*{\string\relax}%
  \LetLtxMacro\NewHbox\hbox
  \Expect*{\meaning\hbox}*{\string\hbox}%
  \Expect*{\meaning\NewHbox}*{\string\hbox}%
  \LetLtxMacro\NewEmpty\empty
  \Expect*{\meaning\empty}{macro:->}%
  \Expect*{\meaning\NewEmpty}{macro:->}%
  \def\TestA{\iffalse}%
  \LetLtxMacro\TestB\TestA
  \TestDef{TestB}{\iffalse }%
  \TestEquals{TestA}{TestB}%
\end{qstest}
%    \end{macrocode}
%
%    \begin{macrocode}
\usepackage{etoolbox}[2008/06/22]

\begin{qstest}{oldnewrobustcmd}{oldnewrobustcmd}
  \begingroup
    \protected\edef\TestA{%
      \noexpand\@protected@testopt
      \noexpand\TestA
      \expandafter\noexpand\csname\string\TestA\endcsname
      {\noexpand\default}%
    }%
    \expandafter\protected\expandafter
    \def\csname\string\TestA\endcsname{TestA}%
    \SaveA
    \LetLtxMacro\TestB\TestA
    \TestDef{TestB}{\@protected@testopt \TestB \\TestB {\default }}%
    \protected\edef\TestC{%
      \noexpand\@protected@testopt
      \noexpand\TestB
      \expandafter\noexpand\csname\string\TestB\endcsname
      {\noexpand\default}%
    }%
    \TestEquals{TestB}{TestC}%
    \CheckA
    \GlobalLetLtxMacro\TestD\TestA
    \CheckA
  \endgroup
  \TestDef{TestD}{\@protected@testopt \TestD \\TestD {\default }}%
  \protected\edef\TestC{%
    \noexpand\@protected@testopt
    \noexpand\TestD
    \expandafter\noexpand\csname\string\TestD\endcsname
    {\noexpand\default}%
  }%
  \TestEquals{TestD}{TestC}%
\end{qstest}

\begin{qstest}{newrobustcmd}{newrobustcmd}
  \begingroup
    \newrobustcmd{\TestA}[1][\default]{TestA}%
    \SaveA
    \LetLtxMacro\TestB\TestA
    \TestDef{TestB}{\@testopt \\TestB {\default }}%
    \protected\edef\TestC{%
      \noexpand\@testopt
      \expandafter\noexpand\csname\string\TestB\endcsname
      {\noexpand\default}%
    }%
    \TestEquals{TestB}{TestC}%
    \CheckA
    \GlobalLetLtxMacro\TestD\TestA
    \CheckA
  \endgroup
  \TestDef{TestD}{\@testopt \\TestD {\default }}%
  \protected\edef\TestC{%
    \noexpand\@testopt
    \expandafter\noexpand\csname\string\TestD\endcsname
    {\noexpand\default}%
  }%
  \TestEquals{TestD}{TestC}%
\end{qstest}

\begin{qstest}{robustifyopt}{robustifyopt}
  \newcommand{\TestA}[2][\default]{}%
  \TestDef{TestA}{\@protected@testopt \TestA \\TestA {\default }}%
  \robustify\TestA
  \TestDef{TestA}{\@protected@testopt \TestA \\TestA {\default }}%
  \protected\edef\TestC{%
    \noexpand\@protected@testopt
    \noexpand\TestA
    \expandafter\noexpand\csname\string\TestA\endcsname
    {\noexpand\default}%
  }%
  \TestEquals{TestA}{TestC}%
  \SaveA
  \LetLtxMacro\TestB\TestA
  \TestDef{TestB}{\@protected@testopt \TestB \\TestB {\default }}%
  \protected\edef\TestC{%
    \noexpand\@protected@testopt
    \noexpand\TestB
    \expandafter\noexpand\csname\string\TestB\endcsname
    {\noexpand\default}%
  }%
  \TestEquals{TestB}{TestC}%
  \CheckA
\end{qstest}

\begin{qstest}{robustifydeclare}{robustifydeclare}
  \DeclareRobustCommand\TestA{\iffalse}%
  \robustify\TestA
  \SaveA
  \LetLtxMacro\TestB\TestA
  \TestDef{TestB}{\iffalse }%
  % before etoolbox 2008/06/22: \protected\long\def
  % since etoolbox 2008/06/22: \protected\def
  \protected\def\TestC{\iffalse}%
  \TestEquals{TestB}{TestC}%
  \CheckA
\end{qstest}

\begin{qstest}{robustifydeclarelong}{robustifydeclarelong}
  \DeclareRobustCommand\TestA[1]{\iffalse}%
  \robustify\TestA
  \SaveA
  \LetLtxMacro\TestB\TestA
  \TestDef{TestB}{\iffalse }%
  % before etoolbox 2008/06/22: \protected\long\def
  % since etoolbox 2008/06/22: \protected\def
  \protected\long\def\TestC#1{\iffalse}%
  \TestEquals{TestB}{TestC}%
  \CheckA
\end{qstest}
%    \end{macrocode}
%
%    \begin{macrocode}
\csname @@end\endcsname
%    \end{macrocode}
%    \begin{macrocode}
%</test2>
%    \end{macrocode}
%
% \section{Installation}
%
% \subsection{Download}
%
% \paragraph{Package.} This package is available on
% CTAN\footnote{\url{ftp://ftp.ctan.org/tex-archive/}}:
% \begin{description}
% \item[\CTAN{macros/latex/contrib/oberdiek/letltxmacro.dtx}] The source file.
% \item[\CTAN{macros/latex/contrib/oberdiek/letltxmacro.pdf}] Documentation.
% \end{description}
%
%
% \paragraph{Bundle.} All the packages of the bundle `oberdiek'
% are also available in a TDS compliant ZIP archive. There
% the packages are already unpacked and the documentation files
% are generated. The files and directories obey the TDS standard.
% \begin{description}
% \item[\CTAN{install/macros/latex/contrib/oberdiek.tds.zip}]
% \end{description}
% \emph{TDS} refers to the standard ``A Directory Structure
% for \TeX\ Files'' (\CTAN{tds/tds.pdf}). Directories
% with \xfile{texmf} in their name are usually organized this way.
%
% \subsection{Bundle installation}
%
% \paragraph{Unpacking.} Unpack the \xfile{oberdiek.tds.zip} in the
% TDS tree (also known as \xfile{texmf} tree) of your choice.
% Example (linux):
% \begin{quote}
%   |unzip oberdiek.tds.zip -d ~/texmf|
% \end{quote}
%
% \paragraph{Script installation.}
% Check the directory \xfile{TDS:scripts/oberdiek/} for
% scripts that need further installation steps.
% Package \xpackage{attachfile2} comes with the Perl script
% \xfile{pdfatfi.pl} that should be installed in such a way
% that it can be called as \texttt{pdfatfi}.
% Example (linux):
% \begin{quote}
%   |chmod +x scripts/oberdiek/pdfatfi.pl|\\
%   |cp scripts/oberdiek/pdfatfi.pl /usr/local/bin/|
% \end{quote}
%
% \subsection{Package installation}
%
% \paragraph{Unpacking.} The \xfile{.dtx} file is a self-extracting
% \docstrip\ archive. The files are extracted by running the
% \xfile{.dtx} through \plainTeX:
% \begin{quote}
%   \verb|tex letltxmacro.dtx|
% \end{quote}
%
% \paragraph{TDS.} Now the different files must be moved into
% the different directories in your installation TDS tree
% (also known as \xfile{texmf} tree):
% \begin{quote}
% \def\t{^^A
% \begin{tabular}{@{}>{\ttfamily}l@{ $\rightarrow$ }>{\ttfamily}l@{}}
%   letltxmacro.sty & tex/latex/oberdiek/letltxmacro.sty\\
%   letltxmacro.pdf & doc/latex/oberdiek/letltxmacro.pdf\\
%   letltxmacro-showcases.tex & doc/latex/oberdiek/letltxmacro-showcases.tex\\
%   test/letltxmacro-test1.tex & doc/latex/oberdiek/test/letltxmacro-test1.tex\\
%   test/letltxmacro-test2.tex & doc/latex/oberdiek/test/letltxmacro-test2.tex\\
%   letltxmacro.dtx & source/latex/oberdiek/letltxmacro.dtx\\
% \end{tabular}^^A
% }^^A
% \sbox0{\t}^^A
% \ifdim\wd0>\linewidth
%   \begingroup
%     \advance\linewidth by\leftmargin
%     \advance\linewidth by\rightmargin
%   \edef\x{\endgroup
%     \def\noexpand\lw{\the\linewidth}^^A
%   }\x
%   \def\lwbox{^^A
%     \leavevmode
%     \hbox to \linewidth{^^A
%       \kern-\leftmargin\relax
%       \hss
%       \usebox0
%       \hss
%       \kern-\rightmargin\relax
%     }^^A
%   }^^A
%   \ifdim\wd0>\lw
%     \sbox0{\small\t}^^A
%     \ifdim\wd0>\linewidth
%       \ifdim\wd0>\lw
%         \sbox0{\footnotesize\t}^^A
%         \ifdim\wd0>\linewidth
%           \ifdim\wd0>\lw
%             \sbox0{\scriptsize\t}^^A
%             \ifdim\wd0>\linewidth
%               \ifdim\wd0>\lw
%                 \sbox0{\tiny\t}^^A
%                 \ifdim\wd0>\linewidth
%                   \lwbox
%                 \else
%                   \usebox0
%                 \fi
%               \else
%                 \lwbox
%               \fi
%             \else
%               \usebox0
%             \fi
%           \else
%             \lwbox
%           \fi
%         \else
%           \usebox0
%         \fi
%       \else
%         \lwbox
%       \fi
%     \else
%       \usebox0
%     \fi
%   \else
%     \lwbox
%   \fi
% \else
%   \usebox0
% \fi
% \end{quote}
% If you have a \xfile{docstrip.cfg} that configures and enables \docstrip's
% TDS installing feature, then some files can already be in the right
% place, see the documentation of \docstrip.
%
% \subsection{Refresh file name databases}
%
% If your \TeX~distribution
% (\teTeX, \mikTeX, \dots) relies on file name databases, you must refresh
% these. For example, \teTeX\ users run \verb|texhash| or
% \verb|mktexlsr|.
%
% \subsection{Some details for the interested}
%
% \paragraph{Attached source.}
%
% The PDF documentation on CTAN also includes the
% \xfile{.dtx} source file. It can be extracted by
% AcrobatReader 6 or higher. Another option is \textsf{pdftk},
% e.g. unpack the file into the current directory:
% \begin{quote}
%   \verb|pdftk letltxmacro.pdf unpack_files output .|
% \end{quote}
%
% \paragraph{Unpacking with \LaTeX.}
% The \xfile{.dtx} chooses its action depending on the format:
% \begin{description}
% \item[\plainTeX:] Run \docstrip\ and extract the files.
% \item[\LaTeX:] Generate the documentation.
% \end{description}
% If you insist on using \LaTeX\ for \docstrip\ (really,
% \docstrip\ does not need \LaTeX), then inform the autodetect routine
% about your intention:
% \begin{quote}
%   \verb|latex \let\install=y\input{letltxmacro.dtx}|
% \end{quote}
% Do not forget to quote the argument according to the demands
% of your shell.
%
% \paragraph{Generating the documentation.}
% You can use both the \xfile{.dtx} or the \xfile{.drv} to generate
% the documentation. The process can be configured by the
% configuration file \xfile{ltxdoc.cfg}. For instance, put this
% line into this file, if you want to have A4 as paper format:
% \begin{quote}
%   \verb|\PassOptionsToClass{a4paper}{article}|
% \end{quote}
% An example follows how to generate the
% documentation with pdf\LaTeX:
% \begin{quote}
%\begin{verbatim}
%pdflatex letltxmacro.dtx
%makeindex -s gind.ist letltxmacro.idx
%pdflatex letltxmacro.dtx
%makeindex -s gind.ist letltxmacro.idx
%pdflatex letltxmacro.dtx
%\end{verbatim}
% \end{quote}
%
% \section{Catalogue}
%
% The following XML file can be used as source for the
% \href{http://mirror.ctan.org/help/Catalogue/catalogue.html}{\TeX\ Catalogue}.
% The elements \texttt{caption} and \texttt{description} are imported
% from the original XML file from the Catalogue.
% The name of the XML file in the Catalogue is \xfile{letltxmacro.xml}.
%    \begin{macrocode}
%<*catalogue>
<?xml version='1.0' encoding='us-ascii'?>
<!DOCTYPE entry SYSTEM 'catalogue.dtd'>
<entry datestamp='$Date$' modifier='$Author$' id='letltxmacro'>
  <name>letltxmacro</name>
  <caption>Let assignment for LaTeX macros.</caption>
  <authorref id='auth:oberdiek'/>
  <copyright owner='Heiko Oberdiek' year='2008,2010'/>
  <license type='lppl1.3'/>
  <version number='1.4'/>
  <description>
    TeX&#x2019;s <tt>\let</tt> assignment does not work for LaTeX macros
    with optional arguments or for macros that are defined
    as robust macros by <tt>\DeclareRobustCommand</tt>.  This package
    defines <tt>\LetLtxMacro</tt> that also takes care of the involved
    internal macros.
    <p/>
    The package is part of the <xref refid='oberdiek'>oberdiek</xref>
    bundle.
  </description>
  <documentation details='Package documentation'
      href='ctan:/macros/latex/contrib/oberdiek/letltxmacro.pdf'/>
  <ctan file='true' path='/macros/latex/contrib/oberdiek/letltxmacro.dtx'/>
  <miktex location='oberdiek'/>
  <texlive location='oberdiek'/>
  <install path='/macros/latex/contrib/oberdiek/oberdiek.tds.zip'/>
</entry>
%</catalogue>
%    \end{macrocode}
%
% \begin{History}
%   \begin{Version}{2008/06/09 v1.0}
%   \item
%     First version.
%   \end{Version}
%   \begin{Version}{2008/06/12 v1.1}
%   \item
%     Support for \xpackage{etoolbox}'s \cs{newrobustcmd} added.
%   \end{Version}
%   \begin{Version}{2008/06/13 v1.2}
%   \item
%     Support for \xpackage{etoolbox}'s \cs{robustify} added.
%   \end{Version}
%   \begin{Version}{2008/06/24 v1.3}
%   \item
%     Test file adapted for etoolbox 2008/06/22 v1.6.
%   \end{Version}
%   \begin{Version}{2010/09/02 v1.4}
%   \item
%     \cs{GlobalLetLtxMacro} added.
%   \end{Version}
% \end{History}
%
% \PrintIndex
%
% \Finale
\endinput

%        (quote the arguments according to the demands of your shell)
%
% Documentation:
%    (a) If letltxmacro.drv is present:
%           latex letltxmacro.drv
%    (b) Without letltxmacro.drv:
%           latex letltxmacro.dtx; ...
%    The class ltxdoc loads the configuration file ltxdoc.cfg
%    if available. Here you can specify further options, e.g.
%    use A4 as paper format:
%       \PassOptionsToClass{a4paper}{article}
%
%    Programm calls to get the documentation (example):
%       pdflatex letltxmacro.dtx
%       makeindex -s gind.ist letltxmacro.idx
%       pdflatex letltxmacro.dtx
%       makeindex -s gind.ist letltxmacro.idx
%       pdflatex letltxmacro.dtx
%
% Installation:
%    TDS:tex/latex/oberdiek/letltxmacro.sty
%    TDS:doc/latex/oberdiek/letltxmacro.pdf
%    TDS:doc/latex/oberdiek/letltxmacro-showcases.tex
%    TDS:doc/latex/oberdiek/test/letltxmacro-test1.tex
%    TDS:doc/latex/oberdiek/test/letltxmacro-test2.tex
%    TDS:source/latex/oberdiek/letltxmacro.dtx
%
%<*ignore>
\begingroup
  \catcode123=1 %
  \catcode125=2 %
  \def\x{LaTeX2e}%
\expandafter\endgroup
\ifcase 0\ifx\install y1\fi\expandafter
         \ifx\csname processbatchFile\endcsname\relax\else1\fi
         \ifx\fmtname\x\else 1\fi\relax
\else\csname fi\endcsname
%</ignore>
%<*install>
\input docstrip.tex
\Msg{************************************************************************}
\Msg{* Installation}
\Msg{* Package: letltxmacro 2010/09/02 v1.4 Let assignment for LaTeX macros (HO)}
\Msg{************************************************************************}

\keepsilent
\askforoverwritefalse

\let\MetaPrefix\relax
\preamble

This is a generated file.

Project: letltxmacro
Version: 2010/09/02 v1.4

Copyright (C) 2008, 2010 by
   Heiko Oberdiek <heiko.oberdiek at googlemail.com>

This work may be distributed and/or modified under the
conditions of the LaTeX Project Public License, either
version 1.3c of this license or (at your option) any later
version. This version of this license is in
   http://www.latex-project.org/lppl/lppl-1-3c.txt
and the latest version of this license is in
   http://www.latex-project.org/lppl.txt
and version 1.3 or later is part of all distributions of
LaTeX version 2005/12/01 or later.

This work has the LPPL maintenance status "maintained".

This Current Maintainer of this work is Heiko Oberdiek.

This work consists of the main source file letltxmacro.dtx
and the derived files
   letltxmacro.sty, letltxmacro.pdf, letltxmacro.ins, letltxmacro.drv,
   letltxmacro-showcases.tex, letltxmacro-test1.tex,
   letltxmacro-test2.tex.

\endpreamble
\let\MetaPrefix\DoubleperCent

\generate{%
  \file{letltxmacro.ins}{\from{letltxmacro.dtx}{install}}%
  \file{letltxmacro.drv}{\from{letltxmacro.dtx}{driver}}%
  \usedir{tex/latex/oberdiek}%
  \file{letltxmacro.sty}{\from{letltxmacro.dtx}{package}}%
  \usedir{doc/latex/oberdiek}%
  \file{letltxmacro-showcases.tex}{\from{letltxmacro.dtx}{showcases}}%
  \usedir{doc/latex/oberdiek/test}%
  \file{letltxmacro-test1.tex}{\from{letltxmacro.dtx}{test1}}%
  \file{letltxmacro-test2.tex}{\from{letltxmacro.dtx}{test2}}%
  \nopreamble
  \nopostamble
  \usedir{source/latex/oberdiek/catalogue}%
  \file{letltxmacro.xml}{\from{letltxmacro.dtx}{catalogue}}%
}

\catcode32=13\relax% active space
\let =\space%
\Msg{************************************************************************}
\Msg{*}
\Msg{* To finish the installation you have to move the following}
\Msg{* file into a directory searched by TeX:}
\Msg{*}
\Msg{*     letltxmacro.sty}
\Msg{*}
\Msg{* To produce the documentation run the file `letltxmacro.drv'}
\Msg{* through LaTeX.}
\Msg{*}
\Msg{* Happy TeXing!}
\Msg{*}
\Msg{************************************************************************}

\endbatchfile
%</install>
%<*ignore>
\fi
%</ignore>
%<*driver>
\NeedsTeXFormat{LaTeX2e}
\ProvidesFile{letltxmacro.drv}%
  [2010/09/02 v1.4 Let assignment for LaTeX macros (HO)]%
\documentclass{ltxdoc}
\usepackage{holtxdoc}[2011/11/22]
\begin{document}
  \DocInput{letltxmacro.dtx}%
\end{document}
%</driver>
% \fi
%
% \CheckSum{990}
%
% \CharacterTable
%  {Upper-case    \A\B\C\D\E\F\G\H\I\J\K\L\M\N\O\P\Q\R\S\T\U\V\W\X\Y\Z
%   Lower-case    \a\b\c\d\e\f\g\h\i\j\k\l\m\n\o\p\q\r\s\t\u\v\w\x\y\z
%   Digits        \0\1\2\3\4\5\6\7\8\9
%   Exclamation   \!     Double quote  \"     Hash (number) \#
%   Dollar        \$     Percent       \%     Ampersand     \&
%   Acute accent  \'     Left paren    \(     Right paren   \)
%   Asterisk      \*     Plus          \+     Comma         \,
%   Minus         \-     Point         \.     Solidus       \/
%   Colon         \:     Semicolon     \;     Less than     \<
%   Equals        \=     Greater than  \>     Question mark \?
%   Commercial at \@     Left bracket  \[     Backslash     \\
%   Right bracket \]     Circumflex    \^     Underscore    \_
%   Grave accent  \`     Left brace    \{     Vertical bar  \|
%   Right brace   \}     Tilde         \~}
%
% \GetFileInfo{letltxmacro.drv}
%
% \title{The \xpackage{letltxmacro} package}
% \date{2010/09/02 v1.4}
% \author{Heiko Oberdiek\\\xemail{heiko.oberdiek at googlemail.com}}
%
% \maketitle
%
% \begin{abstract}
% \TeX's \cs{let} assignment does not work for \LaTeX\ macros
% with optional arguments or for macros that are defined
% as robust macros by \cs{DeclareRobustCommand}. This package
% defines \cs{LetLtxMacro} that also takes care of the involved
% internal macros.
% \end{abstract}
%
% \tableofcontents
%
% \section{Documentation}
%
% If someone wants to redefine a macro with using the old
% meaning, then one method is \TeX's command \cs{let}:
%\begin{quote}
%\begin{verbatim}
%\newcommand{\Macro}{\typeout{Test Macro}}
%\let\SavedMacro=\Macro
%\renewcommand{\Macro}{%
%  \typeout{Begin}%
%  \SavedMacro
%  \typeout{End}%
%}
%\end{verbatim}
%\end{quote}
% However, this method fails, if \cs{Macro} is defined
% by \cs{DeclareRobustCommand} and/or has an optional argument.
% In both cases \LaTeX\ defines an additional internal macro
% that is forgotten in the simple \cs{let} assignment of
% the example above.
%
% \begin{declcs}{LetLtxMacro} \M{new macro} \M{old macro}
% \end{declcs}
% Macro \cs{LetLtxMacro} behaves similar to \TeX's \cs{let}
% assignment, but it takes care of macros that are
% defined by \cs{DeclareRobustCommand} and/or have optional
% arguments. Example:
%\begin{quote}
%\begin{verbatim}
%\DeclareRobustCommand{\Macro}[1][default]{...}
%\LetLtxMacro{\SavedMacro}{\Macro}
%\end{verbatim}
%\end{quote}
% Then macro \cs{SavedMacro} only uses internal macro names
% that are derived from \cs{SavedMacro}'s macro name. Macro \cs{Macro}
% can now be redefined without affecting \cs{SavedMacro}.
%
% \begin{declcs}{GlobalLetLtxMacro} \M{new macro} \M{old macro}
% \end{declcs}
% Like \cs{LetLtxMacro}, but the \meta{new macro} is defined globally.
% Since version 2010/09/02~v1.4.
%
% \subsection{Supported macro definition commands}
%
% \begin{quote}
%   \begin{tabular}{@{}ll@{}}
%     \cs{newcommand}, \cs{renewcommand} & latex/base\\
%     \cs{newenvironment}, \cs{renewenvironment} & latex/base\\
%     \cs{DeclareRobustCommand}& latex/base\\
%     \cs{newrobustcmd}, \cs{renewrobustcmd} & etoolbox\\
%     \cs{robustify} & etoolbox 2008/06/22 v1.6\\
%   \end{tabular}
% \end{quote}
%
% \StopEventually{
% }
%
% \section{Implementation}
%
% \subsection{Show cases}
%
% \subsubsection{\xfile{letltxmacro-showcases.tex}}
%
%    \begin{macrocode}
%<*showcases>
\NeedsTeXFormat{LaTeX2e}
\makeatletter
%    \end{macrocode}
%    \begin{macro}{\Line}
%    The result is displayed by macro \cs{Line}. The percent symbol
%    at line start allows easy grepping and inserting into the DTX
%    file.
%    \begin{macrocode}
\newcommand*{\Line}[1]{%
  \typeout{\@percentchar#1}%
}
%    \end{macrocode}
%    \end{macro}
%    \begin{macrocode}
\newcommand*{\ShowCmdName}[1]{%
  \@ifundefined{#1}{}{%
    \Line{%
      \space\space(\expandafter\string\csname#1\endcsname) = %
      (\expandafter\meaning\csname#1\endcsname)%
    }%
  }%
}
\newcommand*{\ShowCmds}[1]{%
  \ShowCmdName{#1}%
  \ShowCmdName{#1 }%
  \ShowCmdName{\\#1}%
  \ShowCmdName{\\#1 }%
}
\let\\\@backslashchar
%    \end{macrocode}
%    \begin{macro}{\ShowDef}
%    \begin{macrocode}
\newcommand*{\ShowDef}[2]{%
  \begingroup
    \Line{}%
    \newcommand*{\DefString}{#2}%
    \@onelevel@sanitize\DefString
    \Line{\DefString}%
    #2%
    \ShowCmds{#1}%
  \endgroup
}
%    \end{macrocode}
%    \end{macro}
%    \begin{macrocode}
\typeout{}
\Line{* LaTeX definitions:}
\ShowDef{cmd}{%
  \newcommand{\cmd}[2][default]{}%
}
\ShowDef{cmd}{%
  \DeclareRobustCommand{\cmd}{}%
}
\ShowDef{cmd}{%
  \DeclareRobustCommand{\cmd}[2][default]{}%
}
\typeout{}
%    \end{macrocode}
% The minimal version of package \xpackage{etoolbox} is 2008/06/12 v1.6a
% because it fixes \cs{robustify}.
%    \begin{macrocode}
\RequirePackage{etoolbox}[2008/06/12]%
\Line{}
\Line{* etoolbox's robust definitions:}
\ShowDef{cmd}{%
  \newrobustcmd{\cmd}{}%
}
\ShowDef{cmd}{%
  \newrobustcmd{\cmd}[2][default]{}%
}
\Line{}
\Line{* etoolbox's \string\robustify:}
\ShowDef{cmd}{%
  \newcommand{\cmd}[2][default]{} %
  \robustify{\cmd}%
}
\ShowDef{cmd}{%
  \DeclareRobustCommand{\cmd}{} %
  \robustify{\cmd}%
}
\ShowDef{cmd}{%
  \DeclareRobustCommand{\cmd}[2][default]{} %
  \robustify{\cmd}%
}
\typeout{}
\@@end
%</showcases>
%    \end{macrocode}
%
% \subsubsection{Result}
%
% \begingroup
%   \makeatletter
%   \let\org@verbatim\@verbatim
%   \def\@verbatim{^^A
%     \org@verbatim
%     \catcode`\~=\active
%   }^^A
%   \let~\textvisiblespace
%\begin{verbatim}
%* LaTeX definitions:
%
%\newcommand {\cmd }[2][default]{}
%  (\cmd) = (macro:->\@protected@testopt \cmd \\cmd {default})
%  (\\cmd) = (\long macro:[#1]#2->)
%
%\DeclareRobustCommand {\cmd }{}
%  (\cmd) = (macro:->\protect \cmd~ )
%  (\cmd~) = (\long macro:->)
%
%\DeclareRobustCommand {\cmd }[2][default]{}
%  (\cmd) = (macro:->\protect \cmd~ )
%  (\cmd~) = (macro:->\@protected@testopt \cmd~ \\cmd~ {default})
%  (\\cmd~) = (\long macro:[#1]#2->)
%
%* etoolbox's robust definitions:
%
%\newrobustcmd {\cmd }{}
%  (\cmd) = (\protected\long macro:->)
%
%\newrobustcmd {\cmd }[2][default]{}
%  (\cmd) = (\protected macro:->\@testopt \\cmd {default})
%  (\\cmd) = (\long macro:[#1]#2->)
%
%* etoolbox's \robustify:
%
%\newcommand {\cmd }[2][default]{} \robustify {\cmd }
%  (\cmd) = (\protected macro:->\@protected@testopt \cmd \\cmd {default})
%  (\\cmd) = (\long macro:[#1]#2->)
%
%\DeclareRobustCommand {\cmd }{} \robustify {\cmd }
%  (\cmd) = (\protected macro:->)
%
%\DeclareRobustCommand {\cmd }[2][default]{} \robustify {\cmd }
%  (\cmd) = (\protected macro:->\@protected@testopt \cmd~ \\cmd~ {default})
%  (\cmd~) = (macro:->\@protected@testopt \cmd~ \\cmd~ {default})
%  (\\cmd~) = (\long macro:[#1]#2->)
%\end{verbatim}
% \endgroup
%
% \subsection{Package}
%
%    \begin{macrocode}
%<*package>
%    \end{macrocode}
%
% \subsubsection{Catcodes and identification}
%
%    \begin{macrocode}
\begingroup\catcode61\catcode48\catcode32=10\relax%
  \catcode13=5 % ^^M
  \endlinechar=13 %
  \catcode123=1 % {
  \catcode125=2 % }
  \catcode64=11 % @
  \def\x{\endgroup
    \expandafter\edef\csname llm@AtEnd\endcsname{%
      \endlinechar=\the\endlinechar\relax
      \catcode13=\the\catcode13\relax
      \catcode32=\the\catcode32\relax
      \catcode35=\the\catcode35\relax
      \catcode61=\the\catcode61\relax
      \catcode64=\the\catcode64\relax
      \catcode123=\the\catcode123\relax
      \catcode125=\the\catcode125\relax
    }%
  }%
\x\catcode61\catcode48\catcode32=10\relax%
\catcode13=5 % ^^M
\endlinechar=13 %
\catcode35=6 % #
\catcode64=11 % @
\catcode123=1 % {
\catcode125=2 % }
\def\TMP@EnsureCode#1#2{%
  \edef\llm@AtEnd{%
    \llm@AtEnd
    \catcode#1=\the\catcode#1\relax
  }%
  \catcode#1=#2\relax
}
\TMP@EnsureCode{40}{12}% (
\TMP@EnsureCode{41}{12}% )
\TMP@EnsureCode{42}{12}% *
\TMP@EnsureCode{45}{12}% -
\TMP@EnsureCode{46}{12}% .
\TMP@EnsureCode{47}{12}% /
\TMP@EnsureCode{58}{12}% :
\TMP@EnsureCode{62}{12}% >
\TMP@EnsureCode{91}{12}% [
\TMP@EnsureCode{93}{12}% ]
\edef\llm@AtEnd{%
  \llm@AtEnd
  \escapechar\the\escapechar\relax
  \noexpand\endinput
}
\escapechar=92 % `\\
%    \end{macrocode}
%
%    Package identification.
%    \begin{macrocode}
\NeedsTeXFormat{LaTeX2e}
\ProvidesPackage{letltxmacro}%
  [2010/09/02 v1.4 Let assignment for LaTeX macros (HO)]
%    \end{macrocode}
%
% \subsubsection{Main macros}
%
%    \begin{macro}{\LetLtxMacro}
%    \begin{macrocode}
\newcommand*{\LetLtxMacro}{%
  \llm@ModeLetLtxMacro{}%
}
%    \end{macrocode}
%    \end{macro}
%    \begin{macro}{\GlobalLetLtxMacro}
%    \begin{macrocode}
\newcommand*{\GlobalLetLtxMacro}{%
  \llm@ModeLetLtxMacro\global
}
%    \end{macrocode}
%    \end{macro}
%
%    \begin{macro}{\llm@ModeLetLtxMacro}
%    \begin{macrocode}
\newcommand*{\llm@ModeLetLtxMacro}[3]{%
  \edef\llm@escapechar{\the\escapechar}%
  \escapechar=-1 %
  \edef\reserved@a{%
    \noexpand\protect
    \expandafter\noexpand
    \csname\string#3 \endcsname
  }%
  \ifx\reserved@a#3\relax
    #1\edef#2{%
      \noexpand\protect
      \expandafter\noexpand
      \csname\string#2 \endcsname
    }%
    #1\expandafter\let
    \csname\string#2 \expandafter\endcsname
    \csname\string#3 \endcsname
    \expandafter\llm@LetLtxMacro
        \csname\string#2 \expandafter\endcsname
        \csname\string#3 \endcsname{#1}%
  \else
    \llm@LetLtxMacro{#2}{#3}{#1}%
  \fi
  \escapechar=\llm@escapechar\relax
}
%    \end{macrocode}
%    \end{macro}
%    \begin{macro}{\llm@LetLtxMacro}
%    \begin{macrocode}
\def\llm@LetLtxMacro#1#2#3{%
  \escapechar=92 %
  \expandafter\llm@CheckParams\meaning#2:->\@nil{%
    \begingroup
      \def\@protected@testopt{%
        \expandafter\@testopt\@gobble
      }%
      \def\@testopt##1##2{%
        \toks@={##2}%
      }%
      \let\llm@testopt\@empty
      \edef\x{%
        \noexpand\@protected@testopt
        \noexpand#2%
        \expandafter\noexpand\csname\string#2\endcsname
      }%
      \expandafter\expandafter\expandafter\def
      \expandafter\expandafter\expandafter\y
      \expandafter\expandafter\expandafter{%
        \expandafter\llm@CarThree#2{}{}{}\llm@nil
      }%
      \ifx\x\y
        #2%
        \def\llm@testopt{%
          \noexpand\@protected@testopt
          \noexpand#1%
        }%
      \else
        \edef\x{%
          \noexpand\@testopt
          \expandafter\noexpand
          \csname\string#2\endcsname
        }%
        \expandafter\expandafter\expandafter\def
        \expandafter\expandafter\expandafter\y
        \expandafter\expandafter\expandafter{%
          \expandafter\llm@CarTwo#2{}{}\llm@nil
        }%
        \ifx\x\y
          #2%
          \def\llm@testopt{%
            \noexpand\@testopt
          }%
        \fi
      \fi
      \ifx\llm@testopt\@empty
      \else
        \llm@protected\xdef\llm@GlobalTemp{%
          \llm@testopt
          \expandafter\noexpand
          \csname\string#1\endcsname
          {\the\toks@}%
        }%
      \fi
    \expandafter\endgroup\ifx\llm@testopt\@empty
      #3\let#1=#2\relax
    \else
      #3\let#1=\llm@GlobalTemp
      #3\expandafter\let
          \csname\string#1\expandafter\endcsname
          \csname\string#2\endcsname
    \fi
  }{%
    #3\let#1=#2\relax
  }%
}
%    \end{macrocode}
%    \end{macro}
%    \begin{macro}{\llm@CheckParams}
%    \begin{macrocode}
\def\llm@CheckParams#1:->#2\@nil{%
  \begingroup
    \def\x{#1}%
  \ifx\x\llm@macro
    \endgroup
    \def\llm@protected{}%
    \expandafter\@firstoftwo
  \else
    \ifx\x\llm@protectedmacro
      \endgroup
      \def\llm@protected{\protected}%
      \expandafter\expandafter\expandafter\@firstoftwo
    \else
      \endgroup
      \expandafter\expandafter\expandafter\@secondoftwo
    \fi
  \fi
}
%    \end{macrocode}
%    \end{macro}
%    \begin{macro}{\llm@macro}
%    \begin{macrocode}
\def\llm@macro{macro}
\@onelevel@sanitize\llm@macro
%    \end{macrocode}
%    \end{macro}
%    \begin{macro}{\llm@protectedmacro}
%    \begin{macrocode}
\def\llm@protectedmacro{\protected macro}
\@onelevel@sanitize\llm@protectedmacro
%    \end{macrocode}
%    \end{macro}
%    \begin{macro}{\llm@CarThree}
%    \begin{macrocode}
\def\llm@CarThree#1#2#3#4\llm@nil{#1#2#3}%
%    \end{macrocode}
%    \end{macro}
%    \begin{macro}{\llm@CarTwo}
%    \begin{macrocode}
\def\llm@CarTwo#1#2#3\llm@nil{#1#2}%
%    \end{macrocode}
%    \end{macro}
%
%    \begin{macrocode}
\llm@AtEnd%
%</package>
%    \end{macrocode}
%
% \section{Test}
%
% \subsection{Catcode checks for loading}
%
%    \begin{macrocode}
%<*test1>
%    \end{macrocode}
%    \begin{macrocode}
\NeedsTeXFormat{LaTeX2e}
\documentclass{minimal}
\makeatletter
\def\RestoreCatcodes{}
\count@=0 %
\loop
  \edef\RestoreCatcodes{%
    \RestoreCatcodes
    \catcode\the\count@=\the\catcode\count@\relax
  }%
\ifnum\count@<255 %
  \advance\count@\@ne
\repeat

\def\RangeCatcodeInvalid#1#2{%
  \count@=#1\relax
  \loop
    \catcode\count@=15 %
  \ifnum\count@<#2\relax
    \advance\count@\@ne
  \repeat
}
\def\Test{%
  \RangeCatcodeInvalid{0}{47}%
  \RangeCatcodeInvalid{58}{64}%
  \RangeCatcodeInvalid{91}{96}%
  \RangeCatcodeInvalid{123}{127}%
  \catcode`\@=12 %
  \catcode`\\=0 %
  \catcode`\{=1 %
  \catcode`\}=2 %
  \catcode`\#=6 %
  \catcode`\[=12 %
  \catcode`\]=12 %
  \catcode`\%=14 %
  \catcode`\ =10 %
  \catcode13=5 %
  \RequirePackage{letltxmacro}[2010/09/02]\relax
  \RestoreCatcodes
}
\Test
\csname @@end\endcsname
\end
%    \end{macrocode}
%    \begin{macrocode}
%</test1>
%    \end{macrocode}
%
% \subsection{Package tests}
%
%    \begin{macrocode}
%<*test2>
%    \end{macrocode}
%    \begin{macrocode}
\NeedsTeXFormat{LaTeX2e}
\nofiles
\documentclass{minimal}

\usepackage{letltxmacro}[2010/09/02]

\usepackage{qstest}
\IncludeTests{*}
\LogTests{log}{*}{*}

\makeatletter

\def\TestDef#1{%
  \begingroup
    \@makeother\\%
    \@makeother\ %
    \expandafter\@TestDef\csname #1\endcsname
}
\def\@TestDef#1#2{%
    \edef\a{\expandafter\strip@prefix\meaning#1}%
    \edef\b{\detokenize{#2}}%
    \ifx\a\b
    \else
      \typeout{1 [\a]}%
      \typeout{2 [\b]}%
    \fi
  \endgroup
  \Expect*{\expandafter\strip@prefix\meaning#1}*{\detokenize{#2}}%
}
\def\TestEquals#1#2{%
  \Expect*{%
    \expandafter\ifx\csname#1\expandafter\endcsname
                    \csname#2\endcsname
      equals%
    \else
      wrong%
    \fi
  }{equals}%
}
\def\SaveA{%
  \let\SavedA\TestA
  \expandafter\let\csname\string\SavedA\expandafter\endcsname
                  \csname\string\TestA\endcsname
  \expandafter\let\csname SavedA \expandafter\endcsname
                  \csname TestA \endcsname
}
\def\CheckA{%
  \TestEquals{SavedA}{TestA}%
  \TestEquals{\string\SavedA}{\string\TestA}%
  \TestEquals{SavedA }{TestA }%
}

\begin{qstest}{robust}{robust}
  \begingroup
    \DeclareRobustCommand*{\TestA}{Test}%
    \SaveA
    \TestDef{TestA}{\protect \TestA  }%
    \CheckA
    \DeclareRobustCommand{\TestA}{Test}%
    \SaveA
    \TestDef{TestA}{\protect \TestA  }%
    \LetLtxMacro\TestB\TestA
    \TestDef{TestB}{\protect \TestB  }%
    \TestEquals{TestB }{TestA }%
    \CheckA
    \GlobalLetLtxMacro\TestC\TestA
    \CheckA
  \endgroup
  \TestDef{TestC}{\protect \TestC  }%
  \DeclareRobustCommand{\TestA}{Test}%
  \TestEquals{TestC }{TestA }%
\end{qstest}

\begin{qstest}{default}{default}
  \begingroup
    \newcommand{\TestA}[1][\relax default \empty]{TestA #1}%
    \SaveA
    \TestDef
    {TestA}{\@protected@testopt \TestA \\TestA {\relax default \empty }}%
    \LetLtxMacro\TestB\TestA
    \TestDef
    {TestB}{\@protected@testopt \TestB \\TestB {\relax default \empty }}%
    \CheckA
    \TestEquals{\string\TestB}{\string\TestA}%
    \GlobalLetLtxMacro\TestC\TestA
    \CheckA
  \endgroup
  \TestDef
  {TestC}{\@protected@testopt \TestC \\TestC {\relax default \empty }}%
  \newcommand{\TestA}[1][\relax default \empty]{TestA #1}%
  \TestEquals{\string\TestC}{\string\TestA}%
\end{qstest}

\begin{qstest}{robustdefault}{robustdefault}
  \begingroup
    \DeclareRobustCommand{\TestA}[1][\default]{TestA}%
    \SaveA
    \LetLtxMacro\TestB\TestA
    \TestDef{TestB}{\protect \TestB  }%
    \TestDef{TestB }{\@protected@testopt \TestB  \\TestB  {\default }}%
    \CheckA
    \GlobalLetLtxMacro\TestC\TestA
    \CheckA
  \endgroup
  \TestDef{TestC}{\protect \TestC  }%
  \TestDef{TestC }{\@protected@testopt \TestC  \\TestC  {\default }}%
\end{qstest}

\begin{qstest}{plain}{plain}
  \LetLtxMacro\NewRelax\relax
  \Expect*{\meaning\relax}*{\string\relax}%
  \Expect*{\meaning\NewRelax}*{\string\relax}%
  \LetLtxMacro\NewHbox\hbox
  \Expect*{\meaning\hbox}*{\string\hbox}%
  \Expect*{\meaning\NewHbox}*{\string\hbox}%
  \LetLtxMacro\NewEmpty\empty
  \Expect*{\meaning\empty}{macro:->}%
  \Expect*{\meaning\NewEmpty}{macro:->}%
  \def\TestA{\iffalse}%
  \LetLtxMacro\TestB\TestA
  \TestDef{TestB}{\iffalse }%
  \TestEquals{TestA}{TestB}%
\end{qstest}
%    \end{macrocode}
%
%    \begin{macrocode}
\usepackage{etoolbox}[2008/06/22]

\begin{qstest}{oldnewrobustcmd}{oldnewrobustcmd}
  \begingroup
    \protected\edef\TestA{%
      \noexpand\@protected@testopt
      \noexpand\TestA
      \expandafter\noexpand\csname\string\TestA\endcsname
      {\noexpand\default}%
    }%
    \expandafter\protected\expandafter
    \def\csname\string\TestA\endcsname{TestA}%
    \SaveA
    \LetLtxMacro\TestB\TestA
    \TestDef{TestB}{\@protected@testopt \TestB \\TestB {\default }}%
    \protected\edef\TestC{%
      \noexpand\@protected@testopt
      \noexpand\TestB
      \expandafter\noexpand\csname\string\TestB\endcsname
      {\noexpand\default}%
    }%
    \TestEquals{TestB}{TestC}%
    \CheckA
    \GlobalLetLtxMacro\TestD\TestA
    \CheckA
  \endgroup
  \TestDef{TestD}{\@protected@testopt \TestD \\TestD {\default }}%
  \protected\edef\TestC{%
    \noexpand\@protected@testopt
    \noexpand\TestD
    \expandafter\noexpand\csname\string\TestD\endcsname
    {\noexpand\default}%
  }%
  \TestEquals{TestD}{TestC}%
\end{qstest}

\begin{qstest}{newrobustcmd}{newrobustcmd}
  \begingroup
    \newrobustcmd{\TestA}[1][\default]{TestA}%
    \SaveA
    \LetLtxMacro\TestB\TestA
    \TestDef{TestB}{\@testopt \\TestB {\default }}%
    \protected\edef\TestC{%
      \noexpand\@testopt
      \expandafter\noexpand\csname\string\TestB\endcsname
      {\noexpand\default}%
    }%
    \TestEquals{TestB}{TestC}%
    \CheckA
    \GlobalLetLtxMacro\TestD\TestA
    \CheckA
  \endgroup
  \TestDef{TestD}{\@testopt \\TestD {\default }}%
  \protected\edef\TestC{%
    \noexpand\@testopt
    \expandafter\noexpand\csname\string\TestD\endcsname
    {\noexpand\default}%
  }%
  \TestEquals{TestD}{TestC}%
\end{qstest}

\begin{qstest}{robustifyopt}{robustifyopt}
  \newcommand{\TestA}[2][\default]{}%
  \TestDef{TestA}{\@protected@testopt \TestA \\TestA {\default }}%
  \robustify\TestA
  \TestDef{TestA}{\@protected@testopt \TestA \\TestA {\default }}%
  \protected\edef\TestC{%
    \noexpand\@protected@testopt
    \noexpand\TestA
    \expandafter\noexpand\csname\string\TestA\endcsname
    {\noexpand\default}%
  }%
  \TestEquals{TestA}{TestC}%
  \SaveA
  \LetLtxMacro\TestB\TestA
  \TestDef{TestB}{\@protected@testopt \TestB \\TestB {\default }}%
  \protected\edef\TestC{%
    \noexpand\@protected@testopt
    \noexpand\TestB
    \expandafter\noexpand\csname\string\TestB\endcsname
    {\noexpand\default}%
  }%
  \TestEquals{TestB}{TestC}%
  \CheckA
\end{qstest}

\begin{qstest}{robustifydeclare}{robustifydeclare}
  \DeclareRobustCommand\TestA{\iffalse}%
  \robustify\TestA
  \SaveA
  \LetLtxMacro\TestB\TestA
  \TestDef{TestB}{\iffalse }%
  % before etoolbox 2008/06/22: \protected\long\def
  % since etoolbox 2008/06/22: \protected\def
  \protected\def\TestC{\iffalse}%
  \TestEquals{TestB}{TestC}%
  \CheckA
\end{qstest}

\begin{qstest}{robustifydeclarelong}{robustifydeclarelong}
  \DeclareRobustCommand\TestA[1]{\iffalse}%
  \robustify\TestA
  \SaveA
  \LetLtxMacro\TestB\TestA
  \TestDef{TestB}{\iffalse }%
  % before etoolbox 2008/06/22: \protected\long\def
  % since etoolbox 2008/06/22: \protected\def
  \protected\long\def\TestC#1{\iffalse}%
  \TestEquals{TestB}{TestC}%
  \CheckA
\end{qstest}
%    \end{macrocode}
%
%    \begin{macrocode}
\csname @@end\endcsname
%    \end{macrocode}
%    \begin{macrocode}
%</test2>
%    \end{macrocode}
%
% \section{Installation}
%
% \subsection{Download}
%
% \paragraph{Package.} This package is available on
% CTAN\footnote{\url{ftp://ftp.ctan.org/tex-archive/}}:
% \begin{description}
% \item[\CTAN{macros/latex/contrib/oberdiek/letltxmacro.dtx}] The source file.
% \item[\CTAN{macros/latex/contrib/oberdiek/letltxmacro.pdf}] Documentation.
% \end{description}
%
%
% \paragraph{Bundle.} All the packages of the bundle `oberdiek'
% are also available in a TDS compliant ZIP archive. There
% the packages are already unpacked and the documentation files
% are generated. The files and directories obey the TDS standard.
% \begin{description}
% \item[\CTAN{install/macros/latex/contrib/oberdiek.tds.zip}]
% \end{description}
% \emph{TDS} refers to the standard ``A Directory Structure
% for \TeX\ Files'' (\CTAN{tds/tds.pdf}). Directories
% with \xfile{texmf} in their name are usually organized this way.
%
% \subsection{Bundle installation}
%
% \paragraph{Unpacking.} Unpack the \xfile{oberdiek.tds.zip} in the
% TDS tree (also known as \xfile{texmf} tree) of your choice.
% Example (linux):
% \begin{quote}
%   |unzip oberdiek.tds.zip -d ~/texmf|
% \end{quote}
%
% \paragraph{Script installation.}
% Check the directory \xfile{TDS:scripts/oberdiek/} for
% scripts that need further installation steps.
% Package \xpackage{attachfile2} comes with the Perl script
% \xfile{pdfatfi.pl} that should be installed in such a way
% that it can be called as \texttt{pdfatfi}.
% Example (linux):
% \begin{quote}
%   |chmod +x scripts/oberdiek/pdfatfi.pl|\\
%   |cp scripts/oberdiek/pdfatfi.pl /usr/local/bin/|
% \end{quote}
%
% \subsection{Package installation}
%
% \paragraph{Unpacking.} The \xfile{.dtx} file is a self-extracting
% \docstrip\ archive. The files are extracted by running the
% \xfile{.dtx} through \plainTeX:
% \begin{quote}
%   \verb|tex letltxmacro.dtx|
% \end{quote}
%
% \paragraph{TDS.} Now the different files must be moved into
% the different directories in your installation TDS tree
% (also known as \xfile{texmf} tree):
% \begin{quote}
% \def\t{^^A
% \begin{tabular}{@{}>{\ttfamily}l@{ $\rightarrow$ }>{\ttfamily}l@{}}
%   letltxmacro.sty & tex/latex/oberdiek/letltxmacro.sty\\
%   letltxmacro.pdf & doc/latex/oberdiek/letltxmacro.pdf\\
%   letltxmacro-showcases.tex & doc/latex/oberdiek/letltxmacro-showcases.tex\\
%   test/letltxmacro-test1.tex & doc/latex/oberdiek/test/letltxmacro-test1.tex\\
%   test/letltxmacro-test2.tex & doc/latex/oberdiek/test/letltxmacro-test2.tex\\
%   letltxmacro.dtx & source/latex/oberdiek/letltxmacro.dtx\\
% \end{tabular}^^A
% }^^A
% \sbox0{\t}^^A
% \ifdim\wd0>\linewidth
%   \begingroup
%     \advance\linewidth by\leftmargin
%     \advance\linewidth by\rightmargin
%   \edef\x{\endgroup
%     \def\noexpand\lw{\the\linewidth}^^A
%   }\x
%   \def\lwbox{^^A
%     \leavevmode
%     \hbox to \linewidth{^^A
%       \kern-\leftmargin\relax
%       \hss
%       \usebox0
%       \hss
%       \kern-\rightmargin\relax
%     }^^A
%   }^^A
%   \ifdim\wd0>\lw
%     \sbox0{\small\t}^^A
%     \ifdim\wd0>\linewidth
%       \ifdim\wd0>\lw
%         \sbox0{\footnotesize\t}^^A
%         \ifdim\wd0>\linewidth
%           \ifdim\wd0>\lw
%             \sbox0{\scriptsize\t}^^A
%             \ifdim\wd0>\linewidth
%               \ifdim\wd0>\lw
%                 \sbox0{\tiny\t}^^A
%                 \ifdim\wd0>\linewidth
%                   \lwbox
%                 \else
%                   \usebox0
%                 \fi
%               \else
%                 \lwbox
%               \fi
%             \else
%               \usebox0
%             \fi
%           \else
%             \lwbox
%           \fi
%         \else
%           \usebox0
%         \fi
%       \else
%         \lwbox
%       \fi
%     \else
%       \usebox0
%     \fi
%   \else
%     \lwbox
%   \fi
% \else
%   \usebox0
% \fi
% \end{quote}
% If you have a \xfile{docstrip.cfg} that configures and enables \docstrip's
% TDS installing feature, then some files can already be in the right
% place, see the documentation of \docstrip.
%
% \subsection{Refresh file name databases}
%
% If your \TeX~distribution
% (\teTeX, \mikTeX, \dots) relies on file name databases, you must refresh
% these. For example, \teTeX\ users run \verb|texhash| or
% \verb|mktexlsr|.
%
% \subsection{Some details for the interested}
%
% \paragraph{Attached source.}
%
% The PDF documentation on CTAN also includes the
% \xfile{.dtx} source file. It can be extracted by
% AcrobatReader 6 or higher. Another option is \textsf{pdftk},
% e.g. unpack the file into the current directory:
% \begin{quote}
%   \verb|pdftk letltxmacro.pdf unpack_files output .|
% \end{quote}
%
% \paragraph{Unpacking with \LaTeX.}
% The \xfile{.dtx} chooses its action depending on the format:
% \begin{description}
% \item[\plainTeX:] Run \docstrip\ and extract the files.
% \item[\LaTeX:] Generate the documentation.
% \end{description}
% If you insist on using \LaTeX\ for \docstrip\ (really,
% \docstrip\ does not need \LaTeX), then inform the autodetect routine
% about your intention:
% \begin{quote}
%   \verb|latex \let\install=y% \iffalse meta-comment
%
% File: letltxmacro.dtx
% Version: 2010/09/02 v1.4
% Info: Let assignment for LaTeX macros
%
% Copyright (C) 2008, 2010 by
%    Heiko Oberdiek <heiko.oberdiek at googlemail.com>
%
% This work may be distributed and/or modified under the
% conditions of the LaTeX Project Public License, either
% version 1.3c of this license or (at your option) any later
% version. This version of this license is in
%    http://www.latex-project.org/lppl/lppl-1-3c.txt
% and the latest version of this license is in
%    http://www.latex-project.org/lppl.txt
% and version 1.3 or later is part of all distributions of
% LaTeX version 2005/12/01 or later.
%
% This work has the LPPL maintenance status "maintained".
%
% This Current Maintainer of this work is Heiko Oberdiek.
%
% This work consists of the main source file letltxmacro.dtx
% and the derived files
%    letltxmacro.sty, letltxmacro.pdf, letltxmacro.ins, letltxmacro.drv,
%    letltxmacro-showcases.tex, letltxmacro-test1.tex,
%    letltxmacro-test2.tex.
%
% Distribution:
%    CTAN:macros/latex/contrib/oberdiek/letltxmacro.dtx
%    CTAN:macros/latex/contrib/oberdiek/letltxmacro.pdf
%
% Unpacking:
%    (a) If letltxmacro.ins is present:
%           tex letltxmacro.ins
%    (b) Without letltxmacro.ins:
%           tex letltxmacro.dtx
%    (c) If you insist on using LaTeX
%           latex \let\install=y\input{letltxmacro.dtx}
%        (quote the arguments according to the demands of your shell)
%
% Documentation:
%    (a) If letltxmacro.drv is present:
%           latex letltxmacro.drv
%    (b) Without letltxmacro.drv:
%           latex letltxmacro.dtx; ...
%    The class ltxdoc loads the configuration file ltxdoc.cfg
%    if available. Here you can specify further options, e.g.
%    use A4 as paper format:
%       \PassOptionsToClass{a4paper}{article}
%
%    Programm calls to get the documentation (example):
%       pdflatex letltxmacro.dtx
%       makeindex -s gind.ist letltxmacro.idx
%       pdflatex letltxmacro.dtx
%       makeindex -s gind.ist letltxmacro.idx
%       pdflatex letltxmacro.dtx
%
% Installation:
%    TDS:tex/latex/oberdiek/letltxmacro.sty
%    TDS:doc/latex/oberdiek/letltxmacro.pdf
%    TDS:doc/latex/oberdiek/letltxmacro-showcases.tex
%    TDS:doc/latex/oberdiek/test/letltxmacro-test1.tex
%    TDS:doc/latex/oberdiek/test/letltxmacro-test2.tex
%    TDS:source/latex/oberdiek/letltxmacro.dtx
%
%<*ignore>
\begingroup
  \catcode123=1 %
  \catcode125=2 %
  \def\x{LaTeX2e}%
\expandafter\endgroup
\ifcase 0\ifx\install y1\fi\expandafter
         \ifx\csname processbatchFile\endcsname\relax\else1\fi
         \ifx\fmtname\x\else 1\fi\relax
\else\csname fi\endcsname
%</ignore>
%<*install>
\input docstrip.tex
\Msg{************************************************************************}
\Msg{* Installation}
\Msg{* Package: letltxmacro 2010/09/02 v1.4 Let assignment for LaTeX macros (HO)}
\Msg{************************************************************************}

\keepsilent
\askforoverwritefalse

\let\MetaPrefix\relax
\preamble

This is a generated file.

Project: letltxmacro
Version: 2010/09/02 v1.4

Copyright (C) 2008, 2010 by
   Heiko Oberdiek <heiko.oberdiek at googlemail.com>

This work may be distributed and/or modified under the
conditions of the LaTeX Project Public License, either
version 1.3c of this license or (at your option) any later
version. This version of this license is in
   http://www.latex-project.org/lppl/lppl-1-3c.txt
and the latest version of this license is in
   http://www.latex-project.org/lppl.txt
and version 1.3 or later is part of all distributions of
LaTeX version 2005/12/01 or later.

This work has the LPPL maintenance status "maintained".

This Current Maintainer of this work is Heiko Oberdiek.

This work consists of the main source file letltxmacro.dtx
and the derived files
   letltxmacro.sty, letltxmacro.pdf, letltxmacro.ins, letltxmacro.drv,
   letltxmacro-showcases.tex, letltxmacro-test1.tex,
   letltxmacro-test2.tex.

\endpreamble
\let\MetaPrefix\DoubleperCent

\generate{%
  \file{letltxmacro.ins}{\from{letltxmacro.dtx}{install}}%
  \file{letltxmacro.drv}{\from{letltxmacro.dtx}{driver}}%
  \usedir{tex/latex/oberdiek}%
  \file{letltxmacro.sty}{\from{letltxmacro.dtx}{package}}%
  \usedir{doc/latex/oberdiek}%
  \file{letltxmacro-showcases.tex}{\from{letltxmacro.dtx}{showcases}}%
  \usedir{doc/latex/oberdiek/test}%
  \file{letltxmacro-test1.tex}{\from{letltxmacro.dtx}{test1}}%
  \file{letltxmacro-test2.tex}{\from{letltxmacro.dtx}{test2}}%
  \nopreamble
  \nopostamble
  \usedir{source/latex/oberdiek/catalogue}%
  \file{letltxmacro.xml}{\from{letltxmacro.dtx}{catalogue}}%
}

\catcode32=13\relax% active space
\let =\space%
\Msg{************************************************************************}
\Msg{*}
\Msg{* To finish the installation you have to move the following}
\Msg{* file into a directory searched by TeX:}
\Msg{*}
\Msg{*     letltxmacro.sty}
\Msg{*}
\Msg{* To produce the documentation run the file `letltxmacro.drv'}
\Msg{* through LaTeX.}
\Msg{*}
\Msg{* Happy TeXing!}
\Msg{*}
\Msg{************************************************************************}

\endbatchfile
%</install>
%<*ignore>
\fi
%</ignore>
%<*driver>
\NeedsTeXFormat{LaTeX2e}
\ProvidesFile{letltxmacro.drv}%
  [2010/09/02 v1.4 Let assignment for LaTeX macros (HO)]%
\documentclass{ltxdoc}
\usepackage{holtxdoc}[2011/11/22]
\begin{document}
  \DocInput{letltxmacro.dtx}%
\end{document}
%</driver>
% \fi
%
% \CheckSum{990}
%
% \CharacterTable
%  {Upper-case    \A\B\C\D\E\F\G\H\I\J\K\L\M\N\O\P\Q\R\S\T\U\V\W\X\Y\Z
%   Lower-case    \a\b\c\d\e\f\g\h\i\j\k\l\m\n\o\p\q\r\s\t\u\v\w\x\y\z
%   Digits        \0\1\2\3\4\5\6\7\8\9
%   Exclamation   \!     Double quote  \"     Hash (number) \#
%   Dollar        \$     Percent       \%     Ampersand     \&
%   Acute accent  \'     Left paren    \(     Right paren   \)
%   Asterisk      \*     Plus          \+     Comma         \,
%   Minus         \-     Point         \.     Solidus       \/
%   Colon         \:     Semicolon     \;     Less than     \<
%   Equals        \=     Greater than  \>     Question mark \?
%   Commercial at \@     Left bracket  \[     Backslash     \\
%   Right bracket \]     Circumflex    \^     Underscore    \_
%   Grave accent  \`     Left brace    \{     Vertical bar  \|
%   Right brace   \}     Tilde         \~}
%
% \GetFileInfo{letltxmacro.drv}
%
% \title{The \xpackage{letltxmacro} package}
% \date{2010/09/02 v1.4}
% \author{Heiko Oberdiek\\\xemail{heiko.oberdiek at googlemail.com}}
%
% \maketitle
%
% \begin{abstract}
% \TeX's \cs{let} assignment does not work for \LaTeX\ macros
% with optional arguments or for macros that are defined
% as robust macros by \cs{DeclareRobustCommand}. This package
% defines \cs{LetLtxMacro} that also takes care of the involved
% internal macros.
% \end{abstract}
%
% \tableofcontents
%
% \section{Documentation}
%
% If someone wants to redefine a macro with using the old
% meaning, then one method is \TeX's command \cs{let}:
%\begin{quote}
%\begin{verbatim}
%\newcommand{\Macro}{\typeout{Test Macro}}
%\let\SavedMacro=\Macro
%\renewcommand{\Macro}{%
%  \typeout{Begin}%
%  \SavedMacro
%  \typeout{End}%
%}
%\end{verbatim}
%\end{quote}
% However, this method fails, if \cs{Macro} is defined
% by \cs{DeclareRobustCommand} and/or has an optional argument.
% In both cases \LaTeX\ defines an additional internal macro
% that is forgotten in the simple \cs{let} assignment of
% the example above.
%
% \begin{declcs}{LetLtxMacro} \M{new macro} \M{old macro}
% \end{declcs}
% Macro \cs{LetLtxMacro} behaves similar to \TeX's \cs{let}
% assignment, but it takes care of macros that are
% defined by \cs{DeclareRobustCommand} and/or have optional
% arguments. Example:
%\begin{quote}
%\begin{verbatim}
%\DeclareRobustCommand{\Macro}[1][default]{...}
%\LetLtxMacro{\SavedMacro}{\Macro}
%\end{verbatim}
%\end{quote}
% Then macro \cs{SavedMacro} only uses internal macro names
% that are derived from \cs{SavedMacro}'s macro name. Macro \cs{Macro}
% can now be redefined without affecting \cs{SavedMacro}.
%
% \begin{declcs}{GlobalLetLtxMacro} \M{new macro} \M{old macro}
% \end{declcs}
% Like \cs{LetLtxMacro}, but the \meta{new macro} is defined globally.
% Since version 2010/09/02~v1.4.
%
% \subsection{Supported macro definition commands}
%
% \begin{quote}
%   \begin{tabular}{@{}ll@{}}
%     \cs{newcommand}, \cs{renewcommand} & latex/base\\
%     \cs{newenvironment}, \cs{renewenvironment} & latex/base\\
%     \cs{DeclareRobustCommand}& latex/base\\
%     \cs{newrobustcmd}, \cs{renewrobustcmd} & etoolbox\\
%     \cs{robustify} & etoolbox 2008/06/22 v1.6\\
%   \end{tabular}
% \end{quote}
%
% \StopEventually{
% }
%
% \section{Implementation}
%
% \subsection{Show cases}
%
% \subsubsection{\xfile{letltxmacro-showcases.tex}}
%
%    \begin{macrocode}
%<*showcases>
\NeedsTeXFormat{LaTeX2e}
\makeatletter
%    \end{macrocode}
%    \begin{macro}{\Line}
%    The result is displayed by macro \cs{Line}. The percent symbol
%    at line start allows easy grepping and inserting into the DTX
%    file.
%    \begin{macrocode}
\newcommand*{\Line}[1]{%
  \typeout{\@percentchar#1}%
}
%    \end{macrocode}
%    \end{macro}
%    \begin{macrocode}
\newcommand*{\ShowCmdName}[1]{%
  \@ifundefined{#1}{}{%
    \Line{%
      \space\space(\expandafter\string\csname#1\endcsname) = %
      (\expandafter\meaning\csname#1\endcsname)%
    }%
  }%
}
\newcommand*{\ShowCmds}[1]{%
  \ShowCmdName{#1}%
  \ShowCmdName{#1 }%
  \ShowCmdName{\\#1}%
  \ShowCmdName{\\#1 }%
}
\let\\\@backslashchar
%    \end{macrocode}
%    \begin{macro}{\ShowDef}
%    \begin{macrocode}
\newcommand*{\ShowDef}[2]{%
  \begingroup
    \Line{}%
    \newcommand*{\DefString}{#2}%
    \@onelevel@sanitize\DefString
    \Line{\DefString}%
    #2%
    \ShowCmds{#1}%
  \endgroup
}
%    \end{macrocode}
%    \end{macro}
%    \begin{macrocode}
\typeout{}
\Line{* LaTeX definitions:}
\ShowDef{cmd}{%
  \newcommand{\cmd}[2][default]{}%
}
\ShowDef{cmd}{%
  \DeclareRobustCommand{\cmd}{}%
}
\ShowDef{cmd}{%
  \DeclareRobustCommand{\cmd}[2][default]{}%
}
\typeout{}
%    \end{macrocode}
% The minimal version of package \xpackage{etoolbox} is 2008/06/12 v1.6a
% because it fixes \cs{robustify}.
%    \begin{macrocode}
\RequirePackage{etoolbox}[2008/06/12]%
\Line{}
\Line{* etoolbox's robust definitions:}
\ShowDef{cmd}{%
  \newrobustcmd{\cmd}{}%
}
\ShowDef{cmd}{%
  \newrobustcmd{\cmd}[2][default]{}%
}
\Line{}
\Line{* etoolbox's \string\robustify:}
\ShowDef{cmd}{%
  \newcommand{\cmd}[2][default]{} %
  \robustify{\cmd}%
}
\ShowDef{cmd}{%
  \DeclareRobustCommand{\cmd}{} %
  \robustify{\cmd}%
}
\ShowDef{cmd}{%
  \DeclareRobustCommand{\cmd}[2][default]{} %
  \robustify{\cmd}%
}
\typeout{}
\@@end
%</showcases>
%    \end{macrocode}
%
% \subsubsection{Result}
%
% \begingroup
%   \makeatletter
%   \let\org@verbatim\@verbatim
%   \def\@verbatim{^^A
%     \org@verbatim
%     \catcode`\~=\active
%   }^^A
%   \let~\textvisiblespace
%\begin{verbatim}
%* LaTeX definitions:
%
%\newcommand {\cmd }[2][default]{}
%  (\cmd) = (macro:->\@protected@testopt \cmd \\cmd {default})
%  (\\cmd) = (\long macro:[#1]#2->)
%
%\DeclareRobustCommand {\cmd }{}
%  (\cmd) = (macro:->\protect \cmd~ )
%  (\cmd~) = (\long macro:->)
%
%\DeclareRobustCommand {\cmd }[2][default]{}
%  (\cmd) = (macro:->\protect \cmd~ )
%  (\cmd~) = (macro:->\@protected@testopt \cmd~ \\cmd~ {default})
%  (\\cmd~) = (\long macro:[#1]#2->)
%
%* etoolbox's robust definitions:
%
%\newrobustcmd {\cmd }{}
%  (\cmd) = (\protected\long macro:->)
%
%\newrobustcmd {\cmd }[2][default]{}
%  (\cmd) = (\protected macro:->\@testopt \\cmd {default})
%  (\\cmd) = (\long macro:[#1]#2->)
%
%* etoolbox's \robustify:
%
%\newcommand {\cmd }[2][default]{} \robustify {\cmd }
%  (\cmd) = (\protected macro:->\@protected@testopt \cmd \\cmd {default})
%  (\\cmd) = (\long macro:[#1]#2->)
%
%\DeclareRobustCommand {\cmd }{} \robustify {\cmd }
%  (\cmd) = (\protected macro:->)
%
%\DeclareRobustCommand {\cmd }[2][default]{} \robustify {\cmd }
%  (\cmd) = (\protected macro:->\@protected@testopt \cmd~ \\cmd~ {default})
%  (\cmd~) = (macro:->\@protected@testopt \cmd~ \\cmd~ {default})
%  (\\cmd~) = (\long macro:[#1]#2->)
%\end{verbatim}
% \endgroup
%
% \subsection{Package}
%
%    \begin{macrocode}
%<*package>
%    \end{macrocode}
%
% \subsubsection{Catcodes and identification}
%
%    \begin{macrocode}
\begingroup\catcode61\catcode48\catcode32=10\relax%
  \catcode13=5 % ^^M
  \endlinechar=13 %
  \catcode123=1 % {
  \catcode125=2 % }
  \catcode64=11 % @
  \def\x{\endgroup
    \expandafter\edef\csname llm@AtEnd\endcsname{%
      \endlinechar=\the\endlinechar\relax
      \catcode13=\the\catcode13\relax
      \catcode32=\the\catcode32\relax
      \catcode35=\the\catcode35\relax
      \catcode61=\the\catcode61\relax
      \catcode64=\the\catcode64\relax
      \catcode123=\the\catcode123\relax
      \catcode125=\the\catcode125\relax
    }%
  }%
\x\catcode61\catcode48\catcode32=10\relax%
\catcode13=5 % ^^M
\endlinechar=13 %
\catcode35=6 % #
\catcode64=11 % @
\catcode123=1 % {
\catcode125=2 % }
\def\TMP@EnsureCode#1#2{%
  \edef\llm@AtEnd{%
    \llm@AtEnd
    \catcode#1=\the\catcode#1\relax
  }%
  \catcode#1=#2\relax
}
\TMP@EnsureCode{40}{12}% (
\TMP@EnsureCode{41}{12}% )
\TMP@EnsureCode{42}{12}% *
\TMP@EnsureCode{45}{12}% -
\TMP@EnsureCode{46}{12}% .
\TMP@EnsureCode{47}{12}% /
\TMP@EnsureCode{58}{12}% :
\TMP@EnsureCode{62}{12}% >
\TMP@EnsureCode{91}{12}% [
\TMP@EnsureCode{93}{12}% ]
\edef\llm@AtEnd{%
  \llm@AtEnd
  \escapechar\the\escapechar\relax
  \noexpand\endinput
}
\escapechar=92 % `\\
%    \end{macrocode}
%
%    Package identification.
%    \begin{macrocode}
\NeedsTeXFormat{LaTeX2e}
\ProvidesPackage{letltxmacro}%
  [2010/09/02 v1.4 Let assignment for LaTeX macros (HO)]
%    \end{macrocode}
%
% \subsubsection{Main macros}
%
%    \begin{macro}{\LetLtxMacro}
%    \begin{macrocode}
\newcommand*{\LetLtxMacro}{%
  \llm@ModeLetLtxMacro{}%
}
%    \end{macrocode}
%    \end{macro}
%    \begin{macro}{\GlobalLetLtxMacro}
%    \begin{macrocode}
\newcommand*{\GlobalLetLtxMacro}{%
  \llm@ModeLetLtxMacro\global
}
%    \end{macrocode}
%    \end{macro}
%
%    \begin{macro}{\llm@ModeLetLtxMacro}
%    \begin{macrocode}
\newcommand*{\llm@ModeLetLtxMacro}[3]{%
  \edef\llm@escapechar{\the\escapechar}%
  \escapechar=-1 %
  \edef\reserved@a{%
    \noexpand\protect
    \expandafter\noexpand
    \csname\string#3 \endcsname
  }%
  \ifx\reserved@a#3\relax
    #1\edef#2{%
      \noexpand\protect
      \expandafter\noexpand
      \csname\string#2 \endcsname
    }%
    #1\expandafter\let
    \csname\string#2 \expandafter\endcsname
    \csname\string#3 \endcsname
    \expandafter\llm@LetLtxMacro
        \csname\string#2 \expandafter\endcsname
        \csname\string#3 \endcsname{#1}%
  \else
    \llm@LetLtxMacro{#2}{#3}{#1}%
  \fi
  \escapechar=\llm@escapechar\relax
}
%    \end{macrocode}
%    \end{macro}
%    \begin{macro}{\llm@LetLtxMacro}
%    \begin{macrocode}
\def\llm@LetLtxMacro#1#2#3{%
  \escapechar=92 %
  \expandafter\llm@CheckParams\meaning#2:->\@nil{%
    \begingroup
      \def\@protected@testopt{%
        \expandafter\@testopt\@gobble
      }%
      \def\@testopt##1##2{%
        \toks@={##2}%
      }%
      \let\llm@testopt\@empty
      \edef\x{%
        \noexpand\@protected@testopt
        \noexpand#2%
        \expandafter\noexpand\csname\string#2\endcsname
      }%
      \expandafter\expandafter\expandafter\def
      \expandafter\expandafter\expandafter\y
      \expandafter\expandafter\expandafter{%
        \expandafter\llm@CarThree#2{}{}{}\llm@nil
      }%
      \ifx\x\y
        #2%
        \def\llm@testopt{%
          \noexpand\@protected@testopt
          \noexpand#1%
        }%
      \else
        \edef\x{%
          \noexpand\@testopt
          \expandafter\noexpand
          \csname\string#2\endcsname
        }%
        \expandafter\expandafter\expandafter\def
        \expandafter\expandafter\expandafter\y
        \expandafter\expandafter\expandafter{%
          \expandafter\llm@CarTwo#2{}{}\llm@nil
        }%
        \ifx\x\y
          #2%
          \def\llm@testopt{%
            \noexpand\@testopt
          }%
        \fi
      \fi
      \ifx\llm@testopt\@empty
      \else
        \llm@protected\xdef\llm@GlobalTemp{%
          \llm@testopt
          \expandafter\noexpand
          \csname\string#1\endcsname
          {\the\toks@}%
        }%
      \fi
    \expandafter\endgroup\ifx\llm@testopt\@empty
      #3\let#1=#2\relax
    \else
      #3\let#1=\llm@GlobalTemp
      #3\expandafter\let
          \csname\string#1\expandafter\endcsname
          \csname\string#2\endcsname
    \fi
  }{%
    #3\let#1=#2\relax
  }%
}
%    \end{macrocode}
%    \end{macro}
%    \begin{macro}{\llm@CheckParams}
%    \begin{macrocode}
\def\llm@CheckParams#1:->#2\@nil{%
  \begingroup
    \def\x{#1}%
  \ifx\x\llm@macro
    \endgroup
    \def\llm@protected{}%
    \expandafter\@firstoftwo
  \else
    \ifx\x\llm@protectedmacro
      \endgroup
      \def\llm@protected{\protected}%
      \expandafter\expandafter\expandafter\@firstoftwo
    \else
      \endgroup
      \expandafter\expandafter\expandafter\@secondoftwo
    \fi
  \fi
}
%    \end{macrocode}
%    \end{macro}
%    \begin{macro}{\llm@macro}
%    \begin{macrocode}
\def\llm@macro{macro}
\@onelevel@sanitize\llm@macro
%    \end{macrocode}
%    \end{macro}
%    \begin{macro}{\llm@protectedmacro}
%    \begin{macrocode}
\def\llm@protectedmacro{\protected macro}
\@onelevel@sanitize\llm@protectedmacro
%    \end{macrocode}
%    \end{macro}
%    \begin{macro}{\llm@CarThree}
%    \begin{macrocode}
\def\llm@CarThree#1#2#3#4\llm@nil{#1#2#3}%
%    \end{macrocode}
%    \end{macro}
%    \begin{macro}{\llm@CarTwo}
%    \begin{macrocode}
\def\llm@CarTwo#1#2#3\llm@nil{#1#2}%
%    \end{macrocode}
%    \end{macro}
%
%    \begin{macrocode}
\llm@AtEnd%
%</package>
%    \end{macrocode}
%
% \section{Test}
%
% \subsection{Catcode checks for loading}
%
%    \begin{macrocode}
%<*test1>
%    \end{macrocode}
%    \begin{macrocode}
\NeedsTeXFormat{LaTeX2e}
\documentclass{minimal}
\makeatletter
\def\RestoreCatcodes{}
\count@=0 %
\loop
  \edef\RestoreCatcodes{%
    \RestoreCatcodes
    \catcode\the\count@=\the\catcode\count@\relax
  }%
\ifnum\count@<255 %
  \advance\count@\@ne
\repeat

\def\RangeCatcodeInvalid#1#2{%
  \count@=#1\relax
  \loop
    \catcode\count@=15 %
  \ifnum\count@<#2\relax
    \advance\count@\@ne
  \repeat
}
\def\Test{%
  \RangeCatcodeInvalid{0}{47}%
  \RangeCatcodeInvalid{58}{64}%
  \RangeCatcodeInvalid{91}{96}%
  \RangeCatcodeInvalid{123}{127}%
  \catcode`\@=12 %
  \catcode`\\=0 %
  \catcode`\{=1 %
  \catcode`\}=2 %
  \catcode`\#=6 %
  \catcode`\[=12 %
  \catcode`\]=12 %
  \catcode`\%=14 %
  \catcode`\ =10 %
  \catcode13=5 %
  \RequirePackage{letltxmacro}[2010/09/02]\relax
  \RestoreCatcodes
}
\Test
\csname @@end\endcsname
\end
%    \end{macrocode}
%    \begin{macrocode}
%</test1>
%    \end{macrocode}
%
% \subsection{Package tests}
%
%    \begin{macrocode}
%<*test2>
%    \end{macrocode}
%    \begin{macrocode}
\NeedsTeXFormat{LaTeX2e}
\nofiles
\documentclass{minimal}

\usepackage{letltxmacro}[2010/09/02]

\usepackage{qstest}
\IncludeTests{*}
\LogTests{log}{*}{*}

\makeatletter

\def\TestDef#1{%
  \begingroup
    \@makeother\\%
    \@makeother\ %
    \expandafter\@TestDef\csname #1\endcsname
}
\def\@TestDef#1#2{%
    \edef\a{\expandafter\strip@prefix\meaning#1}%
    \edef\b{\detokenize{#2}}%
    \ifx\a\b
    \else
      \typeout{1 [\a]}%
      \typeout{2 [\b]}%
    \fi
  \endgroup
  \Expect*{\expandafter\strip@prefix\meaning#1}*{\detokenize{#2}}%
}
\def\TestEquals#1#2{%
  \Expect*{%
    \expandafter\ifx\csname#1\expandafter\endcsname
                    \csname#2\endcsname
      equals%
    \else
      wrong%
    \fi
  }{equals}%
}
\def\SaveA{%
  \let\SavedA\TestA
  \expandafter\let\csname\string\SavedA\expandafter\endcsname
                  \csname\string\TestA\endcsname
  \expandafter\let\csname SavedA \expandafter\endcsname
                  \csname TestA \endcsname
}
\def\CheckA{%
  \TestEquals{SavedA}{TestA}%
  \TestEquals{\string\SavedA}{\string\TestA}%
  \TestEquals{SavedA }{TestA }%
}

\begin{qstest}{robust}{robust}
  \begingroup
    \DeclareRobustCommand*{\TestA}{Test}%
    \SaveA
    \TestDef{TestA}{\protect \TestA  }%
    \CheckA
    \DeclareRobustCommand{\TestA}{Test}%
    \SaveA
    \TestDef{TestA}{\protect \TestA  }%
    \LetLtxMacro\TestB\TestA
    \TestDef{TestB}{\protect \TestB  }%
    \TestEquals{TestB }{TestA }%
    \CheckA
    \GlobalLetLtxMacro\TestC\TestA
    \CheckA
  \endgroup
  \TestDef{TestC}{\protect \TestC  }%
  \DeclareRobustCommand{\TestA}{Test}%
  \TestEquals{TestC }{TestA }%
\end{qstest}

\begin{qstest}{default}{default}
  \begingroup
    \newcommand{\TestA}[1][\relax default \empty]{TestA #1}%
    \SaveA
    \TestDef
    {TestA}{\@protected@testopt \TestA \\TestA {\relax default \empty }}%
    \LetLtxMacro\TestB\TestA
    \TestDef
    {TestB}{\@protected@testopt \TestB \\TestB {\relax default \empty }}%
    \CheckA
    \TestEquals{\string\TestB}{\string\TestA}%
    \GlobalLetLtxMacro\TestC\TestA
    \CheckA
  \endgroup
  \TestDef
  {TestC}{\@protected@testopt \TestC \\TestC {\relax default \empty }}%
  \newcommand{\TestA}[1][\relax default \empty]{TestA #1}%
  \TestEquals{\string\TestC}{\string\TestA}%
\end{qstest}

\begin{qstest}{robustdefault}{robustdefault}
  \begingroup
    \DeclareRobustCommand{\TestA}[1][\default]{TestA}%
    \SaveA
    \LetLtxMacro\TestB\TestA
    \TestDef{TestB}{\protect \TestB  }%
    \TestDef{TestB }{\@protected@testopt \TestB  \\TestB  {\default }}%
    \CheckA
    \GlobalLetLtxMacro\TestC\TestA
    \CheckA
  \endgroup
  \TestDef{TestC}{\protect \TestC  }%
  \TestDef{TestC }{\@protected@testopt \TestC  \\TestC  {\default }}%
\end{qstest}

\begin{qstest}{plain}{plain}
  \LetLtxMacro\NewRelax\relax
  \Expect*{\meaning\relax}*{\string\relax}%
  \Expect*{\meaning\NewRelax}*{\string\relax}%
  \LetLtxMacro\NewHbox\hbox
  \Expect*{\meaning\hbox}*{\string\hbox}%
  \Expect*{\meaning\NewHbox}*{\string\hbox}%
  \LetLtxMacro\NewEmpty\empty
  \Expect*{\meaning\empty}{macro:->}%
  \Expect*{\meaning\NewEmpty}{macro:->}%
  \def\TestA{\iffalse}%
  \LetLtxMacro\TestB\TestA
  \TestDef{TestB}{\iffalse }%
  \TestEquals{TestA}{TestB}%
\end{qstest}
%    \end{macrocode}
%
%    \begin{macrocode}
\usepackage{etoolbox}[2008/06/22]

\begin{qstest}{oldnewrobustcmd}{oldnewrobustcmd}
  \begingroup
    \protected\edef\TestA{%
      \noexpand\@protected@testopt
      \noexpand\TestA
      \expandafter\noexpand\csname\string\TestA\endcsname
      {\noexpand\default}%
    }%
    \expandafter\protected\expandafter
    \def\csname\string\TestA\endcsname{TestA}%
    \SaveA
    \LetLtxMacro\TestB\TestA
    \TestDef{TestB}{\@protected@testopt \TestB \\TestB {\default }}%
    \protected\edef\TestC{%
      \noexpand\@protected@testopt
      \noexpand\TestB
      \expandafter\noexpand\csname\string\TestB\endcsname
      {\noexpand\default}%
    }%
    \TestEquals{TestB}{TestC}%
    \CheckA
    \GlobalLetLtxMacro\TestD\TestA
    \CheckA
  \endgroup
  \TestDef{TestD}{\@protected@testopt \TestD \\TestD {\default }}%
  \protected\edef\TestC{%
    \noexpand\@protected@testopt
    \noexpand\TestD
    \expandafter\noexpand\csname\string\TestD\endcsname
    {\noexpand\default}%
  }%
  \TestEquals{TestD}{TestC}%
\end{qstest}

\begin{qstest}{newrobustcmd}{newrobustcmd}
  \begingroup
    \newrobustcmd{\TestA}[1][\default]{TestA}%
    \SaveA
    \LetLtxMacro\TestB\TestA
    \TestDef{TestB}{\@testopt \\TestB {\default }}%
    \protected\edef\TestC{%
      \noexpand\@testopt
      \expandafter\noexpand\csname\string\TestB\endcsname
      {\noexpand\default}%
    }%
    \TestEquals{TestB}{TestC}%
    \CheckA
    \GlobalLetLtxMacro\TestD\TestA
    \CheckA
  \endgroup
  \TestDef{TestD}{\@testopt \\TestD {\default }}%
  \protected\edef\TestC{%
    \noexpand\@testopt
    \expandafter\noexpand\csname\string\TestD\endcsname
    {\noexpand\default}%
  }%
  \TestEquals{TestD}{TestC}%
\end{qstest}

\begin{qstest}{robustifyopt}{robustifyopt}
  \newcommand{\TestA}[2][\default]{}%
  \TestDef{TestA}{\@protected@testopt \TestA \\TestA {\default }}%
  \robustify\TestA
  \TestDef{TestA}{\@protected@testopt \TestA \\TestA {\default }}%
  \protected\edef\TestC{%
    \noexpand\@protected@testopt
    \noexpand\TestA
    \expandafter\noexpand\csname\string\TestA\endcsname
    {\noexpand\default}%
  }%
  \TestEquals{TestA}{TestC}%
  \SaveA
  \LetLtxMacro\TestB\TestA
  \TestDef{TestB}{\@protected@testopt \TestB \\TestB {\default }}%
  \protected\edef\TestC{%
    \noexpand\@protected@testopt
    \noexpand\TestB
    \expandafter\noexpand\csname\string\TestB\endcsname
    {\noexpand\default}%
  }%
  \TestEquals{TestB}{TestC}%
  \CheckA
\end{qstest}

\begin{qstest}{robustifydeclare}{robustifydeclare}
  \DeclareRobustCommand\TestA{\iffalse}%
  \robustify\TestA
  \SaveA
  \LetLtxMacro\TestB\TestA
  \TestDef{TestB}{\iffalse }%
  % before etoolbox 2008/06/22: \protected\long\def
  % since etoolbox 2008/06/22: \protected\def
  \protected\def\TestC{\iffalse}%
  \TestEquals{TestB}{TestC}%
  \CheckA
\end{qstest}

\begin{qstest}{robustifydeclarelong}{robustifydeclarelong}
  \DeclareRobustCommand\TestA[1]{\iffalse}%
  \robustify\TestA
  \SaveA
  \LetLtxMacro\TestB\TestA
  \TestDef{TestB}{\iffalse }%
  % before etoolbox 2008/06/22: \protected\long\def
  % since etoolbox 2008/06/22: \protected\def
  \protected\long\def\TestC#1{\iffalse}%
  \TestEquals{TestB}{TestC}%
  \CheckA
\end{qstest}
%    \end{macrocode}
%
%    \begin{macrocode}
\csname @@end\endcsname
%    \end{macrocode}
%    \begin{macrocode}
%</test2>
%    \end{macrocode}
%
% \section{Installation}
%
% \subsection{Download}
%
% \paragraph{Package.} This package is available on
% CTAN\footnote{\url{ftp://ftp.ctan.org/tex-archive/}}:
% \begin{description}
% \item[\CTAN{macros/latex/contrib/oberdiek/letltxmacro.dtx}] The source file.
% \item[\CTAN{macros/latex/contrib/oberdiek/letltxmacro.pdf}] Documentation.
% \end{description}
%
%
% \paragraph{Bundle.} All the packages of the bundle `oberdiek'
% are also available in a TDS compliant ZIP archive. There
% the packages are already unpacked and the documentation files
% are generated. The files and directories obey the TDS standard.
% \begin{description}
% \item[\CTAN{install/macros/latex/contrib/oberdiek.tds.zip}]
% \end{description}
% \emph{TDS} refers to the standard ``A Directory Structure
% for \TeX\ Files'' (\CTAN{tds/tds.pdf}). Directories
% with \xfile{texmf} in their name are usually organized this way.
%
% \subsection{Bundle installation}
%
% \paragraph{Unpacking.} Unpack the \xfile{oberdiek.tds.zip} in the
% TDS tree (also known as \xfile{texmf} tree) of your choice.
% Example (linux):
% \begin{quote}
%   |unzip oberdiek.tds.zip -d ~/texmf|
% \end{quote}
%
% \paragraph{Script installation.}
% Check the directory \xfile{TDS:scripts/oberdiek/} for
% scripts that need further installation steps.
% Package \xpackage{attachfile2} comes with the Perl script
% \xfile{pdfatfi.pl} that should be installed in such a way
% that it can be called as \texttt{pdfatfi}.
% Example (linux):
% \begin{quote}
%   |chmod +x scripts/oberdiek/pdfatfi.pl|\\
%   |cp scripts/oberdiek/pdfatfi.pl /usr/local/bin/|
% \end{quote}
%
% \subsection{Package installation}
%
% \paragraph{Unpacking.} The \xfile{.dtx} file is a self-extracting
% \docstrip\ archive. The files are extracted by running the
% \xfile{.dtx} through \plainTeX:
% \begin{quote}
%   \verb|tex letltxmacro.dtx|
% \end{quote}
%
% \paragraph{TDS.} Now the different files must be moved into
% the different directories in your installation TDS tree
% (also known as \xfile{texmf} tree):
% \begin{quote}
% \def\t{^^A
% \begin{tabular}{@{}>{\ttfamily}l@{ $\rightarrow$ }>{\ttfamily}l@{}}
%   letltxmacro.sty & tex/latex/oberdiek/letltxmacro.sty\\
%   letltxmacro.pdf & doc/latex/oberdiek/letltxmacro.pdf\\
%   letltxmacro-showcases.tex & doc/latex/oberdiek/letltxmacro-showcases.tex\\
%   test/letltxmacro-test1.tex & doc/latex/oberdiek/test/letltxmacro-test1.tex\\
%   test/letltxmacro-test2.tex & doc/latex/oberdiek/test/letltxmacro-test2.tex\\
%   letltxmacro.dtx & source/latex/oberdiek/letltxmacro.dtx\\
% \end{tabular}^^A
% }^^A
% \sbox0{\t}^^A
% \ifdim\wd0>\linewidth
%   \begingroup
%     \advance\linewidth by\leftmargin
%     \advance\linewidth by\rightmargin
%   \edef\x{\endgroup
%     \def\noexpand\lw{\the\linewidth}^^A
%   }\x
%   \def\lwbox{^^A
%     \leavevmode
%     \hbox to \linewidth{^^A
%       \kern-\leftmargin\relax
%       \hss
%       \usebox0
%       \hss
%       \kern-\rightmargin\relax
%     }^^A
%   }^^A
%   \ifdim\wd0>\lw
%     \sbox0{\small\t}^^A
%     \ifdim\wd0>\linewidth
%       \ifdim\wd0>\lw
%         \sbox0{\footnotesize\t}^^A
%         \ifdim\wd0>\linewidth
%           \ifdim\wd0>\lw
%             \sbox0{\scriptsize\t}^^A
%             \ifdim\wd0>\linewidth
%               \ifdim\wd0>\lw
%                 \sbox0{\tiny\t}^^A
%                 \ifdim\wd0>\linewidth
%                   \lwbox
%                 \else
%                   \usebox0
%                 \fi
%               \else
%                 \lwbox
%               \fi
%             \else
%               \usebox0
%             \fi
%           \else
%             \lwbox
%           \fi
%         \else
%           \usebox0
%         \fi
%       \else
%         \lwbox
%       \fi
%     \else
%       \usebox0
%     \fi
%   \else
%     \lwbox
%   \fi
% \else
%   \usebox0
% \fi
% \end{quote}
% If you have a \xfile{docstrip.cfg} that configures and enables \docstrip's
% TDS installing feature, then some files can already be in the right
% place, see the documentation of \docstrip.
%
% \subsection{Refresh file name databases}
%
% If your \TeX~distribution
% (\teTeX, \mikTeX, \dots) relies on file name databases, you must refresh
% these. For example, \teTeX\ users run \verb|texhash| or
% \verb|mktexlsr|.
%
% \subsection{Some details for the interested}
%
% \paragraph{Attached source.}
%
% The PDF documentation on CTAN also includes the
% \xfile{.dtx} source file. It can be extracted by
% AcrobatReader 6 or higher. Another option is \textsf{pdftk},
% e.g. unpack the file into the current directory:
% \begin{quote}
%   \verb|pdftk letltxmacro.pdf unpack_files output .|
% \end{quote}
%
% \paragraph{Unpacking with \LaTeX.}
% The \xfile{.dtx} chooses its action depending on the format:
% \begin{description}
% \item[\plainTeX:] Run \docstrip\ and extract the files.
% \item[\LaTeX:] Generate the documentation.
% \end{description}
% If you insist on using \LaTeX\ for \docstrip\ (really,
% \docstrip\ does not need \LaTeX), then inform the autodetect routine
% about your intention:
% \begin{quote}
%   \verb|latex \let\install=y\input{letltxmacro.dtx}|
% \end{quote}
% Do not forget to quote the argument according to the demands
% of your shell.
%
% \paragraph{Generating the documentation.}
% You can use both the \xfile{.dtx} or the \xfile{.drv} to generate
% the documentation. The process can be configured by the
% configuration file \xfile{ltxdoc.cfg}. For instance, put this
% line into this file, if you want to have A4 as paper format:
% \begin{quote}
%   \verb|\PassOptionsToClass{a4paper}{article}|
% \end{quote}
% An example follows how to generate the
% documentation with pdf\LaTeX:
% \begin{quote}
%\begin{verbatim}
%pdflatex letltxmacro.dtx
%makeindex -s gind.ist letltxmacro.idx
%pdflatex letltxmacro.dtx
%makeindex -s gind.ist letltxmacro.idx
%pdflatex letltxmacro.dtx
%\end{verbatim}
% \end{quote}
%
% \section{Catalogue}
%
% The following XML file can be used as source for the
% \href{http://mirror.ctan.org/help/Catalogue/catalogue.html}{\TeX\ Catalogue}.
% The elements \texttt{caption} and \texttt{description} are imported
% from the original XML file from the Catalogue.
% The name of the XML file in the Catalogue is \xfile{letltxmacro.xml}.
%    \begin{macrocode}
%<*catalogue>
<?xml version='1.0' encoding='us-ascii'?>
<!DOCTYPE entry SYSTEM 'catalogue.dtd'>
<entry datestamp='$Date$' modifier='$Author$' id='letltxmacro'>
  <name>letltxmacro</name>
  <caption>Let assignment for LaTeX macros.</caption>
  <authorref id='auth:oberdiek'/>
  <copyright owner='Heiko Oberdiek' year='2008,2010'/>
  <license type='lppl1.3'/>
  <version number='1.4'/>
  <description>
    TeX&#x2019;s <tt>\let</tt> assignment does not work for LaTeX macros
    with optional arguments or for macros that are defined
    as robust macros by <tt>\DeclareRobustCommand</tt>.  This package
    defines <tt>\LetLtxMacro</tt> that also takes care of the involved
    internal macros.
    <p/>
    The package is part of the <xref refid='oberdiek'>oberdiek</xref>
    bundle.
  </description>
  <documentation details='Package documentation'
      href='ctan:/macros/latex/contrib/oberdiek/letltxmacro.pdf'/>
  <ctan file='true' path='/macros/latex/contrib/oberdiek/letltxmacro.dtx'/>
  <miktex location='oberdiek'/>
  <texlive location='oberdiek'/>
  <install path='/macros/latex/contrib/oberdiek/oberdiek.tds.zip'/>
</entry>
%</catalogue>
%    \end{macrocode}
%
% \begin{History}
%   \begin{Version}{2008/06/09 v1.0}
%   \item
%     First version.
%   \end{Version}
%   \begin{Version}{2008/06/12 v1.1}
%   \item
%     Support for \xpackage{etoolbox}'s \cs{newrobustcmd} added.
%   \end{Version}
%   \begin{Version}{2008/06/13 v1.2}
%   \item
%     Support for \xpackage{etoolbox}'s \cs{robustify} added.
%   \end{Version}
%   \begin{Version}{2008/06/24 v1.3}
%   \item
%     Test file adapted for etoolbox 2008/06/22 v1.6.
%   \end{Version}
%   \begin{Version}{2010/09/02 v1.4}
%   \item
%     \cs{GlobalLetLtxMacro} added.
%   \end{Version}
% \end{History}
%
% \PrintIndex
%
% \Finale
\endinput
|
% \end{quote}
% Do not forget to quote the argument according to the demands
% of your shell.
%
% \paragraph{Generating the documentation.}
% You can use both the \xfile{.dtx} or the \xfile{.drv} to generate
% the documentation. The process can be configured by the
% configuration file \xfile{ltxdoc.cfg}. For instance, put this
% line into this file, if you want to have A4 as paper format:
% \begin{quote}
%   \verb|\PassOptionsToClass{a4paper}{article}|
% \end{quote}
% An example follows how to generate the
% documentation with pdf\LaTeX:
% \begin{quote}
%\begin{verbatim}
%pdflatex letltxmacro.dtx
%makeindex -s gind.ist letltxmacro.idx
%pdflatex letltxmacro.dtx
%makeindex -s gind.ist letltxmacro.idx
%pdflatex letltxmacro.dtx
%\end{verbatim}
% \end{quote}
%
% \section{Catalogue}
%
% The following XML file can be used as source for the
% \href{http://mirror.ctan.org/help/Catalogue/catalogue.html}{\TeX\ Catalogue}.
% The elements \texttt{caption} and \texttt{description} are imported
% from the original XML file from the Catalogue.
% The name of the XML file in the Catalogue is \xfile{letltxmacro.xml}.
%    \begin{macrocode}
%<*catalogue>
<?xml version='1.0' encoding='us-ascii'?>
<!DOCTYPE entry SYSTEM 'catalogue.dtd'>
<entry datestamp='$Date$' modifier='$Author$' id='letltxmacro'>
  <name>letltxmacro</name>
  <caption>Let assignment for LaTeX macros.</caption>
  <authorref id='auth:oberdiek'/>
  <copyright owner='Heiko Oberdiek' year='2008,2010'/>
  <license type='lppl1.3'/>
  <version number='1.4'/>
  <description>
    TeX&#x2019;s <tt>\let</tt> assignment does not work for LaTeX macros
    with optional arguments or for macros that are defined
    as robust macros by <tt>\DeclareRobustCommand</tt>.  This package
    defines <tt>\LetLtxMacro</tt> that also takes care of the involved
    internal macros.
    <p/>
    The package is part of the <xref refid='oberdiek'>oberdiek</xref>
    bundle.
  </description>
  <documentation details='Package documentation'
      href='ctan:/macros/latex/contrib/oberdiek/letltxmacro.pdf'/>
  <ctan file='true' path='/macros/latex/contrib/oberdiek/letltxmacro.dtx'/>
  <miktex location='oberdiek'/>
  <texlive location='oberdiek'/>
  <install path='/macros/latex/contrib/oberdiek/oberdiek.tds.zip'/>
</entry>
%</catalogue>
%    \end{macrocode}
%
% \begin{History}
%   \begin{Version}{2008/06/09 v1.0}
%   \item
%     First version.
%   \end{Version}
%   \begin{Version}{2008/06/12 v1.1}
%   \item
%     Support for \xpackage{etoolbox}'s \cs{newrobustcmd} added.
%   \end{Version}
%   \begin{Version}{2008/06/13 v1.2}
%   \item
%     Support for \xpackage{etoolbox}'s \cs{robustify} added.
%   \end{Version}
%   \begin{Version}{2008/06/24 v1.3}
%   \item
%     Test file adapted for etoolbox 2008/06/22 v1.6.
%   \end{Version}
%   \begin{Version}{2010/09/02 v1.4}
%   \item
%     \cs{GlobalLetLtxMacro} added.
%   \end{Version}
% \end{History}
%
% \PrintIndex
%
% \Finale
\endinput

%        (quote the arguments according to the demands of your shell)
%
% Documentation:
%    (a) If letltxmacro.drv is present:
%           latex letltxmacro.drv
%    (b) Without letltxmacro.drv:
%           latex letltxmacro.dtx; ...
%    The class ltxdoc loads the configuration file ltxdoc.cfg
%    if available. Here you can specify further options, e.g.
%    use A4 as paper format:
%       \PassOptionsToClass{a4paper}{article}
%
%    Programm calls to get the documentation (example):
%       pdflatex letltxmacro.dtx
%       makeindex -s gind.ist letltxmacro.idx
%       pdflatex letltxmacro.dtx
%       makeindex -s gind.ist letltxmacro.idx
%       pdflatex letltxmacro.dtx
%
% Installation:
%    TDS:tex/latex/oberdiek/letltxmacro.sty
%    TDS:doc/latex/oberdiek/letltxmacro.pdf
%    TDS:doc/latex/oberdiek/letltxmacro-showcases.tex
%    TDS:doc/latex/oberdiek/test/letltxmacro-test1.tex
%    TDS:doc/latex/oberdiek/test/letltxmacro-test2.tex
%    TDS:source/latex/oberdiek/letltxmacro.dtx
%
%<*ignore>
\begingroup
  \catcode123=1 %
  \catcode125=2 %
  \def\x{LaTeX2e}%
\expandafter\endgroup
\ifcase 0\ifx\install y1\fi\expandafter
         \ifx\csname processbatchFile\endcsname\relax\else1\fi
         \ifx\fmtname\x\else 1\fi\relax
\else\csname fi\endcsname
%</ignore>
%<*install>
\input docstrip.tex
\Msg{************************************************************************}
\Msg{* Installation}
\Msg{* Package: letltxmacro 2010/09/02 v1.4 Let assignment for LaTeX macros (HO)}
\Msg{************************************************************************}

\keepsilent
\askforoverwritefalse

\let\MetaPrefix\relax
\preamble

This is a generated file.

Project: letltxmacro
Version: 2010/09/02 v1.4

Copyright (C) 2008, 2010 by
   Heiko Oberdiek <heiko.oberdiek at googlemail.com>

This work may be distributed and/or modified under the
conditions of the LaTeX Project Public License, either
version 1.3c of this license or (at your option) any later
version. This version of this license is in
   http://www.latex-project.org/lppl/lppl-1-3c.txt
and the latest version of this license is in
   http://www.latex-project.org/lppl.txt
and version 1.3 or later is part of all distributions of
LaTeX version 2005/12/01 or later.

This work has the LPPL maintenance status "maintained".

This Current Maintainer of this work is Heiko Oberdiek.

This work consists of the main source file letltxmacro.dtx
and the derived files
   letltxmacro.sty, letltxmacro.pdf, letltxmacro.ins, letltxmacro.drv,
   letltxmacro-showcases.tex, letltxmacro-test1.tex,
   letltxmacro-test2.tex.

\endpreamble
\let\MetaPrefix\DoubleperCent

\generate{%
  \file{letltxmacro.ins}{\from{letltxmacro.dtx}{install}}%
  \file{letltxmacro.drv}{\from{letltxmacro.dtx}{driver}}%
  \usedir{tex/latex/oberdiek}%
  \file{letltxmacro.sty}{\from{letltxmacro.dtx}{package}}%
  \usedir{doc/latex/oberdiek}%
  \file{letltxmacro-showcases.tex}{\from{letltxmacro.dtx}{showcases}}%
  \usedir{doc/latex/oberdiek/test}%
  \file{letltxmacro-test1.tex}{\from{letltxmacro.dtx}{test1}}%
  \file{letltxmacro-test2.tex}{\from{letltxmacro.dtx}{test2}}%
  \nopreamble
  \nopostamble
  \usedir{source/latex/oberdiek/catalogue}%
  \file{letltxmacro.xml}{\from{letltxmacro.dtx}{catalogue}}%
}

\catcode32=13\relax% active space
\let =\space%
\Msg{************************************************************************}
\Msg{*}
\Msg{* To finish the installation you have to move the following}
\Msg{* file into a directory searched by TeX:}
\Msg{*}
\Msg{*     letltxmacro.sty}
\Msg{*}
\Msg{* To produce the documentation run the file `letltxmacro.drv'}
\Msg{* through LaTeX.}
\Msg{*}
\Msg{* Happy TeXing!}
\Msg{*}
\Msg{************************************************************************}

\endbatchfile
%</install>
%<*ignore>
\fi
%</ignore>
%<*driver>
\NeedsTeXFormat{LaTeX2e}
\ProvidesFile{letltxmacro.drv}%
  [2010/09/02 v1.4 Let assignment for LaTeX macros (HO)]%
\documentclass{ltxdoc}
\usepackage{holtxdoc}[2011/11/22]
\begin{document}
  \DocInput{letltxmacro.dtx}%
\end{document}
%</driver>
% \fi
%
% \CheckSum{990}
%
% \CharacterTable
%  {Upper-case    \A\B\C\D\E\F\G\H\I\J\K\L\M\N\O\P\Q\R\S\T\U\V\W\X\Y\Z
%   Lower-case    \a\b\c\d\e\f\g\h\i\j\k\l\m\n\o\p\q\r\s\t\u\v\w\x\y\z
%   Digits        \0\1\2\3\4\5\6\7\8\9
%   Exclamation   \!     Double quote  \"     Hash (number) \#
%   Dollar        \$     Percent       \%     Ampersand     \&
%   Acute accent  \'     Left paren    \(     Right paren   \)
%   Asterisk      \*     Plus          \+     Comma         \,
%   Minus         \-     Point         \.     Solidus       \/
%   Colon         \:     Semicolon     \;     Less than     \<
%   Equals        \=     Greater than  \>     Question mark \?
%   Commercial at \@     Left bracket  \[     Backslash     \\
%   Right bracket \]     Circumflex    \^     Underscore    \_
%   Grave accent  \`     Left brace    \{     Vertical bar  \|
%   Right brace   \}     Tilde         \~}
%
% \GetFileInfo{letltxmacro.drv}
%
% \title{The \xpackage{letltxmacro} package}
% \date{2010/09/02 v1.4}
% \author{Heiko Oberdiek\\\xemail{heiko.oberdiek at googlemail.com}}
%
% \maketitle
%
% \begin{abstract}
% \TeX's \cs{let} assignment does not work for \LaTeX\ macros
% with optional arguments or for macros that are defined
% as robust macros by \cs{DeclareRobustCommand}. This package
% defines \cs{LetLtxMacro} that also takes care of the involved
% internal macros.
% \end{abstract}
%
% \tableofcontents
%
% \section{Documentation}
%
% If someone wants to redefine a macro with using the old
% meaning, then one method is \TeX's command \cs{let}:
%\begin{quote}
%\begin{verbatim}
%\newcommand{\Macro}{\typeout{Test Macro}}
%\let\SavedMacro=\Macro
%\renewcommand{\Macro}{%
%  \typeout{Begin}%
%  \SavedMacro
%  \typeout{End}%
%}
%\end{verbatim}
%\end{quote}
% However, this method fails, if \cs{Macro} is defined
% by \cs{DeclareRobustCommand} and/or has an optional argument.
% In both cases \LaTeX\ defines an additional internal macro
% that is forgotten in the simple \cs{let} assignment of
% the example above.
%
% \begin{declcs}{LetLtxMacro} \M{new macro} \M{old macro}
% \end{declcs}
% Macro \cs{LetLtxMacro} behaves similar to \TeX's \cs{let}
% assignment, but it takes care of macros that are
% defined by \cs{DeclareRobustCommand} and/or have optional
% arguments. Example:
%\begin{quote}
%\begin{verbatim}
%\DeclareRobustCommand{\Macro}[1][default]{...}
%\LetLtxMacro{\SavedMacro}{\Macro}
%\end{verbatim}
%\end{quote}
% Then macro \cs{SavedMacro} only uses internal macro names
% that are derived from \cs{SavedMacro}'s macro name. Macro \cs{Macro}
% can now be redefined without affecting \cs{SavedMacro}.
%
% \begin{declcs}{GlobalLetLtxMacro} \M{new macro} \M{old macro}
% \end{declcs}
% Like \cs{LetLtxMacro}, but the \meta{new macro} is defined globally.
% Since version 2010/09/02~v1.4.
%
% \subsection{Supported macro definition commands}
%
% \begin{quote}
%   \begin{tabular}{@{}ll@{}}
%     \cs{newcommand}, \cs{renewcommand} & latex/base\\
%     \cs{newenvironment}, \cs{renewenvironment} & latex/base\\
%     \cs{DeclareRobustCommand}& latex/base\\
%     \cs{newrobustcmd}, \cs{renewrobustcmd} & etoolbox\\
%     \cs{robustify} & etoolbox 2008/06/22 v1.6\\
%   \end{tabular}
% \end{quote}
%
% \StopEventually{
% }
%
% \section{Implementation}
%
% \subsection{Show cases}
%
% \subsubsection{\xfile{letltxmacro-showcases.tex}}
%
%    \begin{macrocode}
%<*showcases>
\NeedsTeXFormat{LaTeX2e}
\makeatletter
%    \end{macrocode}
%    \begin{macro}{\Line}
%    The result is displayed by macro \cs{Line}. The percent symbol
%    at line start allows easy grepping and inserting into the DTX
%    file.
%    \begin{macrocode}
\newcommand*{\Line}[1]{%
  \typeout{\@percentchar#1}%
}
%    \end{macrocode}
%    \end{macro}
%    \begin{macrocode}
\newcommand*{\ShowCmdName}[1]{%
  \@ifundefined{#1}{}{%
    \Line{%
      \space\space(\expandafter\string\csname#1\endcsname) = %
      (\expandafter\meaning\csname#1\endcsname)%
    }%
  }%
}
\newcommand*{\ShowCmds}[1]{%
  \ShowCmdName{#1}%
  \ShowCmdName{#1 }%
  \ShowCmdName{\\#1}%
  \ShowCmdName{\\#1 }%
}
\let\\\@backslashchar
%    \end{macrocode}
%    \begin{macro}{\ShowDef}
%    \begin{macrocode}
\newcommand*{\ShowDef}[2]{%
  \begingroup
    \Line{}%
    \newcommand*{\DefString}{#2}%
    \@onelevel@sanitize\DefString
    \Line{\DefString}%
    #2%
    \ShowCmds{#1}%
  \endgroup
}
%    \end{macrocode}
%    \end{macro}
%    \begin{macrocode}
\typeout{}
\Line{* LaTeX definitions:}
\ShowDef{cmd}{%
  \newcommand{\cmd}[2][default]{}%
}
\ShowDef{cmd}{%
  \DeclareRobustCommand{\cmd}{}%
}
\ShowDef{cmd}{%
  \DeclareRobustCommand{\cmd}[2][default]{}%
}
\typeout{}
%    \end{macrocode}
% The minimal version of package \xpackage{etoolbox} is 2008/06/12 v1.6a
% because it fixes \cs{robustify}.
%    \begin{macrocode}
\RequirePackage{etoolbox}[2008/06/12]%
\Line{}
\Line{* etoolbox's robust definitions:}
\ShowDef{cmd}{%
  \newrobustcmd{\cmd}{}%
}
\ShowDef{cmd}{%
  \newrobustcmd{\cmd}[2][default]{}%
}
\Line{}
\Line{* etoolbox's \string\robustify:}
\ShowDef{cmd}{%
  \newcommand{\cmd}[2][default]{} %
  \robustify{\cmd}%
}
\ShowDef{cmd}{%
  \DeclareRobustCommand{\cmd}{} %
  \robustify{\cmd}%
}
\ShowDef{cmd}{%
  \DeclareRobustCommand{\cmd}[2][default]{} %
  \robustify{\cmd}%
}
\typeout{}
\@@end
%</showcases>
%    \end{macrocode}
%
% \subsubsection{Result}
%
% \begingroup
%   \makeatletter
%   \let\org@verbatim\@verbatim
%   \def\@verbatim{^^A
%     \org@verbatim
%     \catcode`\~=\active
%   }^^A
%   \let~\textvisiblespace
%\begin{verbatim}
%* LaTeX definitions:
%
%\newcommand {\cmd }[2][default]{}
%  (\cmd) = (macro:->\@protected@testopt \cmd \\cmd {default})
%  (\\cmd) = (\long macro:[#1]#2->)
%
%\DeclareRobustCommand {\cmd }{}
%  (\cmd) = (macro:->\protect \cmd~ )
%  (\cmd~) = (\long macro:->)
%
%\DeclareRobustCommand {\cmd }[2][default]{}
%  (\cmd) = (macro:->\protect \cmd~ )
%  (\cmd~) = (macro:->\@protected@testopt \cmd~ \\cmd~ {default})
%  (\\cmd~) = (\long macro:[#1]#2->)
%
%* etoolbox's robust definitions:
%
%\newrobustcmd {\cmd }{}
%  (\cmd) = (\protected\long macro:->)
%
%\newrobustcmd {\cmd }[2][default]{}
%  (\cmd) = (\protected macro:->\@testopt \\cmd {default})
%  (\\cmd) = (\long macro:[#1]#2->)
%
%* etoolbox's \robustify:
%
%\newcommand {\cmd }[2][default]{} \robustify {\cmd }
%  (\cmd) = (\protected macro:->\@protected@testopt \cmd \\cmd {default})
%  (\\cmd) = (\long macro:[#1]#2->)
%
%\DeclareRobustCommand {\cmd }{} \robustify {\cmd }
%  (\cmd) = (\protected macro:->)
%
%\DeclareRobustCommand {\cmd }[2][default]{} \robustify {\cmd }
%  (\cmd) = (\protected macro:->\@protected@testopt \cmd~ \\cmd~ {default})
%  (\cmd~) = (macro:->\@protected@testopt \cmd~ \\cmd~ {default})
%  (\\cmd~) = (\long macro:[#1]#2->)
%\end{verbatim}
% \endgroup
%
% \subsection{Package}
%
%    \begin{macrocode}
%<*package>
%    \end{macrocode}
%
% \subsubsection{Catcodes and identification}
%
%    \begin{macrocode}
\begingroup\catcode61\catcode48\catcode32=10\relax%
  \catcode13=5 % ^^M
  \endlinechar=13 %
  \catcode123=1 % {
  \catcode125=2 % }
  \catcode64=11 % @
  \def\x{\endgroup
    \expandafter\edef\csname llm@AtEnd\endcsname{%
      \endlinechar=\the\endlinechar\relax
      \catcode13=\the\catcode13\relax
      \catcode32=\the\catcode32\relax
      \catcode35=\the\catcode35\relax
      \catcode61=\the\catcode61\relax
      \catcode64=\the\catcode64\relax
      \catcode123=\the\catcode123\relax
      \catcode125=\the\catcode125\relax
    }%
  }%
\x\catcode61\catcode48\catcode32=10\relax%
\catcode13=5 % ^^M
\endlinechar=13 %
\catcode35=6 % #
\catcode64=11 % @
\catcode123=1 % {
\catcode125=2 % }
\def\TMP@EnsureCode#1#2{%
  \edef\llm@AtEnd{%
    \llm@AtEnd
    \catcode#1=\the\catcode#1\relax
  }%
  \catcode#1=#2\relax
}
\TMP@EnsureCode{40}{12}% (
\TMP@EnsureCode{41}{12}% )
\TMP@EnsureCode{42}{12}% *
\TMP@EnsureCode{45}{12}% -
\TMP@EnsureCode{46}{12}% .
\TMP@EnsureCode{47}{12}% /
\TMP@EnsureCode{58}{12}% :
\TMP@EnsureCode{62}{12}% >
\TMP@EnsureCode{91}{12}% [
\TMP@EnsureCode{93}{12}% ]
\edef\llm@AtEnd{%
  \llm@AtEnd
  \escapechar\the\escapechar\relax
  \noexpand\endinput
}
\escapechar=92 % `\\
%    \end{macrocode}
%
%    Package identification.
%    \begin{macrocode}
\NeedsTeXFormat{LaTeX2e}
\ProvidesPackage{letltxmacro}%
  [2010/09/02 v1.4 Let assignment for LaTeX macros (HO)]
%    \end{macrocode}
%
% \subsubsection{Main macros}
%
%    \begin{macro}{\LetLtxMacro}
%    \begin{macrocode}
\newcommand*{\LetLtxMacro}{%
  \llm@ModeLetLtxMacro{}%
}
%    \end{macrocode}
%    \end{macro}
%    \begin{macro}{\GlobalLetLtxMacro}
%    \begin{macrocode}
\newcommand*{\GlobalLetLtxMacro}{%
  \llm@ModeLetLtxMacro\global
}
%    \end{macrocode}
%    \end{macro}
%
%    \begin{macro}{\llm@ModeLetLtxMacro}
%    \begin{macrocode}
\newcommand*{\llm@ModeLetLtxMacro}[3]{%
  \edef\llm@escapechar{\the\escapechar}%
  \escapechar=-1 %
  \edef\reserved@a{%
    \noexpand\protect
    \expandafter\noexpand
    \csname\string#3 \endcsname
  }%
  \ifx\reserved@a#3\relax
    #1\edef#2{%
      \noexpand\protect
      \expandafter\noexpand
      \csname\string#2 \endcsname
    }%
    #1\expandafter\let
    \csname\string#2 \expandafter\endcsname
    \csname\string#3 \endcsname
    \expandafter\llm@LetLtxMacro
        \csname\string#2 \expandafter\endcsname
        \csname\string#3 \endcsname{#1}%
  \else
    \llm@LetLtxMacro{#2}{#3}{#1}%
  \fi
  \escapechar=\llm@escapechar\relax
}
%    \end{macrocode}
%    \end{macro}
%    \begin{macro}{\llm@LetLtxMacro}
%    \begin{macrocode}
\def\llm@LetLtxMacro#1#2#3{%
  \escapechar=92 %
  \expandafter\llm@CheckParams\meaning#2:->\@nil{%
    \begingroup
      \def\@protected@testopt{%
        \expandafter\@testopt\@gobble
      }%
      \def\@testopt##1##2{%
        \toks@={##2}%
      }%
      \let\llm@testopt\@empty
      \edef\x{%
        \noexpand\@protected@testopt
        \noexpand#2%
        \expandafter\noexpand\csname\string#2\endcsname
      }%
      \expandafter\expandafter\expandafter\def
      \expandafter\expandafter\expandafter\y
      \expandafter\expandafter\expandafter{%
        \expandafter\llm@CarThree#2{}{}{}\llm@nil
      }%
      \ifx\x\y
        #2%
        \def\llm@testopt{%
          \noexpand\@protected@testopt
          \noexpand#1%
        }%
      \else
        \edef\x{%
          \noexpand\@testopt
          \expandafter\noexpand
          \csname\string#2\endcsname
        }%
        \expandafter\expandafter\expandafter\def
        \expandafter\expandafter\expandafter\y
        \expandafter\expandafter\expandafter{%
          \expandafter\llm@CarTwo#2{}{}\llm@nil
        }%
        \ifx\x\y
          #2%
          \def\llm@testopt{%
            \noexpand\@testopt
          }%
        \fi
      \fi
      \ifx\llm@testopt\@empty
      \else
        \llm@protected\xdef\llm@GlobalTemp{%
          \llm@testopt
          \expandafter\noexpand
          \csname\string#1\endcsname
          {\the\toks@}%
        }%
      \fi
    \expandafter\endgroup\ifx\llm@testopt\@empty
      #3\let#1=#2\relax
    \else
      #3\let#1=\llm@GlobalTemp
      #3\expandafter\let
          \csname\string#1\expandafter\endcsname
          \csname\string#2\endcsname
    \fi
  }{%
    #3\let#1=#2\relax
  }%
}
%    \end{macrocode}
%    \end{macro}
%    \begin{macro}{\llm@CheckParams}
%    \begin{macrocode}
\def\llm@CheckParams#1:->#2\@nil{%
  \begingroup
    \def\x{#1}%
  \ifx\x\llm@macro
    \endgroup
    \def\llm@protected{}%
    \expandafter\@firstoftwo
  \else
    \ifx\x\llm@protectedmacro
      \endgroup
      \def\llm@protected{\protected}%
      \expandafter\expandafter\expandafter\@firstoftwo
    \else
      \endgroup
      \expandafter\expandafter\expandafter\@secondoftwo
    \fi
  \fi
}
%    \end{macrocode}
%    \end{macro}
%    \begin{macro}{\llm@macro}
%    \begin{macrocode}
\def\llm@macro{macro}
\@onelevel@sanitize\llm@macro
%    \end{macrocode}
%    \end{macro}
%    \begin{macro}{\llm@protectedmacro}
%    \begin{macrocode}
\def\llm@protectedmacro{\protected macro}
\@onelevel@sanitize\llm@protectedmacro
%    \end{macrocode}
%    \end{macro}
%    \begin{macro}{\llm@CarThree}
%    \begin{macrocode}
\def\llm@CarThree#1#2#3#4\llm@nil{#1#2#3}%
%    \end{macrocode}
%    \end{macro}
%    \begin{macro}{\llm@CarTwo}
%    \begin{macrocode}
\def\llm@CarTwo#1#2#3\llm@nil{#1#2}%
%    \end{macrocode}
%    \end{macro}
%
%    \begin{macrocode}
\llm@AtEnd%
%</package>
%    \end{macrocode}
%
% \section{Test}
%
% \subsection{Catcode checks for loading}
%
%    \begin{macrocode}
%<*test1>
%    \end{macrocode}
%    \begin{macrocode}
\NeedsTeXFormat{LaTeX2e}
\documentclass{minimal}
\makeatletter
\def\RestoreCatcodes{}
\count@=0 %
\loop
  \edef\RestoreCatcodes{%
    \RestoreCatcodes
    \catcode\the\count@=\the\catcode\count@\relax
  }%
\ifnum\count@<255 %
  \advance\count@\@ne
\repeat

\def\RangeCatcodeInvalid#1#2{%
  \count@=#1\relax
  \loop
    \catcode\count@=15 %
  \ifnum\count@<#2\relax
    \advance\count@\@ne
  \repeat
}
\def\Test{%
  \RangeCatcodeInvalid{0}{47}%
  \RangeCatcodeInvalid{58}{64}%
  \RangeCatcodeInvalid{91}{96}%
  \RangeCatcodeInvalid{123}{127}%
  \catcode`\@=12 %
  \catcode`\\=0 %
  \catcode`\{=1 %
  \catcode`\}=2 %
  \catcode`\#=6 %
  \catcode`\[=12 %
  \catcode`\]=12 %
  \catcode`\%=14 %
  \catcode`\ =10 %
  \catcode13=5 %
  \RequirePackage{letltxmacro}[2010/09/02]\relax
  \RestoreCatcodes
}
\Test
\csname @@end\endcsname
\end
%    \end{macrocode}
%    \begin{macrocode}
%</test1>
%    \end{macrocode}
%
% \subsection{Package tests}
%
%    \begin{macrocode}
%<*test2>
%    \end{macrocode}
%    \begin{macrocode}
\NeedsTeXFormat{LaTeX2e}
\nofiles
\documentclass{minimal}

\usepackage{letltxmacro}[2010/09/02]

\usepackage{qstest}
\IncludeTests{*}
\LogTests{log}{*}{*}

\makeatletter

\def\TestDef#1{%
  \begingroup
    \@makeother\\%
    \@makeother\ %
    \expandafter\@TestDef\csname #1\endcsname
}
\def\@TestDef#1#2{%
    \edef\a{\expandafter\strip@prefix\meaning#1}%
    \edef\b{\detokenize{#2}}%
    \ifx\a\b
    \else
      \typeout{1 [\a]}%
      \typeout{2 [\b]}%
    \fi
  \endgroup
  \Expect*{\expandafter\strip@prefix\meaning#1}*{\detokenize{#2}}%
}
\def\TestEquals#1#2{%
  \Expect*{%
    \expandafter\ifx\csname#1\expandafter\endcsname
                    \csname#2\endcsname
      equals%
    \else
      wrong%
    \fi
  }{equals}%
}
\def\SaveA{%
  \let\SavedA\TestA
  \expandafter\let\csname\string\SavedA\expandafter\endcsname
                  \csname\string\TestA\endcsname
  \expandafter\let\csname SavedA \expandafter\endcsname
                  \csname TestA \endcsname
}
\def\CheckA{%
  \TestEquals{SavedA}{TestA}%
  \TestEquals{\string\SavedA}{\string\TestA}%
  \TestEquals{SavedA }{TestA }%
}

\begin{qstest}{robust}{robust}
  \begingroup
    \DeclareRobustCommand*{\TestA}{Test}%
    \SaveA
    \TestDef{TestA}{\protect \TestA  }%
    \CheckA
    \DeclareRobustCommand{\TestA}{Test}%
    \SaveA
    \TestDef{TestA}{\protect \TestA  }%
    \LetLtxMacro\TestB\TestA
    \TestDef{TestB}{\protect \TestB  }%
    \TestEquals{TestB }{TestA }%
    \CheckA
    \GlobalLetLtxMacro\TestC\TestA
    \CheckA
  \endgroup
  \TestDef{TestC}{\protect \TestC  }%
  \DeclareRobustCommand{\TestA}{Test}%
  \TestEquals{TestC }{TestA }%
\end{qstest}

\begin{qstest}{default}{default}
  \begingroup
    \newcommand{\TestA}[1][\relax default \empty]{TestA #1}%
    \SaveA
    \TestDef
    {TestA}{\@protected@testopt \TestA \\TestA {\relax default \empty }}%
    \LetLtxMacro\TestB\TestA
    \TestDef
    {TestB}{\@protected@testopt \TestB \\TestB {\relax default \empty }}%
    \CheckA
    \TestEquals{\string\TestB}{\string\TestA}%
    \GlobalLetLtxMacro\TestC\TestA
    \CheckA
  \endgroup
  \TestDef
  {TestC}{\@protected@testopt \TestC \\TestC {\relax default \empty }}%
  \newcommand{\TestA}[1][\relax default \empty]{TestA #1}%
  \TestEquals{\string\TestC}{\string\TestA}%
\end{qstest}

\begin{qstest}{robustdefault}{robustdefault}
  \begingroup
    \DeclareRobustCommand{\TestA}[1][\default]{TestA}%
    \SaveA
    \LetLtxMacro\TestB\TestA
    \TestDef{TestB}{\protect \TestB  }%
    \TestDef{TestB }{\@protected@testopt \TestB  \\TestB  {\default }}%
    \CheckA
    \GlobalLetLtxMacro\TestC\TestA
    \CheckA
  \endgroup
  \TestDef{TestC}{\protect \TestC  }%
  \TestDef{TestC }{\@protected@testopt \TestC  \\TestC  {\default }}%
\end{qstest}

\begin{qstest}{plain}{plain}
  \LetLtxMacro\NewRelax\relax
  \Expect*{\meaning\relax}*{\string\relax}%
  \Expect*{\meaning\NewRelax}*{\string\relax}%
  \LetLtxMacro\NewHbox\hbox
  \Expect*{\meaning\hbox}*{\string\hbox}%
  \Expect*{\meaning\NewHbox}*{\string\hbox}%
  \LetLtxMacro\NewEmpty\empty
  \Expect*{\meaning\empty}{macro:->}%
  \Expect*{\meaning\NewEmpty}{macro:->}%
  \def\TestA{\iffalse}%
  \LetLtxMacro\TestB\TestA
  \TestDef{TestB}{\iffalse }%
  \TestEquals{TestA}{TestB}%
\end{qstest}
%    \end{macrocode}
%
%    \begin{macrocode}
\usepackage{etoolbox}[2008/06/22]

\begin{qstest}{oldnewrobustcmd}{oldnewrobustcmd}
  \begingroup
    \protected\edef\TestA{%
      \noexpand\@protected@testopt
      \noexpand\TestA
      \expandafter\noexpand\csname\string\TestA\endcsname
      {\noexpand\default}%
    }%
    \expandafter\protected\expandafter
    \def\csname\string\TestA\endcsname{TestA}%
    \SaveA
    \LetLtxMacro\TestB\TestA
    \TestDef{TestB}{\@protected@testopt \TestB \\TestB {\default }}%
    \protected\edef\TestC{%
      \noexpand\@protected@testopt
      \noexpand\TestB
      \expandafter\noexpand\csname\string\TestB\endcsname
      {\noexpand\default}%
    }%
    \TestEquals{TestB}{TestC}%
    \CheckA
    \GlobalLetLtxMacro\TestD\TestA
    \CheckA
  \endgroup
  \TestDef{TestD}{\@protected@testopt \TestD \\TestD {\default }}%
  \protected\edef\TestC{%
    \noexpand\@protected@testopt
    \noexpand\TestD
    \expandafter\noexpand\csname\string\TestD\endcsname
    {\noexpand\default}%
  }%
  \TestEquals{TestD}{TestC}%
\end{qstest}

\begin{qstest}{newrobustcmd}{newrobustcmd}
  \begingroup
    \newrobustcmd{\TestA}[1][\default]{TestA}%
    \SaveA
    \LetLtxMacro\TestB\TestA
    \TestDef{TestB}{\@testopt \\TestB {\default }}%
    \protected\edef\TestC{%
      \noexpand\@testopt
      \expandafter\noexpand\csname\string\TestB\endcsname
      {\noexpand\default}%
    }%
    \TestEquals{TestB}{TestC}%
    \CheckA
    \GlobalLetLtxMacro\TestD\TestA
    \CheckA
  \endgroup
  \TestDef{TestD}{\@testopt \\TestD {\default }}%
  \protected\edef\TestC{%
    \noexpand\@testopt
    \expandafter\noexpand\csname\string\TestD\endcsname
    {\noexpand\default}%
  }%
  \TestEquals{TestD}{TestC}%
\end{qstest}

\begin{qstest}{robustifyopt}{robustifyopt}
  \newcommand{\TestA}[2][\default]{}%
  \TestDef{TestA}{\@protected@testopt \TestA \\TestA {\default }}%
  \robustify\TestA
  \TestDef{TestA}{\@protected@testopt \TestA \\TestA {\default }}%
  \protected\edef\TestC{%
    \noexpand\@protected@testopt
    \noexpand\TestA
    \expandafter\noexpand\csname\string\TestA\endcsname
    {\noexpand\default}%
  }%
  \TestEquals{TestA}{TestC}%
  \SaveA
  \LetLtxMacro\TestB\TestA
  \TestDef{TestB}{\@protected@testopt \TestB \\TestB {\default }}%
  \protected\edef\TestC{%
    \noexpand\@protected@testopt
    \noexpand\TestB
    \expandafter\noexpand\csname\string\TestB\endcsname
    {\noexpand\default}%
  }%
  \TestEquals{TestB}{TestC}%
  \CheckA
\end{qstest}

\begin{qstest}{robustifydeclare}{robustifydeclare}
  \DeclareRobustCommand\TestA{\iffalse}%
  \robustify\TestA
  \SaveA
  \LetLtxMacro\TestB\TestA
  \TestDef{TestB}{\iffalse }%
  % before etoolbox 2008/06/22: \protected\long\def
  % since etoolbox 2008/06/22: \protected\def
  \protected\def\TestC{\iffalse}%
  \TestEquals{TestB}{TestC}%
  \CheckA
\end{qstest}

\begin{qstest}{robustifydeclarelong}{robustifydeclarelong}
  \DeclareRobustCommand\TestA[1]{\iffalse}%
  \robustify\TestA
  \SaveA
  \LetLtxMacro\TestB\TestA
  \TestDef{TestB}{\iffalse }%
  % before etoolbox 2008/06/22: \protected\long\def
  % since etoolbox 2008/06/22: \protected\def
  \protected\long\def\TestC#1{\iffalse}%
  \TestEquals{TestB}{TestC}%
  \CheckA
\end{qstest}
%    \end{macrocode}
%
%    \begin{macrocode}
\csname @@end\endcsname
%    \end{macrocode}
%    \begin{macrocode}
%</test2>
%    \end{macrocode}
%
% \section{Installation}
%
% \subsection{Download}
%
% \paragraph{Package.} This package is available on
% CTAN\footnote{\url{ftp://ftp.ctan.org/tex-archive/}}:
% \begin{description}
% \item[\CTAN{macros/latex/contrib/oberdiek/letltxmacro.dtx}] The source file.
% \item[\CTAN{macros/latex/contrib/oberdiek/letltxmacro.pdf}] Documentation.
% \end{description}
%
%
% \paragraph{Bundle.} All the packages of the bundle `oberdiek'
% are also available in a TDS compliant ZIP archive. There
% the packages are already unpacked and the documentation files
% are generated. The files and directories obey the TDS standard.
% \begin{description}
% \item[\CTAN{install/macros/latex/contrib/oberdiek.tds.zip}]
% \end{description}
% \emph{TDS} refers to the standard ``A Directory Structure
% for \TeX\ Files'' (\CTAN{tds/tds.pdf}). Directories
% with \xfile{texmf} in their name are usually organized this way.
%
% \subsection{Bundle installation}
%
% \paragraph{Unpacking.} Unpack the \xfile{oberdiek.tds.zip} in the
% TDS tree (also known as \xfile{texmf} tree) of your choice.
% Example (linux):
% \begin{quote}
%   |unzip oberdiek.tds.zip -d ~/texmf|
% \end{quote}
%
% \paragraph{Script installation.}
% Check the directory \xfile{TDS:scripts/oberdiek/} for
% scripts that need further installation steps.
% Package \xpackage{attachfile2} comes with the Perl script
% \xfile{pdfatfi.pl} that should be installed in such a way
% that it can be called as \texttt{pdfatfi}.
% Example (linux):
% \begin{quote}
%   |chmod +x scripts/oberdiek/pdfatfi.pl|\\
%   |cp scripts/oberdiek/pdfatfi.pl /usr/local/bin/|
% \end{quote}
%
% \subsection{Package installation}
%
% \paragraph{Unpacking.} The \xfile{.dtx} file is a self-extracting
% \docstrip\ archive. The files are extracted by running the
% \xfile{.dtx} through \plainTeX:
% \begin{quote}
%   \verb|tex letltxmacro.dtx|
% \end{quote}
%
% \paragraph{TDS.} Now the different files must be moved into
% the different directories in your installation TDS tree
% (also known as \xfile{texmf} tree):
% \begin{quote}
% \def\t{^^A
% \begin{tabular}{@{}>{\ttfamily}l@{ $\rightarrow$ }>{\ttfamily}l@{}}
%   letltxmacro.sty & tex/latex/oberdiek/letltxmacro.sty\\
%   letltxmacro.pdf & doc/latex/oberdiek/letltxmacro.pdf\\
%   letltxmacro-showcases.tex & doc/latex/oberdiek/letltxmacro-showcases.tex\\
%   test/letltxmacro-test1.tex & doc/latex/oberdiek/test/letltxmacro-test1.tex\\
%   test/letltxmacro-test2.tex & doc/latex/oberdiek/test/letltxmacro-test2.tex\\
%   letltxmacro.dtx & source/latex/oberdiek/letltxmacro.dtx\\
% \end{tabular}^^A
% }^^A
% \sbox0{\t}^^A
% \ifdim\wd0>\linewidth
%   \begingroup
%     \advance\linewidth by\leftmargin
%     \advance\linewidth by\rightmargin
%   \edef\x{\endgroup
%     \def\noexpand\lw{\the\linewidth}^^A
%   }\x
%   \def\lwbox{^^A
%     \leavevmode
%     \hbox to \linewidth{^^A
%       \kern-\leftmargin\relax
%       \hss
%       \usebox0
%       \hss
%       \kern-\rightmargin\relax
%     }^^A
%   }^^A
%   \ifdim\wd0>\lw
%     \sbox0{\small\t}^^A
%     \ifdim\wd0>\linewidth
%       \ifdim\wd0>\lw
%         \sbox0{\footnotesize\t}^^A
%         \ifdim\wd0>\linewidth
%           \ifdim\wd0>\lw
%             \sbox0{\scriptsize\t}^^A
%             \ifdim\wd0>\linewidth
%               \ifdim\wd0>\lw
%                 \sbox0{\tiny\t}^^A
%                 \ifdim\wd0>\linewidth
%                   \lwbox
%                 \else
%                   \usebox0
%                 \fi
%               \else
%                 \lwbox
%               \fi
%             \else
%               \usebox0
%             \fi
%           \else
%             \lwbox
%           \fi
%         \else
%           \usebox0
%         \fi
%       \else
%         \lwbox
%       \fi
%     \else
%       \usebox0
%     \fi
%   \else
%     \lwbox
%   \fi
% \else
%   \usebox0
% \fi
% \end{quote}
% If you have a \xfile{docstrip.cfg} that configures and enables \docstrip's
% TDS installing feature, then some files can already be in the right
% place, see the documentation of \docstrip.
%
% \subsection{Refresh file name databases}
%
% If your \TeX~distribution
% (\teTeX, \mikTeX, \dots) relies on file name databases, you must refresh
% these. For example, \teTeX\ users run \verb|texhash| or
% \verb|mktexlsr|.
%
% \subsection{Some details for the interested}
%
% \paragraph{Attached source.}
%
% The PDF documentation on CTAN also includes the
% \xfile{.dtx} source file. It can be extracted by
% AcrobatReader 6 or higher. Another option is \textsf{pdftk},
% e.g. unpack the file into the current directory:
% \begin{quote}
%   \verb|pdftk letltxmacro.pdf unpack_files output .|
% \end{quote}
%
% \paragraph{Unpacking with \LaTeX.}
% The \xfile{.dtx} chooses its action depending on the format:
% \begin{description}
% \item[\plainTeX:] Run \docstrip\ and extract the files.
% \item[\LaTeX:] Generate the documentation.
% \end{description}
% If you insist on using \LaTeX\ for \docstrip\ (really,
% \docstrip\ does not need \LaTeX), then inform the autodetect routine
% about your intention:
% \begin{quote}
%   \verb|latex \let\install=y% \iffalse meta-comment
%
% File: letltxmacro.dtx
% Version: 2010/09/02 v1.4
% Info: Let assignment for LaTeX macros
%
% Copyright (C) 2008, 2010 by
%    Heiko Oberdiek <heiko.oberdiek at googlemail.com>
%
% This work may be distributed and/or modified under the
% conditions of the LaTeX Project Public License, either
% version 1.3c of this license or (at your option) any later
% version. This version of this license is in
%    http://www.latex-project.org/lppl/lppl-1-3c.txt
% and the latest version of this license is in
%    http://www.latex-project.org/lppl.txt
% and version 1.3 or later is part of all distributions of
% LaTeX version 2005/12/01 or later.
%
% This work has the LPPL maintenance status "maintained".
%
% This Current Maintainer of this work is Heiko Oberdiek.
%
% This work consists of the main source file letltxmacro.dtx
% and the derived files
%    letltxmacro.sty, letltxmacro.pdf, letltxmacro.ins, letltxmacro.drv,
%    letltxmacro-showcases.tex, letltxmacro-test1.tex,
%    letltxmacro-test2.tex.
%
% Distribution:
%    CTAN:macros/latex/contrib/oberdiek/letltxmacro.dtx
%    CTAN:macros/latex/contrib/oberdiek/letltxmacro.pdf
%
% Unpacking:
%    (a) If letltxmacro.ins is present:
%           tex letltxmacro.ins
%    (b) Without letltxmacro.ins:
%           tex letltxmacro.dtx
%    (c) If you insist on using LaTeX
%           latex \let\install=y% \iffalse meta-comment
%
% File: letltxmacro.dtx
% Version: 2010/09/02 v1.4
% Info: Let assignment for LaTeX macros
%
% Copyright (C) 2008, 2010 by
%    Heiko Oberdiek <heiko.oberdiek at googlemail.com>
%
% This work may be distributed and/or modified under the
% conditions of the LaTeX Project Public License, either
% version 1.3c of this license or (at your option) any later
% version. This version of this license is in
%    http://www.latex-project.org/lppl/lppl-1-3c.txt
% and the latest version of this license is in
%    http://www.latex-project.org/lppl.txt
% and version 1.3 or later is part of all distributions of
% LaTeX version 2005/12/01 or later.
%
% This work has the LPPL maintenance status "maintained".
%
% This Current Maintainer of this work is Heiko Oberdiek.
%
% This work consists of the main source file letltxmacro.dtx
% and the derived files
%    letltxmacro.sty, letltxmacro.pdf, letltxmacro.ins, letltxmacro.drv,
%    letltxmacro-showcases.tex, letltxmacro-test1.tex,
%    letltxmacro-test2.tex.
%
% Distribution:
%    CTAN:macros/latex/contrib/oberdiek/letltxmacro.dtx
%    CTAN:macros/latex/contrib/oberdiek/letltxmacro.pdf
%
% Unpacking:
%    (a) If letltxmacro.ins is present:
%           tex letltxmacro.ins
%    (b) Without letltxmacro.ins:
%           tex letltxmacro.dtx
%    (c) If you insist on using LaTeX
%           latex \let\install=y\input{letltxmacro.dtx}
%        (quote the arguments according to the demands of your shell)
%
% Documentation:
%    (a) If letltxmacro.drv is present:
%           latex letltxmacro.drv
%    (b) Without letltxmacro.drv:
%           latex letltxmacro.dtx; ...
%    The class ltxdoc loads the configuration file ltxdoc.cfg
%    if available. Here you can specify further options, e.g.
%    use A4 as paper format:
%       \PassOptionsToClass{a4paper}{article}
%
%    Programm calls to get the documentation (example):
%       pdflatex letltxmacro.dtx
%       makeindex -s gind.ist letltxmacro.idx
%       pdflatex letltxmacro.dtx
%       makeindex -s gind.ist letltxmacro.idx
%       pdflatex letltxmacro.dtx
%
% Installation:
%    TDS:tex/latex/oberdiek/letltxmacro.sty
%    TDS:doc/latex/oberdiek/letltxmacro.pdf
%    TDS:doc/latex/oberdiek/letltxmacro-showcases.tex
%    TDS:doc/latex/oberdiek/test/letltxmacro-test1.tex
%    TDS:doc/latex/oberdiek/test/letltxmacro-test2.tex
%    TDS:source/latex/oberdiek/letltxmacro.dtx
%
%<*ignore>
\begingroup
  \catcode123=1 %
  \catcode125=2 %
  \def\x{LaTeX2e}%
\expandafter\endgroup
\ifcase 0\ifx\install y1\fi\expandafter
         \ifx\csname processbatchFile\endcsname\relax\else1\fi
         \ifx\fmtname\x\else 1\fi\relax
\else\csname fi\endcsname
%</ignore>
%<*install>
\input docstrip.tex
\Msg{************************************************************************}
\Msg{* Installation}
\Msg{* Package: letltxmacro 2010/09/02 v1.4 Let assignment for LaTeX macros (HO)}
\Msg{************************************************************************}

\keepsilent
\askforoverwritefalse

\let\MetaPrefix\relax
\preamble

This is a generated file.

Project: letltxmacro
Version: 2010/09/02 v1.4

Copyright (C) 2008, 2010 by
   Heiko Oberdiek <heiko.oberdiek at googlemail.com>

This work may be distributed and/or modified under the
conditions of the LaTeX Project Public License, either
version 1.3c of this license or (at your option) any later
version. This version of this license is in
   http://www.latex-project.org/lppl/lppl-1-3c.txt
and the latest version of this license is in
   http://www.latex-project.org/lppl.txt
and version 1.3 or later is part of all distributions of
LaTeX version 2005/12/01 or later.

This work has the LPPL maintenance status "maintained".

This Current Maintainer of this work is Heiko Oberdiek.

This work consists of the main source file letltxmacro.dtx
and the derived files
   letltxmacro.sty, letltxmacro.pdf, letltxmacro.ins, letltxmacro.drv,
   letltxmacro-showcases.tex, letltxmacro-test1.tex,
   letltxmacro-test2.tex.

\endpreamble
\let\MetaPrefix\DoubleperCent

\generate{%
  \file{letltxmacro.ins}{\from{letltxmacro.dtx}{install}}%
  \file{letltxmacro.drv}{\from{letltxmacro.dtx}{driver}}%
  \usedir{tex/latex/oberdiek}%
  \file{letltxmacro.sty}{\from{letltxmacro.dtx}{package}}%
  \usedir{doc/latex/oberdiek}%
  \file{letltxmacro-showcases.tex}{\from{letltxmacro.dtx}{showcases}}%
  \usedir{doc/latex/oberdiek/test}%
  \file{letltxmacro-test1.tex}{\from{letltxmacro.dtx}{test1}}%
  \file{letltxmacro-test2.tex}{\from{letltxmacro.dtx}{test2}}%
  \nopreamble
  \nopostamble
  \usedir{source/latex/oberdiek/catalogue}%
  \file{letltxmacro.xml}{\from{letltxmacro.dtx}{catalogue}}%
}

\catcode32=13\relax% active space
\let =\space%
\Msg{************************************************************************}
\Msg{*}
\Msg{* To finish the installation you have to move the following}
\Msg{* file into a directory searched by TeX:}
\Msg{*}
\Msg{*     letltxmacro.sty}
\Msg{*}
\Msg{* To produce the documentation run the file `letltxmacro.drv'}
\Msg{* through LaTeX.}
\Msg{*}
\Msg{* Happy TeXing!}
\Msg{*}
\Msg{************************************************************************}

\endbatchfile
%</install>
%<*ignore>
\fi
%</ignore>
%<*driver>
\NeedsTeXFormat{LaTeX2e}
\ProvidesFile{letltxmacro.drv}%
  [2010/09/02 v1.4 Let assignment for LaTeX macros (HO)]%
\documentclass{ltxdoc}
\usepackage{holtxdoc}[2011/11/22]
\begin{document}
  \DocInput{letltxmacro.dtx}%
\end{document}
%</driver>
% \fi
%
% \CheckSum{990}
%
% \CharacterTable
%  {Upper-case    \A\B\C\D\E\F\G\H\I\J\K\L\M\N\O\P\Q\R\S\T\U\V\W\X\Y\Z
%   Lower-case    \a\b\c\d\e\f\g\h\i\j\k\l\m\n\o\p\q\r\s\t\u\v\w\x\y\z
%   Digits        \0\1\2\3\4\5\6\7\8\9
%   Exclamation   \!     Double quote  \"     Hash (number) \#
%   Dollar        \$     Percent       \%     Ampersand     \&
%   Acute accent  \'     Left paren    \(     Right paren   \)
%   Asterisk      \*     Plus          \+     Comma         \,
%   Minus         \-     Point         \.     Solidus       \/
%   Colon         \:     Semicolon     \;     Less than     \<
%   Equals        \=     Greater than  \>     Question mark \?
%   Commercial at \@     Left bracket  \[     Backslash     \\
%   Right bracket \]     Circumflex    \^     Underscore    \_
%   Grave accent  \`     Left brace    \{     Vertical bar  \|
%   Right brace   \}     Tilde         \~}
%
% \GetFileInfo{letltxmacro.drv}
%
% \title{The \xpackage{letltxmacro} package}
% \date{2010/09/02 v1.4}
% \author{Heiko Oberdiek\\\xemail{heiko.oberdiek at googlemail.com}}
%
% \maketitle
%
% \begin{abstract}
% \TeX's \cs{let} assignment does not work for \LaTeX\ macros
% with optional arguments or for macros that are defined
% as robust macros by \cs{DeclareRobustCommand}. This package
% defines \cs{LetLtxMacro} that also takes care of the involved
% internal macros.
% \end{abstract}
%
% \tableofcontents
%
% \section{Documentation}
%
% If someone wants to redefine a macro with using the old
% meaning, then one method is \TeX's command \cs{let}:
%\begin{quote}
%\begin{verbatim}
%\newcommand{\Macro}{\typeout{Test Macro}}
%\let\SavedMacro=\Macro
%\renewcommand{\Macro}{%
%  \typeout{Begin}%
%  \SavedMacro
%  \typeout{End}%
%}
%\end{verbatim}
%\end{quote}
% However, this method fails, if \cs{Macro} is defined
% by \cs{DeclareRobustCommand} and/or has an optional argument.
% In both cases \LaTeX\ defines an additional internal macro
% that is forgotten in the simple \cs{let} assignment of
% the example above.
%
% \begin{declcs}{LetLtxMacro} \M{new macro} \M{old macro}
% \end{declcs}
% Macro \cs{LetLtxMacro} behaves similar to \TeX's \cs{let}
% assignment, but it takes care of macros that are
% defined by \cs{DeclareRobustCommand} and/or have optional
% arguments. Example:
%\begin{quote}
%\begin{verbatim}
%\DeclareRobustCommand{\Macro}[1][default]{...}
%\LetLtxMacro{\SavedMacro}{\Macro}
%\end{verbatim}
%\end{quote}
% Then macro \cs{SavedMacro} only uses internal macro names
% that are derived from \cs{SavedMacro}'s macro name. Macro \cs{Macro}
% can now be redefined without affecting \cs{SavedMacro}.
%
% \begin{declcs}{GlobalLetLtxMacro} \M{new macro} \M{old macro}
% \end{declcs}
% Like \cs{LetLtxMacro}, but the \meta{new macro} is defined globally.
% Since version 2010/09/02~v1.4.
%
% \subsection{Supported macro definition commands}
%
% \begin{quote}
%   \begin{tabular}{@{}ll@{}}
%     \cs{newcommand}, \cs{renewcommand} & latex/base\\
%     \cs{newenvironment}, \cs{renewenvironment} & latex/base\\
%     \cs{DeclareRobustCommand}& latex/base\\
%     \cs{newrobustcmd}, \cs{renewrobustcmd} & etoolbox\\
%     \cs{robustify} & etoolbox 2008/06/22 v1.6\\
%   \end{tabular}
% \end{quote}
%
% \StopEventually{
% }
%
% \section{Implementation}
%
% \subsection{Show cases}
%
% \subsubsection{\xfile{letltxmacro-showcases.tex}}
%
%    \begin{macrocode}
%<*showcases>
\NeedsTeXFormat{LaTeX2e}
\makeatletter
%    \end{macrocode}
%    \begin{macro}{\Line}
%    The result is displayed by macro \cs{Line}. The percent symbol
%    at line start allows easy grepping and inserting into the DTX
%    file.
%    \begin{macrocode}
\newcommand*{\Line}[1]{%
  \typeout{\@percentchar#1}%
}
%    \end{macrocode}
%    \end{macro}
%    \begin{macrocode}
\newcommand*{\ShowCmdName}[1]{%
  \@ifundefined{#1}{}{%
    \Line{%
      \space\space(\expandafter\string\csname#1\endcsname) = %
      (\expandafter\meaning\csname#1\endcsname)%
    }%
  }%
}
\newcommand*{\ShowCmds}[1]{%
  \ShowCmdName{#1}%
  \ShowCmdName{#1 }%
  \ShowCmdName{\\#1}%
  \ShowCmdName{\\#1 }%
}
\let\\\@backslashchar
%    \end{macrocode}
%    \begin{macro}{\ShowDef}
%    \begin{macrocode}
\newcommand*{\ShowDef}[2]{%
  \begingroup
    \Line{}%
    \newcommand*{\DefString}{#2}%
    \@onelevel@sanitize\DefString
    \Line{\DefString}%
    #2%
    \ShowCmds{#1}%
  \endgroup
}
%    \end{macrocode}
%    \end{macro}
%    \begin{macrocode}
\typeout{}
\Line{* LaTeX definitions:}
\ShowDef{cmd}{%
  \newcommand{\cmd}[2][default]{}%
}
\ShowDef{cmd}{%
  \DeclareRobustCommand{\cmd}{}%
}
\ShowDef{cmd}{%
  \DeclareRobustCommand{\cmd}[2][default]{}%
}
\typeout{}
%    \end{macrocode}
% The minimal version of package \xpackage{etoolbox} is 2008/06/12 v1.6a
% because it fixes \cs{robustify}.
%    \begin{macrocode}
\RequirePackage{etoolbox}[2008/06/12]%
\Line{}
\Line{* etoolbox's robust definitions:}
\ShowDef{cmd}{%
  \newrobustcmd{\cmd}{}%
}
\ShowDef{cmd}{%
  \newrobustcmd{\cmd}[2][default]{}%
}
\Line{}
\Line{* etoolbox's \string\robustify:}
\ShowDef{cmd}{%
  \newcommand{\cmd}[2][default]{} %
  \robustify{\cmd}%
}
\ShowDef{cmd}{%
  \DeclareRobustCommand{\cmd}{} %
  \robustify{\cmd}%
}
\ShowDef{cmd}{%
  \DeclareRobustCommand{\cmd}[2][default]{} %
  \robustify{\cmd}%
}
\typeout{}
\@@end
%</showcases>
%    \end{macrocode}
%
% \subsubsection{Result}
%
% \begingroup
%   \makeatletter
%   \let\org@verbatim\@verbatim
%   \def\@verbatim{^^A
%     \org@verbatim
%     \catcode`\~=\active
%   }^^A
%   \let~\textvisiblespace
%\begin{verbatim}
%* LaTeX definitions:
%
%\newcommand {\cmd }[2][default]{}
%  (\cmd) = (macro:->\@protected@testopt \cmd \\cmd {default})
%  (\\cmd) = (\long macro:[#1]#2->)
%
%\DeclareRobustCommand {\cmd }{}
%  (\cmd) = (macro:->\protect \cmd~ )
%  (\cmd~) = (\long macro:->)
%
%\DeclareRobustCommand {\cmd }[2][default]{}
%  (\cmd) = (macro:->\protect \cmd~ )
%  (\cmd~) = (macro:->\@protected@testopt \cmd~ \\cmd~ {default})
%  (\\cmd~) = (\long macro:[#1]#2->)
%
%* etoolbox's robust definitions:
%
%\newrobustcmd {\cmd }{}
%  (\cmd) = (\protected\long macro:->)
%
%\newrobustcmd {\cmd }[2][default]{}
%  (\cmd) = (\protected macro:->\@testopt \\cmd {default})
%  (\\cmd) = (\long macro:[#1]#2->)
%
%* etoolbox's \robustify:
%
%\newcommand {\cmd }[2][default]{} \robustify {\cmd }
%  (\cmd) = (\protected macro:->\@protected@testopt \cmd \\cmd {default})
%  (\\cmd) = (\long macro:[#1]#2->)
%
%\DeclareRobustCommand {\cmd }{} \robustify {\cmd }
%  (\cmd) = (\protected macro:->)
%
%\DeclareRobustCommand {\cmd }[2][default]{} \robustify {\cmd }
%  (\cmd) = (\protected macro:->\@protected@testopt \cmd~ \\cmd~ {default})
%  (\cmd~) = (macro:->\@protected@testopt \cmd~ \\cmd~ {default})
%  (\\cmd~) = (\long macro:[#1]#2->)
%\end{verbatim}
% \endgroup
%
% \subsection{Package}
%
%    \begin{macrocode}
%<*package>
%    \end{macrocode}
%
% \subsubsection{Catcodes and identification}
%
%    \begin{macrocode}
\begingroup\catcode61\catcode48\catcode32=10\relax%
  \catcode13=5 % ^^M
  \endlinechar=13 %
  \catcode123=1 % {
  \catcode125=2 % }
  \catcode64=11 % @
  \def\x{\endgroup
    \expandafter\edef\csname llm@AtEnd\endcsname{%
      \endlinechar=\the\endlinechar\relax
      \catcode13=\the\catcode13\relax
      \catcode32=\the\catcode32\relax
      \catcode35=\the\catcode35\relax
      \catcode61=\the\catcode61\relax
      \catcode64=\the\catcode64\relax
      \catcode123=\the\catcode123\relax
      \catcode125=\the\catcode125\relax
    }%
  }%
\x\catcode61\catcode48\catcode32=10\relax%
\catcode13=5 % ^^M
\endlinechar=13 %
\catcode35=6 % #
\catcode64=11 % @
\catcode123=1 % {
\catcode125=2 % }
\def\TMP@EnsureCode#1#2{%
  \edef\llm@AtEnd{%
    \llm@AtEnd
    \catcode#1=\the\catcode#1\relax
  }%
  \catcode#1=#2\relax
}
\TMP@EnsureCode{40}{12}% (
\TMP@EnsureCode{41}{12}% )
\TMP@EnsureCode{42}{12}% *
\TMP@EnsureCode{45}{12}% -
\TMP@EnsureCode{46}{12}% .
\TMP@EnsureCode{47}{12}% /
\TMP@EnsureCode{58}{12}% :
\TMP@EnsureCode{62}{12}% >
\TMP@EnsureCode{91}{12}% [
\TMP@EnsureCode{93}{12}% ]
\edef\llm@AtEnd{%
  \llm@AtEnd
  \escapechar\the\escapechar\relax
  \noexpand\endinput
}
\escapechar=92 % `\\
%    \end{macrocode}
%
%    Package identification.
%    \begin{macrocode}
\NeedsTeXFormat{LaTeX2e}
\ProvidesPackage{letltxmacro}%
  [2010/09/02 v1.4 Let assignment for LaTeX macros (HO)]
%    \end{macrocode}
%
% \subsubsection{Main macros}
%
%    \begin{macro}{\LetLtxMacro}
%    \begin{macrocode}
\newcommand*{\LetLtxMacro}{%
  \llm@ModeLetLtxMacro{}%
}
%    \end{macrocode}
%    \end{macro}
%    \begin{macro}{\GlobalLetLtxMacro}
%    \begin{macrocode}
\newcommand*{\GlobalLetLtxMacro}{%
  \llm@ModeLetLtxMacro\global
}
%    \end{macrocode}
%    \end{macro}
%
%    \begin{macro}{\llm@ModeLetLtxMacro}
%    \begin{macrocode}
\newcommand*{\llm@ModeLetLtxMacro}[3]{%
  \edef\llm@escapechar{\the\escapechar}%
  \escapechar=-1 %
  \edef\reserved@a{%
    \noexpand\protect
    \expandafter\noexpand
    \csname\string#3 \endcsname
  }%
  \ifx\reserved@a#3\relax
    #1\edef#2{%
      \noexpand\protect
      \expandafter\noexpand
      \csname\string#2 \endcsname
    }%
    #1\expandafter\let
    \csname\string#2 \expandafter\endcsname
    \csname\string#3 \endcsname
    \expandafter\llm@LetLtxMacro
        \csname\string#2 \expandafter\endcsname
        \csname\string#3 \endcsname{#1}%
  \else
    \llm@LetLtxMacro{#2}{#3}{#1}%
  \fi
  \escapechar=\llm@escapechar\relax
}
%    \end{macrocode}
%    \end{macro}
%    \begin{macro}{\llm@LetLtxMacro}
%    \begin{macrocode}
\def\llm@LetLtxMacro#1#2#3{%
  \escapechar=92 %
  \expandafter\llm@CheckParams\meaning#2:->\@nil{%
    \begingroup
      \def\@protected@testopt{%
        \expandafter\@testopt\@gobble
      }%
      \def\@testopt##1##2{%
        \toks@={##2}%
      }%
      \let\llm@testopt\@empty
      \edef\x{%
        \noexpand\@protected@testopt
        \noexpand#2%
        \expandafter\noexpand\csname\string#2\endcsname
      }%
      \expandafter\expandafter\expandafter\def
      \expandafter\expandafter\expandafter\y
      \expandafter\expandafter\expandafter{%
        \expandafter\llm@CarThree#2{}{}{}\llm@nil
      }%
      \ifx\x\y
        #2%
        \def\llm@testopt{%
          \noexpand\@protected@testopt
          \noexpand#1%
        }%
      \else
        \edef\x{%
          \noexpand\@testopt
          \expandafter\noexpand
          \csname\string#2\endcsname
        }%
        \expandafter\expandafter\expandafter\def
        \expandafter\expandafter\expandafter\y
        \expandafter\expandafter\expandafter{%
          \expandafter\llm@CarTwo#2{}{}\llm@nil
        }%
        \ifx\x\y
          #2%
          \def\llm@testopt{%
            \noexpand\@testopt
          }%
        \fi
      \fi
      \ifx\llm@testopt\@empty
      \else
        \llm@protected\xdef\llm@GlobalTemp{%
          \llm@testopt
          \expandafter\noexpand
          \csname\string#1\endcsname
          {\the\toks@}%
        }%
      \fi
    \expandafter\endgroup\ifx\llm@testopt\@empty
      #3\let#1=#2\relax
    \else
      #3\let#1=\llm@GlobalTemp
      #3\expandafter\let
          \csname\string#1\expandafter\endcsname
          \csname\string#2\endcsname
    \fi
  }{%
    #3\let#1=#2\relax
  }%
}
%    \end{macrocode}
%    \end{macro}
%    \begin{macro}{\llm@CheckParams}
%    \begin{macrocode}
\def\llm@CheckParams#1:->#2\@nil{%
  \begingroup
    \def\x{#1}%
  \ifx\x\llm@macro
    \endgroup
    \def\llm@protected{}%
    \expandafter\@firstoftwo
  \else
    \ifx\x\llm@protectedmacro
      \endgroup
      \def\llm@protected{\protected}%
      \expandafter\expandafter\expandafter\@firstoftwo
    \else
      \endgroup
      \expandafter\expandafter\expandafter\@secondoftwo
    \fi
  \fi
}
%    \end{macrocode}
%    \end{macro}
%    \begin{macro}{\llm@macro}
%    \begin{macrocode}
\def\llm@macro{macro}
\@onelevel@sanitize\llm@macro
%    \end{macrocode}
%    \end{macro}
%    \begin{macro}{\llm@protectedmacro}
%    \begin{macrocode}
\def\llm@protectedmacro{\protected macro}
\@onelevel@sanitize\llm@protectedmacro
%    \end{macrocode}
%    \end{macro}
%    \begin{macro}{\llm@CarThree}
%    \begin{macrocode}
\def\llm@CarThree#1#2#3#4\llm@nil{#1#2#3}%
%    \end{macrocode}
%    \end{macro}
%    \begin{macro}{\llm@CarTwo}
%    \begin{macrocode}
\def\llm@CarTwo#1#2#3\llm@nil{#1#2}%
%    \end{macrocode}
%    \end{macro}
%
%    \begin{macrocode}
\llm@AtEnd%
%</package>
%    \end{macrocode}
%
% \section{Test}
%
% \subsection{Catcode checks for loading}
%
%    \begin{macrocode}
%<*test1>
%    \end{macrocode}
%    \begin{macrocode}
\NeedsTeXFormat{LaTeX2e}
\documentclass{minimal}
\makeatletter
\def\RestoreCatcodes{}
\count@=0 %
\loop
  \edef\RestoreCatcodes{%
    \RestoreCatcodes
    \catcode\the\count@=\the\catcode\count@\relax
  }%
\ifnum\count@<255 %
  \advance\count@\@ne
\repeat

\def\RangeCatcodeInvalid#1#2{%
  \count@=#1\relax
  \loop
    \catcode\count@=15 %
  \ifnum\count@<#2\relax
    \advance\count@\@ne
  \repeat
}
\def\Test{%
  \RangeCatcodeInvalid{0}{47}%
  \RangeCatcodeInvalid{58}{64}%
  \RangeCatcodeInvalid{91}{96}%
  \RangeCatcodeInvalid{123}{127}%
  \catcode`\@=12 %
  \catcode`\\=0 %
  \catcode`\{=1 %
  \catcode`\}=2 %
  \catcode`\#=6 %
  \catcode`\[=12 %
  \catcode`\]=12 %
  \catcode`\%=14 %
  \catcode`\ =10 %
  \catcode13=5 %
  \RequirePackage{letltxmacro}[2010/09/02]\relax
  \RestoreCatcodes
}
\Test
\csname @@end\endcsname
\end
%    \end{macrocode}
%    \begin{macrocode}
%</test1>
%    \end{macrocode}
%
% \subsection{Package tests}
%
%    \begin{macrocode}
%<*test2>
%    \end{macrocode}
%    \begin{macrocode}
\NeedsTeXFormat{LaTeX2e}
\nofiles
\documentclass{minimal}

\usepackage{letltxmacro}[2010/09/02]

\usepackage{qstest}
\IncludeTests{*}
\LogTests{log}{*}{*}

\makeatletter

\def\TestDef#1{%
  \begingroup
    \@makeother\\%
    \@makeother\ %
    \expandafter\@TestDef\csname #1\endcsname
}
\def\@TestDef#1#2{%
    \edef\a{\expandafter\strip@prefix\meaning#1}%
    \edef\b{\detokenize{#2}}%
    \ifx\a\b
    \else
      \typeout{1 [\a]}%
      \typeout{2 [\b]}%
    \fi
  \endgroup
  \Expect*{\expandafter\strip@prefix\meaning#1}*{\detokenize{#2}}%
}
\def\TestEquals#1#2{%
  \Expect*{%
    \expandafter\ifx\csname#1\expandafter\endcsname
                    \csname#2\endcsname
      equals%
    \else
      wrong%
    \fi
  }{equals}%
}
\def\SaveA{%
  \let\SavedA\TestA
  \expandafter\let\csname\string\SavedA\expandafter\endcsname
                  \csname\string\TestA\endcsname
  \expandafter\let\csname SavedA \expandafter\endcsname
                  \csname TestA \endcsname
}
\def\CheckA{%
  \TestEquals{SavedA}{TestA}%
  \TestEquals{\string\SavedA}{\string\TestA}%
  \TestEquals{SavedA }{TestA }%
}

\begin{qstest}{robust}{robust}
  \begingroup
    \DeclareRobustCommand*{\TestA}{Test}%
    \SaveA
    \TestDef{TestA}{\protect \TestA  }%
    \CheckA
    \DeclareRobustCommand{\TestA}{Test}%
    \SaveA
    \TestDef{TestA}{\protect \TestA  }%
    \LetLtxMacro\TestB\TestA
    \TestDef{TestB}{\protect \TestB  }%
    \TestEquals{TestB }{TestA }%
    \CheckA
    \GlobalLetLtxMacro\TestC\TestA
    \CheckA
  \endgroup
  \TestDef{TestC}{\protect \TestC  }%
  \DeclareRobustCommand{\TestA}{Test}%
  \TestEquals{TestC }{TestA }%
\end{qstest}

\begin{qstest}{default}{default}
  \begingroup
    \newcommand{\TestA}[1][\relax default \empty]{TestA #1}%
    \SaveA
    \TestDef
    {TestA}{\@protected@testopt \TestA \\TestA {\relax default \empty }}%
    \LetLtxMacro\TestB\TestA
    \TestDef
    {TestB}{\@protected@testopt \TestB \\TestB {\relax default \empty }}%
    \CheckA
    \TestEquals{\string\TestB}{\string\TestA}%
    \GlobalLetLtxMacro\TestC\TestA
    \CheckA
  \endgroup
  \TestDef
  {TestC}{\@protected@testopt \TestC \\TestC {\relax default \empty }}%
  \newcommand{\TestA}[1][\relax default \empty]{TestA #1}%
  \TestEquals{\string\TestC}{\string\TestA}%
\end{qstest}

\begin{qstest}{robustdefault}{robustdefault}
  \begingroup
    \DeclareRobustCommand{\TestA}[1][\default]{TestA}%
    \SaveA
    \LetLtxMacro\TestB\TestA
    \TestDef{TestB}{\protect \TestB  }%
    \TestDef{TestB }{\@protected@testopt \TestB  \\TestB  {\default }}%
    \CheckA
    \GlobalLetLtxMacro\TestC\TestA
    \CheckA
  \endgroup
  \TestDef{TestC}{\protect \TestC  }%
  \TestDef{TestC }{\@protected@testopt \TestC  \\TestC  {\default }}%
\end{qstest}

\begin{qstest}{plain}{plain}
  \LetLtxMacro\NewRelax\relax
  \Expect*{\meaning\relax}*{\string\relax}%
  \Expect*{\meaning\NewRelax}*{\string\relax}%
  \LetLtxMacro\NewHbox\hbox
  \Expect*{\meaning\hbox}*{\string\hbox}%
  \Expect*{\meaning\NewHbox}*{\string\hbox}%
  \LetLtxMacro\NewEmpty\empty
  \Expect*{\meaning\empty}{macro:->}%
  \Expect*{\meaning\NewEmpty}{macro:->}%
  \def\TestA{\iffalse}%
  \LetLtxMacro\TestB\TestA
  \TestDef{TestB}{\iffalse }%
  \TestEquals{TestA}{TestB}%
\end{qstest}
%    \end{macrocode}
%
%    \begin{macrocode}
\usepackage{etoolbox}[2008/06/22]

\begin{qstest}{oldnewrobustcmd}{oldnewrobustcmd}
  \begingroup
    \protected\edef\TestA{%
      \noexpand\@protected@testopt
      \noexpand\TestA
      \expandafter\noexpand\csname\string\TestA\endcsname
      {\noexpand\default}%
    }%
    \expandafter\protected\expandafter
    \def\csname\string\TestA\endcsname{TestA}%
    \SaveA
    \LetLtxMacro\TestB\TestA
    \TestDef{TestB}{\@protected@testopt \TestB \\TestB {\default }}%
    \protected\edef\TestC{%
      \noexpand\@protected@testopt
      \noexpand\TestB
      \expandafter\noexpand\csname\string\TestB\endcsname
      {\noexpand\default}%
    }%
    \TestEquals{TestB}{TestC}%
    \CheckA
    \GlobalLetLtxMacro\TestD\TestA
    \CheckA
  \endgroup
  \TestDef{TestD}{\@protected@testopt \TestD \\TestD {\default }}%
  \protected\edef\TestC{%
    \noexpand\@protected@testopt
    \noexpand\TestD
    \expandafter\noexpand\csname\string\TestD\endcsname
    {\noexpand\default}%
  }%
  \TestEquals{TestD}{TestC}%
\end{qstest}

\begin{qstest}{newrobustcmd}{newrobustcmd}
  \begingroup
    \newrobustcmd{\TestA}[1][\default]{TestA}%
    \SaveA
    \LetLtxMacro\TestB\TestA
    \TestDef{TestB}{\@testopt \\TestB {\default }}%
    \protected\edef\TestC{%
      \noexpand\@testopt
      \expandafter\noexpand\csname\string\TestB\endcsname
      {\noexpand\default}%
    }%
    \TestEquals{TestB}{TestC}%
    \CheckA
    \GlobalLetLtxMacro\TestD\TestA
    \CheckA
  \endgroup
  \TestDef{TestD}{\@testopt \\TestD {\default }}%
  \protected\edef\TestC{%
    \noexpand\@testopt
    \expandafter\noexpand\csname\string\TestD\endcsname
    {\noexpand\default}%
  }%
  \TestEquals{TestD}{TestC}%
\end{qstest}

\begin{qstest}{robustifyopt}{robustifyopt}
  \newcommand{\TestA}[2][\default]{}%
  \TestDef{TestA}{\@protected@testopt \TestA \\TestA {\default }}%
  \robustify\TestA
  \TestDef{TestA}{\@protected@testopt \TestA \\TestA {\default }}%
  \protected\edef\TestC{%
    \noexpand\@protected@testopt
    \noexpand\TestA
    \expandafter\noexpand\csname\string\TestA\endcsname
    {\noexpand\default}%
  }%
  \TestEquals{TestA}{TestC}%
  \SaveA
  \LetLtxMacro\TestB\TestA
  \TestDef{TestB}{\@protected@testopt \TestB \\TestB {\default }}%
  \protected\edef\TestC{%
    \noexpand\@protected@testopt
    \noexpand\TestB
    \expandafter\noexpand\csname\string\TestB\endcsname
    {\noexpand\default}%
  }%
  \TestEquals{TestB}{TestC}%
  \CheckA
\end{qstest}

\begin{qstest}{robustifydeclare}{robustifydeclare}
  \DeclareRobustCommand\TestA{\iffalse}%
  \robustify\TestA
  \SaveA
  \LetLtxMacro\TestB\TestA
  \TestDef{TestB}{\iffalse }%
  % before etoolbox 2008/06/22: \protected\long\def
  % since etoolbox 2008/06/22: \protected\def
  \protected\def\TestC{\iffalse}%
  \TestEquals{TestB}{TestC}%
  \CheckA
\end{qstest}

\begin{qstest}{robustifydeclarelong}{robustifydeclarelong}
  \DeclareRobustCommand\TestA[1]{\iffalse}%
  \robustify\TestA
  \SaveA
  \LetLtxMacro\TestB\TestA
  \TestDef{TestB}{\iffalse }%
  % before etoolbox 2008/06/22: \protected\long\def
  % since etoolbox 2008/06/22: \protected\def
  \protected\long\def\TestC#1{\iffalse}%
  \TestEquals{TestB}{TestC}%
  \CheckA
\end{qstest}
%    \end{macrocode}
%
%    \begin{macrocode}
\csname @@end\endcsname
%    \end{macrocode}
%    \begin{macrocode}
%</test2>
%    \end{macrocode}
%
% \section{Installation}
%
% \subsection{Download}
%
% \paragraph{Package.} This package is available on
% CTAN\footnote{\url{ftp://ftp.ctan.org/tex-archive/}}:
% \begin{description}
% \item[\CTAN{macros/latex/contrib/oberdiek/letltxmacro.dtx}] The source file.
% \item[\CTAN{macros/latex/contrib/oberdiek/letltxmacro.pdf}] Documentation.
% \end{description}
%
%
% \paragraph{Bundle.} All the packages of the bundle `oberdiek'
% are also available in a TDS compliant ZIP archive. There
% the packages are already unpacked and the documentation files
% are generated. The files and directories obey the TDS standard.
% \begin{description}
% \item[\CTAN{install/macros/latex/contrib/oberdiek.tds.zip}]
% \end{description}
% \emph{TDS} refers to the standard ``A Directory Structure
% for \TeX\ Files'' (\CTAN{tds/tds.pdf}). Directories
% with \xfile{texmf} in their name are usually organized this way.
%
% \subsection{Bundle installation}
%
% \paragraph{Unpacking.} Unpack the \xfile{oberdiek.tds.zip} in the
% TDS tree (also known as \xfile{texmf} tree) of your choice.
% Example (linux):
% \begin{quote}
%   |unzip oberdiek.tds.zip -d ~/texmf|
% \end{quote}
%
% \paragraph{Script installation.}
% Check the directory \xfile{TDS:scripts/oberdiek/} for
% scripts that need further installation steps.
% Package \xpackage{attachfile2} comes with the Perl script
% \xfile{pdfatfi.pl} that should be installed in such a way
% that it can be called as \texttt{pdfatfi}.
% Example (linux):
% \begin{quote}
%   |chmod +x scripts/oberdiek/pdfatfi.pl|\\
%   |cp scripts/oberdiek/pdfatfi.pl /usr/local/bin/|
% \end{quote}
%
% \subsection{Package installation}
%
% \paragraph{Unpacking.} The \xfile{.dtx} file is a self-extracting
% \docstrip\ archive. The files are extracted by running the
% \xfile{.dtx} through \plainTeX:
% \begin{quote}
%   \verb|tex letltxmacro.dtx|
% \end{quote}
%
% \paragraph{TDS.} Now the different files must be moved into
% the different directories in your installation TDS tree
% (also known as \xfile{texmf} tree):
% \begin{quote}
% \def\t{^^A
% \begin{tabular}{@{}>{\ttfamily}l@{ $\rightarrow$ }>{\ttfamily}l@{}}
%   letltxmacro.sty & tex/latex/oberdiek/letltxmacro.sty\\
%   letltxmacro.pdf & doc/latex/oberdiek/letltxmacro.pdf\\
%   letltxmacro-showcases.tex & doc/latex/oberdiek/letltxmacro-showcases.tex\\
%   test/letltxmacro-test1.tex & doc/latex/oberdiek/test/letltxmacro-test1.tex\\
%   test/letltxmacro-test2.tex & doc/latex/oberdiek/test/letltxmacro-test2.tex\\
%   letltxmacro.dtx & source/latex/oberdiek/letltxmacro.dtx\\
% \end{tabular}^^A
% }^^A
% \sbox0{\t}^^A
% \ifdim\wd0>\linewidth
%   \begingroup
%     \advance\linewidth by\leftmargin
%     \advance\linewidth by\rightmargin
%   \edef\x{\endgroup
%     \def\noexpand\lw{\the\linewidth}^^A
%   }\x
%   \def\lwbox{^^A
%     \leavevmode
%     \hbox to \linewidth{^^A
%       \kern-\leftmargin\relax
%       \hss
%       \usebox0
%       \hss
%       \kern-\rightmargin\relax
%     }^^A
%   }^^A
%   \ifdim\wd0>\lw
%     \sbox0{\small\t}^^A
%     \ifdim\wd0>\linewidth
%       \ifdim\wd0>\lw
%         \sbox0{\footnotesize\t}^^A
%         \ifdim\wd0>\linewidth
%           \ifdim\wd0>\lw
%             \sbox0{\scriptsize\t}^^A
%             \ifdim\wd0>\linewidth
%               \ifdim\wd0>\lw
%                 \sbox0{\tiny\t}^^A
%                 \ifdim\wd0>\linewidth
%                   \lwbox
%                 \else
%                   \usebox0
%                 \fi
%               \else
%                 \lwbox
%               \fi
%             \else
%               \usebox0
%             \fi
%           \else
%             \lwbox
%           \fi
%         \else
%           \usebox0
%         \fi
%       \else
%         \lwbox
%       \fi
%     \else
%       \usebox0
%     \fi
%   \else
%     \lwbox
%   \fi
% \else
%   \usebox0
% \fi
% \end{quote}
% If you have a \xfile{docstrip.cfg} that configures and enables \docstrip's
% TDS installing feature, then some files can already be in the right
% place, see the documentation of \docstrip.
%
% \subsection{Refresh file name databases}
%
% If your \TeX~distribution
% (\teTeX, \mikTeX, \dots) relies on file name databases, you must refresh
% these. For example, \teTeX\ users run \verb|texhash| or
% \verb|mktexlsr|.
%
% \subsection{Some details for the interested}
%
% \paragraph{Attached source.}
%
% The PDF documentation on CTAN also includes the
% \xfile{.dtx} source file. It can be extracted by
% AcrobatReader 6 or higher. Another option is \textsf{pdftk},
% e.g. unpack the file into the current directory:
% \begin{quote}
%   \verb|pdftk letltxmacro.pdf unpack_files output .|
% \end{quote}
%
% \paragraph{Unpacking with \LaTeX.}
% The \xfile{.dtx} chooses its action depending on the format:
% \begin{description}
% \item[\plainTeX:] Run \docstrip\ and extract the files.
% \item[\LaTeX:] Generate the documentation.
% \end{description}
% If you insist on using \LaTeX\ for \docstrip\ (really,
% \docstrip\ does not need \LaTeX), then inform the autodetect routine
% about your intention:
% \begin{quote}
%   \verb|latex \let\install=y\input{letltxmacro.dtx}|
% \end{quote}
% Do not forget to quote the argument according to the demands
% of your shell.
%
% \paragraph{Generating the documentation.}
% You can use both the \xfile{.dtx} or the \xfile{.drv} to generate
% the documentation. The process can be configured by the
% configuration file \xfile{ltxdoc.cfg}. For instance, put this
% line into this file, if you want to have A4 as paper format:
% \begin{quote}
%   \verb|\PassOptionsToClass{a4paper}{article}|
% \end{quote}
% An example follows how to generate the
% documentation with pdf\LaTeX:
% \begin{quote}
%\begin{verbatim}
%pdflatex letltxmacro.dtx
%makeindex -s gind.ist letltxmacro.idx
%pdflatex letltxmacro.dtx
%makeindex -s gind.ist letltxmacro.idx
%pdflatex letltxmacro.dtx
%\end{verbatim}
% \end{quote}
%
% \section{Catalogue}
%
% The following XML file can be used as source for the
% \href{http://mirror.ctan.org/help/Catalogue/catalogue.html}{\TeX\ Catalogue}.
% The elements \texttt{caption} and \texttt{description} are imported
% from the original XML file from the Catalogue.
% The name of the XML file in the Catalogue is \xfile{letltxmacro.xml}.
%    \begin{macrocode}
%<*catalogue>
<?xml version='1.0' encoding='us-ascii'?>
<!DOCTYPE entry SYSTEM 'catalogue.dtd'>
<entry datestamp='$Date$' modifier='$Author$' id='letltxmacro'>
  <name>letltxmacro</name>
  <caption>Let assignment for LaTeX macros.</caption>
  <authorref id='auth:oberdiek'/>
  <copyright owner='Heiko Oberdiek' year='2008,2010'/>
  <license type='lppl1.3'/>
  <version number='1.4'/>
  <description>
    TeX&#x2019;s <tt>\let</tt> assignment does not work for LaTeX macros
    with optional arguments or for macros that are defined
    as robust macros by <tt>\DeclareRobustCommand</tt>.  This package
    defines <tt>\LetLtxMacro</tt> that also takes care of the involved
    internal macros.
    <p/>
    The package is part of the <xref refid='oberdiek'>oberdiek</xref>
    bundle.
  </description>
  <documentation details='Package documentation'
      href='ctan:/macros/latex/contrib/oberdiek/letltxmacro.pdf'/>
  <ctan file='true' path='/macros/latex/contrib/oberdiek/letltxmacro.dtx'/>
  <miktex location='oberdiek'/>
  <texlive location='oberdiek'/>
  <install path='/macros/latex/contrib/oberdiek/oberdiek.tds.zip'/>
</entry>
%</catalogue>
%    \end{macrocode}
%
% \begin{History}
%   \begin{Version}{2008/06/09 v1.0}
%   \item
%     First version.
%   \end{Version}
%   \begin{Version}{2008/06/12 v1.1}
%   \item
%     Support for \xpackage{etoolbox}'s \cs{newrobustcmd} added.
%   \end{Version}
%   \begin{Version}{2008/06/13 v1.2}
%   \item
%     Support for \xpackage{etoolbox}'s \cs{robustify} added.
%   \end{Version}
%   \begin{Version}{2008/06/24 v1.3}
%   \item
%     Test file adapted for etoolbox 2008/06/22 v1.6.
%   \end{Version}
%   \begin{Version}{2010/09/02 v1.4}
%   \item
%     \cs{GlobalLetLtxMacro} added.
%   \end{Version}
% \end{History}
%
% \PrintIndex
%
% \Finale
\endinput

%        (quote the arguments according to the demands of your shell)
%
% Documentation:
%    (a) If letltxmacro.drv is present:
%           latex letltxmacro.drv
%    (b) Without letltxmacro.drv:
%           latex letltxmacro.dtx; ...
%    The class ltxdoc loads the configuration file ltxdoc.cfg
%    if available. Here you can specify further options, e.g.
%    use A4 as paper format:
%       \PassOptionsToClass{a4paper}{article}
%
%    Programm calls to get the documentation (example):
%       pdflatex letltxmacro.dtx
%       makeindex -s gind.ist letltxmacro.idx
%       pdflatex letltxmacro.dtx
%       makeindex -s gind.ist letltxmacro.idx
%       pdflatex letltxmacro.dtx
%
% Installation:
%    TDS:tex/latex/oberdiek/letltxmacro.sty
%    TDS:doc/latex/oberdiek/letltxmacro.pdf
%    TDS:doc/latex/oberdiek/letltxmacro-showcases.tex
%    TDS:doc/latex/oberdiek/test/letltxmacro-test1.tex
%    TDS:doc/latex/oberdiek/test/letltxmacro-test2.tex
%    TDS:source/latex/oberdiek/letltxmacro.dtx
%
%<*ignore>
\begingroup
  \catcode123=1 %
  \catcode125=2 %
  \def\x{LaTeX2e}%
\expandafter\endgroup
\ifcase 0\ifx\install y1\fi\expandafter
         \ifx\csname processbatchFile\endcsname\relax\else1\fi
         \ifx\fmtname\x\else 1\fi\relax
\else\csname fi\endcsname
%</ignore>
%<*install>
\input docstrip.tex
\Msg{************************************************************************}
\Msg{* Installation}
\Msg{* Package: letltxmacro 2010/09/02 v1.4 Let assignment for LaTeX macros (HO)}
\Msg{************************************************************************}

\keepsilent
\askforoverwritefalse

\let\MetaPrefix\relax
\preamble

This is a generated file.

Project: letltxmacro
Version: 2010/09/02 v1.4

Copyright (C) 2008, 2010 by
   Heiko Oberdiek <heiko.oberdiek at googlemail.com>

This work may be distributed and/or modified under the
conditions of the LaTeX Project Public License, either
version 1.3c of this license or (at your option) any later
version. This version of this license is in
   http://www.latex-project.org/lppl/lppl-1-3c.txt
and the latest version of this license is in
   http://www.latex-project.org/lppl.txt
and version 1.3 or later is part of all distributions of
LaTeX version 2005/12/01 or later.

This work has the LPPL maintenance status "maintained".

This Current Maintainer of this work is Heiko Oberdiek.

This work consists of the main source file letltxmacro.dtx
and the derived files
   letltxmacro.sty, letltxmacro.pdf, letltxmacro.ins, letltxmacro.drv,
   letltxmacro-showcases.tex, letltxmacro-test1.tex,
   letltxmacro-test2.tex.

\endpreamble
\let\MetaPrefix\DoubleperCent

\generate{%
  \file{letltxmacro.ins}{\from{letltxmacro.dtx}{install}}%
  \file{letltxmacro.drv}{\from{letltxmacro.dtx}{driver}}%
  \usedir{tex/latex/oberdiek}%
  \file{letltxmacro.sty}{\from{letltxmacro.dtx}{package}}%
  \usedir{doc/latex/oberdiek}%
  \file{letltxmacro-showcases.tex}{\from{letltxmacro.dtx}{showcases}}%
  \usedir{doc/latex/oberdiek/test}%
  \file{letltxmacro-test1.tex}{\from{letltxmacro.dtx}{test1}}%
  \file{letltxmacro-test2.tex}{\from{letltxmacro.dtx}{test2}}%
  \nopreamble
  \nopostamble
  \usedir{source/latex/oberdiek/catalogue}%
  \file{letltxmacro.xml}{\from{letltxmacro.dtx}{catalogue}}%
}

\catcode32=13\relax% active space
\let =\space%
\Msg{************************************************************************}
\Msg{*}
\Msg{* To finish the installation you have to move the following}
\Msg{* file into a directory searched by TeX:}
\Msg{*}
\Msg{*     letltxmacro.sty}
\Msg{*}
\Msg{* To produce the documentation run the file `letltxmacro.drv'}
\Msg{* through LaTeX.}
\Msg{*}
\Msg{* Happy TeXing!}
\Msg{*}
\Msg{************************************************************************}

\endbatchfile
%</install>
%<*ignore>
\fi
%</ignore>
%<*driver>
\NeedsTeXFormat{LaTeX2e}
\ProvidesFile{letltxmacro.drv}%
  [2010/09/02 v1.4 Let assignment for LaTeX macros (HO)]%
\documentclass{ltxdoc}
\usepackage{holtxdoc}[2011/11/22]
\begin{document}
  \DocInput{letltxmacro.dtx}%
\end{document}
%</driver>
% \fi
%
% \CheckSum{990}
%
% \CharacterTable
%  {Upper-case    \A\B\C\D\E\F\G\H\I\J\K\L\M\N\O\P\Q\R\S\T\U\V\W\X\Y\Z
%   Lower-case    \a\b\c\d\e\f\g\h\i\j\k\l\m\n\o\p\q\r\s\t\u\v\w\x\y\z
%   Digits        \0\1\2\3\4\5\6\7\8\9
%   Exclamation   \!     Double quote  \"     Hash (number) \#
%   Dollar        \$     Percent       \%     Ampersand     \&
%   Acute accent  \'     Left paren    \(     Right paren   \)
%   Asterisk      \*     Plus          \+     Comma         \,
%   Minus         \-     Point         \.     Solidus       \/
%   Colon         \:     Semicolon     \;     Less than     \<
%   Equals        \=     Greater than  \>     Question mark \?
%   Commercial at \@     Left bracket  \[     Backslash     \\
%   Right bracket \]     Circumflex    \^     Underscore    \_
%   Grave accent  \`     Left brace    \{     Vertical bar  \|
%   Right brace   \}     Tilde         \~}
%
% \GetFileInfo{letltxmacro.drv}
%
% \title{The \xpackage{letltxmacro} package}
% \date{2010/09/02 v1.4}
% \author{Heiko Oberdiek\\\xemail{heiko.oberdiek at googlemail.com}}
%
% \maketitle
%
% \begin{abstract}
% \TeX's \cs{let} assignment does not work for \LaTeX\ macros
% with optional arguments or for macros that are defined
% as robust macros by \cs{DeclareRobustCommand}. This package
% defines \cs{LetLtxMacro} that also takes care of the involved
% internal macros.
% \end{abstract}
%
% \tableofcontents
%
% \section{Documentation}
%
% If someone wants to redefine a macro with using the old
% meaning, then one method is \TeX's command \cs{let}:
%\begin{quote}
%\begin{verbatim}
%\newcommand{\Macro}{\typeout{Test Macro}}
%\let\SavedMacro=\Macro
%\renewcommand{\Macro}{%
%  \typeout{Begin}%
%  \SavedMacro
%  \typeout{End}%
%}
%\end{verbatim}
%\end{quote}
% However, this method fails, if \cs{Macro} is defined
% by \cs{DeclareRobustCommand} and/or has an optional argument.
% In both cases \LaTeX\ defines an additional internal macro
% that is forgotten in the simple \cs{let} assignment of
% the example above.
%
% \begin{declcs}{LetLtxMacro} \M{new macro} \M{old macro}
% \end{declcs}
% Macro \cs{LetLtxMacro} behaves similar to \TeX's \cs{let}
% assignment, but it takes care of macros that are
% defined by \cs{DeclareRobustCommand} and/or have optional
% arguments. Example:
%\begin{quote}
%\begin{verbatim}
%\DeclareRobustCommand{\Macro}[1][default]{...}
%\LetLtxMacro{\SavedMacro}{\Macro}
%\end{verbatim}
%\end{quote}
% Then macro \cs{SavedMacro} only uses internal macro names
% that are derived from \cs{SavedMacro}'s macro name. Macro \cs{Macro}
% can now be redefined without affecting \cs{SavedMacro}.
%
% \begin{declcs}{GlobalLetLtxMacro} \M{new macro} \M{old macro}
% \end{declcs}
% Like \cs{LetLtxMacro}, but the \meta{new macro} is defined globally.
% Since version 2010/09/02~v1.4.
%
% \subsection{Supported macro definition commands}
%
% \begin{quote}
%   \begin{tabular}{@{}ll@{}}
%     \cs{newcommand}, \cs{renewcommand} & latex/base\\
%     \cs{newenvironment}, \cs{renewenvironment} & latex/base\\
%     \cs{DeclareRobustCommand}& latex/base\\
%     \cs{newrobustcmd}, \cs{renewrobustcmd} & etoolbox\\
%     \cs{robustify} & etoolbox 2008/06/22 v1.6\\
%   \end{tabular}
% \end{quote}
%
% \StopEventually{
% }
%
% \section{Implementation}
%
% \subsection{Show cases}
%
% \subsubsection{\xfile{letltxmacro-showcases.tex}}
%
%    \begin{macrocode}
%<*showcases>
\NeedsTeXFormat{LaTeX2e}
\makeatletter
%    \end{macrocode}
%    \begin{macro}{\Line}
%    The result is displayed by macro \cs{Line}. The percent symbol
%    at line start allows easy grepping and inserting into the DTX
%    file.
%    \begin{macrocode}
\newcommand*{\Line}[1]{%
  \typeout{\@percentchar#1}%
}
%    \end{macrocode}
%    \end{macro}
%    \begin{macrocode}
\newcommand*{\ShowCmdName}[1]{%
  \@ifundefined{#1}{}{%
    \Line{%
      \space\space(\expandafter\string\csname#1\endcsname) = %
      (\expandafter\meaning\csname#1\endcsname)%
    }%
  }%
}
\newcommand*{\ShowCmds}[1]{%
  \ShowCmdName{#1}%
  \ShowCmdName{#1 }%
  \ShowCmdName{\\#1}%
  \ShowCmdName{\\#1 }%
}
\let\\\@backslashchar
%    \end{macrocode}
%    \begin{macro}{\ShowDef}
%    \begin{macrocode}
\newcommand*{\ShowDef}[2]{%
  \begingroup
    \Line{}%
    \newcommand*{\DefString}{#2}%
    \@onelevel@sanitize\DefString
    \Line{\DefString}%
    #2%
    \ShowCmds{#1}%
  \endgroup
}
%    \end{macrocode}
%    \end{macro}
%    \begin{macrocode}
\typeout{}
\Line{* LaTeX definitions:}
\ShowDef{cmd}{%
  \newcommand{\cmd}[2][default]{}%
}
\ShowDef{cmd}{%
  \DeclareRobustCommand{\cmd}{}%
}
\ShowDef{cmd}{%
  \DeclareRobustCommand{\cmd}[2][default]{}%
}
\typeout{}
%    \end{macrocode}
% The minimal version of package \xpackage{etoolbox} is 2008/06/12 v1.6a
% because it fixes \cs{robustify}.
%    \begin{macrocode}
\RequirePackage{etoolbox}[2008/06/12]%
\Line{}
\Line{* etoolbox's robust definitions:}
\ShowDef{cmd}{%
  \newrobustcmd{\cmd}{}%
}
\ShowDef{cmd}{%
  \newrobustcmd{\cmd}[2][default]{}%
}
\Line{}
\Line{* etoolbox's \string\robustify:}
\ShowDef{cmd}{%
  \newcommand{\cmd}[2][default]{} %
  \robustify{\cmd}%
}
\ShowDef{cmd}{%
  \DeclareRobustCommand{\cmd}{} %
  \robustify{\cmd}%
}
\ShowDef{cmd}{%
  \DeclareRobustCommand{\cmd}[2][default]{} %
  \robustify{\cmd}%
}
\typeout{}
\@@end
%</showcases>
%    \end{macrocode}
%
% \subsubsection{Result}
%
% \begingroup
%   \makeatletter
%   \let\org@verbatim\@verbatim
%   \def\@verbatim{^^A
%     \org@verbatim
%     \catcode`\~=\active
%   }^^A
%   \let~\textvisiblespace
%\begin{verbatim}
%* LaTeX definitions:
%
%\newcommand {\cmd }[2][default]{}
%  (\cmd) = (macro:->\@protected@testopt \cmd \\cmd {default})
%  (\\cmd) = (\long macro:[#1]#2->)
%
%\DeclareRobustCommand {\cmd }{}
%  (\cmd) = (macro:->\protect \cmd~ )
%  (\cmd~) = (\long macro:->)
%
%\DeclareRobustCommand {\cmd }[2][default]{}
%  (\cmd) = (macro:->\protect \cmd~ )
%  (\cmd~) = (macro:->\@protected@testopt \cmd~ \\cmd~ {default})
%  (\\cmd~) = (\long macro:[#1]#2->)
%
%* etoolbox's robust definitions:
%
%\newrobustcmd {\cmd }{}
%  (\cmd) = (\protected\long macro:->)
%
%\newrobustcmd {\cmd }[2][default]{}
%  (\cmd) = (\protected macro:->\@testopt \\cmd {default})
%  (\\cmd) = (\long macro:[#1]#2->)
%
%* etoolbox's \robustify:
%
%\newcommand {\cmd }[2][default]{} \robustify {\cmd }
%  (\cmd) = (\protected macro:->\@protected@testopt \cmd \\cmd {default})
%  (\\cmd) = (\long macro:[#1]#2->)
%
%\DeclareRobustCommand {\cmd }{} \robustify {\cmd }
%  (\cmd) = (\protected macro:->)
%
%\DeclareRobustCommand {\cmd }[2][default]{} \robustify {\cmd }
%  (\cmd) = (\protected macro:->\@protected@testopt \cmd~ \\cmd~ {default})
%  (\cmd~) = (macro:->\@protected@testopt \cmd~ \\cmd~ {default})
%  (\\cmd~) = (\long macro:[#1]#2->)
%\end{verbatim}
% \endgroup
%
% \subsection{Package}
%
%    \begin{macrocode}
%<*package>
%    \end{macrocode}
%
% \subsubsection{Catcodes and identification}
%
%    \begin{macrocode}
\begingroup\catcode61\catcode48\catcode32=10\relax%
  \catcode13=5 % ^^M
  \endlinechar=13 %
  \catcode123=1 % {
  \catcode125=2 % }
  \catcode64=11 % @
  \def\x{\endgroup
    \expandafter\edef\csname llm@AtEnd\endcsname{%
      \endlinechar=\the\endlinechar\relax
      \catcode13=\the\catcode13\relax
      \catcode32=\the\catcode32\relax
      \catcode35=\the\catcode35\relax
      \catcode61=\the\catcode61\relax
      \catcode64=\the\catcode64\relax
      \catcode123=\the\catcode123\relax
      \catcode125=\the\catcode125\relax
    }%
  }%
\x\catcode61\catcode48\catcode32=10\relax%
\catcode13=5 % ^^M
\endlinechar=13 %
\catcode35=6 % #
\catcode64=11 % @
\catcode123=1 % {
\catcode125=2 % }
\def\TMP@EnsureCode#1#2{%
  \edef\llm@AtEnd{%
    \llm@AtEnd
    \catcode#1=\the\catcode#1\relax
  }%
  \catcode#1=#2\relax
}
\TMP@EnsureCode{40}{12}% (
\TMP@EnsureCode{41}{12}% )
\TMP@EnsureCode{42}{12}% *
\TMP@EnsureCode{45}{12}% -
\TMP@EnsureCode{46}{12}% .
\TMP@EnsureCode{47}{12}% /
\TMP@EnsureCode{58}{12}% :
\TMP@EnsureCode{62}{12}% >
\TMP@EnsureCode{91}{12}% [
\TMP@EnsureCode{93}{12}% ]
\edef\llm@AtEnd{%
  \llm@AtEnd
  \escapechar\the\escapechar\relax
  \noexpand\endinput
}
\escapechar=92 % `\\
%    \end{macrocode}
%
%    Package identification.
%    \begin{macrocode}
\NeedsTeXFormat{LaTeX2e}
\ProvidesPackage{letltxmacro}%
  [2010/09/02 v1.4 Let assignment for LaTeX macros (HO)]
%    \end{macrocode}
%
% \subsubsection{Main macros}
%
%    \begin{macro}{\LetLtxMacro}
%    \begin{macrocode}
\newcommand*{\LetLtxMacro}{%
  \llm@ModeLetLtxMacro{}%
}
%    \end{macrocode}
%    \end{macro}
%    \begin{macro}{\GlobalLetLtxMacro}
%    \begin{macrocode}
\newcommand*{\GlobalLetLtxMacro}{%
  \llm@ModeLetLtxMacro\global
}
%    \end{macrocode}
%    \end{macro}
%
%    \begin{macro}{\llm@ModeLetLtxMacro}
%    \begin{macrocode}
\newcommand*{\llm@ModeLetLtxMacro}[3]{%
  \edef\llm@escapechar{\the\escapechar}%
  \escapechar=-1 %
  \edef\reserved@a{%
    \noexpand\protect
    \expandafter\noexpand
    \csname\string#3 \endcsname
  }%
  \ifx\reserved@a#3\relax
    #1\edef#2{%
      \noexpand\protect
      \expandafter\noexpand
      \csname\string#2 \endcsname
    }%
    #1\expandafter\let
    \csname\string#2 \expandafter\endcsname
    \csname\string#3 \endcsname
    \expandafter\llm@LetLtxMacro
        \csname\string#2 \expandafter\endcsname
        \csname\string#3 \endcsname{#1}%
  \else
    \llm@LetLtxMacro{#2}{#3}{#1}%
  \fi
  \escapechar=\llm@escapechar\relax
}
%    \end{macrocode}
%    \end{macro}
%    \begin{macro}{\llm@LetLtxMacro}
%    \begin{macrocode}
\def\llm@LetLtxMacro#1#2#3{%
  \escapechar=92 %
  \expandafter\llm@CheckParams\meaning#2:->\@nil{%
    \begingroup
      \def\@protected@testopt{%
        \expandafter\@testopt\@gobble
      }%
      \def\@testopt##1##2{%
        \toks@={##2}%
      }%
      \let\llm@testopt\@empty
      \edef\x{%
        \noexpand\@protected@testopt
        \noexpand#2%
        \expandafter\noexpand\csname\string#2\endcsname
      }%
      \expandafter\expandafter\expandafter\def
      \expandafter\expandafter\expandafter\y
      \expandafter\expandafter\expandafter{%
        \expandafter\llm@CarThree#2{}{}{}\llm@nil
      }%
      \ifx\x\y
        #2%
        \def\llm@testopt{%
          \noexpand\@protected@testopt
          \noexpand#1%
        }%
      \else
        \edef\x{%
          \noexpand\@testopt
          \expandafter\noexpand
          \csname\string#2\endcsname
        }%
        \expandafter\expandafter\expandafter\def
        \expandafter\expandafter\expandafter\y
        \expandafter\expandafter\expandafter{%
          \expandafter\llm@CarTwo#2{}{}\llm@nil
        }%
        \ifx\x\y
          #2%
          \def\llm@testopt{%
            \noexpand\@testopt
          }%
        \fi
      \fi
      \ifx\llm@testopt\@empty
      \else
        \llm@protected\xdef\llm@GlobalTemp{%
          \llm@testopt
          \expandafter\noexpand
          \csname\string#1\endcsname
          {\the\toks@}%
        }%
      \fi
    \expandafter\endgroup\ifx\llm@testopt\@empty
      #3\let#1=#2\relax
    \else
      #3\let#1=\llm@GlobalTemp
      #3\expandafter\let
          \csname\string#1\expandafter\endcsname
          \csname\string#2\endcsname
    \fi
  }{%
    #3\let#1=#2\relax
  }%
}
%    \end{macrocode}
%    \end{macro}
%    \begin{macro}{\llm@CheckParams}
%    \begin{macrocode}
\def\llm@CheckParams#1:->#2\@nil{%
  \begingroup
    \def\x{#1}%
  \ifx\x\llm@macro
    \endgroup
    \def\llm@protected{}%
    \expandafter\@firstoftwo
  \else
    \ifx\x\llm@protectedmacro
      \endgroup
      \def\llm@protected{\protected}%
      \expandafter\expandafter\expandafter\@firstoftwo
    \else
      \endgroup
      \expandafter\expandafter\expandafter\@secondoftwo
    \fi
  \fi
}
%    \end{macrocode}
%    \end{macro}
%    \begin{macro}{\llm@macro}
%    \begin{macrocode}
\def\llm@macro{macro}
\@onelevel@sanitize\llm@macro
%    \end{macrocode}
%    \end{macro}
%    \begin{macro}{\llm@protectedmacro}
%    \begin{macrocode}
\def\llm@protectedmacro{\protected macro}
\@onelevel@sanitize\llm@protectedmacro
%    \end{macrocode}
%    \end{macro}
%    \begin{macro}{\llm@CarThree}
%    \begin{macrocode}
\def\llm@CarThree#1#2#3#4\llm@nil{#1#2#3}%
%    \end{macrocode}
%    \end{macro}
%    \begin{macro}{\llm@CarTwo}
%    \begin{macrocode}
\def\llm@CarTwo#1#2#3\llm@nil{#1#2}%
%    \end{macrocode}
%    \end{macro}
%
%    \begin{macrocode}
\llm@AtEnd%
%</package>
%    \end{macrocode}
%
% \section{Test}
%
% \subsection{Catcode checks for loading}
%
%    \begin{macrocode}
%<*test1>
%    \end{macrocode}
%    \begin{macrocode}
\NeedsTeXFormat{LaTeX2e}
\documentclass{minimal}
\makeatletter
\def\RestoreCatcodes{}
\count@=0 %
\loop
  \edef\RestoreCatcodes{%
    \RestoreCatcodes
    \catcode\the\count@=\the\catcode\count@\relax
  }%
\ifnum\count@<255 %
  \advance\count@\@ne
\repeat

\def\RangeCatcodeInvalid#1#2{%
  \count@=#1\relax
  \loop
    \catcode\count@=15 %
  \ifnum\count@<#2\relax
    \advance\count@\@ne
  \repeat
}
\def\Test{%
  \RangeCatcodeInvalid{0}{47}%
  \RangeCatcodeInvalid{58}{64}%
  \RangeCatcodeInvalid{91}{96}%
  \RangeCatcodeInvalid{123}{127}%
  \catcode`\@=12 %
  \catcode`\\=0 %
  \catcode`\{=1 %
  \catcode`\}=2 %
  \catcode`\#=6 %
  \catcode`\[=12 %
  \catcode`\]=12 %
  \catcode`\%=14 %
  \catcode`\ =10 %
  \catcode13=5 %
  \RequirePackage{letltxmacro}[2010/09/02]\relax
  \RestoreCatcodes
}
\Test
\csname @@end\endcsname
\end
%    \end{macrocode}
%    \begin{macrocode}
%</test1>
%    \end{macrocode}
%
% \subsection{Package tests}
%
%    \begin{macrocode}
%<*test2>
%    \end{macrocode}
%    \begin{macrocode}
\NeedsTeXFormat{LaTeX2e}
\nofiles
\documentclass{minimal}

\usepackage{letltxmacro}[2010/09/02]

\usepackage{qstest}
\IncludeTests{*}
\LogTests{log}{*}{*}

\makeatletter

\def\TestDef#1{%
  \begingroup
    \@makeother\\%
    \@makeother\ %
    \expandafter\@TestDef\csname #1\endcsname
}
\def\@TestDef#1#2{%
    \edef\a{\expandafter\strip@prefix\meaning#1}%
    \edef\b{\detokenize{#2}}%
    \ifx\a\b
    \else
      \typeout{1 [\a]}%
      \typeout{2 [\b]}%
    \fi
  \endgroup
  \Expect*{\expandafter\strip@prefix\meaning#1}*{\detokenize{#2}}%
}
\def\TestEquals#1#2{%
  \Expect*{%
    \expandafter\ifx\csname#1\expandafter\endcsname
                    \csname#2\endcsname
      equals%
    \else
      wrong%
    \fi
  }{equals}%
}
\def\SaveA{%
  \let\SavedA\TestA
  \expandafter\let\csname\string\SavedA\expandafter\endcsname
                  \csname\string\TestA\endcsname
  \expandafter\let\csname SavedA \expandafter\endcsname
                  \csname TestA \endcsname
}
\def\CheckA{%
  \TestEquals{SavedA}{TestA}%
  \TestEquals{\string\SavedA}{\string\TestA}%
  \TestEquals{SavedA }{TestA }%
}

\begin{qstest}{robust}{robust}
  \begingroup
    \DeclareRobustCommand*{\TestA}{Test}%
    \SaveA
    \TestDef{TestA}{\protect \TestA  }%
    \CheckA
    \DeclareRobustCommand{\TestA}{Test}%
    \SaveA
    \TestDef{TestA}{\protect \TestA  }%
    \LetLtxMacro\TestB\TestA
    \TestDef{TestB}{\protect \TestB  }%
    \TestEquals{TestB }{TestA }%
    \CheckA
    \GlobalLetLtxMacro\TestC\TestA
    \CheckA
  \endgroup
  \TestDef{TestC}{\protect \TestC  }%
  \DeclareRobustCommand{\TestA}{Test}%
  \TestEquals{TestC }{TestA }%
\end{qstest}

\begin{qstest}{default}{default}
  \begingroup
    \newcommand{\TestA}[1][\relax default \empty]{TestA #1}%
    \SaveA
    \TestDef
    {TestA}{\@protected@testopt \TestA \\TestA {\relax default \empty }}%
    \LetLtxMacro\TestB\TestA
    \TestDef
    {TestB}{\@protected@testopt \TestB \\TestB {\relax default \empty }}%
    \CheckA
    \TestEquals{\string\TestB}{\string\TestA}%
    \GlobalLetLtxMacro\TestC\TestA
    \CheckA
  \endgroup
  \TestDef
  {TestC}{\@protected@testopt \TestC \\TestC {\relax default \empty }}%
  \newcommand{\TestA}[1][\relax default \empty]{TestA #1}%
  \TestEquals{\string\TestC}{\string\TestA}%
\end{qstest}

\begin{qstest}{robustdefault}{robustdefault}
  \begingroup
    \DeclareRobustCommand{\TestA}[1][\default]{TestA}%
    \SaveA
    \LetLtxMacro\TestB\TestA
    \TestDef{TestB}{\protect \TestB  }%
    \TestDef{TestB }{\@protected@testopt \TestB  \\TestB  {\default }}%
    \CheckA
    \GlobalLetLtxMacro\TestC\TestA
    \CheckA
  \endgroup
  \TestDef{TestC}{\protect \TestC  }%
  \TestDef{TestC }{\@protected@testopt \TestC  \\TestC  {\default }}%
\end{qstest}

\begin{qstest}{plain}{plain}
  \LetLtxMacro\NewRelax\relax
  \Expect*{\meaning\relax}*{\string\relax}%
  \Expect*{\meaning\NewRelax}*{\string\relax}%
  \LetLtxMacro\NewHbox\hbox
  \Expect*{\meaning\hbox}*{\string\hbox}%
  \Expect*{\meaning\NewHbox}*{\string\hbox}%
  \LetLtxMacro\NewEmpty\empty
  \Expect*{\meaning\empty}{macro:->}%
  \Expect*{\meaning\NewEmpty}{macro:->}%
  \def\TestA{\iffalse}%
  \LetLtxMacro\TestB\TestA
  \TestDef{TestB}{\iffalse }%
  \TestEquals{TestA}{TestB}%
\end{qstest}
%    \end{macrocode}
%
%    \begin{macrocode}
\usepackage{etoolbox}[2008/06/22]

\begin{qstest}{oldnewrobustcmd}{oldnewrobustcmd}
  \begingroup
    \protected\edef\TestA{%
      \noexpand\@protected@testopt
      \noexpand\TestA
      \expandafter\noexpand\csname\string\TestA\endcsname
      {\noexpand\default}%
    }%
    \expandafter\protected\expandafter
    \def\csname\string\TestA\endcsname{TestA}%
    \SaveA
    \LetLtxMacro\TestB\TestA
    \TestDef{TestB}{\@protected@testopt \TestB \\TestB {\default }}%
    \protected\edef\TestC{%
      \noexpand\@protected@testopt
      \noexpand\TestB
      \expandafter\noexpand\csname\string\TestB\endcsname
      {\noexpand\default}%
    }%
    \TestEquals{TestB}{TestC}%
    \CheckA
    \GlobalLetLtxMacro\TestD\TestA
    \CheckA
  \endgroup
  \TestDef{TestD}{\@protected@testopt \TestD \\TestD {\default }}%
  \protected\edef\TestC{%
    \noexpand\@protected@testopt
    \noexpand\TestD
    \expandafter\noexpand\csname\string\TestD\endcsname
    {\noexpand\default}%
  }%
  \TestEquals{TestD}{TestC}%
\end{qstest}

\begin{qstest}{newrobustcmd}{newrobustcmd}
  \begingroup
    \newrobustcmd{\TestA}[1][\default]{TestA}%
    \SaveA
    \LetLtxMacro\TestB\TestA
    \TestDef{TestB}{\@testopt \\TestB {\default }}%
    \protected\edef\TestC{%
      \noexpand\@testopt
      \expandafter\noexpand\csname\string\TestB\endcsname
      {\noexpand\default}%
    }%
    \TestEquals{TestB}{TestC}%
    \CheckA
    \GlobalLetLtxMacro\TestD\TestA
    \CheckA
  \endgroup
  \TestDef{TestD}{\@testopt \\TestD {\default }}%
  \protected\edef\TestC{%
    \noexpand\@testopt
    \expandafter\noexpand\csname\string\TestD\endcsname
    {\noexpand\default}%
  }%
  \TestEquals{TestD}{TestC}%
\end{qstest}

\begin{qstest}{robustifyopt}{robustifyopt}
  \newcommand{\TestA}[2][\default]{}%
  \TestDef{TestA}{\@protected@testopt \TestA \\TestA {\default }}%
  \robustify\TestA
  \TestDef{TestA}{\@protected@testopt \TestA \\TestA {\default }}%
  \protected\edef\TestC{%
    \noexpand\@protected@testopt
    \noexpand\TestA
    \expandafter\noexpand\csname\string\TestA\endcsname
    {\noexpand\default}%
  }%
  \TestEquals{TestA}{TestC}%
  \SaveA
  \LetLtxMacro\TestB\TestA
  \TestDef{TestB}{\@protected@testopt \TestB \\TestB {\default }}%
  \protected\edef\TestC{%
    \noexpand\@protected@testopt
    \noexpand\TestB
    \expandafter\noexpand\csname\string\TestB\endcsname
    {\noexpand\default}%
  }%
  \TestEquals{TestB}{TestC}%
  \CheckA
\end{qstest}

\begin{qstest}{robustifydeclare}{robustifydeclare}
  \DeclareRobustCommand\TestA{\iffalse}%
  \robustify\TestA
  \SaveA
  \LetLtxMacro\TestB\TestA
  \TestDef{TestB}{\iffalse }%
  % before etoolbox 2008/06/22: \protected\long\def
  % since etoolbox 2008/06/22: \protected\def
  \protected\def\TestC{\iffalse}%
  \TestEquals{TestB}{TestC}%
  \CheckA
\end{qstest}

\begin{qstest}{robustifydeclarelong}{robustifydeclarelong}
  \DeclareRobustCommand\TestA[1]{\iffalse}%
  \robustify\TestA
  \SaveA
  \LetLtxMacro\TestB\TestA
  \TestDef{TestB}{\iffalse }%
  % before etoolbox 2008/06/22: \protected\long\def
  % since etoolbox 2008/06/22: \protected\def
  \protected\long\def\TestC#1{\iffalse}%
  \TestEquals{TestB}{TestC}%
  \CheckA
\end{qstest}
%    \end{macrocode}
%
%    \begin{macrocode}
\csname @@end\endcsname
%    \end{macrocode}
%    \begin{macrocode}
%</test2>
%    \end{macrocode}
%
% \section{Installation}
%
% \subsection{Download}
%
% \paragraph{Package.} This package is available on
% CTAN\footnote{\url{ftp://ftp.ctan.org/tex-archive/}}:
% \begin{description}
% \item[\CTAN{macros/latex/contrib/oberdiek/letltxmacro.dtx}] The source file.
% \item[\CTAN{macros/latex/contrib/oberdiek/letltxmacro.pdf}] Documentation.
% \end{description}
%
%
% \paragraph{Bundle.} All the packages of the bundle `oberdiek'
% are also available in a TDS compliant ZIP archive. There
% the packages are already unpacked and the documentation files
% are generated. The files and directories obey the TDS standard.
% \begin{description}
% \item[\CTAN{install/macros/latex/contrib/oberdiek.tds.zip}]
% \end{description}
% \emph{TDS} refers to the standard ``A Directory Structure
% for \TeX\ Files'' (\CTAN{tds/tds.pdf}). Directories
% with \xfile{texmf} in their name are usually organized this way.
%
% \subsection{Bundle installation}
%
% \paragraph{Unpacking.} Unpack the \xfile{oberdiek.tds.zip} in the
% TDS tree (also known as \xfile{texmf} tree) of your choice.
% Example (linux):
% \begin{quote}
%   |unzip oberdiek.tds.zip -d ~/texmf|
% \end{quote}
%
% \paragraph{Script installation.}
% Check the directory \xfile{TDS:scripts/oberdiek/} for
% scripts that need further installation steps.
% Package \xpackage{attachfile2} comes with the Perl script
% \xfile{pdfatfi.pl} that should be installed in such a way
% that it can be called as \texttt{pdfatfi}.
% Example (linux):
% \begin{quote}
%   |chmod +x scripts/oberdiek/pdfatfi.pl|\\
%   |cp scripts/oberdiek/pdfatfi.pl /usr/local/bin/|
% \end{quote}
%
% \subsection{Package installation}
%
% \paragraph{Unpacking.} The \xfile{.dtx} file is a self-extracting
% \docstrip\ archive. The files are extracted by running the
% \xfile{.dtx} through \plainTeX:
% \begin{quote}
%   \verb|tex letltxmacro.dtx|
% \end{quote}
%
% \paragraph{TDS.} Now the different files must be moved into
% the different directories in your installation TDS tree
% (also known as \xfile{texmf} tree):
% \begin{quote}
% \def\t{^^A
% \begin{tabular}{@{}>{\ttfamily}l@{ $\rightarrow$ }>{\ttfamily}l@{}}
%   letltxmacro.sty & tex/latex/oberdiek/letltxmacro.sty\\
%   letltxmacro.pdf & doc/latex/oberdiek/letltxmacro.pdf\\
%   letltxmacro-showcases.tex & doc/latex/oberdiek/letltxmacro-showcases.tex\\
%   test/letltxmacro-test1.tex & doc/latex/oberdiek/test/letltxmacro-test1.tex\\
%   test/letltxmacro-test2.tex & doc/latex/oberdiek/test/letltxmacro-test2.tex\\
%   letltxmacro.dtx & source/latex/oberdiek/letltxmacro.dtx\\
% \end{tabular}^^A
% }^^A
% \sbox0{\t}^^A
% \ifdim\wd0>\linewidth
%   \begingroup
%     \advance\linewidth by\leftmargin
%     \advance\linewidth by\rightmargin
%   \edef\x{\endgroup
%     \def\noexpand\lw{\the\linewidth}^^A
%   }\x
%   \def\lwbox{^^A
%     \leavevmode
%     \hbox to \linewidth{^^A
%       \kern-\leftmargin\relax
%       \hss
%       \usebox0
%       \hss
%       \kern-\rightmargin\relax
%     }^^A
%   }^^A
%   \ifdim\wd0>\lw
%     \sbox0{\small\t}^^A
%     \ifdim\wd0>\linewidth
%       \ifdim\wd0>\lw
%         \sbox0{\footnotesize\t}^^A
%         \ifdim\wd0>\linewidth
%           \ifdim\wd0>\lw
%             \sbox0{\scriptsize\t}^^A
%             \ifdim\wd0>\linewidth
%               \ifdim\wd0>\lw
%                 \sbox0{\tiny\t}^^A
%                 \ifdim\wd0>\linewidth
%                   \lwbox
%                 \else
%                   \usebox0
%                 \fi
%               \else
%                 \lwbox
%               \fi
%             \else
%               \usebox0
%             \fi
%           \else
%             \lwbox
%           \fi
%         \else
%           \usebox0
%         \fi
%       \else
%         \lwbox
%       \fi
%     \else
%       \usebox0
%     \fi
%   \else
%     \lwbox
%   \fi
% \else
%   \usebox0
% \fi
% \end{quote}
% If you have a \xfile{docstrip.cfg} that configures and enables \docstrip's
% TDS installing feature, then some files can already be in the right
% place, see the documentation of \docstrip.
%
% \subsection{Refresh file name databases}
%
% If your \TeX~distribution
% (\teTeX, \mikTeX, \dots) relies on file name databases, you must refresh
% these. For example, \teTeX\ users run \verb|texhash| or
% \verb|mktexlsr|.
%
% \subsection{Some details for the interested}
%
% \paragraph{Attached source.}
%
% The PDF documentation on CTAN also includes the
% \xfile{.dtx} source file. It can be extracted by
% AcrobatReader 6 or higher. Another option is \textsf{pdftk},
% e.g. unpack the file into the current directory:
% \begin{quote}
%   \verb|pdftk letltxmacro.pdf unpack_files output .|
% \end{quote}
%
% \paragraph{Unpacking with \LaTeX.}
% The \xfile{.dtx} chooses its action depending on the format:
% \begin{description}
% \item[\plainTeX:] Run \docstrip\ and extract the files.
% \item[\LaTeX:] Generate the documentation.
% \end{description}
% If you insist on using \LaTeX\ for \docstrip\ (really,
% \docstrip\ does not need \LaTeX), then inform the autodetect routine
% about your intention:
% \begin{quote}
%   \verb|latex \let\install=y% \iffalse meta-comment
%
% File: letltxmacro.dtx
% Version: 2010/09/02 v1.4
% Info: Let assignment for LaTeX macros
%
% Copyright (C) 2008, 2010 by
%    Heiko Oberdiek <heiko.oberdiek at googlemail.com>
%
% This work may be distributed and/or modified under the
% conditions of the LaTeX Project Public License, either
% version 1.3c of this license or (at your option) any later
% version. This version of this license is in
%    http://www.latex-project.org/lppl/lppl-1-3c.txt
% and the latest version of this license is in
%    http://www.latex-project.org/lppl.txt
% and version 1.3 or later is part of all distributions of
% LaTeX version 2005/12/01 or later.
%
% This work has the LPPL maintenance status "maintained".
%
% This Current Maintainer of this work is Heiko Oberdiek.
%
% This work consists of the main source file letltxmacro.dtx
% and the derived files
%    letltxmacro.sty, letltxmacro.pdf, letltxmacro.ins, letltxmacro.drv,
%    letltxmacro-showcases.tex, letltxmacro-test1.tex,
%    letltxmacro-test2.tex.
%
% Distribution:
%    CTAN:macros/latex/contrib/oberdiek/letltxmacro.dtx
%    CTAN:macros/latex/contrib/oberdiek/letltxmacro.pdf
%
% Unpacking:
%    (a) If letltxmacro.ins is present:
%           tex letltxmacro.ins
%    (b) Without letltxmacro.ins:
%           tex letltxmacro.dtx
%    (c) If you insist on using LaTeX
%           latex \let\install=y\input{letltxmacro.dtx}
%        (quote the arguments according to the demands of your shell)
%
% Documentation:
%    (a) If letltxmacro.drv is present:
%           latex letltxmacro.drv
%    (b) Without letltxmacro.drv:
%           latex letltxmacro.dtx; ...
%    The class ltxdoc loads the configuration file ltxdoc.cfg
%    if available. Here you can specify further options, e.g.
%    use A4 as paper format:
%       \PassOptionsToClass{a4paper}{article}
%
%    Programm calls to get the documentation (example):
%       pdflatex letltxmacro.dtx
%       makeindex -s gind.ist letltxmacro.idx
%       pdflatex letltxmacro.dtx
%       makeindex -s gind.ist letltxmacro.idx
%       pdflatex letltxmacro.dtx
%
% Installation:
%    TDS:tex/latex/oberdiek/letltxmacro.sty
%    TDS:doc/latex/oberdiek/letltxmacro.pdf
%    TDS:doc/latex/oberdiek/letltxmacro-showcases.tex
%    TDS:doc/latex/oberdiek/test/letltxmacro-test1.tex
%    TDS:doc/latex/oberdiek/test/letltxmacro-test2.tex
%    TDS:source/latex/oberdiek/letltxmacro.dtx
%
%<*ignore>
\begingroup
  \catcode123=1 %
  \catcode125=2 %
  \def\x{LaTeX2e}%
\expandafter\endgroup
\ifcase 0\ifx\install y1\fi\expandafter
         \ifx\csname processbatchFile\endcsname\relax\else1\fi
         \ifx\fmtname\x\else 1\fi\relax
\else\csname fi\endcsname
%</ignore>
%<*install>
\input docstrip.tex
\Msg{************************************************************************}
\Msg{* Installation}
\Msg{* Package: letltxmacro 2010/09/02 v1.4 Let assignment for LaTeX macros (HO)}
\Msg{************************************************************************}

\keepsilent
\askforoverwritefalse

\let\MetaPrefix\relax
\preamble

This is a generated file.

Project: letltxmacro
Version: 2010/09/02 v1.4

Copyright (C) 2008, 2010 by
   Heiko Oberdiek <heiko.oberdiek at googlemail.com>

This work may be distributed and/or modified under the
conditions of the LaTeX Project Public License, either
version 1.3c of this license or (at your option) any later
version. This version of this license is in
   http://www.latex-project.org/lppl/lppl-1-3c.txt
and the latest version of this license is in
   http://www.latex-project.org/lppl.txt
and version 1.3 or later is part of all distributions of
LaTeX version 2005/12/01 or later.

This work has the LPPL maintenance status "maintained".

This Current Maintainer of this work is Heiko Oberdiek.

This work consists of the main source file letltxmacro.dtx
and the derived files
   letltxmacro.sty, letltxmacro.pdf, letltxmacro.ins, letltxmacro.drv,
   letltxmacro-showcases.tex, letltxmacro-test1.tex,
   letltxmacro-test2.tex.

\endpreamble
\let\MetaPrefix\DoubleperCent

\generate{%
  \file{letltxmacro.ins}{\from{letltxmacro.dtx}{install}}%
  \file{letltxmacro.drv}{\from{letltxmacro.dtx}{driver}}%
  \usedir{tex/latex/oberdiek}%
  \file{letltxmacro.sty}{\from{letltxmacro.dtx}{package}}%
  \usedir{doc/latex/oberdiek}%
  \file{letltxmacro-showcases.tex}{\from{letltxmacro.dtx}{showcases}}%
  \usedir{doc/latex/oberdiek/test}%
  \file{letltxmacro-test1.tex}{\from{letltxmacro.dtx}{test1}}%
  \file{letltxmacro-test2.tex}{\from{letltxmacro.dtx}{test2}}%
  \nopreamble
  \nopostamble
  \usedir{source/latex/oberdiek/catalogue}%
  \file{letltxmacro.xml}{\from{letltxmacro.dtx}{catalogue}}%
}

\catcode32=13\relax% active space
\let =\space%
\Msg{************************************************************************}
\Msg{*}
\Msg{* To finish the installation you have to move the following}
\Msg{* file into a directory searched by TeX:}
\Msg{*}
\Msg{*     letltxmacro.sty}
\Msg{*}
\Msg{* To produce the documentation run the file `letltxmacro.drv'}
\Msg{* through LaTeX.}
\Msg{*}
\Msg{* Happy TeXing!}
\Msg{*}
\Msg{************************************************************************}

\endbatchfile
%</install>
%<*ignore>
\fi
%</ignore>
%<*driver>
\NeedsTeXFormat{LaTeX2e}
\ProvidesFile{letltxmacro.drv}%
  [2010/09/02 v1.4 Let assignment for LaTeX macros (HO)]%
\documentclass{ltxdoc}
\usepackage{holtxdoc}[2011/11/22]
\begin{document}
  \DocInput{letltxmacro.dtx}%
\end{document}
%</driver>
% \fi
%
% \CheckSum{990}
%
% \CharacterTable
%  {Upper-case    \A\B\C\D\E\F\G\H\I\J\K\L\M\N\O\P\Q\R\S\T\U\V\W\X\Y\Z
%   Lower-case    \a\b\c\d\e\f\g\h\i\j\k\l\m\n\o\p\q\r\s\t\u\v\w\x\y\z
%   Digits        \0\1\2\3\4\5\6\7\8\9
%   Exclamation   \!     Double quote  \"     Hash (number) \#
%   Dollar        \$     Percent       \%     Ampersand     \&
%   Acute accent  \'     Left paren    \(     Right paren   \)
%   Asterisk      \*     Plus          \+     Comma         \,
%   Minus         \-     Point         \.     Solidus       \/
%   Colon         \:     Semicolon     \;     Less than     \<
%   Equals        \=     Greater than  \>     Question mark \?
%   Commercial at \@     Left bracket  \[     Backslash     \\
%   Right bracket \]     Circumflex    \^     Underscore    \_
%   Grave accent  \`     Left brace    \{     Vertical bar  \|
%   Right brace   \}     Tilde         \~}
%
% \GetFileInfo{letltxmacro.drv}
%
% \title{The \xpackage{letltxmacro} package}
% \date{2010/09/02 v1.4}
% \author{Heiko Oberdiek\\\xemail{heiko.oberdiek at googlemail.com}}
%
% \maketitle
%
% \begin{abstract}
% \TeX's \cs{let} assignment does not work for \LaTeX\ macros
% with optional arguments or for macros that are defined
% as robust macros by \cs{DeclareRobustCommand}. This package
% defines \cs{LetLtxMacro} that also takes care of the involved
% internal macros.
% \end{abstract}
%
% \tableofcontents
%
% \section{Documentation}
%
% If someone wants to redefine a macro with using the old
% meaning, then one method is \TeX's command \cs{let}:
%\begin{quote}
%\begin{verbatim}
%\newcommand{\Macro}{\typeout{Test Macro}}
%\let\SavedMacro=\Macro
%\renewcommand{\Macro}{%
%  \typeout{Begin}%
%  \SavedMacro
%  \typeout{End}%
%}
%\end{verbatim}
%\end{quote}
% However, this method fails, if \cs{Macro} is defined
% by \cs{DeclareRobustCommand} and/or has an optional argument.
% In both cases \LaTeX\ defines an additional internal macro
% that is forgotten in the simple \cs{let} assignment of
% the example above.
%
% \begin{declcs}{LetLtxMacro} \M{new macro} \M{old macro}
% \end{declcs}
% Macro \cs{LetLtxMacro} behaves similar to \TeX's \cs{let}
% assignment, but it takes care of macros that are
% defined by \cs{DeclareRobustCommand} and/or have optional
% arguments. Example:
%\begin{quote}
%\begin{verbatim}
%\DeclareRobustCommand{\Macro}[1][default]{...}
%\LetLtxMacro{\SavedMacro}{\Macro}
%\end{verbatim}
%\end{quote}
% Then macro \cs{SavedMacro} only uses internal macro names
% that are derived from \cs{SavedMacro}'s macro name. Macro \cs{Macro}
% can now be redefined without affecting \cs{SavedMacro}.
%
% \begin{declcs}{GlobalLetLtxMacro} \M{new macro} \M{old macro}
% \end{declcs}
% Like \cs{LetLtxMacro}, but the \meta{new macro} is defined globally.
% Since version 2010/09/02~v1.4.
%
% \subsection{Supported macro definition commands}
%
% \begin{quote}
%   \begin{tabular}{@{}ll@{}}
%     \cs{newcommand}, \cs{renewcommand} & latex/base\\
%     \cs{newenvironment}, \cs{renewenvironment} & latex/base\\
%     \cs{DeclareRobustCommand}& latex/base\\
%     \cs{newrobustcmd}, \cs{renewrobustcmd} & etoolbox\\
%     \cs{robustify} & etoolbox 2008/06/22 v1.6\\
%   \end{tabular}
% \end{quote}
%
% \StopEventually{
% }
%
% \section{Implementation}
%
% \subsection{Show cases}
%
% \subsubsection{\xfile{letltxmacro-showcases.tex}}
%
%    \begin{macrocode}
%<*showcases>
\NeedsTeXFormat{LaTeX2e}
\makeatletter
%    \end{macrocode}
%    \begin{macro}{\Line}
%    The result is displayed by macro \cs{Line}. The percent symbol
%    at line start allows easy grepping and inserting into the DTX
%    file.
%    \begin{macrocode}
\newcommand*{\Line}[1]{%
  \typeout{\@percentchar#1}%
}
%    \end{macrocode}
%    \end{macro}
%    \begin{macrocode}
\newcommand*{\ShowCmdName}[1]{%
  \@ifundefined{#1}{}{%
    \Line{%
      \space\space(\expandafter\string\csname#1\endcsname) = %
      (\expandafter\meaning\csname#1\endcsname)%
    }%
  }%
}
\newcommand*{\ShowCmds}[1]{%
  \ShowCmdName{#1}%
  \ShowCmdName{#1 }%
  \ShowCmdName{\\#1}%
  \ShowCmdName{\\#1 }%
}
\let\\\@backslashchar
%    \end{macrocode}
%    \begin{macro}{\ShowDef}
%    \begin{macrocode}
\newcommand*{\ShowDef}[2]{%
  \begingroup
    \Line{}%
    \newcommand*{\DefString}{#2}%
    \@onelevel@sanitize\DefString
    \Line{\DefString}%
    #2%
    \ShowCmds{#1}%
  \endgroup
}
%    \end{macrocode}
%    \end{macro}
%    \begin{macrocode}
\typeout{}
\Line{* LaTeX definitions:}
\ShowDef{cmd}{%
  \newcommand{\cmd}[2][default]{}%
}
\ShowDef{cmd}{%
  \DeclareRobustCommand{\cmd}{}%
}
\ShowDef{cmd}{%
  \DeclareRobustCommand{\cmd}[2][default]{}%
}
\typeout{}
%    \end{macrocode}
% The minimal version of package \xpackage{etoolbox} is 2008/06/12 v1.6a
% because it fixes \cs{robustify}.
%    \begin{macrocode}
\RequirePackage{etoolbox}[2008/06/12]%
\Line{}
\Line{* etoolbox's robust definitions:}
\ShowDef{cmd}{%
  \newrobustcmd{\cmd}{}%
}
\ShowDef{cmd}{%
  \newrobustcmd{\cmd}[2][default]{}%
}
\Line{}
\Line{* etoolbox's \string\robustify:}
\ShowDef{cmd}{%
  \newcommand{\cmd}[2][default]{} %
  \robustify{\cmd}%
}
\ShowDef{cmd}{%
  \DeclareRobustCommand{\cmd}{} %
  \robustify{\cmd}%
}
\ShowDef{cmd}{%
  \DeclareRobustCommand{\cmd}[2][default]{} %
  \robustify{\cmd}%
}
\typeout{}
\@@end
%</showcases>
%    \end{macrocode}
%
% \subsubsection{Result}
%
% \begingroup
%   \makeatletter
%   \let\org@verbatim\@verbatim
%   \def\@verbatim{^^A
%     \org@verbatim
%     \catcode`\~=\active
%   }^^A
%   \let~\textvisiblespace
%\begin{verbatim}
%* LaTeX definitions:
%
%\newcommand {\cmd }[2][default]{}
%  (\cmd) = (macro:->\@protected@testopt \cmd \\cmd {default})
%  (\\cmd) = (\long macro:[#1]#2->)
%
%\DeclareRobustCommand {\cmd }{}
%  (\cmd) = (macro:->\protect \cmd~ )
%  (\cmd~) = (\long macro:->)
%
%\DeclareRobustCommand {\cmd }[2][default]{}
%  (\cmd) = (macro:->\protect \cmd~ )
%  (\cmd~) = (macro:->\@protected@testopt \cmd~ \\cmd~ {default})
%  (\\cmd~) = (\long macro:[#1]#2->)
%
%* etoolbox's robust definitions:
%
%\newrobustcmd {\cmd }{}
%  (\cmd) = (\protected\long macro:->)
%
%\newrobustcmd {\cmd }[2][default]{}
%  (\cmd) = (\protected macro:->\@testopt \\cmd {default})
%  (\\cmd) = (\long macro:[#1]#2->)
%
%* etoolbox's \robustify:
%
%\newcommand {\cmd }[2][default]{} \robustify {\cmd }
%  (\cmd) = (\protected macro:->\@protected@testopt \cmd \\cmd {default})
%  (\\cmd) = (\long macro:[#1]#2->)
%
%\DeclareRobustCommand {\cmd }{} \robustify {\cmd }
%  (\cmd) = (\protected macro:->)
%
%\DeclareRobustCommand {\cmd }[2][default]{} \robustify {\cmd }
%  (\cmd) = (\protected macro:->\@protected@testopt \cmd~ \\cmd~ {default})
%  (\cmd~) = (macro:->\@protected@testopt \cmd~ \\cmd~ {default})
%  (\\cmd~) = (\long macro:[#1]#2->)
%\end{verbatim}
% \endgroup
%
% \subsection{Package}
%
%    \begin{macrocode}
%<*package>
%    \end{macrocode}
%
% \subsubsection{Catcodes and identification}
%
%    \begin{macrocode}
\begingroup\catcode61\catcode48\catcode32=10\relax%
  \catcode13=5 % ^^M
  \endlinechar=13 %
  \catcode123=1 % {
  \catcode125=2 % }
  \catcode64=11 % @
  \def\x{\endgroup
    \expandafter\edef\csname llm@AtEnd\endcsname{%
      \endlinechar=\the\endlinechar\relax
      \catcode13=\the\catcode13\relax
      \catcode32=\the\catcode32\relax
      \catcode35=\the\catcode35\relax
      \catcode61=\the\catcode61\relax
      \catcode64=\the\catcode64\relax
      \catcode123=\the\catcode123\relax
      \catcode125=\the\catcode125\relax
    }%
  }%
\x\catcode61\catcode48\catcode32=10\relax%
\catcode13=5 % ^^M
\endlinechar=13 %
\catcode35=6 % #
\catcode64=11 % @
\catcode123=1 % {
\catcode125=2 % }
\def\TMP@EnsureCode#1#2{%
  \edef\llm@AtEnd{%
    \llm@AtEnd
    \catcode#1=\the\catcode#1\relax
  }%
  \catcode#1=#2\relax
}
\TMP@EnsureCode{40}{12}% (
\TMP@EnsureCode{41}{12}% )
\TMP@EnsureCode{42}{12}% *
\TMP@EnsureCode{45}{12}% -
\TMP@EnsureCode{46}{12}% .
\TMP@EnsureCode{47}{12}% /
\TMP@EnsureCode{58}{12}% :
\TMP@EnsureCode{62}{12}% >
\TMP@EnsureCode{91}{12}% [
\TMP@EnsureCode{93}{12}% ]
\edef\llm@AtEnd{%
  \llm@AtEnd
  \escapechar\the\escapechar\relax
  \noexpand\endinput
}
\escapechar=92 % `\\
%    \end{macrocode}
%
%    Package identification.
%    \begin{macrocode}
\NeedsTeXFormat{LaTeX2e}
\ProvidesPackage{letltxmacro}%
  [2010/09/02 v1.4 Let assignment for LaTeX macros (HO)]
%    \end{macrocode}
%
% \subsubsection{Main macros}
%
%    \begin{macro}{\LetLtxMacro}
%    \begin{macrocode}
\newcommand*{\LetLtxMacro}{%
  \llm@ModeLetLtxMacro{}%
}
%    \end{macrocode}
%    \end{macro}
%    \begin{macro}{\GlobalLetLtxMacro}
%    \begin{macrocode}
\newcommand*{\GlobalLetLtxMacro}{%
  \llm@ModeLetLtxMacro\global
}
%    \end{macrocode}
%    \end{macro}
%
%    \begin{macro}{\llm@ModeLetLtxMacro}
%    \begin{macrocode}
\newcommand*{\llm@ModeLetLtxMacro}[3]{%
  \edef\llm@escapechar{\the\escapechar}%
  \escapechar=-1 %
  \edef\reserved@a{%
    \noexpand\protect
    \expandafter\noexpand
    \csname\string#3 \endcsname
  }%
  \ifx\reserved@a#3\relax
    #1\edef#2{%
      \noexpand\protect
      \expandafter\noexpand
      \csname\string#2 \endcsname
    }%
    #1\expandafter\let
    \csname\string#2 \expandafter\endcsname
    \csname\string#3 \endcsname
    \expandafter\llm@LetLtxMacro
        \csname\string#2 \expandafter\endcsname
        \csname\string#3 \endcsname{#1}%
  \else
    \llm@LetLtxMacro{#2}{#3}{#1}%
  \fi
  \escapechar=\llm@escapechar\relax
}
%    \end{macrocode}
%    \end{macro}
%    \begin{macro}{\llm@LetLtxMacro}
%    \begin{macrocode}
\def\llm@LetLtxMacro#1#2#3{%
  \escapechar=92 %
  \expandafter\llm@CheckParams\meaning#2:->\@nil{%
    \begingroup
      \def\@protected@testopt{%
        \expandafter\@testopt\@gobble
      }%
      \def\@testopt##1##2{%
        \toks@={##2}%
      }%
      \let\llm@testopt\@empty
      \edef\x{%
        \noexpand\@protected@testopt
        \noexpand#2%
        \expandafter\noexpand\csname\string#2\endcsname
      }%
      \expandafter\expandafter\expandafter\def
      \expandafter\expandafter\expandafter\y
      \expandafter\expandafter\expandafter{%
        \expandafter\llm@CarThree#2{}{}{}\llm@nil
      }%
      \ifx\x\y
        #2%
        \def\llm@testopt{%
          \noexpand\@protected@testopt
          \noexpand#1%
        }%
      \else
        \edef\x{%
          \noexpand\@testopt
          \expandafter\noexpand
          \csname\string#2\endcsname
        }%
        \expandafter\expandafter\expandafter\def
        \expandafter\expandafter\expandafter\y
        \expandafter\expandafter\expandafter{%
          \expandafter\llm@CarTwo#2{}{}\llm@nil
        }%
        \ifx\x\y
          #2%
          \def\llm@testopt{%
            \noexpand\@testopt
          }%
        \fi
      \fi
      \ifx\llm@testopt\@empty
      \else
        \llm@protected\xdef\llm@GlobalTemp{%
          \llm@testopt
          \expandafter\noexpand
          \csname\string#1\endcsname
          {\the\toks@}%
        }%
      \fi
    \expandafter\endgroup\ifx\llm@testopt\@empty
      #3\let#1=#2\relax
    \else
      #3\let#1=\llm@GlobalTemp
      #3\expandafter\let
          \csname\string#1\expandafter\endcsname
          \csname\string#2\endcsname
    \fi
  }{%
    #3\let#1=#2\relax
  }%
}
%    \end{macrocode}
%    \end{macro}
%    \begin{macro}{\llm@CheckParams}
%    \begin{macrocode}
\def\llm@CheckParams#1:->#2\@nil{%
  \begingroup
    \def\x{#1}%
  \ifx\x\llm@macro
    \endgroup
    \def\llm@protected{}%
    \expandafter\@firstoftwo
  \else
    \ifx\x\llm@protectedmacro
      \endgroup
      \def\llm@protected{\protected}%
      \expandafter\expandafter\expandafter\@firstoftwo
    \else
      \endgroup
      \expandafter\expandafter\expandafter\@secondoftwo
    \fi
  \fi
}
%    \end{macrocode}
%    \end{macro}
%    \begin{macro}{\llm@macro}
%    \begin{macrocode}
\def\llm@macro{macro}
\@onelevel@sanitize\llm@macro
%    \end{macrocode}
%    \end{macro}
%    \begin{macro}{\llm@protectedmacro}
%    \begin{macrocode}
\def\llm@protectedmacro{\protected macro}
\@onelevel@sanitize\llm@protectedmacro
%    \end{macrocode}
%    \end{macro}
%    \begin{macro}{\llm@CarThree}
%    \begin{macrocode}
\def\llm@CarThree#1#2#3#4\llm@nil{#1#2#3}%
%    \end{macrocode}
%    \end{macro}
%    \begin{macro}{\llm@CarTwo}
%    \begin{macrocode}
\def\llm@CarTwo#1#2#3\llm@nil{#1#2}%
%    \end{macrocode}
%    \end{macro}
%
%    \begin{macrocode}
\llm@AtEnd%
%</package>
%    \end{macrocode}
%
% \section{Test}
%
% \subsection{Catcode checks for loading}
%
%    \begin{macrocode}
%<*test1>
%    \end{macrocode}
%    \begin{macrocode}
\NeedsTeXFormat{LaTeX2e}
\documentclass{minimal}
\makeatletter
\def\RestoreCatcodes{}
\count@=0 %
\loop
  \edef\RestoreCatcodes{%
    \RestoreCatcodes
    \catcode\the\count@=\the\catcode\count@\relax
  }%
\ifnum\count@<255 %
  \advance\count@\@ne
\repeat

\def\RangeCatcodeInvalid#1#2{%
  \count@=#1\relax
  \loop
    \catcode\count@=15 %
  \ifnum\count@<#2\relax
    \advance\count@\@ne
  \repeat
}
\def\Test{%
  \RangeCatcodeInvalid{0}{47}%
  \RangeCatcodeInvalid{58}{64}%
  \RangeCatcodeInvalid{91}{96}%
  \RangeCatcodeInvalid{123}{127}%
  \catcode`\@=12 %
  \catcode`\\=0 %
  \catcode`\{=1 %
  \catcode`\}=2 %
  \catcode`\#=6 %
  \catcode`\[=12 %
  \catcode`\]=12 %
  \catcode`\%=14 %
  \catcode`\ =10 %
  \catcode13=5 %
  \RequirePackage{letltxmacro}[2010/09/02]\relax
  \RestoreCatcodes
}
\Test
\csname @@end\endcsname
\end
%    \end{macrocode}
%    \begin{macrocode}
%</test1>
%    \end{macrocode}
%
% \subsection{Package tests}
%
%    \begin{macrocode}
%<*test2>
%    \end{macrocode}
%    \begin{macrocode}
\NeedsTeXFormat{LaTeX2e}
\nofiles
\documentclass{minimal}

\usepackage{letltxmacro}[2010/09/02]

\usepackage{qstest}
\IncludeTests{*}
\LogTests{log}{*}{*}

\makeatletter

\def\TestDef#1{%
  \begingroup
    \@makeother\\%
    \@makeother\ %
    \expandafter\@TestDef\csname #1\endcsname
}
\def\@TestDef#1#2{%
    \edef\a{\expandafter\strip@prefix\meaning#1}%
    \edef\b{\detokenize{#2}}%
    \ifx\a\b
    \else
      \typeout{1 [\a]}%
      \typeout{2 [\b]}%
    \fi
  \endgroup
  \Expect*{\expandafter\strip@prefix\meaning#1}*{\detokenize{#2}}%
}
\def\TestEquals#1#2{%
  \Expect*{%
    \expandafter\ifx\csname#1\expandafter\endcsname
                    \csname#2\endcsname
      equals%
    \else
      wrong%
    \fi
  }{equals}%
}
\def\SaveA{%
  \let\SavedA\TestA
  \expandafter\let\csname\string\SavedA\expandafter\endcsname
                  \csname\string\TestA\endcsname
  \expandafter\let\csname SavedA \expandafter\endcsname
                  \csname TestA \endcsname
}
\def\CheckA{%
  \TestEquals{SavedA}{TestA}%
  \TestEquals{\string\SavedA}{\string\TestA}%
  \TestEquals{SavedA }{TestA }%
}

\begin{qstest}{robust}{robust}
  \begingroup
    \DeclareRobustCommand*{\TestA}{Test}%
    \SaveA
    \TestDef{TestA}{\protect \TestA  }%
    \CheckA
    \DeclareRobustCommand{\TestA}{Test}%
    \SaveA
    \TestDef{TestA}{\protect \TestA  }%
    \LetLtxMacro\TestB\TestA
    \TestDef{TestB}{\protect \TestB  }%
    \TestEquals{TestB }{TestA }%
    \CheckA
    \GlobalLetLtxMacro\TestC\TestA
    \CheckA
  \endgroup
  \TestDef{TestC}{\protect \TestC  }%
  \DeclareRobustCommand{\TestA}{Test}%
  \TestEquals{TestC }{TestA }%
\end{qstest}

\begin{qstest}{default}{default}
  \begingroup
    \newcommand{\TestA}[1][\relax default \empty]{TestA #1}%
    \SaveA
    \TestDef
    {TestA}{\@protected@testopt \TestA \\TestA {\relax default \empty }}%
    \LetLtxMacro\TestB\TestA
    \TestDef
    {TestB}{\@protected@testopt \TestB \\TestB {\relax default \empty }}%
    \CheckA
    \TestEquals{\string\TestB}{\string\TestA}%
    \GlobalLetLtxMacro\TestC\TestA
    \CheckA
  \endgroup
  \TestDef
  {TestC}{\@protected@testopt \TestC \\TestC {\relax default \empty }}%
  \newcommand{\TestA}[1][\relax default \empty]{TestA #1}%
  \TestEquals{\string\TestC}{\string\TestA}%
\end{qstest}

\begin{qstest}{robustdefault}{robustdefault}
  \begingroup
    \DeclareRobustCommand{\TestA}[1][\default]{TestA}%
    \SaveA
    \LetLtxMacro\TestB\TestA
    \TestDef{TestB}{\protect \TestB  }%
    \TestDef{TestB }{\@protected@testopt \TestB  \\TestB  {\default }}%
    \CheckA
    \GlobalLetLtxMacro\TestC\TestA
    \CheckA
  \endgroup
  \TestDef{TestC}{\protect \TestC  }%
  \TestDef{TestC }{\@protected@testopt \TestC  \\TestC  {\default }}%
\end{qstest}

\begin{qstest}{plain}{plain}
  \LetLtxMacro\NewRelax\relax
  \Expect*{\meaning\relax}*{\string\relax}%
  \Expect*{\meaning\NewRelax}*{\string\relax}%
  \LetLtxMacro\NewHbox\hbox
  \Expect*{\meaning\hbox}*{\string\hbox}%
  \Expect*{\meaning\NewHbox}*{\string\hbox}%
  \LetLtxMacro\NewEmpty\empty
  \Expect*{\meaning\empty}{macro:->}%
  \Expect*{\meaning\NewEmpty}{macro:->}%
  \def\TestA{\iffalse}%
  \LetLtxMacro\TestB\TestA
  \TestDef{TestB}{\iffalse }%
  \TestEquals{TestA}{TestB}%
\end{qstest}
%    \end{macrocode}
%
%    \begin{macrocode}
\usepackage{etoolbox}[2008/06/22]

\begin{qstest}{oldnewrobustcmd}{oldnewrobustcmd}
  \begingroup
    \protected\edef\TestA{%
      \noexpand\@protected@testopt
      \noexpand\TestA
      \expandafter\noexpand\csname\string\TestA\endcsname
      {\noexpand\default}%
    }%
    \expandafter\protected\expandafter
    \def\csname\string\TestA\endcsname{TestA}%
    \SaveA
    \LetLtxMacro\TestB\TestA
    \TestDef{TestB}{\@protected@testopt \TestB \\TestB {\default }}%
    \protected\edef\TestC{%
      \noexpand\@protected@testopt
      \noexpand\TestB
      \expandafter\noexpand\csname\string\TestB\endcsname
      {\noexpand\default}%
    }%
    \TestEquals{TestB}{TestC}%
    \CheckA
    \GlobalLetLtxMacro\TestD\TestA
    \CheckA
  \endgroup
  \TestDef{TestD}{\@protected@testopt \TestD \\TestD {\default }}%
  \protected\edef\TestC{%
    \noexpand\@protected@testopt
    \noexpand\TestD
    \expandafter\noexpand\csname\string\TestD\endcsname
    {\noexpand\default}%
  }%
  \TestEquals{TestD}{TestC}%
\end{qstest}

\begin{qstest}{newrobustcmd}{newrobustcmd}
  \begingroup
    \newrobustcmd{\TestA}[1][\default]{TestA}%
    \SaveA
    \LetLtxMacro\TestB\TestA
    \TestDef{TestB}{\@testopt \\TestB {\default }}%
    \protected\edef\TestC{%
      \noexpand\@testopt
      \expandafter\noexpand\csname\string\TestB\endcsname
      {\noexpand\default}%
    }%
    \TestEquals{TestB}{TestC}%
    \CheckA
    \GlobalLetLtxMacro\TestD\TestA
    \CheckA
  \endgroup
  \TestDef{TestD}{\@testopt \\TestD {\default }}%
  \protected\edef\TestC{%
    \noexpand\@testopt
    \expandafter\noexpand\csname\string\TestD\endcsname
    {\noexpand\default}%
  }%
  \TestEquals{TestD}{TestC}%
\end{qstest}

\begin{qstest}{robustifyopt}{robustifyopt}
  \newcommand{\TestA}[2][\default]{}%
  \TestDef{TestA}{\@protected@testopt \TestA \\TestA {\default }}%
  \robustify\TestA
  \TestDef{TestA}{\@protected@testopt \TestA \\TestA {\default }}%
  \protected\edef\TestC{%
    \noexpand\@protected@testopt
    \noexpand\TestA
    \expandafter\noexpand\csname\string\TestA\endcsname
    {\noexpand\default}%
  }%
  \TestEquals{TestA}{TestC}%
  \SaveA
  \LetLtxMacro\TestB\TestA
  \TestDef{TestB}{\@protected@testopt \TestB \\TestB {\default }}%
  \protected\edef\TestC{%
    \noexpand\@protected@testopt
    \noexpand\TestB
    \expandafter\noexpand\csname\string\TestB\endcsname
    {\noexpand\default}%
  }%
  \TestEquals{TestB}{TestC}%
  \CheckA
\end{qstest}

\begin{qstest}{robustifydeclare}{robustifydeclare}
  \DeclareRobustCommand\TestA{\iffalse}%
  \robustify\TestA
  \SaveA
  \LetLtxMacro\TestB\TestA
  \TestDef{TestB}{\iffalse }%
  % before etoolbox 2008/06/22: \protected\long\def
  % since etoolbox 2008/06/22: \protected\def
  \protected\def\TestC{\iffalse}%
  \TestEquals{TestB}{TestC}%
  \CheckA
\end{qstest}

\begin{qstest}{robustifydeclarelong}{robustifydeclarelong}
  \DeclareRobustCommand\TestA[1]{\iffalse}%
  \robustify\TestA
  \SaveA
  \LetLtxMacro\TestB\TestA
  \TestDef{TestB}{\iffalse }%
  % before etoolbox 2008/06/22: \protected\long\def
  % since etoolbox 2008/06/22: \protected\def
  \protected\long\def\TestC#1{\iffalse}%
  \TestEquals{TestB}{TestC}%
  \CheckA
\end{qstest}
%    \end{macrocode}
%
%    \begin{macrocode}
\csname @@end\endcsname
%    \end{macrocode}
%    \begin{macrocode}
%</test2>
%    \end{macrocode}
%
% \section{Installation}
%
% \subsection{Download}
%
% \paragraph{Package.} This package is available on
% CTAN\footnote{\url{ftp://ftp.ctan.org/tex-archive/}}:
% \begin{description}
% \item[\CTAN{macros/latex/contrib/oberdiek/letltxmacro.dtx}] The source file.
% \item[\CTAN{macros/latex/contrib/oberdiek/letltxmacro.pdf}] Documentation.
% \end{description}
%
%
% \paragraph{Bundle.} All the packages of the bundle `oberdiek'
% are also available in a TDS compliant ZIP archive. There
% the packages are already unpacked and the documentation files
% are generated. The files and directories obey the TDS standard.
% \begin{description}
% \item[\CTAN{install/macros/latex/contrib/oberdiek.tds.zip}]
% \end{description}
% \emph{TDS} refers to the standard ``A Directory Structure
% for \TeX\ Files'' (\CTAN{tds/tds.pdf}). Directories
% with \xfile{texmf} in their name are usually organized this way.
%
% \subsection{Bundle installation}
%
% \paragraph{Unpacking.} Unpack the \xfile{oberdiek.tds.zip} in the
% TDS tree (also known as \xfile{texmf} tree) of your choice.
% Example (linux):
% \begin{quote}
%   |unzip oberdiek.tds.zip -d ~/texmf|
% \end{quote}
%
% \paragraph{Script installation.}
% Check the directory \xfile{TDS:scripts/oberdiek/} for
% scripts that need further installation steps.
% Package \xpackage{attachfile2} comes with the Perl script
% \xfile{pdfatfi.pl} that should be installed in such a way
% that it can be called as \texttt{pdfatfi}.
% Example (linux):
% \begin{quote}
%   |chmod +x scripts/oberdiek/pdfatfi.pl|\\
%   |cp scripts/oberdiek/pdfatfi.pl /usr/local/bin/|
% \end{quote}
%
% \subsection{Package installation}
%
% \paragraph{Unpacking.} The \xfile{.dtx} file is a self-extracting
% \docstrip\ archive. The files are extracted by running the
% \xfile{.dtx} through \plainTeX:
% \begin{quote}
%   \verb|tex letltxmacro.dtx|
% \end{quote}
%
% \paragraph{TDS.} Now the different files must be moved into
% the different directories in your installation TDS tree
% (also known as \xfile{texmf} tree):
% \begin{quote}
% \def\t{^^A
% \begin{tabular}{@{}>{\ttfamily}l@{ $\rightarrow$ }>{\ttfamily}l@{}}
%   letltxmacro.sty & tex/latex/oberdiek/letltxmacro.sty\\
%   letltxmacro.pdf & doc/latex/oberdiek/letltxmacro.pdf\\
%   letltxmacro-showcases.tex & doc/latex/oberdiek/letltxmacro-showcases.tex\\
%   test/letltxmacro-test1.tex & doc/latex/oberdiek/test/letltxmacro-test1.tex\\
%   test/letltxmacro-test2.tex & doc/latex/oberdiek/test/letltxmacro-test2.tex\\
%   letltxmacro.dtx & source/latex/oberdiek/letltxmacro.dtx\\
% \end{tabular}^^A
% }^^A
% \sbox0{\t}^^A
% \ifdim\wd0>\linewidth
%   \begingroup
%     \advance\linewidth by\leftmargin
%     \advance\linewidth by\rightmargin
%   \edef\x{\endgroup
%     \def\noexpand\lw{\the\linewidth}^^A
%   }\x
%   \def\lwbox{^^A
%     \leavevmode
%     \hbox to \linewidth{^^A
%       \kern-\leftmargin\relax
%       \hss
%       \usebox0
%       \hss
%       \kern-\rightmargin\relax
%     }^^A
%   }^^A
%   \ifdim\wd0>\lw
%     \sbox0{\small\t}^^A
%     \ifdim\wd0>\linewidth
%       \ifdim\wd0>\lw
%         \sbox0{\footnotesize\t}^^A
%         \ifdim\wd0>\linewidth
%           \ifdim\wd0>\lw
%             \sbox0{\scriptsize\t}^^A
%             \ifdim\wd0>\linewidth
%               \ifdim\wd0>\lw
%                 \sbox0{\tiny\t}^^A
%                 \ifdim\wd0>\linewidth
%                   \lwbox
%                 \else
%                   \usebox0
%                 \fi
%               \else
%                 \lwbox
%               \fi
%             \else
%               \usebox0
%             \fi
%           \else
%             \lwbox
%           \fi
%         \else
%           \usebox0
%         \fi
%       \else
%         \lwbox
%       \fi
%     \else
%       \usebox0
%     \fi
%   \else
%     \lwbox
%   \fi
% \else
%   \usebox0
% \fi
% \end{quote}
% If you have a \xfile{docstrip.cfg} that configures and enables \docstrip's
% TDS installing feature, then some files can already be in the right
% place, see the documentation of \docstrip.
%
% \subsection{Refresh file name databases}
%
% If your \TeX~distribution
% (\teTeX, \mikTeX, \dots) relies on file name databases, you must refresh
% these. For example, \teTeX\ users run \verb|texhash| or
% \verb|mktexlsr|.
%
% \subsection{Some details for the interested}
%
% \paragraph{Attached source.}
%
% The PDF documentation on CTAN also includes the
% \xfile{.dtx} source file. It can be extracted by
% AcrobatReader 6 or higher. Another option is \textsf{pdftk},
% e.g. unpack the file into the current directory:
% \begin{quote}
%   \verb|pdftk letltxmacro.pdf unpack_files output .|
% \end{quote}
%
% \paragraph{Unpacking with \LaTeX.}
% The \xfile{.dtx} chooses its action depending on the format:
% \begin{description}
% \item[\plainTeX:] Run \docstrip\ and extract the files.
% \item[\LaTeX:] Generate the documentation.
% \end{description}
% If you insist on using \LaTeX\ for \docstrip\ (really,
% \docstrip\ does not need \LaTeX), then inform the autodetect routine
% about your intention:
% \begin{quote}
%   \verb|latex \let\install=y\input{letltxmacro.dtx}|
% \end{quote}
% Do not forget to quote the argument according to the demands
% of your shell.
%
% \paragraph{Generating the documentation.}
% You can use both the \xfile{.dtx} or the \xfile{.drv} to generate
% the documentation. The process can be configured by the
% configuration file \xfile{ltxdoc.cfg}. For instance, put this
% line into this file, if you want to have A4 as paper format:
% \begin{quote}
%   \verb|\PassOptionsToClass{a4paper}{article}|
% \end{quote}
% An example follows how to generate the
% documentation with pdf\LaTeX:
% \begin{quote}
%\begin{verbatim}
%pdflatex letltxmacro.dtx
%makeindex -s gind.ist letltxmacro.idx
%pdflatex letltxmacro.dtx
%makeindex -s gind.ist letltxmacro.idx
%pdflatex letltxmacro.dtx
%\end{verbatim}
% \end{quote}
%
% \section{Catalogue}
%
% The following XML file can be used as source for the
% \href{http://mirror.ctan.org/help/Catalogue/catalogue.html}{\TeX\ Catalogue}.
% The elements \texttt{caption} and \texttt{description} are imported
% from the original XML file from the Catalogue.
% The name of the XML file in the Catalogue is \xfile{letltxmacro.xml}.
%    \begin{macrocode}
%<*catalogue>
<?xml version='1.0' encoding='us-ascii'?>
<!DOCTYPE entry SYSTEM 'catalogue.dtd'>
<entry datestamp='$Date$' modifier='$Author$' id='letltxmacro'>
  <name>letltxmacro</name>
  <caption>Let assignment for LaTeX macros.</caption>
  <authorref id='auth:oberdiek'/>
  <copyright owner='Heiko Oberdiek' year='2008,2010'/>
  <license type='lppl1.3'/>
  <version number='1.4'/>
  <description>
    TeX&#x2019;s <tt>\let</tt> assignment does not work for LaTeX macros
    with optional arguments or for macros that are defined
    as robust macros by <tt>\DeclareRobustCommand</tt>.  This package
    defines <tt>\LetLtxMacro</tt> that also takes care of the involved
    internal macros.
    <p/>
    The package is part of the <xref refid='oberdiek'>oberdiek</xref>
    bundle.
  </description>
  <documentation details='Package documentation'
      href='ctan:/macros/latex/contrib/oberdiek/letltxmacro.pdf'/>
  <ctan file='true' path='/macros/latex/contrib/oberdiek/letltxmacro.dtx'/>
  <miktex location='oberdiek'/>
  <texlive location='oberdiek'/>
  <install path='/macros/latex/contrib/oberdiek/oberdiek.tds.zip'/>
</entry>
%</catalogue>
%    \end{macrocode}
%
% \begin{History}
%   \begin{Version}{2008/06/09 v1.0}
%   \item
%     First version.
%   \end{Version}
%   \begin{Version}{2008/06/12 v1.1}
%   \item
%     Support for \xpackage{etoolbox}'s \cs{newrobustcmd} added.
%   \end{Version}
%   \begin{Version}{2008/06/13 v1.2}
%   \item
%     Support for \xpackage{etoolbox}'s \cs{robustify} added.
%   \end{Version}
%   \begin{Version}{2008/06/24 v1.3}
%   \item
%     Test file adapted for etoolbox 2008/06/22 v1.6.
%   \end{Version}
%   \begin{Version}{2010/09/02 v1.4}
%   \item
%     \cs{GlobalLetLtxMacro} added.
%   \end{Version}
% \end{History}
%
% \PrintIndex
%
% \Finale
\endinput
|
% \end{quote}
% Do not forget to quote the argument according to the demands
% of your shell.
%
% \paragraph{Generating the documentation.}
% You can use both the \xfile{.dtx} or the \xfile{.drv} to generate
% the documentation. The process can be configured by the
% configuration file \xfile{ltxdoc.cfg}. For instance, put this
% line into this file, if you want to have A4 as paper format:
% \begin{quote}
%   \verb|\PassOptionsToClass{a4paper}{article}|
% \end{quote}
% An example follows how to generate the
% documentation with pdf\LaTeX:
% \begin{quote}
%\begin{verbatim}
%pdflatex letltxmacro.dtx
%makeindex -s gind.ist letltxmacro.idx
%pdflatex letltxmacro.dtx
%makeindex -s gind.ist letltxmacro.idx
%pdflatex letltxmacro.dtx
%\end{verbatim}
% \end{quote}
%
% \section{Catalogue}
%
% The following XML file can be used as source for the
% \href{http://mirror.ctan.org/help/Catalogue/catalogue.html}{\TeX\ Catalogue}.
% The elements \texttt{caption} and \texttt{description} are imported
% from the original XML file from the Catalogue.
% The name of the XML file in the Catalogue is \xfile{letltxmacro.xml}.
%    \begin{macrocode}
%<*catalogue>
<?xml version='1.0' encoding='us-ascii'?>
<!DOCTYPE entry SYSTEM 'catalogue.dtd'>
<entry datestamp='$Date$' modifier='$Author$' id='letltxmacro'>
  <name>letltxmacro</name>
  <caption>Let assignment for LaTeX macros.</caption>
  <authorref id='auth:oberdiek'/>
  <copyright owner='Heiko Oberdiek' year='2008,2010'/>
  <license type='lppl1.3'/>
  <version number='1.4'/>
  <description>
    TeX&#x2019;s <tt>\let</tt> assignment does not work for LaTeX macros
    with optional arguments or for macros that are defined
    as robust macros by <tt>\DeclareRobustCommand</tt>.  This package
    defines <tt>\LetLtxMacro</tt> that also takes care of the involved
    internal macros.
    <p/>
    The package is part of the <xref refid='oberdiek'>oberdiek</xref>
    bundle.
  </description>
  <documentation details='Package documentation'
      href='ctan:/macros/latex/contrib/oberdiek/letltxmacro.pdf'/>
  <ctan file='true' path='/macros/latex/contrib/oberdiek/letltxmacro.dtx'/>
  <miktex location='oberdiek'/>
  <texlive location='oberdiek'/>
  <install path='/macros/latex/contrib/oberdiek/oberdiek.tds.zip'/>
</entry>
%</catalogue>
%    \end{macrocode}
%
% \begin{History}
%   \begin{Version}{2008/06/09 v1.0}
%   \item
%     First version.
%   \end{Version}
%   \begin{Version}{2008/06/12 v1.1}
%   \item
%     Support for \xpackage{etoolbox}'s \cs{newrobustcmd} added.
%   \end{Version}
%   \begin{Version}{2008/06/13 v1.2}
%   \item
%     Support for \xpackage{etoolbox}'s \cs{robustify} added.
%   \end{Version}
%   \begin{Version}{2008/06/24 v1.3}
%   \item
%     Test file adapted for etoolbox 2008/06/22 v1.6.
%   \end{Version}
%   \begin{Version}{2010/09/02 v1.4}
%   \item
%     \cs{GlobalLetLtxMacro} added.
%   \end{Version}
% \end{History}
%
% \PrintIndex
%
% \Finale
\endinput
|
% \end{quote}
% Do not forget to quote the argument according to the demands
% of your shell.
%
% \paragraph{Generating the documentation.}
% You can use both the \xfile{.dtx} or the \xfile{.drv} to generate
% the documentation. The process can be configured by the
% configuration file \xfile{ltxdoc.cfg}. For instance, put this
% line into this file, if you want to have A4 as paper format:
% \begin{quote}
%   \verb|\PassOptionsToClass{a4paper}{article}|
% \end{quote}
% An example follows how to generate the
% documentation with pdf\LaTeX:
% \begin{quote}
%\begin{verbatim}
%pdflatex letltxmacro.dtx
%makeindex -s gind.ist letltxmacro.idx
%pdflatex letltxmacro.dtx
%makeindex -s gind.ist letltxmacro.idx
%pdflatex letltxmacro.dtx
%\end{verbatim}
% \end{quote}
%
% \section{Catalogue}
%
% The following XML file can be used as source for the
% \href{http://mirror.ctan.org/help/Catalogue/catalogue.html}{\TeX\ Catalogue}.
% The elements \texttt{caption} and \texttt{description} are imported
% from the original XML file from the Catalogue.
% The name of the XML file in the Catalogue is \xfile{letltxmacro.xml}.
%    \begin{macrocode}
%<*catalogue>
<?xml version='1.0' encoding='us-ascii'?>
<!DOCTYPE entry SYSTEM 'catalogue.dtd'>
<entry datestamp='$Date$' modifier='$Author$' id='letltxmacro'>
  <name>letltxmacro</name>
  <caption>Let assignment for LaTeX macros.</caption>
  <authorref id='auth:oberdiek'/>
  <copyright owner='Heiko Oberdiek' year='2008,2010'/>
  <license type='lppl1.3'/>
  <version number='1.4'/>
  <description>
    TeX&#x2019;s <tt>\let</tt> assignment does not work for LaTeX macros
    with optional arguments or for macros that are defined
    as robust macros by <tt>\DeclareRobustCommand</tt>.  This package
    defines <tt>\LetLtxMacro</tt> that also takes care of the involved
    internal macros.
    <p/>
    The package is part of the <xref refid='oberdiek'>oberdiek</xref>
    bundle.
  </description>
  <documentation details='Package documentation'
      href='ctan:/macros/latex/contrib/oberdiek/letltxmacro.pdf'/>
  <ctan file='true' path='/macros/latex/contrib/oberdiek/letltxmacro.dtx'/>
  <miktex location='oberdiek'/>
  <texlive location='oberdiek'/>
  <install path='/macros/latex/contrib/oberdiek/oberdiek.tds.zip'/>
</entry>
%</catalogue>
%    \end{macrocode}
%
% \begin{History}
%   \begin{Version}{2008/06/09 v1.0}
%   \item
%     First version.
%   \end{Version}
%   \begin{Version}{2008/06/12 v1.1}
%   \item
%     Support for \xpackage{etoolbox}'s \cs{newrobustcmd} added.
%   \end{Version}
%   \begin{Version}{2008/06/13 v1.2}
%   \item
%     Support for \xpackage{etoolbox}'s \cs{robustify} added.
%   \end{Version}
%   \begin{Version}{2008/06/24 v1.3}
%   \item
%     Test file adapted for etoolbox 2008/06/22 v1.6.
%   \end{Version}
%   \begin{Version}{2010/09/02 v1.4}
%   \item
%     \cs{GlobalLetLtxMacro} added.
%   \end{Version}
% \end{History}
%
% \PrintIndex
%
% \Finale
\endinput

%        (quote the arguments according to the demands of your shell)
%
% Documentation:
%    (a) If letltxmacro.drv is present:
%           latex letltxmacro.drv
%    (b) Without letltxmacro.drv:
%           latex letltxmacro.dtx; ...
%    The class ltxdoc loads the configuration file ltxdoc.cfg
%    if available. Here you can specify further options, e.g.
%    use A4 as paper format:
%       \PassOptionsToClass{a4paper}{article}
%
%    Programm calls to get the documentation (example):
%       pdflatex letltxmacro.dtx
%       makeindex -s gind.ist letltxmacro.idx
%       pdflatex letltxmacro.dtx
%       makeindex -s gind.ist letltxmacro.idx
%       pdflatex letltxmacro.dtx
%
% Installation:
%    TDS:tex/latex/oberdiek/letltxmacro.sty
%    TDS:doc/latex/oberdiek/letltxmacro.pdf
%    TDS:doc/latex/oberdiek/letltxmacro-showcases.tex
%    TDS:doc/latex/oberdiek/test/letltxmacro-test1.tex
%    TDS:doc/latex/oberdiek/test/letltxmacro-test2.tex
%    TDS:source/latex/oberdiek/letltxmacro.dtx
%
%<*ignore>
\begingroup
  \catcode123=1 %
  \catcode125=2 %
  \def\x{LaTeX2e}%
\expandafter\endgroup
\ifcase 0\ifx\install y1\fi\expandafter
         \ifx\csname processbatchFile\endcsname\relax\else1\fi
         \ifx\fmtname\x\else 1\fi\relax
\else\csname fi\endcsname
%</ignore>
%<*install>
\input docstrip.tex
\Msg{************************************************************************}
\Msg{* Installation}
\Msg{* Package: letltxmacro 2010/09/02 v1.4 Let assignment for LaTeX macros (HO)}
\Msg{************************************************************************}

\keepsilent
\askforoverwritefalse

\let\MetaPrefix\relax
\preamble

This is a generated file.

Project: letltxmacro
Version: 2010/09/02 v1.4

Copyright (C) 2008, 2010 by
   Heiko Oberdiek <heiko.oberdiek at googlemail.com>

This work may be distributed and/or modified under the
conditions of the LaTeX Project Public License, either
version 1.3c of this license or (at your option) any later
version. This version of this license is in
   http://www.latex-project.org/lppl/lppl-1-3c.txt
and the latest version of this license is in
   http://www.latex-project.org/lppl.txt
and version 1.3 or later is part of all distributions of
LaTeX version 2005/12/01 or later.

This work has the LPPL maintenance status "maintained".

This Current Maintainer of this work is Heiko Oberdiek.

This work consists of the main source file letltxmacro.dtx
and the derived files
   letltxmacro.sty, letltxmacro.pdf, letltxmacro.ins, letltxmacro.drv,
   letltxmacro-showcases.tex, letltxmacro-test1.tex,
   letltxmacro-test2.tex.

\endpreamble
\let\MetaPrefix\DoubleperCent

\generate{%
  \file{letltxmacro.ins}{\from{letltxmacro.dtx}{install}}%
  \file{letltxmacro.drv}{\from{letltxmacro.dtx}{driver}}%
  \usedir{tex/latex/oberdiek}%
  \file{letltxmacro.sty}{\from{letltxmacro.dtx}{package}}%
  \usedir{doc/latex/oberdiek}%
  \file{letltxmacro-showcases.tex}{\from{letltxmacro.dtx}{showcases}}%
  \usedir{doc/latex/oberdiek/test}%
  \file{letltxmacro-test1.tex}{\from{letltxmacro.dtx}{test1}}%
  \file{letltxmacro-test2.tex}{\from{letltxmacro.dtx}{test2}}%
  \nopreamble
  \nopostamble
  \usedir{source/latex/oberdiek/catalogue}%
  \file{letltxmacro.xml}{\from{letltxmacro.dtx}{catalogue}}%
}

\catcode32=13\relax% active space
\let =\space%
\Msg{************************************************************************}
\Msg{*}
\Msg{* To finish the installation you have to move the following}
\Msg{* file into a directory searched by TeX:}
\Msg{*}
\Msg{*     letltxmacro.sty}
\Msg{*}
\Msg{* To produce the documentation run the file `letltxmacro.drv'}
\Msg{* through LaTeX.}
\Msg{*}
\Msg{* Happy TeXing!}
\Msg{*}
\Msg{************************************************************************}

\endbatchfile
%</install>
%<*ignore>
\fi
%</ignore>
%<*driver>
\NeedsTeXFormat{LaTeX2e}
\ProvidesFile{letltxmacro.drv}%
  [2010/09/02 v1.4 Let assignment for LaTeX macros (HO)]%
\documentclass{ltxdoc}
\usepackage{holtxdoc}[2011/11/22]
\begin{document}
  \DocInput{letltxmacro.dtx}%
\end{document}
%</driver>
% \fi
%
% \CheckSum{990}
%
% \CharacterTable
%  {Upper-case    \A\B\C\D\E\F\G\H\I\J\K\L\M\N\O\P\Q\R\S\T\U\V\W\X\Y\Z
%   Lower-case    \a\b\c\d\e\f\g\h\i\j\k\l\m\n\o\p\q\r\s\t\u\v\w\x\y\z
%   Digits        \0\1\2\3\4\5\6\7\8\9
%   Exclamation   \!     Double quote  \"     Hash (number) \#
%   Dollar        \$     Percent       \%     Ampersand     \&
%   Acute accent  \'     Left paren    \(     Right paren   \)
%   Asterisk      \*     Plus          \+     Comma         \,
%   Minus         \-     Point         \.     Solidus       \/
%   Colon         \:     Semicolon     \;     Less than     \<
%   Equals        \=     Greater than  \>     Question mark \?
%   Commercial at \@     Left bracket  \[     Backslash     \\
%   Right bracket \]     Circumflex    \^     Underscore    \_
%   Grave accent  \`     Left brace    \{     Vertical bar  \|
%   Right brace   \}     Tilde         \~}
%
% \GetFileInfo{letltxmacro.drv}
%
% \title{The \xpackage{letltxmacro} package}
% \date{2010/09/02 v1.4}
% \author{Heiko Oberdiek\\\xemail{heiko.oberdiek at googlemail.com}}
%
% \maketitle
%
% \begin{abstract}
% \TeX's \cs{let} assignment does not work for \LaTeX\ macros
% with optional arguments or for macros that are defined
% as robust macros by \cs{DeclareRobustCommand}. This package
% defines \cs{LetLtxMacro} that also takes care of the involved
% internal macros.
% \end{abstract}
%
% \tableofcontents
%
% \section{Documentation}
%
% If someone wants to redefine a macro with using the old
% meaning, then one method is \TeX's command \cs{let}:
%\begin{quote}
%\begin{verbatim}
%\newcommand{\Macro}{\typeout{Test Macro}}
%\let\SavedMacro=\Macro
%\renewcommand{\Macro}{%
%  \typeout{Begin}%
%  \SavedMacro
%  \typeout{End}%
%}
%\end{verbatim}
%\end{quote}
% However, this method fails, if \cs{Macro} is defined
% by \cs{DeclareRobustCommand} and/or has an optional argument.
% In both cases \LaTeX\ defines an additional internal macro
% that is forgotten in the simple \cs{let} assignment of
% the example above.
%
% \begin{declcs}{LetLtxMacro} \M{new macro} \M{old macro}
% \end{declcs}
% Macro \cs{LetLtxMacro} behaves similar to \TeX's \cs{let}
% assignment, but it takes care of macros that are
% defined by \cs{DeclareRobustCommand} and/or have optional
% arguments. Example:
%\begin{quote}
%\begin{verbatim}
%\DeclareRobustCommand{\Macro}[1][default]{...}
%\LetLtxMacro{\SavedMacro}{\Macro}
%\end{verbatim}
%\end{quote}
% Then macro \cs{SavedMacro} only uses internal macro names
% that are derived from \cs{SavedMacro}'s macro name. Macro \cs{Macro}
% can now be redefined without affecting \cs{SavedMacro}.
%
% \begin{declcs}{GlobalLetLtxMacro} \M{new macro} \M{old macro}
% \end{declcs}
% Like \cs{LetLtxMacro}, but the \meta{new macro} is defined globally.
% Since version 2010/09/02~v1.4.
%
% \subsection{Supported macro definition commands}
%
% \begin{quote}
%   \begin{tabular}{@{}ll@{}}
%     \cs{newcommand}, \cs{renewcommand} & latex/base\\
%     \cs{newenvironment}, \cs{renewenvironment} & latex/base\\
%     \cs{DeclareRobustCommand}& latex/base\\
%     \cs{newrobustcmd}, \cs{renewrobustcmd} & etoolbox\\
%     \cs{robustify} & etoolbox 2008/06/22 v1.6\\
%   \end{tabular}
% \end{quote}
%
% \StopEventually{
% }
%
% \section{Implementation}
%
% \subsection{Show cases}
%
% \subsubsection{\xfile{letltxmacro-showcases.tex}}
%
%    \begin{macrocode}
%<*showcases>
\NeedsTeXFormat{LaTeX2e}
\makeatletter
%    \end{macrocode}
%    \begin{macro}{\Line}
%    The result is displayed by macro \cs{Line}. The percent symbol
%    at line start allows easy grepping and inserting into the DTX
%    file.
%    \begin{macrocode}
\newcommand*{\Line}[1]{%
  \typeout{\@percentchar#1}%
}
%    \end{macrocode}
%    \end{macro}
%    \begin{macrocode}
\newcommand*{\ShowCmdName}[1]{%
  \@ifundefined{#1}{}{%
    \Line{%
      \space\space(\expandafter\string\csname#1\endcsname) = %
      (\expandafter\meaning\csname#1\endcsname)%
    }%
  }%
}
\newcommand*{\ShowCmds}[1]{%
  \ShowCmdName{#1}%
  \ShowCmdName{#1 }%
  \ShowCmdName{\\#1}%
  \ShowCmdName{\\#1 }%
}
\let\\\@backslashchar
%    \end{macrocode}
%    \begin{macro}{\ShowDef}
%    \begin{macrocode}
\newcommand*{\ShowDef}[2]{%
  \begingroup
    \Line{}%
    \newcommand*{\DefString}{#2}%
    \@onelevel@sanitize\DefString
    \Line{\DefString}%
    #2%
    \ShowCmds{#1}%
  \endgroup
}
%    \end{macrocode}
%    \end{macro}
%    \begin{macrocode}
\typeout{}
\Line{* LaTeX definitions:}
\ShowDef{cmd}{%
  \newcommand{\cmd}[2][default]{}%
}
\ShowDef{cmd}{%
  \DeclareRobustCommand{\cmd}{}%
}
\ShowDef{cmd}{%
  \DeclareRobustCommand{\cmd}[2][default]{}%
}
\typeout{}
%    \end{macrocode}
% The minimal version of package \xpackage{etoolbox} is 2008/06/12 v1.6a
% because it fixes \cs{robustify}.
%    \begin{macrocode}
\RequirePackage{etoolbox}[2008/06/12]%
\Line{}
\Line{* etoolbox's robust definitions:}
\ShowDef{cmd}{%
  \newrobustcmd{\cmd}{}%
}
\ShowDef{cmd}{%
  \newrobustcmd{\cmd}[2][default]{}%
}
\Line{}
\Line{* etoolbox's \string\robustify:}
\ShowDef{cmd}{%
  \newcommand{\cmd}[2][default]{} %
  \robustify{\cmd}%
}
\ShowDef{cmd}{%
  \DeclareRobustCommand{\cmd}{} %
  \robustify{\cmd}%
}
\ShowDef{cmd}{%
  \DeclareRobustCommand{\cmd}[2][default]{} %
  \robustify{\cmd}%
}
\typeout{}
\@@end
%</showcases>
%    \end{macrocode}
%
% \subsubsection{Result}
%
% \begingroup
%   \makeatletter
%   \let\org@verbatim\@verbatim
%   \def\@verbatim{^^A
%     \org@verbatim
%     \catcode`\~=\active
%   }^^A
%   \let~\textvisiblespace
%\begin{verbatim}
%* LaTeX definitions:
%
%\newcommand {\cmd }[2][default]{}
%  (\cmd) = (macro:->\@protected@testopt \cmd \\cmd {default})
%  (\\cmd) = (\long macro:[#1]#2->)
%
%\DeclareRobustCommand {\cmd }{}
%  (\cmd) = (macro:->\protect \cmd~ )
%  (\cmd~) = (\long macro:->)
%
%\DeclareRobustCommand {\cmd }[2][default]{}
%  (\cmd) = (macro:->\protect \cmd~ )
%  (\cmd~) = (macro:->\@protected@testopt \cmd~ \\cmd~ {default})
%  (\\cmd~) = (\long macro:[#1]#2->)
%
%* etoolbox's robust definitions:
%
%\newrobustcmd {\cmd }{}
%  (\cmd) = (\protected\long macro:->)
%
%\newrobustcmd {\cmd }[2][default]{}
%  (\cmd) = (\protected macro:->\@testopt \\cmd {default})
%  (\\cmd) = (\long macro:[#1]#2->)
%
%* etoolbox's \robustify:
%
%\newcommand {\cmd }[2][default]{} \robustify {\cmd }
%  (\cmd) = (\protected macro:->\@protected@testopt \cmd \\cmd {default})
%  (\\cmd) = (\long macro:[#1]#2->)
%
%\DeclareRobustCommand {\cmd }{} \robustify {\cmd }
%  (\cmd) = (\protected macro:->)
%
%\DeclareRobustCommand {\cmd }[2][default]{} \robustify {\cmd }
%  (\cmd) = (\protected macro:->\@protected@testopt \cmd~ \\cmd~ {default})
%  (\cmd~) = (macro:->\@protected@testopt \cmd~ \\cmd~ {default})
%  (\\cmd~) = (\long macro:[#1]#2->)
%\end{verbatim}
% \endgroup
%
% \subsection{Package}
%
%    \begin{macrocode}
%<*package>
%    \end{macrocode}
%
% \subsubsection{Catcodes and identification}
%
%    \begin{macrocode}
\begingroup\catcode61\catcode48\catcode32=10\relax%
  \catcode13=5 % ^^M
  \endlinechar=13 %
  \catcode123=1 % {
  \catcode125=2 % }
  \catcode64=11 % @
  \def\x{\endgroup
    \expandafter\edef\csname llm@AtEnd\endcsname{%
      \endlinechar=\the\endlinechar\relax
      \catcode13=\the\catcode13\relax
      \catcode32=\the\catcode32\relax
      \catcode35=\the\catcode35\relax
      \catcode61=\the\catcode61\relax
      \catcode64=\the\catcode64\relax
      \catcode123=\the\catcode123\relax
      \catcode125=\the\catcode125\relax
    }%
  }%
\x\catcode61\catcode48\catcode32=10\relax%
\catcode13=5 % ^^M
\endlinechar=13 %
\catcode35=6 % #
\catcode64=11 % @
\catcode123=1 % {
\catcode125=2 % }
\def\TMP@EnsureCode#1#2{%
  \edef\llm@AtEnd{%
    \llm@AtEnd
    \catcode#1=\the\catcode#1\relax
  }%
  \catcode#1=#2\relax
}
\TMP@EnsureCode{40}{12}% (
\TMP@EnsureCode{41}{12}% )
\TMP@EnsureCode{42}{12}% *
\TMP@EnsureCode{45}{12}% -
\TMP@EnsureCode{46}{12}% .
\TMP@EnsureCode{47}{12}% /
\TMP@EnsureCode{58}{12}% :
\TMP@EnsureCode{62}{12}% >
\TMP@EnsureCode{91}{12}% [
\TMP@EnsureCode{93}{12}% ]
\edef\llm@AtEnd{%
  \llm@AtEnd
  \escapechar\the\escapechar\relax
  \noexpand\endinput
}
\escapechar=92 % `\\
%    \end{macrocode}
%
%    Package identification.
%    \begin{macrocode}
\NeedsTeXFormat{LaTeX2e}
\ProvidesPackage{letltxmacro}%
  [2010/09/02 v1.4 Let assignment for LaTeX macros (HO)]
%    \end{macrocode}
%
% \subsubsection{Main macros}
%
%    \begin{macro}{\LetLtxMacro}
%    \begin{macrocode}
\newcommand*{\LetLtxMacro}{%
  \llm@ModeLetLtxMacro{}%
}
%    \end{macrocode}
%    \end{macro}
%    \begin{macro}{\GlobalLetLtxMacro}
%    \begin{macrocode}
\newcommand*{\GlobalLetLtxMacro}{%
  \llm@ModeLetLtxMacro\global
}
%    \end{macrocode}
%    \end{macro}
%
%    \begin{macro}{\llm@ModeLetLtxMacro}
%    \begin{macrocode}
\newcommand*{\llm@ModeLetLtxMacro}[3]{%
  \edef\llm@escapechar{\the\escapechar}%
  \escapechar=-1 %
  \edef\reserved@a{%
    \noexpand\protect
    \expandafter\noexpand
    \csname\string#3 \endcsname
  }%
  \ifx\reserved@a#3\relax
    #1\edef#2{%
      \noexpand\protect
      \expandafter\noexpand
      \csname\string#2 \endcsname
    }%
    #1\expandafter\let
    \csname\string#2 \expandafter\endcsname
    \csname\string#3 \endcsname
    \expandafter\llm@LetLtxMacro
        \csname\string#2 \expandafter\endcsname
        \csname\string#3 \endcsname{#1}%
  \else
    \llm@LetLtxMacro{#2}{#3}{#1}%
  \fi
  \escapechar=\llm@escapechar\relax
}
%    \end{macrocode}
%    \end{macro}
%    \begin{macro}{\llm@LetLtxMacro}
%    \begin{macrocode}
\def\llm@LetLtxMacro#1#2#3{%
  \escapechar=92 %
  \expandafter\llm@CheckParams\meaning#2:->\@nil{%
    \begingroup
      \def\@protected@testopt{%
        \expandafter\@testopt\@gobble
      }%
      \def\@testopt##1##2{%
        \toks@={##2}%
      }%
      \let\llm@testopt\@empty
      \edef\x{%
        \noexpand\@protected@testopt
        \noexpand#2%
        \expandafter\noexpand\csname\string#2\endcsname
      }%
      \expandafter\expandafter\expandafter\def
      \expandafter\expandafter\expandafter\y
      \expandafter\expandafter\expandafter{%
        \expandafter\llm@CarThree#2{}{}{}\llm@nil
      }%
      \ifx\x\y
        #2%
        \def\llm@testopt{%
          \noexpand\@protected@testopt
          \noexpand#1%
        }%
      \else
        \edef\x{%
          \noexpand\@testopt
          \expandafter\noexpand
          \csname\string#2\endcsname
        }%
        \expandafter\expandafter\expandafter\def
        \expandafter\expandafter\expandafter\y
        \expandafter\expandafter\expandafter{%
          \expandafter\llm@CarTwo#2{}{}\llm@nil
        }%
        \ifx\x\y
          #2%
          \def\llm@testopt{%
            \noexpand\@testopt
          }%
        \fi
      \fi
      \ifx\llm@testopt\@empty
      \else
        \llm@protected\xdef\llm@GlobalTemp{%
          \llm@testopt
          \expandafter\noexpand
          \csname\string#1\endcsname
          {\the\toks@}%
        }%
      \fi
    \expandafter\endgroup\ifx\llm@testopt\@empty
      #3\let#1=#2\relax
    \else
      #3\let#1=\llm@GlobalTemp
      #3\expandafter\let
          \csname\string#1\expandafter\endcsname
          \csname\string#2\endcsname
    \fi
  }{%
    #3\let#1=#2\relax
  }%
}
%    \end{macrocode}
%    \end{macro}
%    \begin{macro}{\llm@CheckParams}
%    \begin{macrocode}
\def\llm@CheckParams#1:->#2\@nil{%
  \begingroup
    \def\x{#1}%
  \ifx\x\llm@macro
    \endgroup
    \def\llm@protected{}%
    \expandafter\@firstoftwo
  \else
    \ifx\x\llm@protectedmacro
      \endgroup
      \def\llm@protected{\protected}%
      \expandafter\expandafter\expandafter\@firstoftwo
    \else
      \endgroup
      \expandafter\expandafter\expandafter\@secondoftwo
    \fi
  \fi
}
%    \end{macrocode}
%    \end{macro}
%    \begin{macro}{\llm@macro}
%    \begin{macrocode}
\def\llm@macro{macro}
\@onelevel@sanitize\llm@macro
%    \end{macrocode}
%    \end{macro}
%    \begin{macro}{\llm@protectedmacro}
%    \begin{macrocode}
\def\llm@protectedmacro{\protected macro}
\@onelevel@sanitize\llm@protectedmacro
%    \end{macrocode}
%    \end{macro}
%    \begin{macro}{\llm@CarThree}
%    \begin{macrocode}
\def\llm@CarThree#1#2#3#4\llm@nil{#1#2#3}%
%    \end{macrocode}
%    \end{macro}
%    \begin{macro}{\llm@CarTwo}
%    \begin{macrocode}
\def\llm@CarTwo#1#2#3\llm@nil{#1#2}%
%    \end{macrocode}
%    \end{macro}
%
%    \begin{macrocode}
\llm@AtEnd%
%</package>
%    \end{macrocode}
%
% \section{Test}
%
% \subsection{Catcode checks for loading}
%
%    \begin{macrocode}
%<*test1>
%    \end{macrocode}
%    \begin{macrocode}
\NeedsTeXFormat{LaTeX2e}
\documentclass{minimal}
\makeatletter
\def\RestoreCatcodes{}
\count@=0 %
\loop
  \edef\RestoreCatcodes{%
    \RestoreCatcodes
    \catcode\the\count@=\the\catcode\count@\relax
  }%
\ifnum\count@<255 %
  \advance\count@\@ne
\repeat

\def\RangeCatcodeInvalid#1#2{%
  \count@=#1\relax
  \loop
    \catcode\count@=15 %
  \ifnum\count@<#2\relax
    \advance\count@\@ne
  \repeat
}
\def\Test{%
  \RangeCatcodeInvalid{0}{47}%
  \RangeCatcodeInvalid{58}{64}%
  \RangeCatcodeInvalid{91}{96}%
  \RangeCatcodeInvalid{123}{127}%
  \catcode`\@=12 %
  \catcode`\\=0 %
  \catcode`\{=1 %
  \catcode`\}=2 %
  \catcode`\#=6 %
  \catcode`\[=12 %
  \catcode`\]=12 %
  \catcode`\%=14 %
  \catcode`\ =10 %
  \catcode13=5 %
  \RequirePackage{letltxmacro}[2010/09/02]\relax
  \RestoreCatcodes
}
\Test
\csname @@end\endcsname
\end
%    \end{macrocode}
%    \begin{macrocode}
%</test1>
%    \end{macrocode}
%
% \subsection{Package tests}
%
%    \begin{macrocode}
%<*test2>
%    \end{macrocode}
%    \begin{macrocode}
\NeedsTeXFormat{LaTeX2e}
\nofiles
\documentclass{minimal}

\usepackage{letltxmacro}[2010/09/02]

\usepackage{qstest}
\IncludeTests{*}
\LogTests{log}{*}{*}

\makeatletter

\def\TestDef#1{%
  \begingroup
    \@makeother\\%
    \@makeother\ %
    \expandafter\@TestDef\csname #1\endcsname
}
\def\@TestDef#1#2{%
    \edef\a{\expandafter\strip@prefix\meaning#1}%
    \edef\b{\detokenize{#2}}%
    \ifx\a\b
    \else
      \typeout{1 [\a]}%
      \typeout{2 [\b]}%
    \fi
  \endgroup
  \Expect*{\expandafter\strip@prefix\meaning#1}*{\detokenize{#2}}%
}
\def\TestEquals#1#2{%
  \Expect*{%
    \expandafter\ifx\csname#1\expandafter\endcsname
                    \csname#2\endcsname
      equals%
    \else
      wrong%
    \fi
  }{equals}%
}
\def\SaveA{%
  \let\SavedA\TestA
  \expandafter\let\csname\string\SavedA\expandafter\endcsname
                  \csname\string\TestA\endcsname
  \expandafter\let\csname SavedA \expandafter\endcsname
                  \csname TestA \endcsname
}
\def\CheckA{%
  \TestEquals{SavedA}{TestA}%
  \TestEquals{\string\SavedA}{\string\TestA}%
  \TestEquals{SavedA }{TestA }%
}

\begin{qstest}{robust}{robust}
  \begingroup
    \DeclareRobustCommand*{\TestA}{Test}%
    \SaveA
    \TestDef{TestA}{\protect \TestA  }%
    \CheckA
    \DeclareRobustCommand{\TestA}{Test}%
    \SaveA
    \TestDef{TestA}{\protect \TestA  }%
    \LetLtxMacro\TestB\TestA
    \TestDef{TestB}{\protect \TestB  }%
    \TestEquals{TestB }{TestA }%
    \CheckA
    \GlobalLetLtxMacro\TestC\TestA
    \CheckA
  \endgroup
  \TestDef{TestC}{\protect \TestC  }%
  \DeclareRobustCommand{\TestA}{Test}%
  \TestEquals{TestC }{TestA }%
\end{qstest}

\begin{qstest}{default}{default}
  \begingroup
    \newcommand{\TestA}[1][\relax default \empty]{TestA #1}%
    \SaveA
    \TestDef
    {TestA}{\@protected@testopt \TestA \\TestA {\relax default \empty }}%
    \LetLtxMacro\TestB\TestA
    \TestDef
    {TestB}{\@protected@testopt \TestB \\TestB {\relax default \empty }}%
    \CheckA
    \TestEquals{\string\TestB}{\string\TestA}%
    \GlobalLetLtxMacro\TestC\TestA
    \CheckA
  \endgroup
  \TestDef
  {TestC}{\@protected@testopt \TestC \\TestC {\relax default \empty }}%
  \newcommand{\TestA}[1][\relax default \empty]{TestA #1}%
  \TestEquals{\string\TestC}{\string\TestA}%
\end{qstest}

\begin{qstest}{robustdefault}{robustdefault}
  \begingroup
    \DeclareRobustCommand{\TestA}[1][\default]{TestA}%
    \SaveA
    \LetLtxMacro\TestB\TestA
    \TestDef{TestB}{\protect \TestB  }%
    \TestDef{TestB }{\@protected@testopt \TestB  \\TestB  {\default }}%
    \CheckA
    \GlobalLetLtxMacro\TestC\TestA
    \CheckA
  \endgroup
  \TestDef{TestC}{\protect \TestC  }%
  \TestDef{TestC }{\@protected@testopt \TestC  \\TestC  {\default }}%
\end{qstest}

\begin{qstest}{plain}{plain}
  \LetLtxMacro\NewRelax\relax
  \Expect*{\meaning\relax}*{\string\relax}%
  \Expect*{\meaning\NewRelax}*{\string\relax}%
  \LetLtxMacro\NewHbox\hbox
  \Expect*{\meaning\hbox}*{\string\hbox}%
  \Expect*{\meaning\NewHbox}*{\string\hbox}%
  \LetLtxMacro\NewEmpty\empty
  \Expect*{\meaning\empty}{macro:->}%
  \Expect*{\meaning\NewEmpty}{macro:->}%
  \def\TestA{\iffalse}%
  \LetLtxMacro\TestB\TestA
  \TestDef{TestB}{\iffalse }%
  \TestEquals{TestA}{TestB}%
\end{qstest}
%    \end{macrocode}
%
%    \begin{macrocode}
\usepackage{etoolbox}[2008/06/22]

\begin{qstest}{oldnewrobustcmd}{oldnewrobustcmd}
  \begingroup
    \protected\edef\TestA{%
      \noexpand\@protected@testopt
      \noexpand\TestA
      \expandafter\noexpand\csname\string\TestA\endcsname
      {\noexpand\default}%
    }%
    \expandafter\protected\expandafter
    \def\csname\string\TestA\endcsname{TestA}%
    \SaveA
    \LetLtxMacro\TestB\TestA
    \TestDef{TestB}{\@protected@testopt \TestB \\TestB {\default }}%
    \protected\edef\TestC{%
      \noexpand\@protected@testopt
      \noexpand\TestB
      \expandafter\noexpand\csname\string\TestB\endcsname
      {\noexpand\default}%
    }%
    \TestEquals{TestB}{TestC}%
    \CheckA
    \GlobalLetLtxMacro\TestD\TestA
    \CheckA
  \endgroup
  \TestDef{TestD}{\@protected@testopt \TestD \\TestD {\default }}%
  \protected\edef\TestC{%
    \noexpand\@protected@testopt
    \noexpand\TestD
    \expandafter\noexpand\csname\string\TestD\endcsname
    {\noexpand\default}%
  }%
  \TestEquals{TestD}{TestC}%
\end{qstest}

\begin{qstest}{newrobustcmd}{newrobustcmd}
  \begingroup
    \newrobustcmd{\TestA}[1][\default]{TestA}%
    \SaveA
    \LetLtxMacro\TestB\TestA
    \TestDef{TestB}{\@testopt \\TestB {\default }}%
    \protected\edef\TestC{%
      \noexpand\@testopt
      \expandafter\noexpand\csname\string\TestB\endcsname
      {\noexpand\default}%
    }%
    \TestEquals{TestB}{TestC}%
    \CheckA
    \GlobalLetLtxMacro\TestD\TestA
    \CheckA
  \endgroup
  \TestDef{TestD}{\@testopt \\TestD {\default }}%
  \protected\edef\TestC{%
    \noexpand\@testopt
    \expandafter\noexpand\csname\string\TestD\endcsname
    {\noexpand\default}%
  }%
  \TestEquals{TestD}{TestC}%
\end{qstest}

\begin{qstest}{robustifyopt}{robustifyopt}
  \newcommand{\TestA}[2][\default]{}%
  \TestDef{TestA}{\@protected@testopt \TestA \\TestA {\default }}%
  \robustify\TestA
  \TestDef{TestA}{\@protected@testopt \TestA \\TestA {\default }}%
  \protected\edef\TestC{%
    \noexpand\@protected@testopt
    \noexpand\TestA
    \expandafter\noexpand\csname\string\TestA\endcsname
    {\noexpand\default}%
  }%
  \TestEquals{TestA}{TestC}%
  \SaveA
  \LetLtxMacro\TestB\TestA
  \TestDef{TestB}{\@protected@testopt \TestB \\TestB {\default }}%
  \protected\edef\TestC{%
    \noexpand\@protected@testopt
    \noexpand\TestB
    \expandafter\noexpand\csname\string\TestB\endcsname
    {\noexpand\default}%
  }%
  \TestEquals{TestB}{TestC}%
  \CheckA
\end{qstest}

\begin{qstest}{robustifydeclare}{robustifydeclare}
  \DeclareRobustCommand\TestA{\iffalse}%
  \robustify\TestA
  \SaveA
  \LetLtxMacro\TestB\TestA
  \TestDef{TestB}{\iffalse }%
  % before etoolbox 2008/06/22: \protected\long\def
  % since etoolbox 2008/06/22: \protected\def
  \protected\def\TestC{\iffalse}%
  \TestEquals{TestB}{TestC}%
  \CheckA
\end{qstest}

\begin{qstest}{robustifydeclarelong}{robustifydeclarelong}
  \DeclareRobustCommand\TestA[1]{\iffalse}%
  \robustify\TestA
  \SaveA
  \LetLtxMacro\TestB\TestA
  \TestDef{TestB}{\iffalse }%
  % before etoolbox 2008/06/22: \protected\long\def
  % since etoolbox 2008/06/22: \protected\def
  \protected\long\def\TestC#1{\iffalse}%
  \TestEquals{TestB}{TestC}%
  \CheckA
\end{qstest}
%    \end{macrocode}
%
%    \begin{macrocode}
\csname @@end\endcsname
%    \end{macrocode}
%    \begin{macrocode}
%</test2>
%    \end{macrocode}
%
% \section{Installation}
%
% \subsection{Download}
%
% \paragraph{Package.} This package is available on
% CTAN\footnote{\url{ftp://ftp.ctan.org/tex-archive/}}:
% \begin{description}
% \item[\CTAN{macros/latex/contrib/oberdiek/letltxmacro.dtx}] The source file.
% \item[\CTAN{macros/latex/contrib/oberdiek/letltxmacro.pdf}] Documentation.
% \end{description}
%
%
% \paragraph{Bundle.} All the packages of the bundle `oberdiek'
% are also available in a TDS compliant ZIP archive. There
% the packages are already unpacked and the documentation files
% are generated. The files and directories obey the TDS standard.
% \begin{description}
% \item[\CTAN{install/macros/latex/contrib/oberdiek.tds.zip}]
% \end{description}
% \emph{TDS} refers to the standard ``A Directory Structure
% for \TeX\ Files'' (\CTAN{tds/tds.pdf}). Directories
% with \xfile{texmf} in their name are usually organized this way.
%
% \subsection{Bundle installation}
%
% \paragraph{Unpacking.} Unpack the \xfile{oberdiek.tds.zip} in the
% TDS tree (also known as \xfile{texmf} tree) of your choice.
% Example (linux):
% \begin{quote}
%   |unzip oberdiek.tds.zip -d ~/texmf|
% \end{quote}
%
% \paragraph{Script installation.}
% Check the directory \xfile{TDS:scripts/oberdiek/} for
% scripts that need further installation steps.
% Package \xpackage{attachfile2} comes with the Perl script
% \xfile{pdfatfi.pl} that should be installed in such a way
% that it can be called as \texttt{pdfatfi}.
% Example (linux):
% \begin{quote}
%   |chmod +x scripts/oberdiek/pdfatfi.pl|\\
%   |cp scripts/oberdiek/pdfatfi.pl /usr/local/bin/|
% \end{quote}
%
% \subsection{Package installation}
%
% \paragraph{Unpacking.} The \xfile{.dtx} file is a self-extracting
% \docstrip\ archive. The files are extracted by running the
% \xfile{.dtx} through \plainTeX:
% \begin{quote}
%   \verb|tex letltxmacro.dtx|
% \end{quote}
%
% \paragraph{TDS.} Now the different files must be moved into
% the different directories in your installation TDS tree
% (also known as \xfile{texmf} tree):
% \begin{quote}
% \def\t{^^A
% \begin{tabular}{@{}>{\ttfamily}l@{ $\rightarrow$ }>{\ttfamily}l@{}}
%   letltxmacro.sty & tex/latex/oberdiek/letltxmacro.sty\\
%   letltxmacro.pdf & doc/latex/oberdiek/letltxmacro.pdf\\
%   letltxmacro-showcases.tex & doc/latex/oberdiek/letltxmacro-showcases.tex\\
%   test/letltxmacro-test1.tex & doc/latex/oberdiek/test/letltxmacro-test1.tex\\
%   test/letltxmacro-test2.tex & doc/latex/oberdiek/test/letltxmacro-test2.tex\\
%   letltxmacro.dtx & source/latex/oberdiek/letltxmacro.dtx\\
% \end{tabular}^^A
% }^^A
% \sbox0{\t}^^A
% \ifdim\wd0>\linewidth
%   \begingroup
%     \advance\linewidth by\leftmargin
%     \advance\linewidth by\rightmargin
%   \edef\x{\endgroup
%     \def\noexpand\lw{\the\linewidth}^^A
%   }\x
%   \def\lwbox{^^A
%     \leavevmode
%     \hbox to \linewidth{^^A
%       \kern-\leftmargin\relax
%       \hss
%       \usebox0
%       \hss
%       \kern-\rightmargin\relax
%     }^^A
%   }^^A
%   \ifdim\wd0>\lw
%     \sbox0{\small\t}^^A
%     \ifdim\wd0>\linewidth
%       \ifdim\wd0>\lw
%         \sbox0{\footnotesize\t}^^A
%         \ifdim\wd0>\linewidth
%           \ifdim\wd0>\lw
%             \sbox0{\scriptsize\t}^^A
%             \ifdim\wd0>\linewidth
%               \ifdim\wd0>\lw
%                 \sbox0{\tiny\t}^^A
%                 \ifdim\wd0>\linewidth
%                   \lwbox
%                 \else
%                   \usebox0
%                 \fi
%               \else
%                 \lwbox
%               \fi
%             \else
%               \usebox0
%             \fi
%           \else
%             \lwbox
%           \fi
%         \else
%           \usebox0
%         \fi
%       \else
%         \lwbox
%       \fi
%     \else
%       \usebox0
%     \fi
%   \else
%     \lwbox
%   \fi
% \else
%   \usebox0
% \fi
% \end{quote}
% If you have a \xfile{docstrip.cfg} that configures and enables \docstrip's
% TDS installing feature, then some files can already be in the right
% place, see the documentation of \docstrip.
%
% \subsection{Refresh file name databases}
%
% If your \TeX~distribution
% (\teTeX, \mikTeX, \dots) relies on file name databases, you must refresh
% these. For example, \teTeX\ users run \verb|texhash| or
% \verb|mktexlsr|.
%
% \subsection{Some details for the interested}
%
% \paragraph{Attached source.}
%
% The PDF documentation on CTAN also includes the
% \xfile{.dtx} source file. It can be extracted by
% AcrobatReader 6 or higher. Another option is \textsf{pdftk},
% e.g. unpack the file into the current directory:
% \begin{quote}
%   \verb|pdftk letltxmacro.pdf unpack_files output .|
% \end{quote}
%
% \paragraph{Unpacking with \LaTeX.}
% The \xfile{.dtx} chooses its action depending on the format:
% \begin{description}
% \item[\plainTeX:] Run \docstrip\ and extract the files.
% \item[\LaTeX:] Generate the documentation.
% \end{description}
% If you insist on using \LaTeX\ for \docstrip\ (really,
% \docstrip\ does not need \LaTeX), then inform the autodetect routine
% about your intention:
% \begin{quote}
%   \verb|latex \let\install=y% \iffalse meta-comment
%
% File: letltxmacro.dtx
% Version: 2010/09/02 v1.4
% Info: Let assignment for LaTeX macros
%
% Copyright (C) 2008, 2010 by
%    Heiko Oberdiek <heiko.oberdiek at googlemail.com>
%
% This work may be distributed and/or modified under the
% conditions of the LaTeX Project Public License, either
% version 1.3c of this license or (at your option) any later
% version. This version of this license is in
%    http://www.latex-project.org/lppl/lppl-1-3c.txt
% and the latest version of this license is in
%    http://www.latex-project.org/lppl.txt
% and version 1.3 or later is part of all distributions of
% LaTeX version 2005/12/01 or later.
%
% This work has the LPPL maintenance status "maintained".
%
% This Current Maintainer of this work is Heiko Oberdiek.
%
% This work consists of the main source file letltxmacro.dtx
% and the derived files
%    letltxmacro.sty, letltxmacro.pdf, letltxmacro.ins, letltxmacro.drv,
%    letltxmacro-showcases.tex, letltxmacro-test1.tex,
%    letltxmacro-test2.tex.
%
% Distribution:
%    CTAN:macros/latex/contrib/oberdiek/letltxmacro.dtx
%    CTAN:macros/latex/contrib/oberdiek/letltxmacro.pdf
%
% Unpacking:
%    (a) If letltxmacro.ins is present:
%           tex letltxmacro.ins
%    (b) Without letltxmacro.ins:
%           tex letltxmacro.dtx
%    (c) If you insist on using LaTeX
%           latex \let\install=y% \iffalse meta-comment
%
% File: letltxmacro.dtx
% Version: 2010/09/02 v1.4
% Info: Let assignment for LaTeX macros
%
% Copyright (C) 2008, 2010 by
%    Heiko Oberdiek <heiko.oberdiek at googlemail.com>
%
% This work may be distributed and/or modified under the
% conditions of the LaTeX Project Public License, either
% version 1.3c of this license or (at your option) any later
% version. This version of this license is in
%    http://www.latex-project.org/lppl/lppl-1-3c.txt
% and the latest version of this license is in
%    http://www.latex-project.org/lppl.txt
% and version 1.3 or later is part of all distributions of
% LaTeX version 2005/12/01 or later.
%
% This work has the LPPL maintenance status "maintained".
%
% This Current Maintainer of this work is Heiko Oberdiek.
%
% This work consists of the main source file letltxmacro.dtx
% and the derived files
%    letltxmacro.sty, letltxmacro.pdf, letltxmacro.ins, letltxmacro.drv,
%    letltxmacro-showcases.tex, letltxmacro-test1.tex,
%    letltxmacro-test2.tex.
%
% Distribution:
%    CTAN:macros/latex/contrib/oberdiek/letltxmacro.dtx
%    CTAN:macros/latex/contrib/oberdiek/letltxmacro.pdf
%
% Unpacking:
%    (a) If letltxmacro.ins is present:
%           tex letltxmacro.ins
%    (b) Without letltxmacro.ins:
%           tex letltxmacro.dtx
%    (c) If you insist on using LaTeX
%           latex \let\install=y% \iffalse meta-comment
%
% File: letltxmacro.dtx
% Version: 2010/09/02 v1.4
% Info: Let assignment for LaTeX macros
%
% Copyright (C) 2008, 2010 by
%    Heiko Oberdiek <heiko.oberdiek at googlemail.com>
%
% This work may be distributed and/or modified under the
% conditions of the LaTeX Project Public License, either
% version 1.3c of this license or (at your option) any later
% version. This version of this license is in
%    http://www.latex-project.org/lppl/lppl-1-3c.txt
% and the latest version of this license is in
%    http://www.latex-project.org/lppl.txt
% and version 1.3 or later is part of all distributions of
% LaTeX version 2005/12/01 or later.
%
% This work has the LPPL maintenance status "maintained".
%
% This Current Maintainer of this work is Heiko Oberdiek.
%
% This work consists of the main source file letltxmacro.dtx
% and the derived files
%    letltxmacro.sty, letltxmacro.pdf, letltxmacro.ins, letltxmacro.drv,
%    letltxmacro-showcases.tex, letltxmacro-test1.tex,
%    letltxmacro-test2.tex.
%
% Distribution:
%    CTAN:macros/latex/contrib/oberdiek/letltxmacro.dtx
%    CTAN:macros/latex/contrib/oberdiek/letltxmacro.pdf
%
% Unpacking:
%    (a) If letltxmacro.ins is present:
%           tex letltxmacro.ins
%    (b) Without letltxmacro.ins:
%           tex letltxmacro.dtx
%    (c) If you insist on using LaTeX
%           latex \let\install=y\input{letltxmacro.dtx}
%        (quote the arguments according to the demands of your shell)
%
% Documentation:
%    (a) If letltxmacro.drv is present:
%           latex letltxmacro.drv
%    (b) Without letltxmacro.drv:
%           latex letltxmacro.dtx; ...
%    The class ltxdoc loads the configuration file ltxdoc.cfg
%    if available. Here you can specify further options, e.g.
%    use A4 as paper format:
%       \PassOptionsToClass{a4paper}{article}
%
%    Programm calls to get the documentation (example):
%       pdflatex letltxmacro.dtx
%       makeindex -s gind.ist letltxmacro.idx
%       pdflatex letltxmacro.dtx
%       makeindex -s gind.ist letltxmacro.idx
%       pdflatex letltxmacro.dtx
%
% Installation:
%    TDS:tex/latex/oberdiek/letltxmacro.sty
%    TDS:doc/latex/oberdiek/letltxmacro.pdf
%    TDS:doc/latex/oberdiek/letltxmacro-showcases.tex
%    TDS:doc/latex/oberdiek/test/letltxmacro-test1.tex
%    TDS:doc/latex/oberdiek/test/letltxmacro-test2.tex
%    TDS:source/latex/oberdiek/letltxmacro.dtx
%
%<*ignore>
\begingroup
  \catcode123=1 %
  \catcode125=2 %
  \def\x{LaTeX2e}%
\expandafter\endgroup
\ifcase 0\ifx\install y1\fi\expandafter
         \ifx\csname processbatchFile\endcsname\relax\else1\fi
         \ifx\fmtname\x\else 1\fi\relax
\else\csname fi\endcsname
%</ignore>
%<*install>
\input docstrip.tex
\Msg{************************************************************************}
\Msg{* Installation}
\Msg{* Package: letltxmacro 2010/09/02 v1.4 Let assignment for LaTeX macros (HO)}
\Msg{************************************************************************}

\keepsilent
\askforoverwritefalse

\let\MetaPrefix\relax
\preamble

This is a generated file.

Project: letltxmacro
Version: 2010/09/02 v1.4

Copyright (C) 2008, 2010 by
   Heiko Oberdiek <heiko.oberdiek at googlemail.com>

This work may be distributed and/or modified under the
conditions of the LaTeX Project Public License, either
version 1.3c of this license or (at your option) any later
version. This version of this license is in
   http://www.latex-project.org/lppl/lppl-1-3c.txt
and the latest version of this license is in
   http://www.latex-project.org/lppl.txt
and version 1.3 or later is part of all distributions of
LaTeX version 2005/12/01 or later.

This work has the LPPL maintenance status "maintained".

This Current Maintainer of this work is Heiko Oberdiek.

This work consists of the main source file letltxmacro.dtx
and the derived files
   letltxmacro.sty, letltxmacro.pdf, letltxmacro.ins, letltxmacro.drv,
   letltxmacro-showcases.tex, letltxmacro-test1.tex,
   letltxmacro-test2.tex.

\endpreamble
\let\MetaPrefix\DoubleperCent

\generate{%
  \file{letltxmacro.ins}{\from{letltxmacro.dtx}{install}}%
  \file{letltxmacro.drv}{\from{letltxmacro.dtx}{driver}}%
  \usedir{tex/latex/oberdiek}%
  \file{letltxmacro.sty}{\from{letltxmacro.dtx}{package}}%
  \usedir{doc/latex/oberdiek}%
  \file{letltxmacro-showcases.tex}{\from{letltxmacro.dtx}{showcases}}%
  \usedir{doc/latex/oberdiek/test}%
  \file{letltxmacro-test1.tex}{\from{letltxmacro.dtx}{test1}}%
  \file{letltxmacro-test2.tex}{\from{letltxmacro.dtx}{test2}}%
  \nopreamble
  \nopostamble
  \usedir{source/latex/oberdiek/catalogue}%
  \file{letltxmacro.xml}{\from{letltxmacro.dtx}{catalogue}}%
}

\catcode32=13\relax% active space
\let =\space%
\Msg{************************************************************************}
\Msg{*}
\Msg{* To finish the installation you have to move the following}
\Msg{* file into a directory searched by TeX:}
\Msg{*}
\Msg{*     letltxmacro.sty}
\Msg{*}
\Msg{* To produce the documentation run the file `letltxmacro.drv'}
\Msg{* through LaTeX.}
\Msg{*}
\Msg{* Happy TeXing!}
\Msg{*}
\Msg{************************************************************************}

\endbatchfile
%</install>
%<*ignore>
\fi
%</ignore>
%<*driver>
\NeedsTeXFormat{LaTeX2e}
\ProvidesFile{letltxmacro.drv}%
  [2010/09/02 v1.4 Let assignment for LaTeX macros (HO)]%
\documentclass{ltxdoc}
\usepackage{holtxdoc}[2011/11/22]
\begin{document}
  \DocInput{letltxmacro.dtx}%
\end{document}
%</driver>
% \fi
%
% \CheckSum{990}
%
% \CharacterTable
%  {Upper-case    \A\B\C\D\E\F\G\H\I\J\K\L\M\N\O\P\Q\R\S\T\U\V\W\X\Y\Z
%   Lower-case    \a\b\c\d\e\f\g\h\i\j\k\l\m\n\o\p\q\r\s\t\u\v\w\x\y\z
%   Digits        \0\1\2\3\4\5\6\7\8\9
%   Exclamation   \!     Double quote  \"     Hash (number) \#
%   Dollar        \$     Percent       \%     Ampersand     \&
%   Acute accent  \'     Left paren    \(     Right paren   \)
%   Asterisk      \*     Plus          \+     Comma         \,
%   Minus         \-     Point         \.     Solidus       \/
%   Colon         \:     Semicolon     \;     Less than     \<
%   Equals        \=     Greater than  \>     Question mark \?
%   Commercial at \@     Left bracket  \[     Backslash     \\
%   Right bracket \]     Circumflex    \^     Underscore    \_
%   Grave accent  \`     Left brace    \{     Vertical bar  \|
%   Right brace   \}     Tilde         \~}
%
% \GetFileInfo{letltxmacro.drv}
%
% \title{The \xpackage{letltxmacro} package}
% \date{2010/09/02 v1.4}
% \author{Heiko Oberdiek\\\xemail{heiko.oberdiek at googlemail.com}}
%
% \maketitle
%
% \begin{abstract}
% \TeX's \cs{let} assignment does not work for \LaTeX\ macros
% with optional arguments or for macros that are defined
% as robust macros by \cs{DeclareRobustCommand}. This package
% defines \cs{LetLtxMacro} that also takes care of the involved
% internal macros.
% \end{abstract}
%
% \tableofcontents
%
% \section{Documentation}
%
% If someone wants to redefine a macro with using the old
% meaning, then one method is \TeX's command \cs{let}:
%\begin{quote}
%\begin{verbatim}
%\newcommand{\Macro}{\typeout{Test Macro}}
%\let\SavedMacro=\Macro
%\renewcommand{\Macro}{%
%  \typeout{Begin}%
%  \SavedMacro
%  \typeout{End}%
%}
%\end{verbatim}
%\end{quote}
% However, this method fails, if \cs{Macro} is defined
% by \cs{DeclareRobustCommand} and/or has an optional argument.
% In both cases \LaTeX\ defines an additional internal macro
% that is forgotten in the simple \cs{let} assignment of
% the example above.
%
% \begin{declcs}{LetLtxMacro} \M{new macro} \M{old macro}
% \end{declcs}
% Macro \cs{LetLtxMacro} behaves similar to \TeX's \cs{let}
% assignment, but it takes care of macros that are
% defined by \cs{DeclareRobustCommand} and/or have optional
% arguments. Example:
%\begin{quote}
%\begin{verbatim}
%\DeclareRobustCommand{\Macro}[1][default]{...}
%\LetLtxMacro{\SavedMacro}{\Macro}
%\end{verbatim}
%\end{quote}
% Then macro \cs{SavedMacro} only uses internal macro names
% that are derived from \cs{SavedMacro}'s macro name. Macro \cs{Macro}
% can now be redefined without affecting \cs{SavedMacro}.
%
% \begin{declcs}{GlobalLetLtxMacro} \M{new macro} \M{old macro}
% \end{declcs}
% Like \cs{LetLtxMacro}, but the \meta{new macro} is defined globally.
% Since version 2010/09/02~v1.4.
%
% \subsection{Supported macro definition commands}
%
% \begin{quote}
%   \begin{tabular}{@{}ll@{}}
%     \cs{newcommand}, \cs{renewcommand} & latex/base\\
%     \cs{newenvironment}, \cs{renewenvironment} & latex/base\\
%     \cs{DeclareRobustCommand}& latex/base\\
%     \cs{newrobustcmd}, \cs{renewrobustcmd} & etoolbox\\
%     \cs{robustify} & etoolbox 2008/06/22 v1.6\\
%   \end{tabular}
% \end{quote}
%
% \StopEventually{
% }
%
% \section{Implementation}
%
% \subsection{Show cases}
%
% \subsubsection{\xfile{letltxmacro-showcases.tex}}
%
%    \begin{macrocode}
%<*showcases>
\NeedsTeXFormat{LaTeX2e}
\makeatletter
%    \end{macrocode}
%    \begin{macro}{\Line}
%    The result is displayed by macro \cs{Line}. The percent symbol
%    at line start allows easy grepping and inserting into the DTX
%    file.
%    \begin{macrocode}
\newcommand*{\Line}[1]{%
  \typeout{\@percentchar#1}%
}
%    \end{macrocode}
%    \end{macro}
%    \begin{macrocode}
\newcommand*{\ShowCmdName}[1]{%
  \@ifundefined{#1}{}{%
    \Line{%
      \space\space(\expandafter\string\csname#1\endcsname) = %
      (\expandafter\meaning\csname#1\endcsname)%
    }%
  }%
}
\newcommand*{\ShowCmds}[1]{%
  \ShowCmdName{#1}%
  \ShowCmdName{#1 }%
  \ShowCmdName{\\#1}%
  \ShowCmdName{\\#1 }%
}
\let\\\@backslashchar
%    \end{macrocode}
%    \begin{macro}{\ShowDef}
%    \begin{macrocode}
\newcommand*{\ShowDef}[2]{%
  \begingroup
    \Line{}%
    \newcommand*{\DefString}{#2}%
    \@onelevel@sanitize\DefString
    \Line{\DefString}%
    #2%
    \ShowCmds{#1}%
  \endgroup
}
%    \end{macrocode}
%    \end{macro}
%    \begin{macrocode}
\typeout{}
\Line{* LaTeX definitions:}
\ShowDef{cmd}{%
  \newcommand{\cmd}[2][default]{}%
}
\ShowDef{cmd}{%
  \DeclareRobustCommand{\cmd}{}%
}
\ShowDef{cmd}{%
  \DeclareRobustCommand{\cmd}[2][default]{}%
}
\typeout{}
%    \end{macrocode}
% The minimal version of package \xpackage{etoolbox} is 2008/06/12 v1.6a
% because it fixes \cs{robustify}.
%    \begin{macrocode}
\RequirePackage{etoolbox}[2008/06/12]%
\Line{}
\Line{* etoolbox's robust definitions:}
\ShowDef{cmd}{%
  \newrobustcmd{\cmd}{}%
}
\ShowDef{cmd}{%
  \newrobustcmd{\cmd}[2][default]{}%
}
\Line{}
\Line{* etoolbox's \string\robustify:}
\ShowDef{cmd}{%
  \newcommand{\cmd}[2][default]{} %
  \robustify{\cmd}%
}
\ShowDef{cmd}{%
  \DeclareRobustCommand{\cmd}{} %
  \robustify{\cmd}%
}
\ShowDef{cmd}{%
  \DeclareRobustCommand{\cmd}[2][default]{} %
  \robustify{\cmd}%
}
\typeout{}
\@@end
%</showcases>
%    \end{macrocode}
%
% \subsubsection{Result}
%
% \begingroup
%   \makeatletter
%   \let\org@verbatim\@verbatim
%   \def\@verbatim{^^A
%     \org@verbatim
%     \catcode`\~=\active
%   }^^A
%   \let~\textvisiblespace
%\begin{verbatim}
%* LaTeX definitions:
%
%\newcommand {\cmd }[2][default]{}
%  (\cmd) = (macro:->\@protected@testopt \cmd \\cmd {default})
%  (\\cmd) = (\long macro:[#1]#2->)
%
%\DeclareRobustCommand {\cmd }{}
%  (\cmd) = (macro:->\protect \cmd~ )
%  (\cmd~) = (\long macro:->)
%
%\DeclareRobustCommand {\cmd }[2][default]{}
%  (\cmd) = (macro:->\protect \cmd~ )
%  (\cmd~) = (macro:->\@protected@testopt \cmd~ \\cmd~ {default})
%  (\\cmd~) = (\long macro:[#1]#2->)
%
%* etoolbox's robust definitions:
%
%\newrobustcmd {\cmd }{}
%  (\cmd) = (\protected\long macro:->)
%
%\newrobustcmd {\cmd }[2][default]{}
%  (\cmd) = (\protected macro:->\@testopt \\cmd {default})
%  (\\cmd) = (\long macro:[#1]#2->)
%
%* etoolbox's \robustify:
%
%\newcommand {\cmd }[2][default]{} \robustify {\cmd }
%  (\cmd) = (\protected macro:->\@protected@testopt \cmd \\cmd {default})
%  (\\cmd) = (\long macro:[#1]#2->)
%
%\DeclareRobustCommand {\cmd }{} \robustify {\cmd }
%  (\cmd) = (\protected macro:->)
%
%\DeclareRobustCommand {\cmd }[2][default]{} \robustify {\cmd }
%  (\cmd) = (\protected macro:->\@protected@testopt \cmd~ \\cmd~ {default})
%  (\cmd~) = (macro:->\@protected@testopt \cmd~ \\cmd~ {default})
%  (\\cmd~) = (\long macro:[#1]#2->)
%\end{verbatim}
% \endgroup
%
% \subsection{Package}
%
%    \begin{macrocode}
%<*package>
%    \end{macrocode}
%
% \subsubsection{Catcodes and identification}
%
%    \begin{macrocode}
\begingroup\catcode61\catcode48\catcode32=10\relax%
  \catcode13=5 % ^^M
  \endlinechar=13 %
  \catcode123=1 % {
  \catcode125=2 % }
  \catcode64=11 % @
  \def\x{\endgroup
    \expandafter\edef\csname llm@AtEnd\endcsname{%
      \endlinechar=\the\endlinechar\relax
      \catcode13=\the\catcode13\relax
      \catcode32=\the\catcode32\relax
      \catcode35=\the\catcode35\relax
      \catcode61=\the\catcode61\relax
      \catcode64=\the\catcode64\relax
      \catcode123=\the\catcode123\relax
      \catcode125=\the\catcode125\relax
    }%
  }%
\x\catcode61\catcode48\catcode32=10\relax%
\catcode13=5 % ^^M
\endlinechar=13 %
\catcode35=6 % #
\catcode64=11 % @
\catcode123=1 % {
\catcode125=2 % }
\def\TMP@EnsureCode#1#2{%
  \edef\llm@AtEnd{%
    \llm@AtEnd
    \catcode#1=\the\catcode#1\relax
  }%
  \catcode#1=#2\relax
}
\TMP@EnsureCode{40}{12}% (
\TMP@EnsureCode{41}{12}% )
\TMP@EnsureCode{42}{12}% *
\TMP@EnsureCode{45}{12}% -
\TMP@EnsureCode{46}{12}% .
\TMP@EnsureCode{47}{12}% /
\TMP@EnsureCode{58}{12}% :
\TMP@EnsureCode{62}{12}% >
\TMP@EnsureCode{91}{12}% [
\TMP@EnsureCode{93}{12}% ]
\edef\llm@AtEnd{%
  \llm@AtEnd
  \escapechar\the\escapechar\relax
  \noexpand\endinput
}
\escapechar=92 % `\\
%    \end{macrocode}
%
%    Package identification.
%    \begin{macrocode}
\NeedsTeXFormat{LaTeX2e}
\ProvidesPackage{letltxmacro}%
  [2010/09/02 v1.4 Let assignment for LaTeX macros (HO)]
%    \end{macrocode}
%
% \subsubsection{Main macros}
%
%    \begin{macro}{\LetLtxMacro}
%    \begin{macrocode}
\newcommand*{\LetLtxMacro}{%
  \llm@ModeLetLtxMacro{}%
}
%    \end{macrocode}
%    \end{macro}
%    \begin{macro}{\GlobalLetLtxMacro}
%    \begin{macrocode}
\newcommand*{\GlobalLetLtxMacro}{%
  \llm@ModeLetLtxMacro\global
}
%    \end{macrocode}
%    \end{macro}
%
%    \begin{macro}{\llm@ModeLetLtxMacro}
%    \begin{macrocode}
\newcommand*{\llm@ModeLetLtxMacro}[3]{%
  \edef\llm@escapechar{\the\escapechar}%
  \escapechar=-1 %
  \edef\reserved@a{%
    \noexpand\protect
    \expandafter\noexpand
    \csname\string#3 \endcsname
  }%
  \ifx\reserved@a#3\relax
    #1\edef#2{%
      \noexpand\protect
      \expandafter\noexpand
      \csname\string#2 \endcsname
    }%
    #1\expandafter\let
    \csname\string#2 \expandafter\endcsname
    \csname\string#3 \endcsname
    \expandafter\llm@LetLtxMacro
        \csname\string#2 \expandafter\endcsname
        \csname\string#3 \endcsname{#1}%
  \else
    \llm@LetLtxMacro{#2}{#3}{#1}%
  \fi
  \escapechar=\llm@escapechar\relax
}
%    \end{macrocode}
%    \end{macro}
%    \begin{macro}{\llm@LetLtxMacro}
%    \begin{macrocode}
\def\llm@LetLtxMacro#1#2#3{%
  \escapechar=92 %
  \expandafter\llm@CheckParams\meaning#2:->\@nil{%
    \begingroup
      \def\@protected@testopt{%
        \expandafter\@testopt\@gobble
      }%
      \def\@testopt##1##2{%
        \toks@={##2}%
      }%
      \let\llm@testopt\@empty
      \edef\x{%
        \noexpand\@protected@testopt
        \noexpand#2%
        \expandafter\noexpand\csname\string#2\endcsname
      }%
      \expandafter\expandafter\expandafter\def
      \expandafter\expandafter\expandafter\y
      \expandafter\expandafter\expandafter{%
        \expandafter\llm@CarThree#2{}{}{}\llm@nil
      }%
      \ifx\x\y
        #2%
        \def\llm@testopt{%
          \noexpand\@protected@testopt
          \noexpand#1%
        }%
      \else
        \edef\x{%
          \noexpand\@testopt
          \expandafter\noexpand
          \csname\string#2\endcsname
        }%
        \expandafter\expandafter\expandafter\def
        \expandafter\expandafter\expandafter\y
        \expandafter\expandafter\expandafter{%
          \expandafter\llm@CarTwo#2{}{}\llm@nil
        }%
        \ifx\x\y
          #2%
          \def\llm@testopt{%
            \noexpand\@testopt
          }%
        \fi
      \fi
      \ifx\llm@testopt\@empty
      \else
        \llm@protected\xdef\llm@GlobalTemp{%
          \llm@testopt
          \expandafter\noexpand
          \csname\string#1\endcsname
          {\the\toks@}%
        }%
      \fi
    \expandafter\endgroup\ifx\llm@testopt\@empty
      #3\let#1=#2\relax
    \else
      #3\let#1=\llm@GlobalTemp
      #3\expandafter\let
          \csname\string#1\expandafter\endcsname
          \csname\string#2\endcsname
    \fi
  }{%
    #3\let#1=#2\relax
  }%
}
%    \end{macrocode}
%    \end{macro}
%    \begin{macro}{\llm@CheckParams}
%    \begin{macrocode}
\def\llm@CheckParams#1:->#2\@nil{%
  \begingroup
    \def\x{#1}%
  \ifx\x\llm@macro
    \endgroup
    \def\llm@protected{}%
    \expandafter\@firstoftwo
  \else
    \ifx\x\llm@protectedmacro
      \endgroup
      \def\llm@protected{\protected}%
      \expandafter\expandafter\expandafter\@firstoftwo
    \else
      \endgroup
      \expandafter\expandafter\expandafter\@secondoftwo
    \fi
  \fi
}
%    \end{macrocode}
%    \end{macro}
%    \begin{macro}{\llm@macro}
%    \begin{macrocode}
\def\llm@macro{macro}
\@onelevel@sanitize\llm@macro
%    \end{macrocode}
%    \end{macro}
%    \begin{macro}{\llm@protectedmacro}
%    \begin{macrocode}
\def\llm@protectedmacro{\protected macro}
\@onelevel@sanitize\llm@protectedmacro
%    \end{macrocode}
%    \end{macro}
%    \begin{macro}{\llm@CarThree}
%    \begin{macrocode}
\def\llm@CarThree#1#2#3#4\llm@nil{#1#2#3}%
%    \end{macrocode}
%    \end{macro}
%    \begin{macro}{\llm@CarTwo}
%    \begin{macrocode}
\def\llm@CarTwo#1#2#3\llm@nil{#1#2}%
%    \end{macrocode}
%    \end{macro}
%
%    \begin{macrocode}
\llm@AtEnd%
%</package>
%    \end{macrocode}
%
% \section{Test}
%
% \subsection{Catcode checks for loading}
%
%    \begin{macrocode}
%<*test1>
%    \end{macrocode}
%    \begin{macrocode}
\NeedsTeXFormat{LaTeX2e}
\documentclass{minimal}
\makeatletter
\def\RestoreCatcodes{}
\count@=0 %
\loop
  \edef\RestoreCatcodes{%
    \RestoreCatcodes
    \catcode\the\count@=\the\catcode\count@\relax
  }%
\ifnum\count@<255 %
  \advance\count@\@ne
\repeat

\def\RangeCatcodeInvalid#1#2{%
  \count@=#1\relax
  \loop
    \catcode\count@=15 %
  \ifnum\count@<#2\relax
    \advance\count@\@ne
  \repeat
}
\def\Test{%
  \RangeCatcodeInvalid{0}{47}%
  \RangeCatcodeInvalid{58}{64}%
  \RangeCatcodeInvalid{91}{96}%
  \RangeCatcodeInvalid{123}{127}%
  \catcode`\@=12 %
  \catcode`\\=0 %
  \catcode`\{=1 %
  \catcode`\}=2 %
  \catcode`\#=6 %
  \catcode`\[=12 %
  \catcode`\]=12 %
  \catcode`\%=14 %
  \catcode`\ =10 %
  \catcode13=5 %
  \RequirePackage{letltxmacro}[2010/09/02]\relax
  \RestoreCatcodes
}
\Test
\csname @@end\endcsname
\end
%    \end{macrocode}
%    \begin{macrocode}
%</test1>
%    \end{macrocode}
%
% \subsection{Package tests}
%
%    \begin{macrocode}
%<*test2>
%    \end{macrocode}
%    \begin{macrocode}
\NeedsTeXFormat{LaTeX2e}
\nofiles
\documentclass{minimal}

\usepackage{letltxmacro}[2010/09/02]

\usepackage{qstest}
\IncludeTests{*}
\LogTests{log}{*}{*}

\makeatletter

\def\TestDef#1{%
  \begingroup
    \@makeother\\%
    \@makeother\ %
    \expandafter\@TestDef\csname #1\endcsname
}
\def\@TestDef#1#2{%
    \edef\a{\expandafter\strip@prefix\meaning#1}%
    \edef\b{\detokenize{#2}}%
    \ifx\a\b
    \else
      \typeout{1 [\a]}%
      \typeout{2 [\b]}%
    \fi
  \endgroup
  \Expect*{\expandafter\strip@prefix\meaning#1}*{\detokenize{#2}}%
}
\def\TestEquals#1#2{%
  \Expect*{%
    \expandafter\ifx\csname#1\expandafter\endcsname
                    \csname#2\endcsname
      equals%
    \else
      wrong%
    \fi
  }{equals}%
}
\def\SaveA{%
  \let\SavedA\TestA
  \expandafter\let\csname\string\SavedA\expandafter\endcsname
                  \csname\string\TestA\endcsname
  \expandafter\let\csname SavedA \expandafter\endcsname
                  \csname TestA \endcsname
}
\def\CheckA{%
  \TestEquals{SavedA}{TestA}%
  \TestEquals{\string\SavedA}{\string\TestA}%
  \TestEquals{SavedA }{TestA }%
}

\begin{qstest}{robust}{robust}
  \begingroup
    \DeclareRobustCommand*{\TestA}{Test}%
    \SaveA
    \TestDef{TestA}{\protect \TestA  }%
    \CheckA
    \DeclareRobustCommand{\TestA}{Test}%
    \SaveA
    \TestDef{TestA}{\protect \TestA  }%
    \LetLtxMacro\TestB\TestA
    \TestDef{TestB}{\protect \TestB  }%
    \TestEquals{TestB }{TestA }%
    \CheckA
    \GlobalLetLtxMacro\TestC\TestA
    \CheckA
  \endgroup
  \TestDef{TestC}{\protect \TestC  }%
  \DeclareRobustCommand{\TestA}{Test}%
  \TestEquals{TestC }{TestA }%
\end{qstest}

\begin{qstest}{default}{default}
  \begingroup
    \newcommand{\TestA}[1][\relax default \empty]{TestA #1}%
    \SaveA
    \TestDef
    {TestA}{\@protected@testopt \TestA \\TestA {\relax default \empty }}%
    \LetLtxMacro\TestB\TestA
    \TestDef
    {TestB}{\@protected@testopt \TestB \\TestB {\relax default \empty }}%
    \CheckA
    \TestEquals{\string\TestB}{\string\TestA}%
    \GlobalLetLtxMacro\TestC\TestA
    \CheckA
  \endgroup
  \TestDef
  {TestC}{\@protected@testopt \TestC \\TestC {\relax default \empty }}%
  \newcommand{\TestA}[1][\relax default \empty]{TestA #1}%
  \TestEquals{\string\TestC}{\string\TestA}%
\end{qstest}

\begin{qstest}{robustdefault}{robustdefault}
  \begingroup
    \DeclareRobustCommand{\TestA}[1][\default]{TestA}%
    \SaveA
    \LetLtxMacro\TestB\TestA
    \TestDef{TestB}{\protect \TestB  }%
    \TestDef{TestB }{\@protected@testopt \TestB  \\TestB  {\default }}%
    \CheckA
    \GlobalLetLtxMacro\TestC\TestA
    \CheckA
  \endgroup
  \TestDef{TestC}{\protect \TestC  }%
  \TestDef{TestC }{\@protected@testopt \TestC  \\TestC  {\default }}%
\end{qstest}

\begin{qstest}{plain}{plain}
  \LetLtxMacro\NewRelax\relax
  \Expect*{\meaning\relax}*{\string\relax}%
  \Expect*{\meaning\NewRelax}*{\string\relax}%
  \LetLtxMacro\NewHbox\hbox
  \Expect*{\meaning\hbox}*{\string\hbox}%
  \Expect*{\meaning\NewHbox}*{\string\hbox}%
  \LetLtxMacro\NewEmpty\empty
  \Expect*{\meaning\empty}{macro:->}%
  \Expect*{\meaning\NewEmpty}{macro:->}%
  \def\TestA{\iffalse}%
  \LetLtxMacro\TestB\TestA
  \TestDef{TestB}{\iffalse }%
  \TestEquals{TestA}{TestB}%
\end{qstest}
%    \end{macrocode}
%
%    \begin{macrocode}
\usepackage{etoolbox}[2008/06/22]

\begin{qstest}{oldnewrobustcmd}{oldnewrobustcmd}
  \begingroup
    \protected\edef\TestA{%
      \noexpand\@protected@testopt
      \noexpand\TestA
      \expandafter\noexpand\csname\string\TestA\endcsname
      {\noexpand\default}%
    }%
    \expandafter\protected\expandafter
    \def\csname\string\TestA\endcsname{TestA}%
    \SaveA
    \LetLtxMacro\TestB\TestA
    \TestDef{TestB}{\@protected@testopt \TestB \\TestB {\default }}%
    \protected\edef\TestC{%
      \noexpand\@protected@testopt
      \noexpand\TestB
      \expandafter\noexpand\csname\string\TestB\endcsname
      {\noexpand\default}%
    }%
    \TestEquals{TestB}{TestC}%
    \CheckA
    \GlobalLetLtxMacro\TestD\TestA
    \CheckA
  \endgroup
  \TestDef{TestD}{\@protected@testopt \TestD \\TestD {\default }}%
  \protected\edef\TestC{%
    \noexpand\@protected@testopt
    \noexpand\TestD
    \expandafter\noexpand\csname\string\TestD\endcsname
    {\noexpand\default}%
  }%
  \TestEquals{TestD}{TestC}%
\end{qstest}

\begin{qstest}{newrobustcmd}{newrobustcmd}
  \begingroup
    \newrobustcmd{\TestA}[1][\default]{TestA}%
    \SaveA
    \LetLtxMacro\TestB\TestA
    \TestDef{TestB}{\@testopt \\TestB {\default }}%
    \protected\edef\TestC{%
      \noexpand\@testopt
      \expandafter\noexpand\csname\string\TestB\endcsname
      {\noexpand\default}%
    }%
    \TestEquals{TestB}{TestC}%
    \CheckA
    \GlobalLetLtxMacro\TestD\TestA
    \CheckA
  \endgroup
  \TestDef{TestD}{\@testopt \\TestD {\default }}%
  \protected\edef\TestC{%
    \noexpand\@testopt
    \expandafter\noexpand\csname\string\TestD\endcsname
    {\noexpand\default}%
  }%
  \TestEquals{TestD}{TestC}%
\end{qstest}

\begin{qstest}{robustifyopt}{robustifyopt}
  \newcommand{\TestA}[2][\default]{}%
  \TestDef{TestA}{\@protected@testopt \TestA \\TestA {\default }}%
  \robustify\TestA
  \TestDef{TestA}{\@protected@testopt \TestA \\TestA {\default }}%
  \protected\edef\TestC{%
    \noexpand\@protected@testopt
    \noexpand\TestA
    \expandafter\noexpand\csname\string\TestA\endcsname
    {\noexpand\default}%
  }%
  \TestEquals{TestA}{TestC}%
  \SaveA
  \LetLtxMacro\TestB\TestA
  \TestDef{TestB}{\@protected@testopt \TestB \\TestB {\default }}%
  \protected\edef\TestC{%
    \noexpand\@protected@testopt
    \noexpand\TestB
    \expandafter\noexpand\csname\string\TestB\endcsname
    {\noexpand\default}%
  }%
  \TestEquals{TestB}{TestC}%
  \CheckA
\end{qstest}

\begin{qstest}{robustifydeclare}{robustifydeclare}
  \DeclareRobustCommand\TestA{\iffalse}%
  \robustify\TestA
  \SaveA
  \LetLtxMacro\TestB\TestA
  \TestDef{TestB}{\iffalse }%
  % before etoolbox 2008/06/22: \protected\long\def
  % since etoolbox 2008/06/22: \protected\def
  \protected\def\TestC{\iffalse}%
  \TestEquals{TestB}{TestC}%
  \CheckA
\end{qstest}

\begin{qstest}{robustifydeclarelong}{robustifydeclarelong}
  \DeclareRobustCommand\TestA[1]{\iffalse}%
  \robustify\TestA
  \SaveA
  \LetLtxMacro\TestB\TestA
  \TestDef{TestB}{\iffalse }%
  % before etoolbox 2008/06/22: \protected\long\def
  % since etoolbox 2008/06/22: \protected\def
  \protected\long\def\TestC#1{\iffalse}%
  \TestEquals{TestB}{TestC}%
  \CheckA
\end{qstest}
%    \end{macrocode}
%
%    \begin{macrocode}
\csname @@end\endcsname
%    \end{macrocode}
%    \begin{macrocode}
%</test2>
%    \end{macrocode}
%
% \section{Installation}
%
% \subsection{Download}
%
% \paragraph{Package.} This package is available on
% CTAN\footnote{\url{ftp://ftp.ctan.org/tex-archive/}}:
% \begin{description}
% \item[\CTAN{macros/latex/contrib/oberdiek/letltxmacro.dtx}] The source file.
% \item[\CTAN{macros/latex/contrib/oberdiek/letltxmacro.pdf}] Documentation.
% \end{description}
%
%
% \paragraph{Bundle.} All the packages of the bundle `oberdiek'
% are also available in a TDS compliant ZIP archive. There
% the packages are already unpacked and the documentation files
% are generated. The files and directories obey the TDS standard.
% \begin{description}
% \item[\CTAN{install/macros/latex/contrib/oberdiek.tds.zip}]
% \end{description}
% \emph{TDS} refers to the standard ``A Directory Structure
% for \TeX\ Files'' (\CTAN{tds/tds.pdf}). Directories
% with \xfile{texmf} in their name are usually organized this way.
%
% \subsection{Bundle installation}
%
% \paragraph{Unpacking.} Unpack the \xfile{oberdiek.tds.zip} in the
% TDS tree (also known as \xfile{texmf} tree) of your choice.
% Example (linux):
% \begin{quote}
%   |unzip oberdiek.tds.zip -d ~/texmf|
% \end{quote}
%
% \paragraph{Script installation.}
% Check the directory \xfile{TDS:scripts/oberdiek/} for
% scripts that need further installation steps.
% Package \xpackage{attachfile2} comes with the Perl script
% \xfile{pdfatfi.pl} that should be installed in such a way
% that it can be called as \texttt{pdfatfi}.
% Example (linux):
% \begin{quote}
%   |chmod +x scripts/oberdiek/pdfatfi.pl|\\
%   |cp scripts/oberdiek/pdfatfi.pl /usr/local/bin/|
% \end{quote}
%
% \subsection{Package installation}
%
% \paragraph{Unpacking.} The \xfile{.dtx} file is a self-extracting
% \docstrip\ archive. The files are extracted by running the
% \xfile{.dtx} through \plainTeX:
% \begin{quote}
%   \verb|tex letltxmacro.dtx|
% \end{quote}
%
% \paragraph{TDS.} Now the different files must be moved into
% the different directories in your installation TDS tree
% (also known as \xfile{texmf} tree):
% \begin{quote}
% \def\t{^^A
% \begin{tabular}{@{}>{\ttfamily}l@{ $\rightarrow$ }>{\ttfamily}l@{}}
%   letltxmacro.sty & tex/latex/oberdiek/letltxmacro.sty\\
%   letltxmacro.pdf & doc/latex/oberdiek/letltxmacro.pdf\\
%   letltxmacro-showcases.tex & doc/latex/oberdiek/letltxmacro-showcases.tex\\
%   test/letltxmacro-test1.tex & doc/latex/oberdiek/test/letltxmacro-test1.tex\\
%   test/letltxmacro-test2.tex & doc/latex/oberdiek/test/letltxmacro-test2.tex\\
%   letltxmacro.dtx & source/latex/oberdiek/letltxmacro.dtx\\
% \end{tabular}^^A
% }^^A
% \sbox0{\t}^^A
% \ifdim\wd0>\linewidth
%   \begingroup
%     \advance\linewidth by\leftmargin
%     \advance\linewidth by\rightmargin
%   \edef\x{\endgroup
%     \def\noexpand\lw{\the\linewidth}^^A
%   }\x
%   \def\lwbox{^^A
%     \leavevmode
%     \hbox to \linewidth{^^A
%       \kern-\leftmargin\relax
%       \hss
%       \usebox0
%       \hss
%       \kern-\rightmargin\relax
%     }^^A
%   }^^A
%   \ifdim\wd0>\lw
%     \sbox0{\small\t}^^A
%     \ifdim\wd0>\linewidth
%       \ifdim\wd0>\lw
%         \sbox0{\footnotesize\t}^^A
%         \ifdim\wd0>\linewidth
%           \ifdim\wd0>\lw
%             \sbox0{\scriptsize\t}^^A
%             \ifdim\wd0>\linewidth
%               \ifdim\wd0>\lw
%                 \sbox0{\tiny\t}^^A
%                 \ifdim\wd0>\linewidth
%                   \lwbox
%                 \else
%                   \usebox0
%                 \fi
%               \else
%                 \lwbox
%               \fi
%             \else
%               \usebox0
%             \fi
%           \else
%             \lwbox
%           \fi
%         \else
%           \usebox0
%         \fi
%       \else
%         \lwbox
%       \fi
%     \else
%       \usebox0
%     \fi
%   \else
%     \lwbox
%   \fi
% \else
%   \usebox0
% \fi
% \end{quote}
% If you have a \xfile{docstrip.cfg} that configures and enables \docstrip's
% TDS installing feature, then some files can already be in the right
% place, see the documentation of \docstrip.
%
% \subsection{Refresh file name databases}
%
% If your \TeX~distribution
% (\teTeX, \mikTeX, \dots) relies on file name databases, you must refresh
% these. For example, \teTeX\ users run \verb|texhash| or
% \verb|mktexlsr|.
%
% \subsection{Some details for the interested}
%
% \paragraph{Attached source.}
%
% The PDF documentation on CTAN also includes the
% \xfile{.dtx} source file. It can be extracted by
% AcrobatReader 6 or higher. Another option is \textsf{pdftk},
% e.g. unpack the file into the current directory:
% \begin{quote}
%   \verb|pdftk letltxmacro.pdf unpack_files output .|
% \end{quote}
%
% \paragraph{Unpacking with \LaTeX.}
% The \xfile{.dtx} chooses its action depending on the format:
% \begin{description}
% \item[\plainTeX:] Run \docstrip\ and extract the files.
% \item[\LaTeX:] Generate the documentation.
% \end{description}
% If you insist on using \LaTeX\ for \docstrip\ (really,
% \docstrip\ does not need \LaTeX), then inform the autodetect routine
% about your intention:
% \begin{quote}
%   \verb|latex \let\install=y\input{letltxmacro.dtx}|
% \end{quote}
% Do not forget to quote the argument according to the demands
% of your shell.
%
% \paragraph{Generating the documentation.}
% You can use both the \xfile{.dtx} or the \xfile{.drv} to generate
% the documentation. The process can be configured by the
% configuration file \xfile{ltxdoc.cfg}. For instance, put this
% line into this file, if you want to have A4 as paper format:
% \begin{quote}
%   \verb|\PassOptionsToClass{a4paper}{article}|
% \end{quote}
% An example follows how to generate the
% documentation with pdf\LaTeX:
% \begin{quote}
%\begin{verbatim}
%pdflatex letltxmacro.dtx
%makeindex -s gind.ist letltxmacro.idx
%pdflatex letltxmacro.dtx
%makeindex -s gind.ist letltxmacro.idx
%pdflatex letltxmacro.dtx
%\end{verbatim}
% \end{quote}
%
% \section{Catalogue}
%
% The following XML file can be used as source for the
% \href{http://mirror.ctan.org/help/Catalogue/catalogue.html}{\TeX\ Catalogue}.
% The elements \texttt{caption} and \texttt{description} are imported
% from the original XML file from the Catalogue.
% The name of the XML file in the Catalogue is \xfile{letltxmacro.xml}.
%    \begin{macrocode}
%<*catalogue>
<?xml version='1.0' encoding='us-ascii'?>
<!DOCTYPE entry SYSTEM 'catalogue.dtd'>
<entry datestamp='$Date$' modifier='$Author$' id='letltxmacro'>
  <name>letltxmacro</name>
  <caption>Let assignment for LaTeX macros.</caption>
  <authorref id='auth:oberdiek'/>
  <copyright owner='Heiko Oberdiek' year='2008,2010'/>
  <license type='lppl1.3'/>
  <version number='1.4'/>
  <description>
    TeX&#x2019;s <tt>\let</tt> assignment does not work for LaTeX macros
    with optional arguments or for macros that are defined
    as robust macros by <tt>\DeclareRobustCommand</tt>.  This package
    defines <tt>\LetLtxMacro</tt> that also takes care of the involved
    internal macros.
    <p/>
    The package is part of the <xref refid='oberdiek'>oberdiek</xref>
    bundle.
  </description>
  <documentation details='Package documentation'
      href='ctan:/macros/latex/contrib/oberdiek/letltxmacro.pdf'/>
  <ctan file='true' path='/macros/latex/contrib/oberdiek/letltxmacro.dtx'/>
  <miktex location='oberdiek'/>
  <texlive location='oberdiek'/>
  <install path='/macros/latex/contrib/oberdiek/oberdiek.tds.zip'/>
</entry>
%</catalogue>
%    \end{macrocode}
%
% \begin{History}
%   \begin{Version}{2008/06/09 v1.0}
%   \item
%     First version.
%   \end{Version}
%   \begin{Version}{2008/06/12 v1.1}
%   \item
%     Support for \xpackage{etoolbox}'s \cs{newrobustcmd} added.
%   \end{Version}
%   \begin{Version}{2008/06/13 v1.2}
%   \item
%     Support for \xpackage{etoolbox}'s \cs{robustify} added.
%   \end{Version}
%   \begin{Version}{2008/06/24 v1.3}
%   \item
%     Test file adapted for etoolbox 2008/06/22 v1.6.
%   \end{Version}
%   \begin{Version}{2010/09/02 v1.4}
%   \item
%     \cs{GlobalLetLtxMacro} added.
%   \end{Version}
% \end{History}
%
% \PrintIndex
%
% \Finale
\endinput

%        (quote the arguments according to the demands of your shell)
%
% Documentation:
%    (a) If letltxmacro.drv is present:
%           latex letltxmacro.drv
%    (b) Without letltxmacro.drv:
%           latex letltxmacro.dtx; ...
%    The class ltxdoc loads the configuration file ltxdoc.cfg
%    if available. Here you can specify further options, e.g.
%    use A4 as paper format:
%       \PassOptionsToClass{a4paper}{article}
%
%    Programm calls to get the documentation (example):
%       pdflatex letltxmacro.dtx
%       makeindex -s gind.ist letltxmacro.idx
%       pdflatex letltxmacro.dtx
%       makeindex -s gind.ist letltxmacro.idx
%       pdflatex letltxmacro.dtx
%
% Installation:
%    TDS:tex/latex/oberdiek/letltxmacro.sty
%    TDS:doc/latex/oberdiek/letltxmacro.pdf
%    TDS:doc/latex/oberdiek/letltxmacro-showcases.tex
%    TDS:doc/latex/oberdiek/test/letltxmacro-test1.tex
%    TDS:doc/latex/oberdiek/test/letltxmacro-test2.tex
%    TDS:source/latex/oberdiek/letltxmacro.dtx
%
%<*ignore>
\begingroup
  \catcode123=1 %
  \catcode125=2 %
  \def\x{LaTeX2e}%
\expandafter\endgroup
\ifcase 0\ifx\install y1\fi\expandafter
         \ifx\csname processbatchFile\endcsname\relax\else1\fi
         \ifx\fmtname\x\else 1\fi\relax
\else\csname fi\endcsname
%</ignore>
%<*install>
\input docstrip.tex
\Msg{************************************************************************}
\Msg{* Installation}
\Msg{* Package: letltxmacro 2010/09/02 v1.4 Let assignment for LaTeX macros (HO)}
\Msg{************************************************************************}

\keepsilent
\askforoverwritefalse

\let\MetaPrefix\relax
\preamble

This is a generated file.

Project: letltxmacro
Version: 2010/09/02 v1.4

Copyright (C) 2008, 2010 by
   Heiko Oberdiek <heiko.oberdiek at googlemail.com>

This work may be distributed and/or modified under the
conditions of the LaTeX Project Public License, either
version 1.3c of this license or (at your option) any later
version. This version of this license is in
   http://www.latex-project.org/lppl/lppl-1-3c.txt
and the latest version of this license is in
   http://www.latex-project.org/lppl.txt
and version 1.3 or later is part of all distributions of
LaTeX version 2005/12/01 or later.

This work has the LPPL maintenance status "maintained".

This Current Maintainer of this work is Heiko Oberdiek.

This work consists of the main source file letltxmacro.dtx
and the derived files
   letltxmacro.sty, letltxmacro.pdf, letltxmacro.ins, letltxmacro.drv,
   letltxmacro-showcases.tex, letltxmacro-test1.tex,
   letltxmacro-test2.tex.

\endpreamble
\let\MetaPrefix\DoubleperCent

\generate{%
  \file{letltxmacro.ins}{\from{letltxmacro.dtx}{install}}%
  \file{letltxmacro.drv}{\from{letltxmacro.dtx}{driver}}%
  \usedir{tex/latex/oberdiek}%
  \file{letltxmacro.sty}{\from{letltxmacro.dtx}{package}}%
  \usedir{doc/latex/oberdiek}%
  \file{letltxmacro-showcases.tex}{\from{letltxmacro.dtx}{showcases}}%
  \usedir{doc/latex/oberdiek/test}%
  \file{letltxmacro-test1.tex}{\from{letltxmacro.dtx}{test1}}%
  \file{letltxmacro-test2.tex}{\from{letltxmacro.dtx}{test2}}%
  \nopreamble
  \nopostamble
  \usedir{source/latex/oberdiek/catalogue}%
  \file{letltxmacro.xml}{\from{letltxmacro.dtx}{catalogue}}%
}

\catcode32=13\relax% active space
\let =\space%
\Msg{************************************************************************}
\Msg{*}
\Msg{* To finish the installation you have to move the following}
\Msg{* file into a directory searched by TeX:}
\Msg{*}
\Msg{*     letltxmacro.sty}
\Msg{*}
\Msg{* To produce the documentation run the file `letltxmacro.drv'}
\Msg{* through LaTeX.}
\Msg{*}
\Msg{* Happy TeXing!}
\Msg{*}
\Msg{************************************************************************}

\endbatchfile
%</install>
%<*ignore>
\fi
%</ignore>
%<*driver>
\NeedsTeXFormat{LaTeX2e}
\ProvidesFile{letltxmacro.drv}%
  [2010/09/02 v1.4 Let assignment for LaTeX macros (HO)]%
\documentclass{ltxdoc}
\usepackage{holtxdoc}[2011/11/22]
\begin{document}
  \DocInput{letltxmacro.dtx}%
\end{document}
%</driver>
% \fi
%
% \CheckSum{990}
%
% \CharacterTable
%  {Upper-case    \A\B\C\D\E\F\G\H\I\J\K\L\M\N\O\P\Q\R\S\T\U\V\W\X\Y\Z
%   Lower-case    \a\b\c\d\e\f\g\h\i\j\k\l\m\n\o\p\q\r\s\t\u\v\w\x\y\z
%   Digits        \0\1\2\3\4\5\6\7\8\9
%   Exclamation   \!     Double quote  \"     Hash (number) \#
%   Dollar        \$     Percent       \%     Ampersand     \&
%   Acute accent  \'     Left paren    \(     Right paren   \)
%   Asterisk      \*     Plus          \+     Comma         \,
%   Minus         \-     Point         \.     Solidus       \/
%   Colon         \:     Semicolon     \;     Less than     \<
%   Equals        \=     Greater than  \>     Question mark \?
%   Commercial at \@     Left bracket  \[     Backslash     \\
%   Right bracket \]     Circumflex    \^     Underscore    \_
%   Grave accent  \`     Left brace    \{     Vertical bar  \|
%   Right brace   \}     Tilde         \~}
%
% \GetFileInfo{letltxmacro.drv}
%
% \title{The \xpackage{letltxmacro} package}
% \date{2010/09/02 v1.4}
% \author{Heiko Oberdiek\\\xemail{heiko.oberdiek at googlemail.com}}
%
% \maketitle
%
% \begin{abstract}
% \TeX's \cs{let} assignment does not work for \LaTeX\ macros
% with optional arguments or for macros that are defined
% as robust macros by \cs{DeclareRobustCommand}. This package
% defines \cs{LetLtxMacro} that also takes care of the involved
% internal macros.
% \end{abstract}
%
% \tableofcontents
%
% \section{Documentation}
%
% If someone wants to redefine a macro with using the old
% meaning, then one method is \TeX's command \cs{let}:
%\begin{quote}
%\begin{verbatim}
%\newcommand{\Macro}{\typeout{Test Macro}}
%\let\SavedMacro=\Macro
%\renewcommand{\Macro}{%
%  \typeout{Begin}%
%  \SavedMacro
%  \typeout{End}%
%}
%\end{verbatim}
%\end{quote}
% However, this method fails, if \cs{Macro} is defined
% by \cs{DeclareRobustCommand} and/or has an optional argument.
% In both cases \LaTeX\ defines an additional internal macro
% that is forgotten in the simple \cs{let} assignment of
% the example above.
%
% \begin{declcs}{LetLtxMacro} \M{new macro} \M{old macro}
% \end{declcs}
% Macro \cs{LetLtxMacro} behaves similar to \TeX's \cs{let}
% assignment, but it takes care of macros that are
% defined by \cs{DeclareRobustCommand} and/or have optional
% arguments. Example:
%\begin{quote}
%\begin{verbatim}
%\DeclareRobustCommand{\Macro}[1][default]{...}
%\LetLtxMacro{\SavedMacro}{\Macro}
%\end{verbatim}
%\end{quote}
% Then macro \cs{SavedMacro} only uses internal macro names
% that are derived from \cs{SavedMacro}'s macro name. Macro \cs{Macro}
% can now be redefined without affecting \cs{SavedMacro}.
%
% \begin{declcs}{GlobalLetLtxMacro} \M{new macro} \M{old macro}
% \end{declcs}
% Like \cs{LetLtxMacro}, but the \meta{new macro} is defined globally.
% Since version 2010/09/02~v1.4.
%
% \subsection{Supported macro definition commands}
%
% \begin{quote}
%   \begin{tabular}{@{}ll@{}}
%     \cs{newcommand}, \cs{renewcommand} & latex/base\\
%     \cs{newenvironment}, \cs{renewenvironment} & latex/base\\
%     \cs{DeclareRobustCommand}& latex/base\\
%     \cs{newrobustcmd}, \cs{renewrobustcmd} & etoolbox\\
%     \cs{robustify} & etoolbox 2008/06/22 v1.6\\
%   \end{tabular}
% \end{quote}
%
% \StopEventually{
% }
%
% \section{Implementation}
%
% \subsection{Show cases}
%
% \subsubsection{\xfile{letltxmacro-showcases.tex}}
%
%    \begin{macrocode}
%<*showcases>
\NeedsTeXFormat{LaTeX2e}
\makeatletter
%    \end{macrocode}
%    \begin{macro}{\Line}
%    The result is displayed by macro \cs{Line}. The percent symbol
%    at line start allows easy grepping and inserting into the DTX
%    file.
%    \begin{macrocode}
\newcommand*{\Line}[1]{%
  \typeout{\@percentchar#1}%
}
%    \end{macrocode}
%    \end{macro}
%    \begin{macrocode}
\newcommand*{\ShowCmdName}[1]{%
  \@ifundefined{#1}{}{%
    \Line{%
      \space\space(\expandafter\string\csname#1\endcsname) = %
      (\expandafter\meaning\csname#1\endcsname)%
    }%
  }%
}
\newcommand*{\ShowCmds}[1]{%
  \ShowCmdName{#1}%
  \ShowCmdName{#1 }%
  \ShowCmdName{\\#1}%
  \ShowCmdName{\\#1 }%
}
\let\\\@backslashchar
%    \end{macrocode}
%    \begin{macro}{\ShowDef}
%    \begin{macrocode}
\newcommand*{\ShowDef}[2]{%
  \begingroup
    \Line{}%
    \newcommand*{\DefString}{#2}%
    \@onelevel@sanitize\DefString
    \Line{\DefString}%
    #2%
    \ShowCmds{#1}%
  \endgroup
}
%    \end{macrocode}
%    \end{macro}
%    \begin{macrocode}
\typeout{}
\Line{* LaTeX definitions:}
\ShowDef{cmd}{%
  \newcommand{\cmd}[2][default]{}%
}
\ShowDef{cmd}{%
  \DeclareRobustCommand{\cmd}{}%
}
\ShowDef{cmd}{%
  \DeclareRobustCommand{\cmd}[2][default]{}%
}
\typeout{}
%    \end{macrocode}
% The minimal version of package \xpackage{etoolbox} is 2008/06/12 v1.6a
% because it fixes \cs{robustify}.
%    \begin{macrocode}
\RequirePackage{etoolbox}[2008/06/12]%
\Line{}
\Line{* etoolbox's robust definitions:}
\ShowDef{cmd}{%
  \newrobustcmd{\cmd}{}%
}
\ShowDef{cmd}{%
  \newrobustcmd{\cmd}[2][default]{}%
}
\Line{}
\Line{* etoolbox's \string\robustify:}
\ShowDef{cmd}{%
  \newcommand{\cmd}[2][default]{} %
  \robustify{\cmd}%
}
\ShowDef{cmd}{%
  \DeclareRobustCommand{\cmd}{} %
  \robustify{\cmd}%
}
\ShowDef{cmd}{%
  \DeclareRobustCommand{\cmd}[2][default]{} %
  \robustify{\cmd}%
}
\typeout{}
\@@end
%</showcases>
%    \end{macrocode}
%
% \subsubsection{Result}
%
% \begingroup
%   \makeatletter
%   \let\org@verbatim\@verbatim
%   \def\@verbatim{^^A
%     \org@verbatim
%     \catcode`\~=\active
%   }^^A
%   \let~\textvisiblespace
%\begin{verbatim}
%* LaTeX definitions:
%
%\newcommand {\cmd }[2][default]{}
%  (\cmd) = (macro:->\@protected@testopt \cmd \\cmd {default})
%  (\\cmd) = (\long macro:[#1]#2->)
%
%\DeclareRobustCommand {\cmd }{}
%  (\cmd) = (macro:->\protect \cmd~ )
%  (\cmd~) = (\long macro:->)
%
%\DeclareRobustCommand {\cmd }[2][default]{}
%  (\cmd) = (macro:->\protect \cmd~ )
%  (\cmd~) = (macro:->\@protected@testopt \cmd~ \\cmd~ {default})
%  (\\cmd~) = (\long macro:[#1]#2->)
%
%* etoolbox's robust definitions:
%
%\newrobustcmd {\cmd }{}
%  (\cmd) = (\protected\long macro:->)
%
%\newrobustcmd {\cmd }[2][default]{}
%  (\cmd) = (\protected macro:->\@testopt \\cmd {default})
%  (\\cmd) = (\long macro:[#1]#2->)
%
%* etoolbox's \robustify:
%
%\newcommand {\cmd }[2][default]{} \robustify {\cmd }
%  (\cmd) = (\protected macro:->\@protected@testopt \cmd \\cmd {default})
%  (\\cmd) = (\long macro:[#1]#2->)
%
%\DeclareRobustCommand {\cmd }{} \robustify {\cmd }
%  (\cmd) = (\protected macro:->)
%
%\DeclareRobustCommand {\cmd }[2][default]{} \robustify {\cmd }
%  (\cmd) = (\protected macro:->\@protected@testopt \cmd~ \\cmd~ {default})
%  (\cmd~) = (macro:->\@protected@testopt \cmd~ \\cmd~ {default})
%  (\\cmd~) = (\long macro:[#1]#2->)
%\end{verbatim}
% \endgroup
%
% \subsection{Package}
%
%    \begin{macrocode}
%<*package>
%    \end{macrocode}
%
% \subsubsection{Catcodes and identification}
%
%    \begin{macrocode}
\begingroup\catcode61\catcode48\catcode32=10\relax%
  \catcode13=5 % ^^M
  \endlinechar=13 %
  \catcode123=1 % {
  \catcode125=2 % }
  \catcode64=11 % @
  \def\x{\endgroup
    \expandafter\edef\csname llm@AtEnd\endcsname{%
      \endlinechar=\the\endlinechar\relax
      \catcode13=\the\catcode13\relax
      \catcode32=\the\catcode32\relax
      \catcode35=\the\catcode35\relax
      \catcode61=\the\catcode61\relax
      \catcode64=\the\catcode64\relax
      \catcode123=\the\catcode123\relax
      \catcode125=\the\catcode125\relax
    }%
  }%
\x\catcode61\catcode48\catcode32=10\relax%
\catcode13=5 % ^^M
\endlinechar=13 %
\catcode35=6 % #
\catcode64=11 % @
\catcode123=1 % {
\catcode125=2 % }
\def\TMP@EnsureCode#1#2{%
  \edef\llm@AtEnd{%
    \llm@AtEnd
    \catcode#1=\the\catcode#1\relax
  }%
  \catcode#1=#2\relax
}
\TMP@EnsureCode{40}{12}% (
\TMP@EnsureCode{41}{12}% )
\TMP@EnsureCode{42}{12}% *
\TMP@EnsureCode{45}{12}% -
\TMP@EnsureCode{46}{12}% .
\TMP@EnsureCode{47}{12}% /
\TMP@EnsureCode{58}{12}% :
\TMP@EnsureCode{62}{12}% >
\TMP@EnsureCode{91}{12}% [
\TMP@EnsureCode{93}{12}% ]
\edef\llm@AtEnd{%
  \llm@AtEnd
  \escapechar\the\escapechar\relax
  \noexpand\endinput
}
\escapechar=92 % `\\
%    \end{macrocode}
%
%    Package identification.
%    \begin{macrocode}
\NeedsTeXFormat{LaTeX2e}
\ProvidesPackage{letltxmacro}%
  [2010/09/02 v1.4 Let assignment for LaTeX macros (HO)]
%    \end{macrocode}
%
% \subsubsection{Main macros}
%
%    \begin{macro}{\LetLtxMacro}
%    \begin{macrocode}
\newcommand*{\LetLtxMacro}{%
  \llm@ModeLetLtxMacro{}%
}
%    \end{macrocode}
%    \end{macro}
%    \begin{macro}{\GlobalLetLtxMacro}
%    \begin{macrocode}
\newcommand*{\GlobalLetLtxMacro}{%
  \llm@ModeLetLtxMacro\global
}
%    \end{macrocode}
%    \end{macro}
%
%    \begin{macro}{\llm@ModeLetLtxMacro}
%    \begin{macrocode}
\newcommand*{\llm@ModeLetLtxMacro}[3]{%
  \edef\llm@escapechar{\the\escapechar}%
  \escapechar=-1 %
  \edef\reserved@a{%
    \noexpand\protect
    \expandafter\noexpand
    \csname\string#3 \endcsname
  }%
  \ifx\reserved@a#3\relax
    #1\edef#2{%
      \noexpand\protect
      \expandafter\noexpand
      \csname\string#2 \endcsname
    }%
    #1\expandafter\let
    \csname\string#2 \expandafter\endcsname
    \csname\string#3 \endcsname
    \expandafter\llm@LetLtxMacro
        \csname\string#2 \expandafter\endcsname
        \csname\string#3 \endcsname{#1}%
  \else
    \llm@LetLtxMacro{#2}{#3}{#1}%
  \fi
  \escapechar=\llm@escapechar\relax
}
%    \end{macrocode}
%    \end{macro}
%    \begin{macro}{\llm@LetLtxMacro}
%    \begin{macrocode}
\def\llm@LetLtxMacro#1#2#3{%
  \escapechar=92 %
  \expandafter\llm@CheckParams\meaning#2:->\@nil{%
    \begingroup
      \def\@protected@testopt{%
        \expandafter\@testopt\@gobble
      }%
      \def\@testopt##1##2{%
        \toks@={##2}%
      }%
      \let\llm@testopt\@empty
      \edef\x{%
        \noexpand\@protected@testopt
        \noexpand#2%
        \expandafter\noexpand\csname\string#2\endcsname
      }%
      \expandafter\expandafter\expandafter\def
      \expandafter\expandafter\expandafter\y
      \expandafter\expandafter\expandafter{%
        \expandafter\llm@CarThree#2{}{}{}\llm@nil
      }%
      \ifx\x\y
        #2%
        \def\llm@testopt{%
          \noexpand\@protected@testopt
          \noexpand#1%
        }%
      \else
        \edef\x{%
          \noexpand\@testopt
          \expandafter\noexpand
          \csname\string#2\endcsname
        }%
        \expandafter\expandafter\expandafter\def
        \expandafter\expandafter\expandafter\y
        \expandafter\expandafter\expandafter{%
          \expandafter\llm@CarTwo#2{}{}\llm@nil
        }%
        \ifx\x\y
          #2%
          \def\llm@testopt{%
            \noexpand\@testopt
          }%
        \fi
      \fi
      \ifx\llm@testopt\@empty
      \else
        \llm@protected\xdef\llm@GlobalTemp{%
          \llm@testopt
          \expandafter\noexpand
          \csname\string#1\endcsname
          {\the\toks@}%
        }%
      \fi
    \expandafter\endgroup\ifx\llm@testopt\@empty
      #3\let#1=#2\relax
    \else
      #3\let#1=\llm@GlobalTemp
      #3\expandafter\let
          \csname\string#1\expandafter\endcsname
          \csname\string#2\endcsname
    \fi
  }{%
    #3\let#1=#2\relax
  }%
}
%    \end{macrocode}
%    \end{macro}
%    \begin{macro}{\llm@CheckParams}
%    \begin{macrocode}
\def\llm@CheckParams#1:->#2\@nil{%
  \begingroup
    \def\x{#1}%
  \ifx\x\llm@macro
    \endgroup
    \def\llm@protected{}%
    \expandafter\@firstoftwo
  \else
    \ifx\x\llm@protectedmacro
      \endgroup
      \def\llm@protected{\protected}%
      \expandafter\expandafter\expandafter\@firstoftwo
    \else
      \endgroup
      \expandafter\expandafter\expandafter\@secondoftwo
    \fi
  \fi
}
%    \end{macrocode}
%    \end{macro}
%    \begin{macro}{\llm@macro}
%    \begin{macrocode}
\def\llm@macro{macro}
\@onelevel@sanitize\llm@macro
%    \end{macrocode}
%    \end{macro}
%    \begin{macro}{\llm@protectedmacro}
%    \begin{macrocode}
\def\llm@protectedmacro{\protected macro}
\@onelevel@sanitize\llm@protectedmacro
%    \end{macrocode}
%    \end{macro}
%    \begin{macro}{\llm@CarThree}
%    \begin{macrocode}
\def\llm@CarThree#1#2#3#4\llm@nil{#1#2#3}%
%    \end{macrocode}
%    \end{macro}
%    \begin{macro}{\llm@CarTwo}
%    \begin{macrocode}
\def\llm@CarTwo#1#2#3\llm@nil{#1#2}%
%    \end{macrocode}
%    \end{macro}
%
%    \begin{macrocode}
\llm@AtEnd%
%</package>
%    \end{macrocode}
%
% \section{Test}
%
% \subsection{Catcode checks for loading}
%
%    \begin{macrocode}
%<*test1>
%    \end{macrocode}
%    \begin{macrocode}
\NeedsTeXFormat{LaTeX2e}
\documentclass{minimal}
\makeatletter
\def\RestoreCatcodes{}
\count@=0 %
\loop
  \edef\RestoreCatcodes{%
    \RestoreCatcodes
    \catcode\the\count@=\the\catcode\count@\relax
  }%
\ifnum\count@<255 %
  \advance\count@\@ne
\repeat

\def\RangeCatcodeInvalid#1#2{%
  \count@=#1\relax
  \loop
    \catcode\count@=15 %
  \ifnum\count@<#2\relax
    \advance\count@\@ne
  \repeat
}
\def\Test{%
  \RangeCatcodeInvalid{0}{47}%
  \RangeCatcodeInvalid{58}{64}%
  \RangeCatcodeInvalid{91}{96}%
  \RangeCatcodeInvalid{123}{127}%
  \catcode`\@=12 %
  \catcode`\\=0 %
  \catcode`\{=1 %
  \catcode`\}=2 %
  \catcode`\#=6 %
  \catcode`\[=12 %
  \catcode`\]=12 %
  \catcode`\%=14 %
  \catcode`\ =10 %
  \catcode13=5 %
  \RequirePackage{letltxmacro}[2010/09/02]\relax
  \RestoreCatcodes
}
\Test
\csname @@end\endcsname
\end
%    \end{macrocode}
%    \begin{macrocode}
%</test1>
%    \end{macrocode}
%
% \subsection{Package tests}
%
%    \begin{macrocode}
%<*test2>
%    \end{macrocode}
%    \begin{macrocode}
\NeedsTeXFormat{LaTeX2e}
\nofiles
\documentclass{minimal}

\usepackage{letltxmacro}[2010/09/02]

\usepackage{qstest}
\IncludeTests{*}
\LogTests{log}{*}{*}

\makeatletter

\def\TestDef#1{%
  \begingroup
    \@makeother\\%
    \@makeother\ %
    \expandafter\@TestDef\csname #1\endcsname
}
\def\@TestDef#1#2{%
    \edef\a{\expandafter\strip@prefix\meaning#1}%
    \edef\b{\detokenize{#2}}%
    \ifx\a\b
    \else
      \typeout{1 [\a]}%
      \typeout{2 [\b]}%
    \fi
  \endgroup
  \Expect*{\expandafter\strip@prefix\meaning#1}*{\detokenize{#2}}%
}
\def\TestEquals#1#2{%
  \Expect*{%
    \expandafter\ifx\csname#1\expandafter\endcsname
                    \csname#2\endcsname
      equals%
    \else
      wrong%
    \fi
  }{equals}%
}
\def\SaveA{%
  \let\SavedA\TestA
  \expandafter\let\csname\string\SavedA\expandafter\endcsname
                  \csname\string\TestA\endcsname
  \expandafter\let\csname SavedA \expandafter\endcsname
                  \csname TestA \endcsname
}
\def\CheckA{%
  \TestEquals{SavedA}{TestA}%
  \TestEquals{\string\SavedA}{\string\TestA}%
  \TestEquals{SavedA }{TestA }%
}

\begin{qstest}{robust}{robust}
  \begingroup
    \DeclareRobustCommand*{\TestA}{Test}%
    \SaveA
    \TestDef{TestA}{\protect \TestA  }%
    \CheckA
    \DeclareRobustCommand{\TestA}{Test}%
    \SaveA
    \TestDef{TestA}{\protect \TestA  }%
    \LetLtxMacro\TestB\TestA
    \TestDef{TestB}{\protect \TestB  }%
    \TestEquals{TestB }{TestA }%
    \CheckA
    \GlobalLetLtxMacro\TestC\TestA
    \CheckA
  \endgroup
  \TestDef{TestC}{\protect \TestC  }%
  \DeclareRobustCommand{\TestA}{Test}%
  \TestEquals{TestC }{TestA }%
\end{qstest}

\begin{qstest}{default}{default}
  \begingroup
    \newcommand{\TestA}[1][\relax default \empty]{TestA #1}%
    \SaveA
    \TestDef
    {TestA}{\@protected@testopt \TestA \\TestA {\relax default \empty }}%
    \LetLtxMacro\TestB\TestA
    \TestDef
    {TestB}{\@protected@testopt \TestB \\TestB {\relax default \empty }}%
    \CheckA
    \TestEquals{\string\TestB}{\string\TestA}%
    \GlobalLetLtxMacro\TestC\TestA
    \CheckA
  \endgroup
  \TestDef
  {TestC}{\@protected@testopt \TestC \\TestC {\relax default \empty }}%
  \newcommand{\TestA}[1][\relax default \empty]{TestA #1}%
  \TestEquals{\string\TestC}{\string\TestA}%
\end{qstest}

\begin{qstest}{robustdefault}{robustdefault}
  \begingroup
    \DeclareRobustCommand{\TestA}[1][\default]{TestA}%
    \SaveA
    \LetLtxMacro\TestB\TestA
    \TestDef{TestB}{\protect \TestB  }%
    \TestDef{TestB }{\@protected@testopt \TestB  \\TestB  {\default }}%
    \CheckA
    \GlobalLetLtxMacro\TestC\TestA
    \CheckA
  \endgroup
  \TestDef{TestC}{\protect \TestC  }%
  \TestDef{TestC }{\@protected@testopt \TestC  \\TestC  {\default }}%
\end{qstest}

\begin{qstest}{plain}{plain}
  \LetLtxMacro\NewRelax\relax
  \Expect*{\meaning\relax}*{\string\relax}%
  \Expect*{\meaning\NewRelax}*{\string\relax}%
  \LetLtxMacro\NewHbox\hbox
  \Expect*{\meaning\hbox}*{\string\hbox}%
  \Expect*{\meaning\NewHbox}*{\string\hbox}%
  \LetLtxMacro\NewEmpty\empty
  \Expect*{\meaning\empty}{macro:->}%
  \Expect*{\meaning\NewEmpty}{macro:->}%
  \def\TestA{\iffalse}%
  \LetLtxMacro\TestB\TestA
  \TestDef{TestB}{\iffalse }%
  \TestEquals{TestA}{TestB}%
\end{qstest}
%    \end{macrocode}
%
%    \begin{macrocode}
\usepackage{etoolbox}[2008/06/22]

\begin{qstest}{oldnewrobustcmd}{oldnewrobustcmd}
  \begingroup
    \protected\edef\TestA{%
      \noexpand\@protected@testopt
      \noexpand\TestA
      \expandafter\noexpand\csname\string\TestA\endcsname
      {\noexpand\default}%
    }%
    \expandafter\protected\expandafter
    \def\csname\string\TestA\endcsname{TestA}%
    \SaveA
    \LetLtxMacro\TestB\TestA
    \TestDef{TestB}{\@protected@testopt \TestB \\TestB {\default }}%
    \protected\edef\TestC{%
      \noexpand\@protected@testopt
      \noexpand\TestB
      \expandafter\noexpand\csname\string\TestB\endcsname
      {\noexpand\default}%
    }%
    \TestEquals{TestB}{TestC}%
    \CheckA
    \GlobalLetLtxMacro\TestD\TestA
    \CheckA
  \endgroup
  \TestDef{TestD}{\@protected@testopt \TestD \\TestD {\default }}%
  \protected\edef\TestC{%
    \noexpand\@protected@testopt
    \noexpand\TestD
    \expandafter\noexpand\csname\string\TestD\endcsname
    {\noexpand\default}%
  }%
  \TestEquals{TestD}{TestC}%
\end{qstest}

\begin{qstest}{newrobustcmd}{newrobustcmd}
  \begingroup
    \newrobustcmd{\TestA}[1][\default]{TestA}%
    \SaveA
    \LetLtxMacro\TestB\TestA
    \TestDef{TestB}{\@testopt \\TestB {\default }}%
    \protected\edef\TestC{%
      \noexpand\@testopt
      \expandafter\noexpand\csname\string\TestB\endcsname
      {\noexpand\default}%
    }%
    \TestEquals{TestB}{TestC}%
    \CheckA
    \GlobalLetLtxMacro\TestD\TestA
    \CheckA
  \endgroup
  \TestDef{TestD}{\@testopt \\TestD {\default }}%
  \protected\edef\TestC{%
    \noexpand\@testopt
    \expandafter\noexpand\csname\string\TestD\endcsname
    {\noexpand\default}%
  }%
  \TestEquals{TestD}{TestC}%
\end{qstest}

\begin{qstest}{robustifyopt}{robustifyopt}
  \newcommand{\TestA}[2][\default]{}%
  \TestDef{TestA}{\@protected@testopt \TestA \\TestA {\default }}%
  \robustify\TestA
  \TestDef{TestA}{\@protected@testopt \TestA \\TestA {\default }}%
  \protected\edef\TestC{%
    \noexpand\@protected@testopt
    \noexpand\TestA
    \expandafter\noexpand\csname\string\TestA\endcsname
    {\noexpand\default}%
  }%
  \TestEquals{TestA}{TestC}%
  \SaveA
  \LetLtxMacro\TestB\TestA
  \TestDef{TestB}{\@protected@testopt \TestB \\TestB {\default }}%
  \protected\edef\TestC{%
    \noexpand\@protected@testopt
    \noexpand\TestB
    \expandafter\noexpand\csname\string\TestB\endcsname
    {\noexpand\default}%
  }%
  \TestEquals{TestB}{TestC}%
  \CheckA
\end{qstest}

\begin{qstest}{robustifydeclare}{robustifydeclare}
  \DeclareRobustCommand\TestA{\iffalse}%
  \robustify\TestA
  \SaveA
  \LetLtxMacro\TestB\TestA
  \TestDef{TestB}{\iffalse }%
  % before etoolbox 2008/06/22: \protected\long\def
  % since etoolbox 2008/06/22: \protected\def
  \protected\def\TestC{\iffalse}%
  \TestEquals{TestB}{TestC}%
  \CheckA
\end{qstest}

\begin{qstest}{robustifydeclarelong}{robustifydeclarelong}
  \DeclareRobustCommand\TestA[1]{\iffalse}%
  \robustify\TestA
  \SaveA
  \LetLtxMacro\TestB\TestA
  \TestDef{TestB}{\iffalse }%
  % before etoolbox 2008/06/22: \protected\long\def
  % since etoolbox 2008/06/22: \protected\def
  \protected\long\def\TestC#1{\iffalse}%
  \TestEquals{TestB}{TestC}%
  \CheckA
\end{qstest}
%    \end{macrocode}
%
%    \begin{macrocode}
\csname @@end\endcsname
%    \end{macrocode}
%    \begin{macrocode}
%</test2>
%    \end{macrocode}
%
% \section{Installation}
%
% \subsection{Download}
%
% \paragraph{Package.} This package is available on
% CTAN\footnote{\url{ftp://ftp.ctan.org/tex-archive/}}:
% \begin{description}
% \item[\CTAN{macros/latex/contrib/oberdiek/letltxmacro.dtx}] The source file.
% \item[\CTAN{macros/latex/contrib/oberdiek/letltxmacro.pdf}] Documentation.
% \end{description}
%
%
% \paragraph{Bundle.} All the packages of the bundle `oberdiek'
% are also available in a TDS compliant ZIP archive. There
% the packages are already unpacked and the documentation files
% are generated. The files and directories obey the TDS standard.
% \begin{description}
% \item[\CTAN{install/macros/latex/contrib/oberdiek.tds.zip}]
% \end{description}
% \emph{TDS} refers to the standard ``A Directory Structure
% for \TeX\ Files'' (\CTAN{tds/tds.pdf}). Directories
% with \xfile{texmf} in their name are usually organized this way.
%
% \subsection{Bundle installation}
%
% \paragraph{Unpacking.} Unpack the \xfile{oberdiek.tds.zip} in the
% TDS tree (also known as \xfile{texmf} tree) of your choice.
% Example (linux):
% \begin{quote}
%   |unzip oberdiek.tds.zip -d ~/texmf|
% \end{quote}
%
% \paragraph{Script installation.}
% Check the directory \xfile{TDS:scripts/oberdiek/} for
% scripts that need further installation steps.
% Package \xpackage{attachfile2} comes with the Perl script
% \xfile{pdfatfi.pl} that should be installed in such a way
% that it can be called as \texttt{pdfatfi}.
% Example (linux):
% \begin{quote}
%   |chmod +x scripts/oberdiek/pdfatfi.pl|\\
%   |cp scripts/oberdiek/pdfatfi.pl /usr/local/bin/|
% \end{quote}
%
% \subsection{Package installation}
%
% \paragraph{Unpacking.} The \xfile{.dtx} file is a self-extracting
% \docstrip\ archive. The files are extracted by running the
% \xfile{.dtx} through \plainTeX:
% \begin{quote}
%   \verb|tex letltxmacro.dtx|
% \end{quote}
%
% \paragraph{TDS.} Now the different files must be moved into
% the different directories in your installation TDS tree
% (also known as \xfile{texmf} tree):
% \begin{quote}
% \def\t{^^A
% \begin{tabular}{@{}>{\ttfamily}l@{ $\rightarrow$ }>{\ttfamily}l@{}}
%   letltxmacro.sty & tex/latex/oberdiek/letltxmacro.sty\\
%   letltxmacro.pdf & doc/latex/oberdiek/letltxmacro.pdf\\
%   letltxmacro-showcases.tex & doc/latex/oberdiek/letltxmacro-showcases.tex\\
%   test/letltxmacro-test1.tex & doc/latex/oberdiek/test/letltxmacro-test1.tex\\
%   test/letltxmacro-test2.tex & doc/latex/oberdiek/test/letltxmacro-test2.tex\\
%   letltxmacro.dtx & source/latex/oberdiek/letltxmacro.dtx\\
% \end{tabular}^^A
% }^^A
% \sbox0{\t}^^A
% \ifdim\wd0>\linewidth
%   \begingroup
%     \advance\linewidth by\leftmargin
%     \advance\linewidth by\rightmargin
%   \edef\x{\endgroup
%     \def\noexpand\lw{\the\linewidth}^^A
%   }\x
%   \def\lwbox{^^A
%     \leavevmode
%     \hbox to \linewidth{^^A
%       \kern-\leftmargin\relax
%       \hss
%       \usebox0
%       \hss
%       \kern-\rightmargin\relax
%     }^^A
%   }^^A
%   \ifdim\wd0>\lw
%     \sbox0{\small\t}^^A
%     \ifdim\wd0>\linewidth
%       \ifdim\wd0>\lw
%         \sbox0{\footnotesize\t}^^A
%         \ifdim\wd0>\linewidth
%           \ifdim\wd0>\lw
%             \sbox0{\scriptsize\t}^^A
%             \ifdim\wd0>\linewidth
%               \ifdim\wd0>\lw
%                 \sbox0{\tiny\t}^^A
%                 \ifdim\wd0>\linewidth
%                   \lwbox
%                 \else
%                   \usebox0
%                 \fi
%               \else
%                 \lwbox
%               \fi
%             \else
%               \usebox0
%             \fi
%           \else
%             \lwbox
%           \fi
%         \else
%           \usebox0
%         \fi
%       \else
%         \lwbox
%       \fi
%     \else
%       \usebox0
%     \fi
%   \else
%     \lwbox
%   \fi
% \else
%   \usebox0
% \fi
% \end{quote}
% If you have a \xfile{docstrip.cfg} that configures and enables \docstrip's
% TDS installing feature, then some files can already be in the right
% place, see the documentation of \docstrip.
%
% \subsection{Refresh file name databases}
%
% If your \TeX~distribution
% (\teTeX, \mikTeX, \dots) relies on file name databases, you must refresh
% these. For example, \teTeX\ users run \verb|texhash| or
% \verb|mktexlsr|.
%
% \subsection{Some details for the interested}
%
% \paragraph{Attached source.}
%
% The PDF documentation on CTAN also includes the
% \xfile{.dtx} source file. It can be extracted by
% AcrobatReader 6 or higher. Another option is \textsf{pdftk},
% e.g. unpack the file into the current directory:
% \begin{quote}
%   \verb|pdftk letltxmacro.pdf unpack_files output .|
% \end{quote}
%
% \paragraph{Unpacking with \LaTeX.}
% The \xfile{.dtx} chooses its action depending on the format:
% \begin{description}
% \item[\plainTeX:] Run \docstrip\ and extract the files.
% \item[\LaTeX:] Generate the documentation.
% \end{description}
% If you insist on using \LaTeX\ for \docstrip\ (really,
% \docstrip\ does not need \LaTeX), then inform the autodetect routine
% about your intention:
% \begin{quote}
%   \verb|latex \let\install=y% \iffalse meta-comment
%
% File: letltxmacro.dtx
% Version: 2010/09/02 v1.4
% Info: Let assignment for LaTeX macros
%
% Copyright (C) 2008, 2010 by
%    Heiko Oberdiek <heiko.oberdiek at googlemail.com>
%
% This work may be distributed and/or modified under the
% conditions of the LaTeX Project Public License, either
% version 1.3c of this license or (at your option) any later
% version. This version of this license is in
%    http://www.latex-project.org/lppl/lppl-1-3c.txt
% and the latest version of this license is in
%    http://www.latex-project.org/lppl.txt
% and version 1.3 or later is part of all distributions of
% LaTeX version 2005/12/01 or later.
%
% This work has the LPPL maintenance status "maintained".
%
% This Current Maintainer of this work is Heiko Oberdiek.
%
% This work consists of the main source file letltxmacro.dtx
% and the derived files
%    letltxmacro.sty, letltxmacro.pdf, letltxmacro.ins, letltxmacro.drv,
%    letltxmacro-showcases.tex, letltxmacro-test1.tex,
%    letltxmacro-test2.tex.
%
% Distribution:
%    CTAN:macros/latex/contrib/oberdiek/letltxmacro.dtx
%    CTAN:macros/latex/contrib/oberdiek/letltxmacro.pdf
%
% Unpacking:
%    (a) If letltxmacro.ins is present:
%           tex letltxmacro.ins
%    (b) Without letltxmacro.ins:
%           tex letltxmacro.dtx
%    (c) If you insist on using LaTeX
%           latex \let\install=y\input{letltxmacro.dtx}
%        (quote the arguments according to the demands of your shell)
%
% Documentation:
%    (a) If letltxmacro.drv is present:
%           latex letltxmacro.drv
%    (b) Without letltxmacro.drv:
%           latex letltxmacro.dtx; ...
%    The class ltxdoc loads the configuration file ltxdoc.cfg
%    if available. Here you can specify further options, e.g.
%    use A4 as paper format:
%       \PassOptionsToClass{a4paper}{article}
%
%    Programm calls to get the documentation (example):
%       pdflatex letltxmacro.dtx
%       makeindex -s gind.ist letltxmacro.idx
%       pdflatex letltxmacro.dtx
%       makeindex -s gind.ist letltxmacro.idx
%       pdflatex letltxmacro.dtx
%
% Installation:
%    TDS:tex/latex/oberdiek/letltxmacro.sty
%    TDS:doc/latex/oberdiek/letltxmacro.pdf
%    TDS:doc/latex/oberdiek/letltxmacro-showcases.tex
%    TDS:doc/latex/oberdiek/test/letltxmacro-test1.tex
%    TDS:doc/latex/oberdiek/test/letltxmacro-test2.tex
%    TDS:source/latex/oberdiek/letltxmacro.dtx
%
%<*ignore>
\begingroup
  \catcode123=1 %
  \catcode125=2 %
  \def\x{LaTeX2e}%
\expandafter\endgroup
\ifcase 0\ifx\install y1\fi\expandafter
         \ifx\csname processbatchFile\endcsname\relax\else1\fi
         \ifx\fmtname\x\else 1\fi\relax
\else\csname fi\endcsname
%</ignore>
%<*install>
\input docstrip.tex
\Msg{************************************************************************}
\Msg{* Installation}
\Msg{* Package: letltxmacro 2010/09/02 v1.4 Let assignment for LaTeX macros (HO)}
\Msg{************************************************************************}

\keepsilent
\askforoverwritefalse

\let\MetaPrefix\relax
\preamble

This is a generated file.

Project: letltxmacro
Version: 2010/09/02 v1.4

Copyright (C) 2008, 2010 by
   Heiko Oberdiek <heiko.oberdiek at googlemail.com>

This work may be distributed and/or modified under the
conditions of the LaTeX Project Public License, either
version 1.3c of this license or (at your option) any later
version. This version of this license is in
   http://www.latex-project.org/lppl/lppl-1-3c.txt
and the latest version of this license is in
   http://www.latex-project.org/lppl.txt
and version 1.3 or later is part of all distributions of
LaTeX version 2005/12/01 or later.

This work has the LPPL maintenance status "maintained".

This Current Maintainer of this work is Heiko Oberdiek.

This work consists of the main source file letltxmacro.dtx
and the derived files
   letltxmacro.sty, letltxmacro.pdf, letltxmacro.ins, letltxmacro.drv,
   letltxmacro-showcases.tex, letltxmacro-test1.tex,
   letltxmacro-test2.tex.

\endpreamble
\let\MetaPrefix\DoubleperCent

\generate{%
  \file{letltxmacro.ins}{\from{letltxmacro.dtx}{install}}%
  \file{letltxmacro.drv}{\from{letltxmacro.dtx}{driver}}%
  \usedir{tex/latex/oberdiek}%
  \file{letltxmacro.sty}{\from{letltxmacro.dtx}{package}}%
  \usedir{doc/latex/oberdiek}%
  \file{letltxmacro-showcases.tex}{\from{letltxmacro.dtx}{showcases}}%
  \usedir{doc/latex/oberdiek/test}%
  \file{letltxmacro-test1.tex}{\from{letltxmacro.dtx}{test1}}%
  \file{letltxmacro-test2.tex}{\from{letltxmacro.dtx}{test2}}%
  \nopreamble
  \nopostamble
  \usedir{source/latex/oberdiek/catalogue}%
  \file{letltxmacro.xml}{\from{letltxmacro.dtx}{catalogue}}%
}

\catcode32=13\relax% active space
\let =\space%
\Msg{************************************************************************}
\Msg{*}
\Msg{* To finish the installation you have to move the following}
\Msg{* file into a directory searched by TeX:}
\Msg{*}
\Msg{*     letltxmacro.sty}
\Msg{*}
\Msg{* To produce the documentation run the file `letltxmacro.drv'}
\Msg{* through LaTeX.}
\Msg{*}
\Msg{* Happy TeXing!}
\Msg{*}
\Msg{************************************************************************}

\endbatchfile
%</install>
%<*ignore>
\fi
%</ignore>
%<*driver>
\NeedsTeXFormat{LaTeX2e}
\ProvidesFile{letltxmacro.drv}%
  [2010/09/02 v1.4 Let assignment for LaTeX macros (HO)]%
\documentclass{ltxdoc}
\usepackage{holtxdoc}[2011/11/22]
\begin{document}
  \DocInput{letltxmacro.dtx}%
\end{document}
%</driver>
% \fi
%
% \CheckSum{990}
%
% \CharacterTable
%  {Upper-case    \A\B\C\D\E\F\G\H\I\J\K\L\M\N\O\P\Q\R\S\T\U\V\W\X\Y\Z
%   Lower-case    \a\b\c\d\e\f\g\h\i\j\k\l\m\n\o\p\q\r\s\t\u\v\w\x\y\z
%   Digits        \0\1\2\3\4\5\6\7\8\9
%   Exclamation   \!     Double quote  \"     Hash (number) \#
%   Dollar        \$     Percent       \%     Ampersand     \&
%   Acute accent  \'     Left paren    \(     Right paren   \)
%   Asterisk      \*     Plus          \+     Comma         \,
%   Minus         \-     Point         \.     Solidus       \/
%   Colon         \:     Semicolon     \;     Less than     \<
%   Equals        \=     Greater than  \>     Question mark \?
%   Commercial at \@     Left bracket  \[     Backslash     \\
%   Right bracket \]     Circumflex    \^     Underscore    \_
%   Grave accent  \`     Left brace    \{     Vertical bar  \|
%   Right brace   \}     Tilde         \~}
%
% \GetFileInfo{letltxmacro.drv}
%
% \title{The \xpackage{letltxmacro} package}
% \date{2010/09/02 v1.4}
% \author{Heiko Oberdiek\\\xemail{heiko.oberdiek at googlemail.com}}
%
% \maketitle
%
% \begin{abstract}
% \TeX's \cs{let} assignment does not work for \LaTeX\ macros
% with optional arguments or for macros that are defined
% as robust macros by \cs{DeclareRobustCommand}. This package
% defines \cs{LetLtxMacro} that also takes care of the involved
% internal macros.
% \end{abstract}
%
% \tableofcontents
%
% \section{Documentation}
%
% If someone wants to redefine a macro with using the old
% meaning, then one method is \TeX's command \cs{let}:
%\begin{quote}
%\begin{verbatim}
%\newcommand{\Macro}{\typeout{Test Macro}}
%\let\SavedMacro=\Macro
%\renewcommand{\Macro}{%
%  \typeout{Begin}%
%  \SavedMacro
%  \typeout{End}%
%}
%\end{verbatim}
%\end{quote}
% However, this method fails, if \cs{Macro} is defined
% by \cs{DeclareRobustCommand} and/or has an optional argument.
% In both cases \LaTeX\ defines an additional internal macro
% that is forgotten in the simple \cs{let} assignment of
% the example above.
%
% \begin{declcs}{LetLtxMacro} \M{new macro} \M{old macro}
% \end{declcs}
% Macro \cs{LetLtxMacro} behaves similar to \TeX's \cs{let}
% assignment, but it takes care of macros that are
% defined by \cs{DeclareRobustCommand} and/or have optional
% arguments. Example:
%\begin{quote}
%\begin{verbatim}
%\DeclareRobustCommand{\Macro}[1][default]{...}
%\LetLtxMacro{\SavedMacro}{\Macro}
%\end{verbatim}
%\end{quote}
% Then macro \cs{SavedMacro} only uses internal macro names
% that are derived from \cs{SavedMacro}'s macro name. Macro \cs{Macro}
% can now be redefined without affecting \cs{SavedMacro}.
%
% \begin{declcs}{GlobalLetLtxMacro} \M{new macro} \M{old macro}
% \end{declcs}
% Like \cs{LetLtxMacro}, but the \meta{new macro} is defined globally.
% Since version 2010/09/02~v1.4.
%
% \subsection{Supported macro definition commands}
%
% \begin{quote}
%   \begin{tabular}{@{}ll@{}}
%     \cs{newcommand}, \cs{renewcommand} & latex/base\\
%     \cs{newenvironment}, \cs{renewenvironment} & latex/base\\
%     \cs{DeclareRobustCommand}& latex/base\\
%     \cs{newrobustcmd}, \cs{renewrobustcmd} & etoolbox\\
%     \cs{robustify} & etoolbox 2008/06/22 v1.6\\
%   \end{tabular}
% \end{quote}
%
% \StopEventually{
% }
%
% \section{Implementation}
%
% \subsection{Show cases}
%
% \subsubsection{\xfile{letltxmacro-showcases.tex}}
%
%    \begin{macrocode}
%<*showcases>
\NeedsTeXFormat{LaTeX2e}
\makeatletter
%    \end{macrocode}
%    \begin{macro}{\Line}
%    The result is displayed by macro \cs{Line}. The percent symbol
%    at line start allows easy grepping and inserting into the DTX
%    file.
%    \begin{macrocode}
\newcommand*{\Line}[1]{%
  \typeout{\@percentchar#1}%
}
%    \end{macrocode}
%    \end{macro}
%    \begin{macrocode}
\newcommand*{\ShowCmdName}[1]{%
  \@ifundefined{#1}{}{%
    \Line{%
      \space\space(\expandafter\string\csname#1\endcsname) = %
      (\expandafter\meaning\csname#1\endcsname)%
    }%
  }%
}
\newcommand*{\ShowCmds}[1]{%
  \ShowCmdName{#1}%
  \ShowCmdName{#1 }%
  \ShowCmdName{\\#1}%
  \ShowCmdName{\\#1 }%
}
\let\\\@backslashchar
%    \end{macrocode}
%    \begin{macro}{\ShowDef}
%    \begin{macrocode}
\newcommand*{\ShowDef}[2]{%
  \begingroup
    \Line{}%
    \newcommand*{\DefString}{#2}%
    \@onelevel@sanitize\DefString
    \Line{\DefString}%
    #2%
    \ShowCmds{#1}%
  \endgroup
}
%    \end{macrocode}
%    \end{macro}
%    \begin{macrocode}
\typeout{}
\Line{* LaTeX definitions:}
\ShowDef{cmd}{%
  \newcommand{\cmd}[2][default]{}%
}
\ShowDef{cmd}{%
  \DeclareRobustCommand{\cmd}{}%
}
\ShowDef{cmd}{%
  \DeclareRobustCommand{\cmd}[2][default]{}%
}
\typeout{}
%    \end{macrocode}
% The minimal version of package \xpackage{etoolbox} is 2008/06/12 v1.6a
% because it fixes \cs{robustify}.
%    \begin{macrocode}
\RequirePackage{etoolbox}[2008/06/12]%
\Line{}
\Line{* etoolbox's robust definitions:}
\ShowDef{cmd}{%
  \newrobustcmd{\cmd}{}%
}
\ShowDef{cmd}{%
  \newrobustcmd{\cmd}[2][default]{}%
}
\Line{}
\Line{* etoolbox's \string\robustify:}
\ShowDef{cmd}{%
  \newcommand{\cmd}[2][default]{} %
  \robustify{\cmd}%
}
\ShowDef{cmd}{%
  \DeclareRobustCommand{\cmd}{} %
  \robustify{\cmd}%
}
\ShowDef{cmd}{%
  \DeclareRobustCommand{\cmd}[2][default]{} %
  \robustify{\cmd}%
}
\typeout{}
\@@end
%</showcases>
%    \end{macrocode}
%
% \subsubsection{Result}
%
% \begingroup
%   \makeatletter
%   \let\org@verbatim\@verbatim
%   \def\@verbatim{^^A
%     \org@verbatim
%     \catcode`\~=\active
%   }^^A
%   \let~\textvisiblespace
%\begin{verbatim}
%* LaTeX definitions:
%
%\newcommand {\cmd }[2][default]{}
%  (\cmd) = (macro:->\@protected@testopt \cmd \\cmd {default})
%  (\\cmd) = (\long macro:[#1]#2->)
%
%\DeclareRobustCommand {\cmd }{}
%  (\cmd) = (macro:->\protect \cmd~ )
%  (\cmd~) = (\long macro:->)
%
%\DeclareRobustCommand {\cmd }[2][default]{}
%  (\cmd) = (macro:->\protect \cmd~ )
%  (\cmd~) = (macro:->\@protected@testopt \cmd~ \\cmd~ {default})
%  (\\cmd~) = (\long macro:[#1]#2->)
%
%* etoolbox's robust definitions:
%
%\newrobustcmd {\cmd }{}
%  (\cmd) = (\protected\long macro:->)
%
%\newrobustcmd {\cmd }[2][default]{}
%  (\cmd) = (\protected macro:->\@testopt \\cmd {default})
%  (\\cmd) = (\long macro:[#1]#2->)
%
%* etoolbox's \robustify:
%
%\newcommand {\cmd }[2][default]{} \robustify {\cmd }
%  (\cmd) = (\protected macro:->\@protected@testopt \cmd \\cmd {default})
%  (\\cmd) = (\long macro:[#1]#2->)
%
%\DeclareRobustCommand {\cmd }{} \robustify {\cmd }
%  (\cmd) = (\protected macro:->)
%
%\DeclareRobustCommand {\cmd }[2][default]{} \robustify {\cmd }
%  (\cmd) = (\protected macro:->\@protected@testopt \cmd~ \\cmd~ {default})
%  (\cmd~) = (macro:->\@protected@testopt \cmd~ \\cmd~ {default})
%  (\\cmd~) = (\long macro:[#1]#2->)
%\end{verbatim}
% \endgroup
%
% \subsection{Package}
%
%    \begin{macrocode}
%<*package>
%    \end{macrocode}
%
% \subsubsection{Catcodes and identification}
%
%    \begin{macrocode}
\begingroup\catcode61\catcode48\catcode32=10\relax%
  \catcode13=5 % ^^M
  \endlinechar=13 %
  \catcode123=1 % {
  \catcode125=2 % }
  \catcode64=11 % @
  \def\x{\endgroup
    \expandafter\edef\csname llm@AtEnd\endcsname{%
      \endlinechar=\the\endlinechar\relax
      \catcode13=\the\catcode13\relax
      \catcode32=\the\catcode32\relax
      \catcode35=\the\catcode35\relax
      \catcode61=\the\catcode61\relax
      \catcode64=\the\catcode64\relax
      \catcode123=\the\catcode123\relax
      \catcode125=\the\catcode125\relax
    }%
  }%
\x\catcode61\catcode48\catcode32=10\relax%
\catcode13=5 % ^^M
\endlinechar=13 %
\catcode35=6 % #
\catcode64=11 % @
\catcode123=1 % {
\catcode125=2 % }
\def\TMP@EnsureCode#1#2{%
  \edef\llm@AtEnd{%
    \llm@AtEnd
    \catcode#1=\the\catcode#1\relax
  }%
  \catcode#1=#2\relax
}
\TMP@EnsureCode{40}{12}% (
\TMP@EnsureCode{41}{12}% )
\TMP@EnsureCode{42}{12}% *
\TMP@EnsureCode{45}{12}% -
\TMP@EnsureCode{46}{12}% .
\TMP@EnsureCode{47}{12}% /
\TMP@EnsureCode{58}{12}% :
\TMP@EnsureCode{62}{12}% >
\TMP@EnsureCode{91}{12}% [
\TMP@EnsureCode{93}{12}% ]
\edef\llm@AtEnd{%
  \llm@AtEnd
  \escapechar\the\escapechar\relax
  \noexpand\endinput
}
\escapechar=92 % `\\
%    \end{macrocode}
%
%    Package identification.
%    \begin{macrocode}
\NeedsTeXFormat{LaTeX2e}
\ProvidesPackage{letltxmacro}%
  [2010/09/02 v1.4 Let assignment for LaTeX macros (HO)]
%    \end{macrocode}
%
% \subsubsection{Main macros}
%
%    \begin{macro}{\LetLtxMacro}
%    \begin{macrocode}
\newcommand*{\LetLtxMacro}{%
  \llm@ModeLetLtxMacro{}%
}
%    \end{macrocode}
%    \end{macro}
%    \begin{macro}{\GlobalLetLtxMacro}
%    \begin{macrocode}
\newcommand*{\GlobalLetLtxMacro}{%
  \llm@ModeLetLtxMacro\global
}
%    \end{macrocode}
%    \end{macro}
%
%    \begin{macro}{\llm@ModeLetLtxMacro}
%    \begin{macrocode}
\newcommand*{\llm@ModeLetLtxMacro}[3]{%
  \edef\llm@escapechar{\the\escapechar}%
  \escapechar=-1 %
  \edef\reserved@a{%
    \noexpand\protect
    \expandafter\noexpand
    \csname\string#3 \endcsname
  }%
  \ifx\reserved@a#3\relax
    #1\edef#2{%
      \noexpand\protect
      \expandafter\noexpand
      \csname\string#2 \endcsname
    }%
    #1\expandafter\let
    \csname\string#2 \expandafter\endcsname
    \csname\string#3 \endcsname
    \expandafter\llm@LetLtxMacro
        \csname\string#2 \expandafter\endcsname
        \csname\string#3 \endcsname{#1}%
  \else
    \llm@LetLtxMacro{#2}{#3}{#1}%
  \fi
  \escapechar=\llm@escapechar\relax
}
%    \end{macrocode}
%    \end{macro}
%    \begin{macro}{\llm@LetLtxMacro}
%    \begin{macrocode}
\def\llm@LetLtxMacro#1#2#3{%
  \escapechar=92 %
  \expandafter\llm@CheckParams\meaning#2:->\@nil{%
    \begingroup
      \def\@protected@testopt{%
        \expandafter\@testopt\@gobble
      }%
      \def\@testopt##1##2{%
        \toks@={##2}%
      }%
      \let\llm@testopt\@empty
      \edef\x{%
        \noexpand\@protected@testopt
        \noexpand#2%
        \expandafter\noexpand\csname\string#2\endcsname
      }%
      \expandafter\expandafter\expandafter\def
      \expandafter\expandafter\expandafter\y
      \expandafter\expandafter\expandafter{%
        \expandafter\llm@CarThree#2{}{}{}\llm@nil
      }%
      \ifx\x\y
        #2%
        \def\llm@testopt{%
          \noexpand\@protected@testopt
          \noexpand#1%
        }%
      \else
        \edef\x{%
          \noexpand\@testopt
          \expandafter\noexpand
          \csname\string#2\endcsname
        }%
        \expandafter\expandafter\expandafter\def
        \expandafter\expandafter\expandafter\y
        \expandafter\expandafter\expandafter{%
          \expandafter\llm@CarTwo#2{}{}\llm@nil
        }%
        \ifx\x\y
          #2%
          \def\llm@testopt{%
            \noexpand\@testopt
          }%
        \fi
      \fi
      \ifx\llm@testopt\@empty
      \else
        \llm@protected\xdef\llm@GlobalTemp{%
          \llm@testopt
          \expandafter\noexpand
          \csname\string#1\endcsname
          {\the\toks@}%
        }%
      \fi
    \expandafter\endgroup\ifx\llm@testopt\@empty
      #3\let#1=#2\relax
    \else
      #3\let#1=\llm@GlobalTemp
      #3\expandafter\let
          \csname\string#1\expandafter\endcsname
          \csname\string#2\endcsname
    \fi
  }{%
    #3\let#1=#2\relax
  }%
}
%    \end{macrocode}
%    \end{macro}
%    \begin{macro}{\llm@CheckParams}
%    \begin{macrocode}
\def\llm@CheckParams#1:->#2\@nil{%
  \begingroup
    \def\x{#1}%
  \ifx\x\llm@macro
    \endgroup
    \def\llm@protected{}%
    \expandafter\@firstoftwo
  \else
    \ifx\x\llm@protectedmacro
      \endgroup
      \def\llm@protected{\protected}%
      \expandafter\expandafter\expandafter\@firstoftwo
    \else
      \endgroup
      \expandafter\expandafter\expandafter\@secondoftwo
    \fi
  \fi
}
%    \end{macrocode}
%    \end{macro}
%    \begin{macro}{\llm@macro}
%    \begin{macrocode}
\def\llm@macro{macro}
\@onelevel@sanitize\llm@macro
%    \end{macrocode}
%    \end{macro}
%    \begin{macro}{\llm@protectedmacro}
%    \begin{macrocode}
\def\llm@protectedmacro{\protected macro}
\@onelevel@sanitize\llm@protectedmacro
%    \end{macrocode}
%    \end{macro}
%    \begin{macro}{\llm@CarThree}
%    \begin{macrocode}
\def\llm@CarThree#1#2#3#4\llm@nil{#1#2#3}%
%    \end{macrocode}
%    \end{macro}
%    \begin{macro}{\llm@CarTwo}
%    \begin{macrocode}
\def\llm@CarTwo#1#2#3\llm@nil{#1#2}%
%    \end{macrocode}
%    \end{macro}
%
%    \begin{macrocode}
\llm@AtEnd%
%</package>
%    \end{macrocode}
%
% \section{Test}
%
% \subsection{Catcode checks for loading}
%
%    \begin{macrocode}
%<*test1>
%    \end{macrocode}
%    \begin{macrocode}
\NeedsTeXFormat{LaTeX2e}
\documentclass{minimal}
\makeatletter
\def\RestoreCatcodes{}
\count@=0 %
\loop
  \edef\RestoreCatcodes{%
    \RestoreCatcodes
    \catcode\the\count@=\the\catcode\count@\relax
  }%
\ifnum\count@<255 %
  \advance\count@\@ne
\repeat

\def\RangeCatcodeInvalid#1#2{%
  \count@=#1\relax
  \loop
    \catcode\count@=15 %
  \ifnum\count@<#2\relax
    \advance\count@\@ne
  \repeat
}
\def\Test{%
  \RangeCatcodeInvalid{0}{47}%
  \RangeCatcodeInvalid{58}{64}%
  \RangeCatcodeInvalid{91}{96}%
  \RangeCatcodeInvalid{123}{127}%
  \catcode`\@=12 %
  \catcode`\\=0 %
  \catcode`\{=1 %
  \catcode`\}=2 %
  \catcode`\#=6 %
  \catcode`\[=12 %
  \catcode`\]=12 %
  \catcode`\%=14 %
  \catcode`\ =10 %
  \catcode13=5 %
  \RequirePackage{letltxmacro}[2010/09/02]\relax
  \RestoreCatcodes
}
\Test
\csname @@end\endcsname
\end
%    \end{macrocode}
%    \begin{macrocode}
%</test1>
%    \end{macrocode}
%
% \subsection{Package tests}
%
%    \begin{macrocode}
%<*test2>
%    \end{macrocode}
%    \begin{macrocode}
\NeedsTeXFormat{LaTeX2e}
\nofiles
\documentclass{minimal}

\usepackage{letltxmacro}[2010/09/02]

\usepackage{qstest}
\IncludeTests{*}
\LogTests{log}{*}{*}

\makeatletter

\def\TestDef#1{%
  \begingroup
    \@makeother\\%
    \@makeother\ %
    \expandafter\@TestDef\csname #1\endcsname
}
\def\@TestDef#1#2{%
    \edef\a{\expandafter\strip@prefix\meaning#1}%
    \edef\b{\detokenize{#2}}%
    \ifx\a\b
    \else
      \typeout{1 [\a]}%
      \typeout{2 [\b]}%
    \fi
  \endgroup
  \Expect*{\expandafter\strip@prefix\meaning#1}*{\detokenize{#2}}%
}
\def\TestEquals#1#2{%
  \Expect*{%
    \expandafter\ifx\csname#1\expandafter\endcsname
                    \csname#2\endcsname
      equals%
    \else
      wrong%
    \fi
  }{equals}%
}
\def\SaveA{%
  \let\SavedA\TestA
  \expandafter\let\csname\string\SavedA\expandafter\endcsname
                  \csname\string\TestA\endcsname
  \expandafter\let\csname SavedA \expandafter\endcsname
                  \csname TestA \endcsname
}
\def\CheckA{%
  \TestEquals{SavedA}{TestA}%
  \TestEquals{\string\SavedA}{\string\TestA}%
  \TestEquals{SavedA }{TestA }%
}

\begin{qstest}{robust}{robust}
  \begingroup
    \DeclareRobustCommand*{\TestA}{Test}%
    \SaveA
    \TestDef{TestA}{\protect \TestA  }%
    \CheckA
    \DeclareRobustCommand{\TestA}{Test}%
    \SaveA
    \TestDef{TestA}{\protect \TestA  }%
    \LetLtxMacro\TestB\TestA
    \TestDef{TestB}{\protect \TestB  }%
    \TestEquals{TestB }{TestA }%
    \CheckA
    \GlobalLetLtxMacro\TestC\TestA
    \CheckA
  \endgroup
  \TestDef{TestC}{\protect \TestC  }%
  \DeclareRobustCommand{\TestA}{Test}%
  \TestEquals{TestC }{TestA }%
\end{qstest}

\begin{qstest}{default}{default}
  \begingroup
    \newcommand{\TestA}[1][\relax default \empty]{TestA #1}%
    \SaveA
    \TestDef
    {TestA}{\@protected@testopt \TestA \\TestA {\relax default \empty }}%
    \LetLtxMacro\TestB\TestA
    \TestDef
    {TestB}{\@protected@testopt \TestB \\TestB {\relax default \empty }}%
    \CheckA
    \TestEquals{\string\TestB}{\string\TestA}%
    \GlobalLetLtxMacro\TestC\TestA
    \CheckA
  \endgroup
  \TestDef
  {TestC}{\@protected@testopt \TestC \\TestC {\relax default \empty }}%
  \newcommand{\TestA}[1][\relax default \empty]{TestA #1}%
  \TestEquals{\string\TestC}{\string\TestA}%
\end{qstest}

\begin{qstest}{robustdefault}{robustdefault}
  \begingroup
    \DeclareRobustCommand{\TestA}[1][\default]{TestA}%
    \SaveA
    \LetLtxMacro\TestB\TestA
    \TestDef{TestB}{\protect \TestB  }%
    \TestDef{TestB }{\@protected@testopt \TestB  \\TestB  {\default }}%
    \CheckA
    \GlobalLetLtxMacro\TestC\TestA
    \CheckA
  \endgroup
  \TestDef{TestC}{\protect \TestC  }%
  \TestDef{TestC }{\@protected@testopt \TestC  \\TestC  {\default }}%
\end{qstest}

\begin{qstest}{plain}{plain}
  \LetLtxMacro\NewRelax\relax
  \Expect*{\meaning\relax}*{\string\relax}%
  \Expect*{\meaning\NewRelax}*{\string\relax}%
  \LetLtxMacro\NewHbox\hbox
  \Expect*{\meaning\hbox}*{\string\hbox}%
  \Expect*{\meaning\NewHbox}*{\string\hbox}%
  \LetLtxMacro\NewEmpty\empty
  \Expect*{\meaning\empty}{macro:->}%
  \Expect*{\meaning\NewEmpty}{macro:->}%
  \def\TestA{\iffalse}%
  \LetLtxMacro\TestB\TestA
  \TestDef{TestB}{\iffalse }%
  \TestEquals{TestA}{TestB}%
\end{qstest}
%    \end{macrocode}
%
%    \begin{macrocode}
\usepackage{etoolbox}[2008/06/22]

\begin{qstest}{oldnewrobustcmd}{oldnewrobustcmd}
  \begingroup
    \protected\edef\TestA{%
      \noexpand\@protected@testopt
      \noexpand\TestA
      \expandafter\noexpand\csname\string\TestA\endcsname
      {\noexpand\default}%
    }%
    \expandafter\protected\expandafter
    \def\csname\string\TestA\endcsname{TestA}%
    \SaveA
    \LetLtxMacro\TestB\TestA
    \TestDef{TestB}{\@protected@testopt \TestB \\TestB {\default }}%
    \protected\edef\TestC{%
      \noexpand\@protected@testopt
      \noexpand\TestB
      \expandafter\noexpand\csname\string\TestB\endcsname
      {\noexpand\default}%
    }%
    \TestEquals{TestB}{TestC}%
    \CheckA
    \GlobalLetLtxMacro\TestD\TestA
    \CheckA
  \endgroup
  \TestDef{TestD}{\@protected@testopt \TestD \\TestD {\default }}%
  \protected\edef\TestC{%
    \noexpand\@protected@testopt
    \noexpand\TestD
    \expandafter\noexpand\csname\string\TestD\endcsname
    {\noexpand\default}%
  }%
  \TestEquals{TestD}{TestC}%
\end{qstest}

\begin{qstest}{newrobustcmd}{newrobustcmd}
  \begingroup
    \newrobustcmd{\TestA}[1][\default]{TestA}%
    \SaveA
    \LetLtxMacro\TestB\TestA
    \TestDef{TestB}{\@testopt \\TestB {\default }}%
    \protected\edef\TestC{%
      \noexpand\@testopt
      \expandafter\noexpand\csname\string\TestB\endcsname
      {\noexpand\default}%
    }%
    \TestEquals{TestB}{TestC}%
    \CheckA
    \GlobalLetLtxMacro\TestD\TestA
    \CheckA
  \endgroup
  \TestDef{TestD}{\@testopt \\TestD {\default }}%
  \protected\edef\TestC{%
    \noexpand\@testopt
    \expandafter\noexpand\csname\string\TestD\endcsname
    {\noexpand\default}%
  }%
  \TestEquals{TestD}{TestC}%
\end{qstest}

\begin{qstest}{robustifyopt}{robustifyopt}
  \newcommand{\TestA}[2][\default]{}%
  \TestDef{TestA}{\@protected@testopt \TestA \\TestA {\default }}%
  \robustify\TestA
  \TestDef{TestA}{\@protected@testopt \TestA \\TestA {\default }}%
  \protected\edef\TestC{%
    \noexpand\@protected@testopt
    \noexpand\TestA
    \expandafter\noexpand\csname\string\TestA\endcsname
    {\noexpand\default}%
  }%
  \TestEquals{TestA}{TestC}%
  \SaveA
  \LetLtxMacro\TestB\TestA
  \TestDef{TestB}{\@protected@testopt \TestB \\TestB {\default }}%
  \protected\edef\TestC{%
    \noexpand\@protected@testopt
    \noexpand\TestB
    \expandafter\noexpand\csname\string\TestB\endcsname
    {\noexpand\default}%
  }%
  \TestEquals{TestB}{TestC}%
  \CheckA
\end{qstest}

\begin{qstest}{robustifydeclare}{robustifydeclare}
  \DeclareRobustCommand\TestA{\iffalse}%
  \robustify\TestA
  \SaveA
  \LetLtxMacro\TestB\TestA
  \TestDef{TestB}{\iffalse }%
  % before etoolbox 2008/06/22: \protected\long\def
  % since etoolbox 2008/06/22: \protected\def
  \protected\def\TestC{\iffalse}%
  \TestEquals{TestB}{TestC}%
  \CheckA
\end{qstest}

\begin{qstest}{robustifydeclarelong}{robustifydeclarelong}
  \DeclareRobustCommand\TestA[1]{\iffalse}%
  \robustify\TestA
  \SaveA
  \LetLtxMacro\TestB\TestA
  \TestDef{TestB}{\iffalse }%
  % before etoolbox 2008/06/22: \protected\long\def
  % since etoolbox 2008/06/22: \protected\def
  \protected\long\def\TestC#1{\iffalse}%
  \TestEquals{TestB}{TestC}%
  \CheckA
\end{qstest}
%    \end{macrocode}
%
%    \begin{macrocode}
\csname @@end\endcsname
%    \end{macrocode}
%    \begin{macrocode}
%</test2>
%    \end{macrocode}
%
% \section{Installation}
%
% \subsection{Download}
%
% \paragraph{Package.} This package is available on
% CTAN\footnote{\url{ftp://ftp.ctan.org/tex-archive/}}:
% \begin{description}
% \item[\CTAN{macros/latex/contrib/oberdiek/letltxmacro.dtx}] The source file.
% \item[\CTAN{macros/latex/contrib/oberdiek/letltxmacro.pdf}] Documentation.
% \end{description}
%
%
% \paragraph{Bundle.} All the packages of the bundle `oberdiek'
% are also available in a TDS compliant ZIP archive. There
% the packages are already unpacked and the documentation files
% are generated. The files and directories obey the TDS standard.
% \begin{description}
% \item[\CTAN{install/macros/latex/contrib/oberdiek.tds.zip}]
% \end{description}
% \emph{TDS} refers to the standard ``A Directory Structure
% for \TeX\ Files'' (\CTAN{tds/tds.pdf}). Directories
% with \xfile{texmf} in their name are usually organized this way.
%
% \subsection{Bundle installation}
%
% \paragraph{Unpacking.} Unpack the \xfile{oberdiek.tds.zip} in the
% TDS tree (also known as \xfile{texmf} tree) of your choice.
% Example (linux):
% \begin{quote}
%   |unzip oberdiek.tds.zip -d ~/texmf|
% \end{quote}
%
% \paragraph{Script installation.}
% Check the directory \xfile{TDS:scripts/oberdiek/} for
% scripts that need further installation steps.
% Package \xpackage{attachfile2} comes with the Perl script
% \xfile{pdfatfi.pl} that should be installed in such a way
% that it can be called as \texttt{pdfatfi}.
% Example (linux):
% \begin{quote}
%   |chmod +x scripts/oberdiek/pdfatfi.pl|\\
%   |cp scripts/oberdiek/pdfatfi.pl /usr/local/bin/|
% \end{quote}
%
% \subsection{Package installation}
%
% \paragraph{Unpacking.} The \xfile{.dtx} file is a self-extracting
% \docstrip\ archive. The files are extracted by running the
% \xfile{.dtx} through \plainTeX:
% \begin{quote}
%   \verb|tex letltxmacro.dtx|
% \end{quote}
%
% \paragraph{TDS.} Now the different files must be moved into
% the different directories in your installation TDS tree
% (also known as \xfile{texmf} tree):
% \begin{quote}
% \def\t{^^A
% \begin{tabular}{@{}>{\ttfamily}l@{ $\rightarrow$ }>{\ttfamily}l@{}}
%   letltxmacro.sty & tex/latex/oberdiek/letltxmacro.sty\\
%   letltxmacro.pdf & doc/latex/oberdiek/letltxmacro.pdf\\
%   letltxmacro-showcases.tex & doc/latex/oberdiek/letltxmacro-showcases.tex\\
%   test/letltxmacro-test1.tex & doc/latex/oberdiek/test/letltxmacro-test1.tex\\
%   test/letltxmacro-test2.tex & doc/latex/oberdiek/test/letltxmacro-test2.tex\\
%   letltxmacro.dtx & source/latex/oberdiek/letltxmacro.dtx\\
% \end{tabular}^^A
% }^^A
% \sbox0{\t}^^A
% \ifdim\wd0>\linewidth
%   \begingroup
%     \advance\linewidth by\leftmargin
%     \advance\linewidth by\rightmargin
%   \edef\x{\endgroup
%     \def\noexpand\lw{\the\linewidth}^^A
%   }\x
%   \def\lwbox{^^A
%     \leavevmode
%     \hbox to \linewidth{^^A
%       \kern-\leftmargin\relax
%       \hss
%       \usebox0
%       \hss
%       \kern-\rightmargin\relax
%     }^^A
%   }^^A
%   \ifdim\wd0>\lw
%     \sbox0{\small\t}^^A
%     \ifdim\wd0>\linewidth
%       \ifdim\wd0>\lw
%         \sbox0{\footnotesize\t}^^A
%         \ifdim\wd0>\linewidth
%           \ifdim\wd0>\lw
%             \sbox0{\scriptsize\t}^^A
%             \ifdim\wd0>\linewidth
%               \ifdim\wd0>\lw
%                 \sbox0{\tiny\t}^^A
%                 \ifdim\wd0>\linewidth
%                   \lwbox
%                 \else
%                   \usebox0
%                 \fi
%               \else
%                 \lwbox
%               \fi
%             \else
%               \usebox0
%             \fi
%           \else
%             \lwbox
%           \fi
%         \else
%           \usebox0
%         \fi
%       \else
%         \lwbox
%       \fi
%     \else
%       \usebox0
%     \fi
%   \else
%     \lwbox
%   \fi
% \else
%   \usebox0
% \fi
% \end{quote}
% If you have a \xfile{docstrip.cfg} that configures and enables \docstrip's
% TDS installing feature, then some files can already be in the right
% place, see the documentation of \docstrip.
%
% \subsection{Refresh file name databases}
%
% If your \TeX~distribution
% (\teTeX, \mikTeX, \dots) relies on file name databases, you must refresh
% these. For example, \teTeX\ users run \verb|texhash| or
% \verb|mktexlsr|.
%
% \subsection{Some details for the interested}
%
% \paragraph{Attached source.}
%
% The PDF documentation on CTAN also includes the
% \xfile{.dtx} source file. It can be extracted by
% AcrobatReader 6 or higher. Another option is \textsf{pdftk},
% e.g. unpack the file into the current directory:
% \begin{quote}
%   \verb|pdftk letltxmacro.pdf unpack_files output .|
% \end{quote}
%
% \paragraph{Unpacking with \LaTeX.}
% The \xfile{.dtx} chooses its action depending on the format:
% \begin{description}
% \item[\plainTeX:] Run \docstrip\ and extract the files.
% \item[\LaTeX:] Generate the documentation.
% \end{description}
% If you insist on using \LaTeX\ for \docstrip\ (really,
% \docstrip\ does not need \LaTeX), then inform the autodetect routine
% about your intention:
% \begin{quote}
%   \verb|latex \let\install=y\input{letltxmacro.dtx}|
% \end{quote}
% Do not forget to quote the argument according to the demands
% of your shell.
%
% \paragraph{Generating the documentation.}
% You can use both the \xfile{.dtx} or the \xfile{.drv} to generate
% the documentation. The process can be configured by the
% configuration file \xfile{ltxdoc.cfg}. For instance, put this
% line into this file, if you want to have A4 as paper format:
% \begin{quote}
%   \verb|\PassOptionsToClass{a4paper}{article}|
% \end{quote}
% An example follows how to generate the
% documentation with pdf\LaTeX:
% \begin{quote}
%\begin{verbatim}
%pdflatex letltxmacro.dtx
%makeindex -s gind.ist letltxmacro.idx
%pdflatex letltxmacro.dtx
%makeindex -s gind.ist letltxmacro.idx
%pdflatex letltxmacro.dtx
%\end{verbatim}
% \end{quote}
%
% \section{Catalogue}
%
% The following XML file can be used as source for the
% \href{http://mirror.ctan.org/help/Catalogue/catalogue.html}{\TeX\ Catalogue}.
% The elements \texttt{caption} and \texttt{description} are imported
% from the original XML file from the Catalogue.
% The name of the XML file in the Catalogue is \xfile{letltxmacro.xml}.
%    \begin{macrocode}
%<*catalogue>
<?xml version='1.0' encoding='us-ascii'?>
<!DOCTYPE entry SYSTEM 'catalogue.dtd'>
<entry datestamp='$Date$' modifier='$Author$' id='letltxmacro'>
  <name>letltxmacro</name>
  <caption>Let assignment for LaTeX macros.</caption>
  <authorref id='auth:oberdiek'/>
  <copyright owner='Heiko Oberdiek' year='2008,2010'/>
  <license type='lppl1.3'/>
  <version number='1.4'/>
  <description>
    TeX&#x2019;s <tt>\let</tt> assignment does not work for LaTeX macros
    with optional arguments or for macros that are defined
    as robust macros by <tt>\DeclareRobustCommand</tt>.  This package
    defines <tt>\LetLtxMacro</tt> that also takes care of the involved
    internal macros.
    <p/>
    The package is part of the <xref refid='oberdiek'>oberdiek</xref>
    bundle.
  </description>
  <documentation details='Package documentation'
      href='ctan:/macros/latex/contrib/oberdiek/letltxmacro.pdf'/>
  <ctan file='true' path='/macros/latex/contrib/oberdiek/letltxmacro.dtx'/>
  <miktex location='oberdiek'/>
  <texlive location='oberdiek'/>
  <install path='/macros/latex/contrib/oberdiek/oberdiek.tds.zip'/>
</entry>
%</catalogue>
%    \end{macrocode}
%
% \begin{History}
%   \begin{Version}{2008/06/09 v1.0}
%   \item
%     First version.
%   \end{Version}
%   \begin{Version}{2008/06/12 v1.1}
%   \item
%     Support for \xpackage{etoolbox}'s \cs{newrobustcmd} added.
%   \end{Version}
%   \begin{Version}{2008/06/13 v1.2}
%   \item
%     Support for \xpackage{etoolbox}'s \cs{robustify} added.
%   \end{Version}
%   \begin{Version}{2008/06/24 v1.3}
%   \item
%     Test file adapted for etoolbox 2008/06/22 v1.6.
%   \end{Version}
%   \begin{Version}{2010/09/02 v1.4}
%   \item
%     \cs{GlobalLetLtxMacro} added.
%   \end{Version}
% \end{History}
%
% \PrintIndex
%
% \Finale
\endinput
|
% \end{quote}
% Do not forget to quote the argument according to the demands
% of your shell.
%
% \paragraph{Generating the documentation.}
% You can use both the \xfile{.dtx} or the \xfile{.drv} to generate
% the documentation. The process can be configured by the
% configuration file \xfile{ltxdoc.cfg}. For instance, put this
% line into this file, if you want to have A4 as paper format:
% \begin{quote}
%   \verb|\PassOptionsToClass{a4paper}{article}|
% \end{quote}
% An example follows how to generate the
% documentation with pdf\LaTeX:
% \begin{quote}
%\begin{verbatim}
%pdflatex letltxmacro.dtx
%makeindex -s gind.ist letltxmacro.idx
%pdflatex letltxmacro.dtx
%makeindex -s gind.ist letltxmacro.idx
%pdflatex letltxmacro.dtx
%\end{verbatim}
% \end{quote}
%
% \section{Catalogue}
%
% The following XML file can be used as source for the
% \href{http://mirror.ctan.org/help/Catalogue/catalogue.html}{\TeX\ Catalogue}.
% The elements \texttt{caption} and \texttt{description} are imported
% from the original XML file from the Catalogue.
% The name of the XML file in the Catalogue is \xfile{letltxmacro.xml}.
%    \begin{macrocode}
%<*catalogue>
<?xml version='1.0' encoding='us-ascii'?>
<!DOCTYPE entry SYSTEM 'catalogue.dtd'>
<entry datestamp='$Date$' modifier='$Author$' id='letltxmacro'>
  <name>letltxmacro</name>
  <caption>Let assignment for LaTeX macros.</caption>
  <authorref id='auth:oberdiek'/>
  <copyright owner='Heiko Oberdiek' year='2008,2010'/>
  <license type='lppl1.3'/>
  <version number='1.4'/>
  <description>
    TeX&#x2019;s <tt>\let</tt> assignment does not work for LaTeX macros
    with optional arguments or for macros that are defined
    as robust macros by <tt>\DeclareRobustCommand</tt>.  This package
    defines <tt>\LetLtxMacro</tt> that also takes care of the involved
    internal macros.
    <p/>
    The package is part of the <xref refid='oberdiek'>oberdiek</xref>
    bundle.
  </description>
  <documentation details='Package documentation'
      href='ctan:/macros/latex/contrib/oberdiek/letltxmacro.pdf'/>
  <ctan file='true' path='/macros/latex/contrib/oberdiek/letltxmacro.dtx'/>
  <miktex location='oberdiek'/>
  <texlive location='oberdiek'/>
  <install path='/macros/latex/contrib/oberdiek/oberdiek.tds.zip'/>
</entry>
%</catalogue>
%    \end{macrocode}
%
% \begin{History}
%   \begin{Version}{2008/06/09 v1.0}
%   \item
%     First version.
%   \end{Version}
%   \begin{Version}{2008/06/12 v1.1}
%   \item
%     Support for \xpackage{etoolbox}'s \cs{newrobustcmd} added.
%   \end{Version}
%   \begin{Version}{2008/06/13 v1.2}
%   \item
%     Support for \xpackage{etoolbox}'s \cs{robustify} added.
%   \end{Version}
%   \begin{Version}{2008/06/24 v1.3}
%   \item
%     Test file adapted for etoolbox 2008/06/22 v1.6.
%   \end{Version}
%   \begin{Version}{2010/09/02 v1.4}
%   \item
%     \cs{GlobalLetLtxMacro} added.
%   \end{Version}
% \end{History}
%
% \PrintIndex
%
% \Finale
\endinput

%        (quote the arguments according to the demands of your shell)
%
% Documentation:
%    (a) If letltxmacro.drv is present:
%           latex letltxmacro.drv
%    (b) Without letltxmacro.drv:
%           latex letltxmacro.dtx; ...
%    The class ltxdoc loads the configuration file ltxdoc.cfg
%    if available. Here you can specify further options, e.g.
%    use A4 as paper format:
%       \PassOptionsToClass{a4paper}{article}
%
%    Programm calls to get the documentation (example):
%       pdflatex letltxmacro.dtx
%       makeindex -s gind.ist letltxmacro.idx
%       pdflatex letltxmacro.dtx
%       makeindex -s gind.ist letltxmacro.idx
%       pdflatex letltxmacro.dtx
%
% Installation:
%    TDS:tex/latex/oberdiek/letltxmacro.sty
%    TDS:doc/latex/oberdiek/letltxmacro.pdf
%    TDS:doc/latex/oberdiek/letltxmacro-showcases.tex
%    TDS:doc/latex/oberdiek/test/letltxmacro-test1.tex
%    TDS:doc/latex/oberdiek/test/letltxmacro-test2.tex
%    TDS:source/latex/oberdiek/letltxmacro.dtx
%
%<*ignore>
\begingroup
  \catcode123=1 %
  \catcode125=2 %
  \def\x{LaTeX2e}%
\expandafter\endgroup
\ifcase 0\ifx\install y1\fi\expandafter
         \ifx\csname processbatchFile\endcsname\relax\else1\fi
         \ifx\fmtname\x\else 1\fi\relax
\else\csname fi\endcsname
%</ignore>
%<*install>
\input docstrip.tex
\Msg{************************************************************************}
\Msg{* Installation}
\Msg{* Package: letltxmacro 2010/09/02 v1.4 Let assignment for LaTeX macros (HO)}
\Msg{************************************************************************}

\keepsilent
\askforoverwritefalse

\let\MetaPrefix\relax
\preamble

This is a generated file.

Project: letltxmacro
Version: 2010/09/02 v1.4

Copyright (C) 2008, 2010 by
   Heiko Oberdiek <heiko.oberdiek at googlemail.com>

This work may be distributed and/or modified under the
conditions of the LaTeX Project Public License, either
version 1.3c of this license or (at your option) any later
version. This version of this license is in
   http://www.latex-project.org/lppl/lppl-1-3c.txt
and the latest version of this license is in
   http://www.latex-project.org/lppl.txt
and version 1.3 or later is part of all distributions of
LaTeX version 2005/12/01 or later.

This work has the LPPL maintenance status "maintained".

This Current Maintainer of this work is Heiko Oberdiek.

This work consists of the main source file letltxmacro.dtx
and the derived files
   letltxmacro.sty, letltxmacro.pdf, letltxmacro.ins, letltxmacro.drv,
   letltxmacro-showcases.tex, letltxmacro-test1.tex,
   letltxmacro-test2.tex.

\endpreamble
\let\MetaPrefix\DoubleperCent

\generate{%
  \file{letltxmacro.ins}{\from{letltxmacro.dtx}{install}}%
  \file{letltxmacro.drv}{\from{letltxmacro.dtx}{driver}}%
  \usedir{tex/latex/oberdiek}%
  \file{letltxmacro.sty}{\from{letltxmacro.dtx}{package}}%
  \usedir{doc/latex/oberdiek}%
  \file{letltxmacro-showcases.tex}{\from{letltxmacro.dtx}{showcases}}%
  \usedir{doc/latex/oberdiek/test}%
  \file{letltxmacro-test1.tex}{\from{letltxmacro.dtx}{test1}}%
  \file{letltxmacro-test2.tex}{\from{letltxmacro.dtx}{test2}}%
  \nopreamble
  \nopostamble
  \usedir{source/latex/oberdiek/catalogue}%
  \file{letltxmacro.xml}{\from{letltxmacro.dtx}{catalogue}}%
}

\catcode32=13\relax% active space
\let =\space%
\Msg{************************************************************************}
\Msg{*}
\Msg{* To finish the installation you have to move the following}
\Msg{* file into a directory searched by TeX:}
\Msg{*}
\Msg{*     letltxmacro.sty}
\Msg{*}
\Msg{* To produce the documentation run the file `letltxmacro.drv'}
\Msg{* through LaTeX.}
\Msg{*}
\Msg{* Happy TeXing!}
\Msg{*}
\Msg{************************************************************************}

\endbatchfile
%</install>
%<*ignore>
\fi
%</ignore>
%<*driver>
\NeedsTeXFormat{LaTeX2e}
\ProvidesFile{letltxmacro.drv}%
  [2010/09/02 v1.4 Let assignment for LaTeX macros (HO)]%
\documentclass{ltxdoc}
\usepackage{holtxdoc}[2011/11/22]
\begin{document}
  \DocInput{letltxmacro.dtx}%
\end{document}
%</driver>
% \fi
%
% \CheckSum{990}
%
% \CharacterTable
%  {Upper-case    \A\B\C\D\E\F\G\H\I\J\K\L\M\N\O\P\Q\R\S\T\U\V\W\X\Y\Z
%   Lower-case    \a\b\c\d\e\f\g\h\i\j\k\l\m\n\o\p\q\r\s\t\u\v\w\x\y\z
%   Digits        \0\1\2\3\4\5\6\7\8\9
%   Exclamation   \!     Double quote  \"     Hash (number) \#
%   Dollar        \$     Percent       \%     Ampersand     \&
%   Acute accent  \'     Left paren    \(     Right paren   \)
%   Asterisk      \*     Plus          \+     Comma         \,
%   Minus         \-     Point         \.     Solidus       \/
%   Colon         \:     Semicolon     \;     Less than     \<
%   Equals        \=     Greater than  \>     Question mark \?
%   Commercial at \@     Left bracket  \[     Backslash     \\
%   Right bracket \]     Circumflex    \^     Underscore    \_
%   Grave accent  \`     Left brace    \{     Vertical bar  \|
%   Right brace   \}     Tilde         \~}
%
% \GetFileInfo{letltxmacro.drv}
%
% \title{The \xpackage{letltxmacro} package}
% \date{2010/09/02 v1.4}
% \author{Heiko Oberdiek\\\xemail{heiko.oberdiek at googlemail.com}}
%
% \maketitle
%
% \begin{abstract}
% \TeX's \cs{let} assignment does not work for \LaTeX\ macros
% with optional arguments or for macros that are defined
% as robust macros by \cs{DeclareRobustCommand}. This package
% defines \cs{LetLtxMacro} that also takes care of the involved
% internal macros.
% \end{abstract}
%
% \tableofcontents
%
% \section{Documentation}
%
% If someone wants to redefine a macro with using the old
% meaning, then one method is \TeX's command \cs{let}:
%\begin{quote}
%\begin{verbatim}
%\newcommand{\Macro}{\typeout{Test Macro}}
%\let\SavedMacro=\Macro
%\renewcommand{\Macro}{%
%  \typeout{Begin}%
%  \SavedMacro
%  \typeout{End}%
%}
%\end{verbatim}
%\end{quote}
% However, this method fails, if \cs{Macro} is defined
% by \cs{DeclareRobustCommand} and/or has an optional argument.
% In both cases \LaTeX\ defines an additional internal macro
% that is forgotten in the simple \cs{let} assignment of
% the example above.
%
% \begin{declcs}{LetLtxMacro} \M{new macro} \M{old macro}
% \end{declcs}
% Macro \cs{LetLtxMacro} behaves similar to \TeX's \cs{let}
% assignment, but it takes care of macros that are
% defined by \cs{DeclareRobustCommand} and/or have optional
% arguments. Example:
%\begin{quote}
%\begin{verbatim}
%\DeclareRobustCommand{\Macro}[1][default]{...}
%\LetLtxMacro{\SavedMacro}{\Macro}
%\end{verbatim}
%\end{quote}
% Then macro \cs{SavedMacro} only uses internal macro names
% that are derived from \cs{SavedMacro}'s macro name. Macro \cs{Macro}
% can now be redefined without affecting \cs{SavedMacro}.
%
% \begin{declcs}{GlobalLetLtxMacro} \M{new macro} \M{old macro}
% \end{declcs}
% Like \cs{LetLtxMacro}, but the \meta{new macro} is defined globally.
% Since version 2010/09/02~v1.4.
%
% \subsection{Supported macro definition commands}
%
% \begin{quote}
%   \begin{tabular}{@{}ll@{}}
%     \cs{newcommand}, \cs{renewcommand} & latex/base\\
%     \cs{newenvironment}, \cs{renewenvironment} & latex/base\\
%     \cs{DeclareRobustCommand}& latex/base\\
%     \cs{newrobustcmd}, \cs{renewrobustcmd} & etoolbox\\
%     \cs{robustify} & etoolbox 2008/06/22 v1.6\\
%   \end{tabular}
% \end{quote}
%
% \StopEventually{
% }
%
% \section{Implementation}
%
% \subsection{Show cases}
%
% \subsubsection{\xfile{letltxmacro-showcases.tex}}
%
%    \begin{macrocode}
%<*showcases>
\NeedsTeXFormat{LaTeX2e}
\makeatletter
%    \end{macrocode}
%    \begin{macro}{\Line}
%    The result is displayed by macro \cs{Line}. The percent symbol
%    at line start allows easy grepping and inserting into the DTX
%    file.
%    \begin{macrocode}
\newcommand*{\Line}[1]{%
  \typeout{\@percentchar#1}%
}
%    \end{macrocode}
%    \end{macro}
%    \begin{macrocode}
\newcommand*{\ShowCmdName}[1]{%
  \@ifundefined{#1}{}{%
    \Line{%
      \space\space(\expandafter\string\csname#1\endcsname) = %
      (\expandafter\meaning\csname#1\endcsname)%
    }%
  }%
}
\newcommand*{\ShowCmds}[1]{%
  \ShowCmdName{#1}%
  \ShowCmdName{#1 }%
  \ShowCmdName{\\#1}%
  \ShowCmdName{\\#1 }%
}
\let\\\@backslashchar
%    \end{macrocode}
%    \begin{macro}{\ShowDef}
%    \begin{macrocode}
\newcommand*{\ShowDef}[2]{%
  \begingroup
    \Line{}%
    \newcommand*{\DefString}{#2}%
    \@onelevel@sanitize\DefString
    \Line{\DefString}%
    #2%
    \ShowCmds{#1}%
  \endgroup
}
%    \end{macrocode}
%    \end{macro}
%    \begin{macrocode}
\typeout{}
\Line{* LaTeX definitions:}
\ShowDef{cmd}{%
  \newcommand{\cmd}[2][default]{}%
}
\ShowDef{cmd}{%
  \DeclareRobustCommand{\cmd}{}%
}
\ShowDef{cmd}{%
  \DeclareRobustCommand{\cmd}[2][default]{}%
}
\typeout{}
%    \end{macrocode}
% The minimal version of package \xpackage{etoolbox} is 2008/06/12 v1.6a
% because it fixes \cs{robustify}.
%    \begin{macrocode}
\RequirePackage{etoolbox}[2008/06/12]%
\Line{}
\Line{* etoolbox's robust definitions:}
\ShowDef{cmd}{%
  \newrobustcmd{\cmd}{}%
}
\ShowDef{cmd}{%
  \newrobustcmd{\cmd}[2][default]{}%
}
\Line{}
\Line{* etoolbox's \string\robustify:}
\ShowDef{cmd}{%
  \newcommand{\cmd}[2][default]{} %
  \robustify{\cmd}%
}
\ShowDef{cmd}{%
  \DeclareRobustCommand{\cmd}{} %
  \robustify{\cmd}%
}
\ShowDef{cmd}{%
  \DeclareRobustCommand{\cmd}[2][default]{} %
  \robustify{\cmd}%
}
\typeout{}
\@@end
%</showcases>
%    \end{macrocode}
%
% \subsubsection{Result}
%
% \begingroup
%   \makeatletter
%   \let\org@verbatim\@verbatim
%   \def\@verbatim{^^A
%     \org@verbatim
%     \catcode`\~=\active
%   }^^A
%   \let~\textvisiblespace
%\begin{verbatim}
%* LaTeX definitions:
%
%\newcommand {\cmd }[2][default]{}
%  (\cmd) = (macro:->\@protected@testopt \cmd \\cmd {default})
%  (\\cmd) = (\long macro:[#1]#2->)
%
%\DeclareRobustCommand {\cmd }{}
%  (\cmd) = (macro:->\protect \cmd~ )
%  (\cmd~) = (\long macro:->)
%
%\DeclareRobustCommand {\cmd }[2][default]{}
%  (\cmd) = (macro:->\protect \cmd~ )
%  (\cmd~) = (macro:->\@protected@testopt \cmd~ \\cmd~ {default})
%  (\\cmd~) = (\long macro:[#1]#2->)
%
%* etoolbox's robust definitions:
%
%\newrobustcmd {\cmd }{}
%  (\cmd) = (\protected\long macro:->)
%
%\newrobustcmd {\cmd }[2][default]{}
%  (\cmd) = (\protected macro:->\@testopt \\cmd {default})
%  (\\cmd) = (\long macro:[#1]#2->)
%
%* etoolbox's \robustify:
%
%\newcommand {\cmd }[2][default]{} \robustify {\cmd }
%  (\cmd) = (\protected macro:->\@protected@testopt \cmd \\cmd {default})
%  (\\cmd) = (\long macro:[#1]#2->)
%
%\DeclareRobustCommand {\cmd }{} \robustify {\cmd }
%  (\cmd) = (\protected macro:->)
%
%\DeclareRobustCommand {\cmd }[2][default]{} \robustify {\cmd }
%  (\cmd) = (\protected macro:->\@protected@testopt \cmd~ \\cmd~ {default})
%  (\cmd~) = (macro:->\@protected@testopt \cmd~ \\cmd~ {default})
%  (\\cmd~) = (\long macro:[#1]#2->)
%\end{verbatim}
% \endgroup
%
% \subsection{Package}
%
%    \begin{macrocode}
%<*package>
%    \end{macrocode}
%
% \subsubsection{Catcodes and identification}
%
%    \begin{macrocode}
\begingroup\catcode61\catcode48\catcode32=10\relax%
  \catcode13=5 % ^^M
  \endlinechar=13 %
  \catcode123=1 % {
  \catcode125=2 % }
  \catcode64=11 % @
  \def\x{\endgroup
    \expandafter\edef\csname llm@AtEnd\endcsname{%
      \endlinechar=\the\endlinechar\relax
      \catcode13=\the\catcode13\relax
      \catcode32=\the\catcode32\relax
      \catcode35=\the\catcode35\relax
      \catcode61=\the\catcode61\relax
      \catcode64=\the\catcode64\relax
      \catcode123=\the\catcode123\relax
      \catcode125=\the\catcode125\relax
    }%
  }%
\x\catcode61\catcode48\catcode32=10\relax%
\catcode13=5 % ^^M
\endlinechar=13 %
\catcode35=6 % #
\catcode64=11 % @
\catcode123=1 % {
\catcode125=2 % }
\def\TMP@EnsureCode#1#2{%
  \edef\llm@AtEnd{%
    \llm@AtEnd
    \catcode#1=\the\catcode#1\relax
  }%
  \catcode#1=#2\relax
}
\TMP@EnsureCode{40}{12}% (
\TMP@EnsureCode{41}{12}% )
\TMP@EnsureCode{42}{12}% *
\TMP@EnsureCode{45}{12}% -
\TMP@EnsureCode{46}{12}% .
\TMP@EnsureCode{47}{12}% /
\TMP@EnsureCode{58}{12}% :
\TMP@EnsureCode{62}{12}% >
\TMP@EnsureCode{91}{12}% [
\TMP@EnsureCode{93}{12}% ]
\edef\llm@AtEnd{%
  \llm@AtEnd
  \escapechar\the\escapechar\relax
  \noexpand\endinput
}
\escapechar=92 % `\\
%    \end{macrocode}
%
%    Package identification.
%    \begin{macrocode}
\NeedsTeXFormat{LaTeX2e}
\ProvidesPackage{letltxmacro}%
  [2010/09/02 v1.4 Let assignment for LaTeX macros (HO)]
%    \end{macrocode}
%
% \subsubsection{Main macros}
%
%    \begin{macro}{\LetLtxMacro}
%    \begin{macrocode}
\newcommand*{\LetLtxMacro}{%
  \llm@ModeLetLtxMacro{}%
}
%    \end{macrocode}
%    \end{macro}
%    \begin{macro}{\GlobalLetLtxMacro}
%    \begin{macrocode}
\newcommand*{\GlobalLetLtxMacro}{%
  \llm@ModeLetLtxMacro\global
}
%    \end{macrocode}
%    \end{macro}
%
%    \begin{macro}{\llm@ModeLetLtxMacro}
%    \begin{macrocode}
\newcommand*{\llm@ModeLetLtxMacro}[3]{%
  \edef\llm@escapechar{\the\escapechar}%
  \escapechar=-1 %
  \edef\reserved@a{%
    \noexpand\protect
    \expandafter\noexpand
    \csname\string#3 \endcsname
  }%
  \ifx\reserved@a#3\relax
    #1\edef#2{%
      \noexpand\protect
      \expandafter\noexpand
      \csname\string#2 \endcsname
    }%
    #1\expandafter\let
    \csname\string#2 \expandafter\endcsname
    \csname\string#3 \endcsname
    \expandafter\llm@LetLtxMacro
        \csname\string#2 \expandafter\endcsname
        \csname\string#3 \endcsname{#1}%
  \else
    \llm@LetLtxMacro{#2}{#3}{#1}%
  \fi
  \escapechar=\llm@escapechar\relax
}
%    \end{macrocode}
%    \end{macro}
%    \begin{macro}{\llm@LetLtxMacro}
%    \begin{macrocode}
\def\llm@LetLtxMacro#1#2#3{%
  \escapechar=92 %
  \expandafter\llm@CheckParams\meaning#2:->\@nil{%
    \begingroup
      \def\@protected@testopt{%
        \expandafter\@testopt\@gobble
      }%
      \def\@testopt##1##2{%
        \toks@={##2}%
      }%
      \let\llm@testopt\@empty
      \edef\x{%
        \noexpand\@protected@testopt
        \noexpand#2%
        \expandafter\noexpand\csname\string#2\endcsname
      }%
      \expandafter\expandafter\expandafter\def
      \expandafter\expandafter\expandafter\y
      \expandafter\expandafter\expandafter{%
        \expandafter\llm@CarThree#2{}{}{}\llm@nil
      }%
      \ifx\x\y
        #2%
        \def\llm@testopt{%
          \noexpand\@protected@testopt
          \noexpand#1%
        }%
      \else
        \edef\x{%
          \noexpand\@testopt
          \expandafter\noexpand
          \csname\string#2\endcsname
        }%
        \expandafter\expandafter\expandafter\def
        \expandafter\expandafter\expandafter\y
        \expandafter\expandafter\expandafter{%
          \expandafter\llm@CarTwo#2{}{}\llm@nil
        }%
        \ifx\x\y
          #2%
          \def\llm@testopt{%
            \noexpand\@testopt
          }%
        \fi
      \fi
      \ifx\llm@testopt\@empty
      \else
        \llm@protected\xdef\llm@GlobalTemp{%
          \llm@testopt
          \expandafter\noexpand
          \csname\string#1\endcsname
          {\the\toks@}%
        }%
      \fi
    \expandafter\endgroup\ifx\llm@testopt\@empty
      #3\let#1=#2\relax
    \else
      #3\let#1=\llm@GlobalTemp
      #3\expandafter\let
          \csname\string#1\expandafter\endcsname
          \csname\string#2\endcsname
    \fi
  }{%
    #3\let#1=#2\relax
  }%
}
%    \end{macrocode}
%    \end{macro}
%    \begin{macro}{\llm@CheckParams}
%    \begin{macrocode}
\def\llm@CheckParams#1:->#2\@nil{%
  \begingroup
    \def\x{#1}%
  \ifx\x\llm@macro
    \endgroup
    \def\llm@protected{}%
    \expandafter\@firstoftwo
  \else
    \ifx\x\llm@protectedmacro
      \endgroup
      \def\llm@protected{\protected}%
      \expandafter\expandafter\expandafter\@firstoftwo
    \else
      \endgroup
      \expandafter\expandafter\expandafter\@secondoftwo
    \fi
  \fi
}
%    \end{macrocode}
%    \end{macro}
%    \begin{macro}{\llm@macro}
%    \begin{macrocode}
\def\llm@macro{macro}
\@onelevel@sanitize\llm@macro
%    \end{macrocode}
%    \end{macro}
%    \begin{macro}{\llm@protectedmacro}
%    \begin{macrocode}
\def\llm@protectedmacro{\protected macro}
\@onelevel@sanitize\llm@protectedmacro
%    \end{macrocode}
%    \end{macro}
%    \begin{macro}{\llm@CarThree}
%    \begin{macrocode}
\def\llm@CarThree#1#2#3#4\llm@nil{#1#2#3}%
%    \end{macrocode}
%    \end{macro}
%    \begin{macro}{\llm@CarTwo}
%    \begin{macrocode}
\def\llm@CarTwo#1#2#3\llm@nil{#1#2}%
%    \end{macrocode}
%    \end{macro}
%
%    \begin{macrocode}
\llm@AtEnd%
%</package>
%    \end{macrocode}
%
% \section{Test}
%
% \subsection{Catcode checks for loading}
%
%    \begin{macrocode}
%<*test1>
%    \end{macrocode}
%    \begin{macrocode}
\NeedsTeXFormat{LaTeX2e}
\documentclass{minimal}
\makeatletter
\def\RestoreCatcodes{}
\count@=0 %
\loop
  \edef\RestoreCatcodes{%
    \RestoreCatcodes
    \catcode\the\count@=\the\catcode\count@\relax
  }%
\ifnum\count@<255 %
  \advance\count@\@ne
\repeat

\def\RangeCatcodeInvalid#1#2{%
  \count@=#1\relax
  \loop
    \catcode\count@=15 %
  \ifnum\count@<#2\relax
    \advance\count@\@ne
  \repeat
}
\def\Test{%
  \RangeCatcodeInvalid{0}{47}%
  \RangeCatcodeInvalid{58}{64}%
  \RangeCatcodeInvalid{91}{96}%
  \RangeCatcodeInvalid{123}{127}%
  \catcode`\@=12 %
  \catcode`\\=0 %
  \catcode`\{=1 %
  \catcode`\}=2 %
  \catcode`\#=6 %
  \catcode`\[=12 %
  \catcode`\]=12 %
  \catcode`\%=14 %
  \catcode`\ =10 %
  \catcode13=5 %
  \RequirePackage{letltxmacro}[2010/09/02]\relax
  \RestoreCatcodes
}
\Test
\csname @@end\endcsname
\end
%    \end{macrocode}
%    \begin{macrocode}
%</test1>
%    \end{macrocode}
%
% \subsection{Package tests}
%
%    \begin{macrocode}
%<*test2>
%    \end{macrocode}
%    \begin{macrocode}
\NeedsTeXFormat{LaTeX2e}
\nofiles
\documentclass{minimal}

\usepackage{letltxmacro}[2010/09/02]

\usepackage{qstest}
\IncludeTests{*}
\LogTests{log}{*}{*}

\makeatletter

\def\TestDef#1{%
  \begingroup
    \@makeother\\%
    \@makeother\ %
    \expandafter\@TestDef\csname #1\endcsname
}
\def\@TestDef#1#2{%
    \edef\a{\expandafter\strip@prefix\meaning#1}%
    \edef\b{\detokenize{#2}}%
    \ifx\a\b
    \else
      \typeout{1 [\a]}%
      \typeout{2 [\b]}%
    \fi
  \endgroup
  \Expect*{\expandafter\strip@prefix\meaning#1}*{\detokenize{#2}}%
}
\def\TestEquals#1#2{%
  \Expect*{%
    \expandafter\ifx\csname#1\expandafter\endcsname
                    \csname#2\endcsname
      equals%
    \else
      wrong%
    \fi
  }{equals}%
}
\def\SaveA{%
  \let\SavedA\TestA
  \expandafter\let\csname\string\SavedA\expandafter\endcsname
                  \csname\string\TestA\endcsname
  \expandafter\let\csname SavedA \expandafter\endcsname
                  \csname TestA \endcsname
}
\def\CheckA{%
  \TestEquals{SavedA}{TestA}%
  \TestEquals{\string\SavedA}{\string\TestA}%
  \TestEquals{SavedA }{TestA }%
}

\begin{qstest}{robust}{robust}
  \begingroup
    \DeclareRobustCommand*{\TestA}{Test}%
    \SaveA
    \TestDef{TestA}{\protect \TestA  }%
    \CheckA
    \DeclareRobustCommand{\TestA}{Test}%
    \SaveA
    \TestDef{TestA}{\protect \TestA  }%
    \LetLtxMacro\TestB\TestA
    \TestDef{TestB}{\protect \TestB  }%
    \TestEquals{TestB }{TestA }%
    \CheckA
    \GlobalLetLtxMacro\TestC\TestA
    \CheckA
  \endgroup
  \TestDef{TestC}{\protect \TestC  }%
  \DeclareRobustCommand{\TestA}{Test}%
  \TestEquals{TestC }{TestA }%
\end{qstest}

\begin{qstest}{default}{default}
  \begingroup
    \newcommand{\TestA}[1][\relax default \empty]{TestA #1}%
    \SaveA
    \TestDef
    {TestA}{\@protected@testopt \TestA \\TestA {\relax default \empty }}%
    \LetLtxMacro\TestB\TestA
    \TestDef
    {TestB}{\@protected@testopt \TestB \\TestB {\relax default \empty }}%
    \CheckA
    \TestEquals{\string\TestB}{\string\TestA}%
    \GlobalLetLtxMacro\TestC\TestA
    \CheckA
  \endgroup
  \TestDef
  {TestC}{\@protected@testopt \TestC \\TestC {\relax default \empty }}%
  \newcommand{\TestA}[1][\relax default \empty]{TestA #1}%
  \TestEquals{\string\TestC}{\string\TestA}%
\end{qstest}

\begin{qstest}{robustdefault}{robustdefault}
  \begingroup
    \DeclareRobustCommand{\TestA}[1][\default]{TestA}%
    \SaveA
    \LetLtxMacro\TestB\TestA
    \TestDef{TestB}{\protect \TestB  }%
    \TestDef{TestB }{\@protected@testopt \TestB  \\TestB  {\default }}%
    \CheckA
    \GlobalLetLtxMacro\TestC\TestA
    \CheckA
  \endgroup
  \TestDef{TestC}{\protect \TestC  }%
  \TestDef{TestC }{\@protected@testopt \TestC  \\TestC  {\default }}%
\end{qstest}

\begin{qstest}{plain}{plain}
  \LetLtxMacro\NewRelax\relax
  \Expect*{\meaning\relax}*{\string\relax}%
  \Expect*{\meaning\NewRelax}*{\string\relax}%
  \LetLtxMacro\NewHbox\hbox
  \Expect*{\meaning\hbox}*{\string\hbox}%
  \Expect*{\meaning\NewHbox}*{\string\hbox}%
  \LetLtxMacro\NewEmpty\empty
  \Expect*{\meaning\empty}{macro:->}%
  \Expect*{\meaning\NewEmpty}{macro:->}%
  \def\TestA{\iffalse}%
  \LetLtxMacro\TestB\TestA
  \TestDef{TestB}{\iffalse }%
  \TestEquals{TestA}{TestB}%
\end{qstest}
%    \end{macrocode}
%
%    \begin{macrocode}
\usepackage{etoolbox}[2008/06/22]

\begin{qstest}{oldnewrobustcmd}{oldnewrobustcmd}
  \begingroup
    \protected\edef\TestA{%
      \noexpand\@protected@testopt
      \noexpand\TestA
      \expandafter\noexpand\csname\string\TestA\endcsname
      {\noexpand\default}%
    }%
    \expandafter\protected\expandafter
    \def\csname\string\TestA\endcsname{TestA}%
    \SaveA
    \LetLtxMacro\TestB\TestA
    \TestDef{TestB}{\@protected@testopt \TestB \\TestB {\default }}%
    \protected\edef\TestC{%
      \noexpand\@protected@testopt
      \noexpand\TestB
      \expandafter\noexpand\csname\string\TestB\endcsname
      {\noexpand\default}%
    }%
    \TestEquals{TestB}{TestC}%
    \CheckA
    \GlobalLetLtxMacro\TestD\TestA
    \CheckA
  \endgroup
  \TestDef{TestD}{\@protected@testopt \TestD \\TestD {\default }}%
  \protected\edef\TestC{%
    \noexpand\@protected@testopt
    \noexpand\TestD
    \expandafter\noexpand\csname\string\TestD\endcsname
    {\noexpand\default}%
  }%
  \TestEquals{TestD}{TestC}%
\end{qstest}

\begin{qstest}{newrobustcmd}{newrobustcmd}
  \begingroup
    \newrobustcmd{\TestA}[1][\default]{TestA}%
    \SaveA
    \LetLtxMacro\TestB\TestA
    \TestDef{TestB}{\@testopt \\TestB {\default }}%
    \protected\edef\TestC{%
      \noexpand\@testopt
      \expandafter\noexpand\csname\string\TestB\endcsname
      {\noexpand\default}%
    }%
    \TestEquals{TestB}{TestC}%
    \CheckA
    \GlobalLetLtxMacro\TestD\TestA
    \CheckA
  \endgroup
  \TestDef{TestD}{\@testopt \\TestD {\default }}%
  \protected\edef\TestC{%
    \noexpand\@testopt
    \expandafter\noexpand\csname\string\TestD\endcsname
    {\noexpand\default}%
  }%
  \TestEquals{TestD}{TestC}%
\end{qstest}

\begin{qstest}{robustifyopt}{robustifyopt}
  \newcommand{\TestA}[2][\default]{}%
  \TestDef{TestA}{\@protected@testopt \TestA \\TestA {\default }}%
  \robustify\TestA
  \TestDef{TestA}{\@protected@testopt \TestA \\TestA {\default }}%
  \protected\edef\TestC{%
    \noexpand\@protected@testopt
    \noexpand\TestA
    \expandafter\noexpand\csname\string\TestA\endcsname
    {\noexpand\default}%
  }%
  \TestEquals{TestA}{TestC}%
  \SaveA
  \LetLtxMacro\TestB\TestA
  \TestDef{TestB}{\@protected@testopt \TestB \\TestB {\default }}%
  \protected\edef\TestC{%
    \noexpand\@protected@testopt
    \noexpand\TestB
    \expandafter\noexpand\csname\string\TestB\endcsname
    {\noexpand\default}%
  }%
  \TestEquals{TestB}{TestC}%
  \CheckA
\end{qstest}

\begin{qstest}{robustifydeclare}{robustifydeclare}
  \DeclareRobustCommand\TestA{\iffalse}%
  \robustify\TestA
  \SaveA
  \LetLtxMacro\TestB\TestA
  \TestDef{TestB}{\iffalse }%
  % before etoolbox 2008/06/22: \protected\long\def
  % since etoolbox 2008/06/22: \protected\def
  \protected\def\TestC{\iffalse}%
  \TestEquals{TestB}{TestC}%
  \CheckA
\end{qstest}

\begin{qstest}{robustifydeclarelong}{robustifydeclarelong}
  \DeclareRobustCommand\TestA[1]{\iffalse}%
  \robustify\TestA
  \SaveA
  \LetLtxMacro\TestB\TestA
  \TestDef{TestB}{\iffalse }%
  % before etoolbox 2008/06/22: \protected\long\def
  % since etoolbox 2008/06/22: \protected\def
  \protected\long\def\TestC#1{\iffalse}%
  \TestEquals{TestB}{TestC}%
  \CheckA
\end{qstest}
%    \end{macrocode}
%
%    \begin{macrocode}
\csname @@end\endcsname
%    \end{macrocode}
%    \begin{macrocode}
%</test2>
%    \end{macrocode}
%
% \section{Installation}
%
% \subsection{Download}
%
% \paragraph{Package.} This package is available on
% CTAN\footnote{\url{ftp://ftp.ctan.org/tex-archive/}}:
% \begin{description}
% \item[\CTAN{macros/latex/contrib/oberdiek/letltxmacro.dtx}] The source file.
% \item[\CTAN{macros/latex/contrib/oberdiek/letltxmacro.pdf}] Documentation.
% \end{description}
%
%
% \paragraph{Bundle.} All the packages of the bundle `oberdiek'
% are also available in a TDS compliant ZIP archive. There
% the packages are already unpacked and the documentation files
% are generated. The files and directories obey the TDS standard.
% \begin{description}
% \item[\CTAN{install/macros/latex/contrib/oberdiek.tds.zip}]
% \end{description}
% \emph{TDS} refers to the standard ``A Directory Structure
% for \TeX\ Files'' (\CTAN{tds/tds.pdf}). Directories
% with \xfile{texmf} in their name are usually organized this way.
%
% \subsection{Bundle installation}
%
% \paragraph{Unpacking.} Unpack the \xfile{oberdiek.tds.zip} in the
% TDS tree (also known as \xfile{texmf} tree) of your choice.
% Example (linux):
% \begin{quote}
%   |unzip oberdiek.tds.zip -d ~/texmf|
% \end{quote}
%
% \paragraph{Script installation.}
% Check the directory \xfile{TDS:scripts/oberdiek/} for
% scripts that need further installation steps.
% Package \xpackage{attachfile2} comes with the Perl script
% \xfile{pdfatfi.pl} that should be installed in such a way
% that it can be called as \texttt{pdfatfi}.
% Example (linux):
% \begin{quote}
%   |chmod +x scripts/oberdiek/pdfatfi.pl|\\
%   |cp scripts/oberdiek/pdfatfi.pl /usr/local/bin/|
% \end{quote}
%
% \subsection{Package installation}
%
% \paragraph{Unpacking.} The \xfile{.dtx} file is a self-extracting
% \docstrip\ archive. The files are extracted by running the
% \xfile{.dtx} through \plainTeX:
% \begin{quote}
%   \verb|tex letltxmacro.dtx|
% \end{quote}
%
% \paragraph{TDS.} Now the different files must be moved into
% the different directories in your installation TDS tree
% (also known as \xfile{texmf} tree):
% \begin{quote}
% \def\t{^^A
% \begin{tabular}{@{}>{\ttfamily}l@{ $\rightarrow$ }>{\ttfamily}l@{}}
%   letltxmacro.sty & tex/latex/oberdiek/letltxmacro.sty\\
%   letltxmacro.pdf & doc/latex/oberdiek/letltxmacro.pdf\\
%   letltxmacro-showcases.tex & doc/latex/oberdiek/letltxmacro-showcases.tex\\
%   test/letltxmacro-test1.tex & doc/latex/oberdiek/test/letltxmacro-test1.tex\\
%   test/letltxmacro-test2.tex & doc/latex/oberdiek/test/letltxmacro-test2.tex\\
%   letltxmacro.dtx & source/latex/oberdiek/letltxmacro.dtx\\
% \end{tabular}^^A
% }^^A
% \sbox0{\t}^^A
% \ifdim\wd0>\linewidth
%   \begingroup
%     \advance\linewidth by\leftmargin
%     \advance\linewidth by\rightmargin
%   \edef\x{\endgroup
%     \def\noexpand\lw{\the\linewidth}^^A
%   }\x
%   \def\lwbox{^^A
%     \leavevmode
%     \hbox to \linewidth{^^A
%       \kern-\leftmargin\relax
%       \hss
%       \usebox0
%       \hss
%       \kern-\rightmargin\relax
%     }^^A
%   }^^A
%   \ifdim\wd0>\lw
%     \sbox0{\small\t}^^A
%     \ifdim\wd0>\linewidth
%       \ifdim\wd0>\lw
%         \sbox0{\footnotesize\t}^^A
%         \ifdim\wd0>\linewidth
%           \ifdim\wd0>\lw
%             \sbox0{\scriptsize\t}^^A
%             \ifdim\wd0>\linewidth
%               \ifdim\wd0>\lw
%                 \sbox0{\tiny\t}^^A
%                 \ifdim\wd0>\linewidth
%                   \lwbox
%                 \else
%                   \usebox0
%                 \fi
%               \else
%                 \lwbox
%               \fi
%             \else
%               \usebox0
%             \fi
%           \else
%             \lwbox
%           \fi
%         \else
%           \usebox0
%         \fi
%       \else
%         \lwbox
%       \fi
%     \else
%       \usebox0
%     \fi
%   \else
%     \lwbox
%   \fi
% \else
%   \usebox0
% \fi
% \end{quote}
% If you have a \xfile{docstrip.cfg} that configures and enables \docstrip's
% TDS installing feature, then some files can already be in the right
% place, see the documentation of \docstrip.
%
% \subsection{Refresh file name databases}
%
% If your \TeX~distribution
% (\teTeX, \mikTeX, \dots) relies on file name databases, you must refresh
% these. For example, \teTeX\ users run \verb|texhash| or
% \verb|mktexlsr|.
%
% \subsection{Some details for the interested}
%
% \paragraph{Attached source.}
%
% The PDF documentation on CTAN also includes the
% \xfile{.dtx} source file. It can be extracted by
% AcrobatReader 6 or higher. Another option is \textsf{pdftk},
% e.g. unpack the file into the current directory:
% \begin{quote}
%   \verb|pdftk letltxmacro.pdf unpack_files output .|
% \end{quote}
%
% \paragraph{Unpacking with \LaTeX.}
% The \xfile{.dtx} chooses its action depending on the format:
% \begin{description}
% \item[\plainTeX:] Run \docstrip\ and extract the files.
% \item[\LaTeX:] Generate the documentation.
% \end{description}
% If you insist on using \LaTeX\ for \docstrip\ (really,
% \docstrip\ does not need \LaTeX), then inform the autodetect routine
% about your intention:
% \begin{quote}
%   \verb|latex \let\install=y% \iffalse meta-comment
%
% File: letltxmacro.dtx
% Version: 2010/09/02 v1.4
% Info: Let assignment for LaTeX macros
%
% Copyright (C) 2008, 2010 by
%    Heiko Oberdiek <heiko.oberdiek at googlemail.com>
%
% This work may be distributed and/or modified under the
% conditions of the LaTeX Project Public License, either
% version 1.3c of this license or (at your option) any later
% version. This version of this license is in
%    http://www.latex-project.org/lppl/lppl-1-3c.txt
% and the latest version of this license is in
%    http://www.latex-project.org/lppl.txt
% and version 1.3 or later is part of all distributions of
% LaTeX version 2005/12/01 or later.
%
% This work has the LPPL maintenance status "maintained".
%
% This Current Maintainer of this work is Heiko Oberdiek.
%
% This work consists of the main source file letltxmacro.dtx
% and the derived files
%    letltxmacro.sty, letltxmacro.pdf, letltxmacro.ins, letltxmacro.drv,
%    letltxmacro-showcases.tex, letltxmacro-test1.tex,
%    letltxmacro-test2.tex.
%
% Distribution:
%    CTAN:macros/latex/contrib/oberdiek/letltxmacro.dtx
%    CTAN:macros/latex/contrib/oberdiek/letltxmacro.pdf
%
% Unpacking:
%    (a) If letltxmacro.ins is present:
%           tex letltxmacro.ins
%    (b) Without letltxmacro.ins:
%           tex letltxmacro.dtx
%    (c) If you insist on using LaTeX
%           latex \let\install=y% \iffalse meta-comment
%
% File: letltxmacro.dtx
% Version: 2010/09/02 v1.4
% Info: Let assignment for LaTeX macros
%
% Copyright (C) 2008, 2010 by
%    Heiko Oberdiek <heiko.oberdiek at googlemail.com>
%
% This work may be distributed and/or modified under the
% conditions of the LaTeX Project Public License, either
% version 1.3c of this license or (at your option) any later
% version. This version of this license is in
%    http://www.latex-project.org/lppl/lppl-1-3c.txt
% and the latest version of this license is in
%    http://www.latex-project.org/lppl.txt
% and version 1.3 or later is part of all distributions of
% LaTeX version 2005/12/01 or later.
%
% This work has the LPPL maintenance status "maintained".
%
% This Current Maintainer of this work is Heiko Oberdiek.
%
% This work consists of the main source file letltxmacro.dtx
% and the derived files
%    letltxmacro.sty, letltxmacro.pdf, letltxmacro.ins, letltxmacro.drv,
%    letltxmacro-showcases.tex, letltxmacro-test1.tex,
%    letltxmacro-test2.tex.
%
% Distribution:
%    CTAN:macros/latex/contrib/oberdiek/letltxmacro.dtx
%    CTAN:macros/latex/contrib/oberdiek/letltxmacro.pdf
%
% Unpacking:
%    (a) If letltxmacro.ins is present:
%           tex letltxmacro.ins
%    (b) Without letltxmacro.ins:
%           tex letltxmacro.dtx
%    (c) If you insist on using LaTeX
%           latex \let\install=y\input{letltxmacro.dtx}
%        (quote the arguments according to the demands of your shell)
%
% Documentation:
%    (a) If letltxmacro.drv is present:
%           latex letltxmacro.drv
%    (b) Without letltxmacro.drv:
%           latex letltxmacro.dtx; ...
%    The class ltxdoc loads the configuration file ltxdoc.cfg
%    if available. Here you can specify further options, e.g.
%    use A4 as paper format:
%       \PassOptionsToClass{a4paper}{article}
%
%    Programm calls to get the documentation (example):
%       pdflatex letltxmacro.dtx
%       makeindex -s gind.ist letltxmacro.idx
%       pdflatex letltxmacro.dtx
%       makeindex -s gind.ist letltxmacro.idx
%       pdflatex letltxmacro.dtx
%
% Installation:
%    TDS:tex/latex/oberdiek/letltxmacro.sty
%    TDS:doc/latex/oberdiek/letltxmacro.pdf
%    TDS:doc/latex/oberdiek/letltxmacro-showcases.tex
%    TDS:doc/latex/oberdiek/test/letltxmacro-test1.tex
%    TDS:doc/latex/oberdiek/test/letltxmacro-test2.tex
%    TDS:source/latex/oberdiek/letltxmacro.dtx
%
%<*ignore>
\begingroup
  \catcode123=1 %
  \catcode125=2 %
  \def\x{LaTeX2e}%
\expandafter\endgroup
\ifcase 0\ifx\install y1\fi\expandafter
         \ifx\csname processbatchFile\endcsname\relax\else1\fi
         \ifx\fmtname\x\else 1\fi\relax
\else\csname fi\endcsname
%</ignore>
%<*install>
\input docstrip.tex
\Msg{************************************************************************}
\Msg{* Installation}
\Msg{* Package: letltxmacro 2010/09/02 v1.4 Let assignment for LaTeX macros (HO)}
\Msg{************************************************************************}

\keepsilent
\askforoverwritefalse

\let\MetaPrefix\relax
\preamble

This is a generated file.

Project: letltxmacro
Version: 2010/09/02 v1.4

Copyright (C) 2008, 2010 by
   Heiko Oberdiek <heiko.oberdiek at googlemail.com>

This work may be distributed and/or modified under the
conditions of the LaTeX Project Public License, either
version 1.3c of this license or (at your option) any later
version. This version of this license is in
   http://www.latex-project.org/lppl/lppl-1-3c.txt
and the latest version of this license is in
   http://www.latex-project.org/lppl.txt
and version 1.3 or later is part of all distributions of
LaTeX version 2005/12/01 or later.

This work has the LPPL maintenance status "maintained".

This Current Maintainer of this work is Heiko Oberdiek.

This work consists of the main source file letltxmacro.dtx
and the derived files
   letltxmacro.sty, letltxmacro.pdf, letltxmacro.ins, letltxmacro.drv,
   letltxmacro-showcases.tex, letltxmacro-test1.tex,
   letltxmacro-test2.tex.

\endpreamble
\let\MetaPrefix\DoubleperCent

\generate{%
  \file{letltxmacro.ins}{\from{letltxmacro.dtx}{install}}%
  \file{letltxmacro.drv}{\from{letltxmacro.dtx}{driver}}%
  \usedir{tex/latex/oberdiek}%
  \file{letltxmacro.sty}{\from{letltxmacro.dtx}{package}}%
  \usedir{doc/latex/oberdiek}%
  \file{letltxmacro-showcases.tex}{\from{letltxmacro.dtx}{showcases}}%
  \usedir{doc/latex/oberdiek/test}%
  \file{letltxmacro-test1.tex}{\from{letltxmacro.dtx}{test1}}%
  \file{letltxmacro-test2.tex}{\from{letltxmacro.dtx}{test2}}%
  \nopreamble
  \nopostamble
  \usedir{source/latex/oberdiek/catalogue}%
  \file{letltxmacro.xml}{\from{letltxmacro.dtx}{catalogue}}%
}

\catcode32=13\relax% active space
\let =\space%
\Msg{************************************************************************}
\Msg{*}
\Msg{* To finish the installation you have to move the following}
\Msg{* file into a directory searched by TeX:}
\Msg{*}
\Msg{*     letltxmacro.sty}
\Msg{*}
\Msg{* To produce the documentation run the file `letltxmacro.drv'}
\Msg{* through LaTeX.}
\Msg{*}
\Msg{* Happy TeXing!}
\Msg{*}
\Msg{************************************************************************}

\endbatchfile
%</install>
%<*ignore>
\fi
%</ignore>
%<*driver>
\NeedsTeXFormat{LaTeX2e}
\ProvidesFile{letltxmacro.drv}%
  [2010/09/02 v1.4 Let assignment for LaTeX macros (HO)]%
\documentclass{ltxdoc}
\usepackage{holtxdoc}[2011/11/22]
\begin{document}
  \DocInput{letltxmacro.dtx}%
\end{document}
%</driver>
% \fi
%
% \CheckSum{990}
%
% \CharacterTable
%  {Upper-case    \A\B\C\D\E\F\G\H\I\J\K\L\M\N\O\P\Q\R\S\T\U\V\W\X\Y\Z
%   Lower-case    \a\b\c\d\e\f\g\h\i\j\k\l\m\n\o\p\q\r\s\t\u\v\w\x\y\z
%   Digits        \0\1\2\3\4\5\6\7\8\9
%   Exclamation   \!     Double quote  \"     Hash (number) \#
%   Dollar        \$     Percent       \%     Ampersand     \&
%   Acute accent  \'     Left paren    \(     Right paren   \)
%   Asterisk      \*     Plus          \+     Comma         \,
%   Minus         \-     Point         \.     Solidus       \/
%   Colon         \:     Semicolon     \;     Less than     \<
%   Equals        \=     Greater than  \>     Question mark \?
%   Commercial at \@     Left bracket  \[     Backslash     \\
%   Right bracket \]     Circumflex    \^     Underscore    \_
%   Grave accent  \`     Left brace    \{     Vertical bar  \|
%   Right brace   \}     Tilde         \~}
%
% \GetFileInfo{letltxmacro.drv}
%
% \title{The \xpackage{letltxmacro} package}
% \date{2010/09/02 v1.4}
% \author{Heiko Oberdiek\\\xemail{heiko.oberdiek at googlemail.com}}
%
% \maketitle
%
% \begin{abstract}
% \TeX's \cs{let} assignment does not work for \LaTeX\ macros
% with optional arguments or for macros that are defined
% as robust macros by \cs{DeclareRobustCommand}. This package
% defines \cs{LetLtxMacro} that also takes care of the involved
% internal macros.
% \end{abstract}
%
% \tableofcontents
%
% \section{Documentation}
%
% If someone wants to redefine a macro with using the old
% meaning, then one method is \TeX's command \cs{let}:
%\begin{quote}
%\begin{verbatim}
%\newcommand{\Macro}{\typeout{Test Macro}}
%\let\SavedMacro=\Macro
%\renewcommand{\Macro}{%
%  \typeout{Begin}%
%  \SavedMacro
%  \typeout{End}%
%}
%\end{verbatim}
%\end{quote}
% However, this method fails, if \cs{Macro} is defined
% by \cs{DeclareRobustCommand} and/or has an optional argument.
% In both cases \LaTeX\ defines an additional internal macro
% that is forgotten in the simple \cs{let} assignment of
% the example above.
%
% \begin{declcs}{LetLtxMacro} \M{new macro} \M{old macro}
% \end{declcs}
% Macro \cs{LetLtxMacro} behaves similar to \TeX's \cs{let}
% assignment, but it takes care of macros that are
% defined by \cs{DeclareRobustCommand} and/or have optional
% arguments. Example:
%\begin{quote}
%\begin{verbatim}
%\DeclareRobustCommand{\Macro}[1][default]{...}
%\LetLtxMacro{\SavedMacro}{\Macro}
%\end{verbatim}
%\end{quote}
% Then macro \cs{SavedMacro} only uses internal macro names
% that are derived from \cs{SavedMacro}'s macro name. Macro \cs{Macro}
% can now be redefined without affecting \cs{SavedMacro}.
%
% \begin{declcs}{GlobalLetLtxMacro} \M{new macro} \M{old macro}
% \end{declcs}
% Like \cs{LetLtxMacro}, but the \meta{new macro} is defined globally.
% Since version 2010/09/02~v1.4.
%
% \subsection{Supported macro definition commands}
%
% \begin{quote}
%   \begin{tabular}{@{}ll@{}}
%     \cs{newcommand}, \cs{renewcommand} & latex/base\\
%     \cs{newenvironment}, \cs{renewenvironment} & latex/base\\
%     \cs{DeclareRobustCommand}& latex/base\\
%     \cs{newrobustcmd}, \cs{renewrobustcmd} & etoolbox\\
%     \cs{robustify} & etoolbox 2008/06/22 v1.6\\
%   \end{tabular}
% \end{quote}
%
% \StopEventually{
% }
%
% \section{Implementation}
%
% \subsection{Show cases}
%
% \subsubsection{\xfile{letltxmacro-showcases.tex}}
%
%    \begin{macrocode}
%<*showcases>
\NeedsTeXFormat{LaTeX2e}
\makeatletter
%    \end{macrocode}
%    \begin{macro}{\Line}
%    The result is displayed by macro \cs{Line}. The percent symbol
%    at line start allows easy grepping and inserting into the DTX
%    file.
%    \begin{macrocode}
\newcommand*{\Line}[1]{%
  \typeout{\@percentchar#1}%
}
%    \end{macrocode}
%    \end{macro}
%    \begin{macrocode}
\newcommand*{\ShowCmdName}[1]{%
  \@ifundefined{#1}{}{%
    \Line{%
      \space\space(\expandafter\string\csname#1\endcsname) = %
      (\expandafter\meaning\csname#1\endcsname)%
    }%
  }%
}
\newcommand*{\ShowCmds}[1]{%
  \ShowCmdName{#1}%
  \ShowCmdName{#1 }%
  \ShowCmdName{\\#1}%
  \ShowCmdName{\\#1 }%
}
\let\\\@backslashchar
%    \end{macrocode}
%    \begin{macro}{\ShowDef}
%    \begin{macrocode}
\newcommand*{\ShowDef}[2]{%
  \begingroup
    \Line{}%
    \newcommand*{\DefString}{#2}%
    \@onelevel@sanitize\DefString
    \Line{\DefString}%
    #2%
    \ShowCmds{#1}%
  \endgroup
}
%    \end{macrocode}
%    \end{macro}
%    \begin{macrocode}
\typeout{}
\Line{* LaTeX definitions:}
\ShowDef{cmd}{%
  \newcommand{\cmd}[2][default]{}%
}
\ShowDef{cmd}{%
  \DeclareRobustCommand{\cmd}{}%
}
\ShowDef{cmd}{%
  \DeclareRobustCommand{\cmd}[2][default]{}%
}
\typeout{}
%    \end{macrocode}
% The minimal version of package \xpackage{etoolbox} is 2008/06/12 v1.6a
% because it fixes \cs{robustify}.
%    \begin{macrocode}
\RequirePackage{etoolbox}[2008/06/12]%
\Line{}
\Line{* etoolbox's robust definitions:}
\ShowDef{cmd}{%
  \newrobustcmd{\cmd}{}%
}
\ShowDef{cmd}{%
  \newrobustcmd{\cmd}[2][default]{}%
}
\Line{}
\Line{* etoolbox's \string\robustify:}
\ShowDef{cmd}{%
  \newcommand{\cmd}[2][default]{} %
  \robustify{\cmd}%
}
\ShowDef{cmd}{%
  \DeclareRobustCommand{\cmd}{} %
  \robustify{\cmd}%
}
\ShowDef{cmd}{%
  \DeclareRobustCommand{\cmd}[2][default]{} %
  \robustify{\cmd}%
}
\typeout{}
\@@end
%</showcases>
%    \end{macrocode}
%
% \subsubsection{Result}
%
% \begingroup
%   \makeatletter
%   \let\org@verbatim\@verbatim
%   \def\@verbatim{^^A
%     \org@verbatim
%     \catcode`\~=\active
%   }^^A
%   \let~\textvisiblespace
%\begin{verbatim}
%* LaTeX definitions:
%
%\newcommand {\cmd }[2][default]{}
%  (\cmd) = (macro:->\@protected@testopt \cmd \\cmd {default})
%  (\\cmd) = (\long macro:[#1]#2->)
%
%\DeclareRobustCommand {\cmd }{}
%  (\cmd) = (macro:->\protect \cmd~ )
%  (\cmd~) = (\long macro:->)
%
%\DeclareRobustCommand {\cmd }[2][default]{}
%  (\cmd) = (macro:->\protect \cmd~ )
%  (\cmd~) = (macro:->\@protected@testopt \cmd~ \\cmd~ {default})
%  (\\cmd~) = (\long macro:[#1]#2->)
%
%* etoolbox's robust definitions:
%
%\newrobustcmd {\cmd }{}
%  (\cmd) = (\protected\long macro:->)
%
%\newrobustcmd {\cmd }[2][default]{}
%  (\cmd) = (\protected macro:->\@testopt \\cmd {default})
%  (\\cmd) = (\long macro:[#1]#2->)
%
%* etoolbox's \robustify:
%
%\newcommand {\cmd }[2][default]{} \robustify {\cmd }
%  (\cmd) = (\protected macro:->\@protected@testopt \cmd \\cmd {default})
%  (\\cmd) = (\long macro:[#1]#2->)
%
%\DeclareRobustCommand {\cmd }{} \robustify {\cmd }
%  (\cmd) = (\protected macro:->)
%
%\DeclareRobustCommand {\cmd }[2][default]{} \robustify {\cmd }
%  (\cmd) = (\protected macro:->\@protected@testopt \cmd~ \\cmd~ {default})
%  (\cmd~) = (macro:->\@protected@testopt \cmd~ \\cmd~ {default})
%  (\\cmd~) = (\long macro:[#1]#2->)
%\end{verbatim}
% \endgroup
%
% \subsection{Package}
%
%    \begin{macrocode}
%<*package>
%    \end{macrocode}
%
% \subsubsection{Catcodes and identification}
%
%    \begin{macrocode}
\begingroup\catcode61\catcode48\catcode32=10\relax%
  \catcode13=5 % ^^M
  \endlinechar=13 %
  \catcode123=1 % {
  \catcode125=2 % }
  \catcode64=11 % @
  \def\x{\endgroup
    \expandafter\edef\csname llm@AtEnd\endcsname{%
      \endlinechar=\the\endlinechar\relax
      \catcode13=\the\catcode13\relax
      \catcode32=\the\catcode32\relax
      \catcode35=\the\catcode35\relax
      \catcode61=\the\catcode61\relax
      \catcode64=\the\catcode64\relax
      \catcode123=\the\catcode123\relax
      \catcode125=\the\catcode125\relax
    }%
  }%
\x\catcode61\catcode48\catcode32=10\relax%
\catcode13=5 % ^^M
\endlinechar=13 %
\catcode35=6 % #
\catcode64=11 % @
\catcode123=1 % {
\catcode125=2 % }
\def\TMP@EnsureCode#1#2{%
  \edef\llm@AtEnd{%
    \llm@AtEnd
    \catcode#1=\the\catcode#1\relax
  }%
  \catcode#1=#2\relax
}
\TMP@EnsureCode{40}{12}% (
\TMP@EnsureCode{41}{12}% )
\TMP@EnsureCode{42}{12}% *
\TMP@EnsureCode{45}{12}% -
\TMP@EnsureCode{46}{12}% .
\TMP@EnsureCode{47}{12}% /
\TMP@EnsureCode{58}{12}% :
\TMP@EnsureCode{62}{12}% >
\TMP@EnsureCode{91}{12}% [
\TMP@EnsureCode{93}{12}% ]
\edef\llm@AtEnd{%
  \llm@AtEnd
  \escapechar\the\escapechar\relax
  \noexpand\endinput
}
\escapechar=92 % `\\
%    \end{macrocode}
%
%    Package identification.
%    \begin{macrocode}
\NeedsTeXFormat{LaTeX2e}
\ProvidesPackage{letltxmacro}%
  [2010/09/02 v1.4 Let assignment for LaTeX macros (HO)]
%    \end{macrocode}
%
% \subsubsection{Main macros}
%
%    \begin{macro}{\LetLtxMacro}
%    \begin{macrocode}
\newcommand*{\LetLtxMacro}{%
  \llm@ModeLetLtxMacro{}%
}
%    \end{macrocode}
%    \end{macro}
%    \begin{macro}{\GlobalLetLtxMacro}
%    \begin{macrocode}
\newcommand*{\GlobalLetLtxMacro}{%
  \llm@ModeLetLtxMacro\global
}
%    \end{macrocode}
%    \end{macro}
%
%    \begin{macro}{\llm@ModeLetLtxMacro}
%    \begin{macrocode}
\newcommand*{\llm@ModeLetLtxMacro}[3]{%
  \edef\llm@escapechar{\the\escapechar}%
  \escapechar=-1 %
  \edef\reserved@a{%
    \noexpand\protect
    \expandafter\noexpand
    \csname\string#3 \endcsname
  }%
  \ifx\reserved@a#3\relax
    #1\edef#2{%
      \noexpand\protect
      \expandafter\noexpand
      \csname\string#2 \endcsname
    }%
    #1\expandafter\let
    \csname\string#2 \expandafter\endcsname
    \csname\string#3 \endcsname
    \expandafter\llm@LetLtxMacro
        \csname\string#2 \expandafter\endcsname
        \csname\string#3 \endcsname{#1}%
  \else
    \llm@LetLtxMacro{#2}{#3}{#1}%
  \fi
  \escapechar=\llm@escapechar\relax
}
%    \end{macrocode}
%    \end{macro}
%    \begin{macro}{\llm@LetLtxMacro}
%    \begin{macrocode}
\def\llm@LetLtxMacro#1#2#3{%
  \escapechar=92 %
  \expandafter\llm@CheckParams\meaning#2:->\@nil{%
    \begingroup
      \def\@protected@testopt{%
        \expandafter\@testopt\@gobble
      }%
      \def\@testopt##1##2{%
        \toks@={##2}%
      }%
      \let\llm@testopt\@empty
      \edef\x{%
        \noexpand\@protected@testopt
        \noexpand#2%
        \expandafter\noexpand\csname\string#2\endcsname
      }%
      \expandafter\expandafter\expandafter\def
      \expandafter\expandafter\expandafter\y
      \expandafter\expandafter\expandafter{%
        \expandafter\llm@CarThree#2{}{}{}\llm@nil
      }%
      \ifx\x\y
        #2%
        \def\llm@testopt{%
          \noexpand\@protected@testopt
          \noexpand#1%
        }%
      \else
        \edef\x{%
          \noexpand\@testopt
          \expandafter\noexpand
          \csname\string#2\endcsname
        }%
        \expandafter\expandafter\expandafter\def
        \expandafter\expandafter\expandafter\y
        \expandafter\expandafter\expandafter{%
          \expandafter\llm@CarTwo#2{}{}\llm@nil
        }%
        \ifx\x\y
          #2%
          \def\llm@testopt{%
            \noexpand\@testopt
          }%
        \fi
      \fi
      \ifx\llm@testopt\@empty
      \else
        \llm@protected\xdef\llm@GlobalTemp{%
          \llm@testopt
          \expandafter\noexpand
          \csname\string#1\endcsname
          {\the\toks@}%
        }%
      \fi
    \expandafter\endgroup\ifx\llm@testopt\@empty
      #3\let#1=#2\relax
    \else
      #3\let#1=\llm@GlobalTemp
      #3\expandafter\let
          \csname\string#1\expandafter\endcsname
          \csname\string#2\endcsname
    \fi
  }{%
    #3\let#1=#2\relax
  }%
}
%    \end{macrocode}
%    \end{macro}
%    \begin{macro}{\llm@CheckParams}
%    \begin{macrocode}
\def\llm@CheckParams#1:->#2\@nil{%
  \begingroup
    \def\x{#1}%
  \ifx\x\llm@macro
    \endgroup
    \def\llm@protected{}%
    \expandafter\@firstoftwo
  \else
    \ifx\x\llm@protectedmacro
      \endgroup
      \def\llm@protected{\protected}%
      \expandafter\expandafter\expandafter\@firstoftwo
    \else
      \endgroup
      \expandafter\expandafter\expandafter\@secondoftwo
    \fi
  \fi
}
%    \end{macrocode}
%    \end{macro}
%    \begin{macro}{\llm@macro}
%    \begin{macrocode}
\def\llm@macro{macro}
\@onelevel@sanitize\llm@macro
%    \end{macrocode}
%    \end{macro}
%    \begin{macro}{\llm@protectedmacro}
%    \begin{macrocode}
\def\llm@protectedmacro{\protected macro}
\@onelevel@sanitize\llm@protectedmacro
%    \end{macrocode}
%    \end{macro}
%    \begin{macro}{\llm@CarThree}
%    \begin{macrocode}
\def\llm@CarThree#1#2#3#4\llm@nil{#1#2#3}%
%    \end{macrocode}
%    \end{macro}
%    \begin{macro}{\llm@CarTwo}
%    \begin{macrocode}
\def\llm@CarTwo#1#2#3\llm@nil{#1#2}%
%    \end{macrocode}
%    \end{macro}
%
%    \begin{macrocode}
\llm@AtEnd%
%</package>
%    \end{macrocode}
%
% \section{Test}
%
% \subsection{Catcode checks for loading}
%
%    \begin{macrocode}
%<*test1>
%    \end{macrocode}
%    \begin{macrocode}
\NeedsTeXFormat{LaTeX2e}
\documentclass{minimal}
\makeatletter
\def\RestoreCatcodes{}
\count@=0 %
\loop
  \edef\RestoreCatcodes{%
    \RestoreCatcodes
    \catcode\the\count@=\the\catcode\count@\relax
  }%
\ifnum\count@<255 %
  \advance\count@\@ne
\repeat

\def\RangeCatcodeInvalid#1#2{%
  \count@=#1\relax
  \loop
    \catcode\count@=15 %
  \ifnum\count@<#2\relax
    \advance\count@\@ne
  \repeat
}
\def\Test{%
  \RangeCatcodeInvalid{0}{47}%
  \RangeCatcodeInvalid{58}{64}%
  \RangeCatcodeInvalid{91}{96}%
  \RangeCatcodeInvalid{123}{127}%
  \catcode`\@=12 %
  \catcode`\\=0 %
  \catcode`\{=1 %
  \catcode`\}=2 %
  \catcode`\#=6 %
  \catcode`\[=12 %
  \catcode`\]=12 %
  \catcode`\%=14 %
  \catcode`\ =10 %
  \catcode13=5 %
  \RequirePackage{letltxmacro}[2010/09/02]\relax
  \RestoreCatcodes
}
\Test
\csname @@end\endcsname
\end
%    \end{macrocode}
%    \begin{macrocode}
%</test1>
%    \end{macrocode}
%
% \subsection{Package tests}
%
%    \begin{macrocode}
%<*test2>
%    \end{macrocode}
%    \begin{macrocode}
\NeedsTeXFormat{LaTeX2e}
\nofiles
\documentclass{minimal}

\usepackage{letltxmacro}[2010/09/02]

\usepackage{qstest}
\IncludeTests{*}
\LogTests{log}{*}{*}

\makeatletter

\def\TestDef#1{%
  \begingroup
    \@makeother\\%
    \@makeother\ %
    \expandafter\@TestDef\csname #1\endcsname
}
\def\@TestDef#1#2{%
    \edef\a{\expandafter\strip@prefix\meaning#1}%
    \edef\b{\detokenize{#2}}%
    \ifx\a\b
    \else
      \typeout{1 [\a]}%
      \typeout{2 [\b]}%
    \fi
  \endgroup
  \Expect*{\expandafter\strip@prefix\meaning#1}*{\detokenize{#2}}%
}
\def\TestEquals#1#2{%
  \Expect*{%
    \expandafter\ifx\csname#1\expandafter\endcsname
                    \csname#2\endcsname
      equals%
    \else
      wrong%
    \fi
  }{equals}%
}
\def\SaveA{%
  \let\SavedA\TestA
  \expandafter\let\csname\string\SavedA\expandafter\endcsname
                  \csname\string\TestA\endcsname
  \expandafter\let\csname SavedA \expandafter\endcsname
                  \csname TestA \endcsname
}
\def\CheckA{%
  \TestEquals{SavedA}{TestA}%
  \TestEquals{\string\SavedA}{\string\TestA}%
  \TestEquals{SavedA }{TestA }%
}

\begin{qstest}{robust}{robust}
  \begingroup
    \DeclareRobustCommand*{\TestA}{Test}%
    \SaveA
    \TestDef{TestA}{\protect \TestA  }%
    \CheckA
    \DeclareRobustCommand{\TestA}{Test}%
    \SaveA
    \TestDef{TestA}{\protect \TestA  }%
    \LetLtxMacro\TestB\TestA
    \TestDef{TestB}{\protect \TestB  }%
    \TestEquals{TestB }{TestA }%
    \CheckA
    \GlobalLetLtxMacro\TestC\TestA
    \CheckA
  \endgroup
  \TestDef{TestC}{\protect \TestC  }%
  \DeclareRobustCommand{\TestA}{Test}%
  \TestEquals{TestC }{TestA }%
\end{qstest}

\begin{qstest}{default}{default}
  \begingroup
    \newcommand{\TestA}[1][\relax default \empty]{TestA #1}%
    \SaveA
    \TestDef
    {TestA}{\@protected@testopt \TestA \\TestA {\relax default \empty }}%
    \LetLtxMacro\TestB\TestA
    \TestDef
    {TestB}{\@protected@testopt \TestB \\TestB {\relax default \empty }}%
    \CheckA
    \TestEquals{\string\TestB}{\string\TestA}%
    \GlobalLetLtxMacro\TestC\TestA
    \CheckA
  \endgroup
  \TestDef
  {TestC}{\@protected@testopt \TestC \\TestC {\relax default \empty }}%
  \newcommand{\TestA}[1][\relax default \empty]{TestA #1}%
  \TestEquals{\string\TestC}{\string\TestA}%
\end{qstest}

\begin{qstest}{robustdefault}{robustdefault}
  \begingroup
    \DeclareRobustCommand{\TestA}[1][\default]{TestA}%
    \SaveA
    \LetLtxMacro\TestB\TestA
    \TestDef{TestB}{\protect \TestB  }%
    \TestDef{TestB }{\@protected@testopt \TestB  \\TestB  {\default }}%
    \CheckA
    \GlobalLetLtxMacro\TestC\TestA
    \CheckA
  \endgroup
  \TestDef{TestC}{\protect \TestC  }%
  \TestDef{TestC }{\@protected@testopt \TestC  \\TestC  {\default }}%
\end{qstest}

\begin{qstest}{plain}{plain}
  \LetLtxMacro\NewRelax\relax
  \Expect*{\meaning\relax}*{\string\relax}%
  \Expect*{\meaning\NewRelax}*{\string\relax}%
  \LetLtxMacro\NewHbox\hbox
  \Expect*{\meaning\hbox}*{\string\hbox}%
  \Expect*{\meaning\NewHbox}*{\string\hbox}%
  \LetLtxMacro\NewEmpty\empty
  \Expect*{\meaning\empty}{macro:->}%
  \Expect*{\meaning\NewEmpty}{macro:->}%
  \def\TestA{\iffalse}%
  \LetLtxMacro\TestB\TestA
  \TestDef{TestB}{\iffalse }%
  \TestEquals{TestA}{TestB}%
\end{qstest}
%    \end{macrocode}
%
%    \begin{macrocode}
\usepackage{etoolbox}[2008/06/22]

\begin{qstest}{oldnewrobustcmd}{oldnewrobustcmd}
  \begingroup
    \protected\edef\TestA{%
      \noexpand\@protected@testopt
      \noexpand\TestA
      \expandafter\noexpand\csname\string\TestA\endcsname
      {\noexpand\default}%
    }%
    \expandafter\protected\expandafter
    \def\csname\string\TestA\endcsname{TestA}%
    \SaveA
    \LetLtxMacro\TestB\TestA
    \TestDef{TestB}{\@protected@testopt \TestB \\TestB {\default }}%
    \protected\edef\TestC{%
      \noexpand\@protected@testopt
      \noexpand\TestB
      \expandafter\noexpand\csname\string\TestB\endcsname
      {\noexpand\default}%
    }%
    \TestEquals{TestB}{TestC}%
    \CheckA
    \GlobalLetLtxMacro\TestD\TestA
    \CheckA
  \endgroup
  \TestDef{TestD}{\@protected@testopt \TestD \\TestD {\default }}%
  \protected\edef\TestC{%
    \noexpand\@protected@testopt
    \noexpand\TestD
    \expandafter\noexpand\csname\string\TestD\endcsname
    {\noexpand\default}%
  }%
  \TestEquals{TestD}{TestC}%
\end{qstest}

\begin{qstest}{newrobustcmd}{newrobustcmd}
  \begingroup
    \newrobustcmd{\TestA}[1][\default]{TestA}%
    \SaveA
    \LetLtxMacro\TestB\TestA
    \TestDef{TestB}{\@testopt \\TestB {\default }}%
    \protected\edef\TestC{%
      \noexpand\@testopt
      \expandafter\noexpand\csname\string\TestB\endcsname
      {\noexpand\default}%
    }%
    \TestEquals{TestB}{TestC}%
    \CheckA
    \GlobalLetLtxMacro\TestD\TestA
    \CheckA
  \endgroup
  \TestDef{TestD}{\@testopt \\TestD {\default }}%
  \protected\edef\TestC{%
    \noexpand\@testopt
    \expandafter\noexpand\csname\string\TestD\endcsname
    {\noexpand\default}%
  }%
  \TestEquals{TestD}{TestC}%
\end{qstest}

\begin{qstest}{robustifyopt}{robustifyopt}
  \newcommand{\TestA}[2][\default]{}%
  \TestDef{TestA}{\@protected@testopt \TestA \\TestA {\default }}%
  \robustify\TestA
  \TestDef{TestA}{\@protected@testopt \TestA \\TestA {\default }}%
  \protected\edef\TestC{%
    \noexpand\@protected@testopt
    \noexpand\TestA
    \expandafter\noexpand\csname\string\TestA\endcsname
    {\noexpand\default}%
  }%
  \TestEquals{TestA}{TestC}%
  \SaveA
  \LetLtxMacro\TestB\TestA
  \TestDef{TestB}{\@protected@testopt \TestB \\TestB {\default }}%
  \protected\edef\TestC{%
    \noexpand\@protected@testopt
    \noexpand\TestB
    \expandafter\noexpand\csname\string\TestB\endcsname
    {\noexpand\default}%
  }%
  \TestEquals{TestB}{TestC}%
  \CheckA
\end{qstest}

\begin{qstest}{robustifydeclare}{robustifydeclare}
  \DeclareRobustCommand\TestA{\iffalse}%
  \robustify\TestA
  \SaveA
  \LetLtxMacro\TestB\TestA
  \TestDef{TestB}{\iffalse }%
  % before etoolbox 2008/06/22: \protected\long\def
  % since etoolbox 2008/06/22: \protected\def
  \protected\def\TestC{\iffalse}%
  \TestEquals{TestB}{TestC}%
  \CheckA
\end{qstest}

\begin{qstest}{robustifydeclarelong}{robustifydeclarelong}
  \DeclareRobustCommand\TestA[1]{\iffalse}%
  \robustify\TestA
  \SaveA
  \LetLtxMacro\TestB\TestA
  \TestDef{TestB}{\iffalse }%
  % before etoolbox 2008/06/22: \protected\long\def
  % since etoolbox 2008/06/22: \protected\def
  \protected\long\def\TestC#1{\iffalse}%
  \TestEquals{TestB}{TestC}%
  \CheckA
\end{qstest}
%    \end{macrocode}
%
%    \begin{macrocode}
\csname @@end\endcsname
%    \end{macrocode}
%    \begin{macrocode}
%</test2>
%    \end{macrocode}
%
% \section{Installation}
%
% \subsection{Download}
%
% \paragraph{Package.} This package is available on
% CTAN\footnote{\url{ftp://ftp.ctan.org/tex-archive/}}:
% \begin{description}
% \item[\CTAN{macros/latex/contrib/oberdiek/letltxmacro.dtx}] The source file.
% \item[\CTAN{macros/latex/contrib/oberdiek/letltxmacro.pdf}] Documentation.
% \end{description}
%
%
% \paragraph{Bundle.} All the packages of the bundle `oberdiek'
% are also available in a TDS compliant ZIP archive. There
% the packages are already unpacked and the documentation files
% are generated. The files and directories obey the TDS standard.
% \begin{description}
% \item[\CTAN{install/macros/latex/contrib/oberdiek.tds.zip}]
% \end{description}
% \emph{TDS} refers to the standard ``A Directory Structure
% for \TeX\ Files'' (\CTAN{tds/tds.pdf}). Directories
% with \xfile{texmf} in their name are usually organized this way.
%
% \subsection{Bundle installation}
%
% \paragraph{Unpacking.} Unpack the \xfile{oberdiek.tds.zip} in the
% TDS tree (also known as \xfile{texmf} tree) of your choice.
% Example (linux):
% \begin{quote}
%   |unzip oberdiek.tds.zip -d ~/texmf|
% \end{quote}
%
% \paragraph{Script installation.}
% Check the directory \xfile{TDS:scripts/oberdiek/} for
% scripts that need further installation steps.
% Package \xpackage{attachfile2} comes with the Perl script
% \xfile{pdfatfi.pl} that should be installed in such a way
% that it can be called as \texttt{pdfatfi}.
% Example (linux):
% \begin{quote}
%   |chmod +x scripts/oberdiek/pdfatfi.pl|\\
%   |cp scripts/oberdiek/pdfatfi.pl /usr/local/bin/|
% \end{quote}
%
% \subsection{Package installation}
%
% \paragraph{Unpacking.} The \xfile{.dtx} file is a self-extracting
% \docstrip\ archive. The files are extracted by running the
% \xfile{.dtx} through \plainTeX:
% \begin{quote}
%   \verb|tex letltxmacro.dtx|
% \end{quote}
%
% \paragraph{TDS.} Now the different files must be moved into
% the different directories in your installation TDS tree
% (also known as \xfile{texmf} tree):
% \begin{quote}
% \def\t{^^A
% \begin{tabular}{@{}>{\ttfamily}l@{ $\rightarrow$ }>{\ttfamily}l@{}}
%   letltxmacro.sty & tex/latex/oberdiek/letltxmacro.sty\\
%   letltxmacro.pdf & doc/latex/oberdiek/letltxmacro.pdf\\
%   letltxmacro-showcases.tex & doc/latex/oberdiek/letltxmacro-showcases.tex\\
%   test/letltxmacro-test1.tex & doc/latex/oberdiek/test/letltxmacro-test1.tex\\
%   test/letltxmacro-test2.tex & doc/latex/oberdiek/test/letltxmacro-test2.tex\\
%   letltxmacro.dtx & source/latex/oberdiek/letltxmacro.dtx\\
% \end{tabular}^^A
% }^^A
% \sbox0{\t}^^A
% \ifdim\wd0>\linewidth
%   \begingroup
%     \advance\linewidth by\leftmargin
%     \advance\linewidth by\rightmargin
%   \edef\x{\endgroup
%     \def\noexpand\lw{\the\linewidth}^^A
%   }\x
%   \def\lwbox{^^A
%     \leavevmode
%     \hbox to \linewidth{^^A
%       \kern-\leftmargin\relax
%       \hss
%       \usebox0
%       \hss
%       \kern-\rightmargin\relax
%     }^^A
%   }^^A
%   \ifdim\wd0>\lw
%     \sbox0{\small\t}^^A
%     \ifdim\wd0>\linewidth
%       \ifdim\wd0>\lw
%         \sbox0{\footnotesize\t}^^A
%         \ifdim\wd0>\linewidth
%           \ifdim\wd0>\lw
%             \sbox0{\scriptsize\t}^^A
%             \ifdim\wd0>\linewidth
%               \ifdim\wd0>\lw
%                 \sbox0{\tiny\t}^^A
%                 \ifdim\wd0>\linewidth
%                   \lwbox
%                 \else
%                   \usebox0
%                 \fi
%               \else
%                 \lwbox
%               \fi
%             \else
%               \usebox0
%             \fi
%           \else
%             \lwbox
%           \fi
%         \else
%           \usebox0
%         \fi
%       \else
%         \lwbox
%       \fi
%     \else
%       \usebox0
%     \fi
%   \else
%     \lwbox
%   \fi
% \else
%   \usebox0
% \fi
% \end{quote}
% If you have a \xfile{docstrip.cfg} that configures and enables \docstrip's
% TDS installing feature, then some files can already be in the right
% place, see the documentation of \docstrip.
%
% \subsection{Refresh file name databases}
%
% If your \TeX~distribution
% (\teTeX, \mikTeX, \dots) relies on file name databases, you must refresh
% these. For example, \teTeX\ users run \verb|texhash| or
% \verb|mktexlsr|.
%
% \subsection{Some details for the interested}
%
% \paragraph{Attached source.}
%
% The PDF documentation on CTAN also includes the
% \xfile{.dtx} source file. It can be extracted by
% AcrobatReader 6 or higher. Another option is \textsf{pdftk},
% e.g. unpack the file into the current directory:
% \begin{quote}
%   \verb|pdftk letltxmacro.pdf unpack_files output .|
% \end{quote}
%
% \paragraph{Unpacking with \LaTeX.}
% The \xfile{.dtx} chooses its action depending on the format:
% \begin{description}
% \item[\plainTeX:] Run \docstrip\ and extract the files.
% \item[\LaTeX:] Generate the documentation.
% \end{description}
% If you insist on using \LaTeX\ for \docstrip\ (really,
% \docstrip\ does not need \LaTeX), then inform the autodetect routine
% about your intention:
% \begin{quote}
%   \verb|latex \let\install=y\input{letltxmacro.dtx}|
% \end{quote}
% Do not forget to quote the argument according to the demands
% of your shell.
%
% \paragraph{Generating the documentation.}
% You can use both the \xfile{.dtx} or the \xfile{.drv} to generate
% the documentation. The process can be configured by the
% configuration file \xfile{ltxdoc.cfg}. For instance, put this
% line into this file, if you want to have A4 as paper format:
% \begin{quote}
%   \verb|\PassOptionsToClass{a4paper}{article}|
% \end{quote}
% An example follows how to generate the
% documentation with pdf\LaTeX:
% \begin{quote}
%\begin{verbatim}
%pdflatex letltxmacro.dtx
%makeindex -s gind.ist letltxmacro.idx
%pdflatex letltxmacro.dtx
%makeindex -s gind.ist letltxmacro.idx
%pdflatex letltxmacro.dtx
%\end{verbatim}
% \end{quote}
%
% \section{Catalogue}
%
% The following XML file can be used as source for the
% \href{http://mirror.ctan.org/help/Catalogue/catalogue.html}{\TeX\ Catalogue}.
% The elements \texttt{caption} and \texttt{description} are imported
% from the original XML file from the Catalogue.
% The name of the XML file in the Catalogue is \xfile{letltxmacro.xml}.
%    \begin{macrocode}
%<*catalogue>
<?xml version='1.0' encoding='us-ascii'?>
<!DOCTYPE entry SYSTEM 'catalogue.dtd'>
<entry datestamp='$Date$' modifier='$Author$' id='letltxmacro'>
  <name>letltxmacro</name>
  <caption>Let assignment for LaTeX macros.</caption>
  <authorref id='auth:oberdiek'/>
  <copyright owner='Heiko Oberdiek' year='2008,2010'/>
  <license type='lppl1.3'/>
  <version number='1.4'/>
  <description>
    TeX&#x2019;s <tt>\let</tt> assignment does not work for LaTeX macros
    with optional arguments or for macros that are defined
    as robust macros by <tt>\DeclareRobustCommand</tt>.  This package
    defines <tt>\LetLtxMacro</tt> that also takes care of the involved
    internal macros.
    <p/>
    The package is part of the <xref refid='oberdiek'>oberdiek</xref>
    bundle.
  </description>
  <documentation details='Package documentation'
      href='ctan:/macros/latex/contrib/oberdiek/letltxmacro.pdf'/>
  <ctan file='true' path='/macros/latex/contrib/oberdiek/letltxmacro.dtx'/>
  <miktex location='oberdiek'/>
  <texlive location='oberdiek'/>
  <install path='/macros/latex/contrib/oberdiek/oberdiek.tds.zip'/>
</entry>
%</catalogue>
%    \end{macrocode}
%
% \begin{History}
%   \begin{Version}{2008/06/09 v1.0}
%   \item
%     First version.
%   \end{Version}
%   \begin{Version}{2008/06/12 v1.1}
%   \item
%     Support for \xpackage{etoolbox}'s \cs{newrobustcmd} added.
%   \end{Version}
%   \begin{Version}{2008/06/13 v1.2}
%   \item
%     Support for \xpackage{etoolbox}'s \cs{robustify} added.
%   \end{Version}
%   \begin{Version}{2008/06/24 v1.3}
%   \item
%     Test file adapted for etoolbox 2008/06/22 v1.6.
%   \end{Version}
%   \begin{Version}{2010/09/02 v1.4}
%   \item
%     \cs{GlobalLetLtxMacro} added.
%   \end{Version}
% \end{History}
%
% \PrintIndex
%
% \Finale
\endinput

%        (quote the arguments according to the demands of your shell)
%
% Documentation:
%    (a) If letltxmacro.drv is present:
%           latex letltxmacro.drv
%    (b) Without letltxmacro.drv:
%           latex letltxmacro.dtx; ...
%    The class ltxdoc loads the configuration file ltxdoc.cfg
%    if available. Here you can specify further options, e.g.
%    use A4 as paper format:
%       \PassOptionsToClass{a4paper}{article}
%
%    Programm calls to get the documentation (example):
%       pdflatex letltxmacro.dtx
%       makeindex -s gind.ist letltxmacro.idx
%       pdflatex letltxmacro.dtx
%       makeindex -s gind.ist letltxmacro.idx
%       pdflatex letltxmacro.dtx
%
% Installation:
%    TDS:tex/latex/oberdiek/letltxmacro.sty
%    TDS:doc/latex/oberdiek/letltxmacro.pdf
%    TDS:doc/latex/oberdiek/letltxmacro-showcases.tex
%    TDS:doc/latex/oberdiek/test/letltxmacro-test1.tex
%    TDS:doc/latex/oberdiek/test/letltxmacro-test2.tex
%    TDS:source/latex/oberdiek/letltxmacro.dtx
%
%<*ignore>
\begingroup
  \catcode123=1 %
  \catcode125=2 %
  \def\x{LaTeX2e}%
\expandafter\endgroup
\ifcase 0\ifx\install y1\fi\expandafter
         \ifx\csname processbatchFile\endcsname\relax\else1\fi
         \ifx\fmtname\x\else 1\fi\relax
\else\csname fi\endcsname
%</ignore>
%<*install>
\input docstrip.tex
\Msg{************************************************************************}
\Msg{* Installation}
\Msg{* Package: letltxmacro 2010/09/02 v1.4 Let assignment for LaTeX macros (HO)}
\Msg{************************************************************************}

\keepsilent
\askforoverwritefalse

\let\MetaPrefix\relax
\preamble

This is a generated file.

Project: letltxmacro
Version: 2010/09/02 v1.4

Copyright (C) 2008, 2010 by
   Heiko Oberdiek <heiko.oberdiek at googlemail.com>

This work may be distributed and/or modified under the
conditions of the LaTeX Project Public License, either
version 1.3c of this license or (at your option) any later
version. This version of this license is in
   http://www.latex-project.org/lppl/lppl-1-3c.txt
and the latest version of this license is in
   http://www.latex-project.org/lppl.txt
and version 1.3 or later is part of all distributions of
LaTeX version 2005/12/01 or later.

This work has the LPPL maintenance status "maintained".

This Current Maintainer of this work is Heiko Oberdiek.

This work consists of the main source file letltxmacro.dtx
and the derived files
   letltxmacro.sty, letltxmacro.pdf, letltxmacro.ins, letltxmacro.drv,
   letltxmacro-showcases.tex, letltxmacro-test1.tex,
   letltxmacro-test2.tex.

\endpreamble
\let\MetaPrefix\DoubleperCent

\generate{%
  \file{letltxmacro.ins}{\from{letltxmacro.dtx}{install}}%
  \file{letltxmacro.drv}{\from{letltxmacro.dtx}{driver}}%
  \usedir{tex/latex/oberdiek}%
  \file{letltxmacro.sty}{\from{letltxmacro.dtx}{package}}%
  \usedir{doc/latex/oberdiek}%
  \file{letltxmacro-showcases.tex}{\from{letltxmacro.dtx}{showcases}}%
  \usedir{doc/latex/oberdiek/test}%
  \file{letltxmacro-test1.tex}{\from{letltxmacro.dtx}{test1}}%
  \file{letltxmacro-test2.tex}{\from{letltxmacro.dtx}{test2}}%
  \nopreamble
  \nopostamble
  \usedir{source/latex/oberdiek/catalogue}%
  \file{letltxmacro.xml}{\from{letltxmacro.dtx}{catalogue}}%
}

\catcode32=13\relax% active space
\let =\space%
\Msg{************************************************************************}
\Msg{*}
\Msg{* To finish the installation you have to move the following}
\Msg{* file into a directory searched by TeX:}
\Msg{*}
\Msg{*     letltxmacro.sty}
\Msg{*}
\Msg{* To produce the documentation run the file `letltxmacro.drv'}
\Msg{* through LaTeX.}
\Msg{*}
\Msg{* Happy TeXing!}
\Msg{*}
\Msg{************************************************************************}

\endbatchfile
%</install>
%<*ignore>
\fi
%</ignore>
%<*driver>
\NeedsTeXFormat{LaTeX2e}
\ProvidesFile{letltxmacro.drv}%
  [2010/09/02 v1.4 Let assignment for LaTeX macros (HO)]%
\documentclass{ltxdoc}
\usepackage{holtxdoc}[2011/11/22]
\begin{document}
  \DocInput{letltxmacro.dtx}%
\end{document}
%</driver>
% \fi
%
% \CheckSum{990}
%
% \CharacterTable
%  {Upper-case    \A\B\C\D\E\F\G\H\I\J\K\L\M\N\O\P\Q\R\S\T\U\V\W\X\Y\Z
%   Lower-case    \a\b\c\d\e\f\g\h\i\j\k\l\m\n\o\p\q\r\s\t\u\v\w\x\y\z
%   Digits        \0\1\2\3\4\5\6\7\8\9
%   Exclamation   \!     Double quote  \"     Hash (number) \#
%   Dollar        \$     Percent       \%     Ampersand     \&
%   Acute accent  \'     Left paren    \(     Right paren   \)
%   Asterisk      \*     Plus          \+     Comma         \,
%   Minus         \-     Point         \.     Solidus       \/
%   Colon         \:     Semicolon     \;     Less than     \<
%   Equals        \=     Greater than  \>     Question mark \?
%   Commercial at \@     Left bracket  \[     Backslash     \\
%   Right bracket \]     Circumflex    \^     Underscore    \_
%   Grave accent  \`     Left brace    \{     Vertical bar  \|
%   Right brace   \}     Tilde         \~}
%
% \GetFileInfo{letltxmacro.drv}
%
% \title{The \xpackage{letltxmacro} package}
% \date{2010/09/02 v1.4}
% \author{Heiko Oberdiek\\\xemail{heiko.oberdiek at googlemail.com}}
%
% \maketitle
%
% \begin{abstract}
% \TeX's \cs{let} assignment does not work for \LaTeX\ macros
% with optional arguments or for macros that are defined
% as robust macros by \cs{DeclareRobustCommand}. This package
% defines \cs{LetLtxMacro} that also takes care of the involved
% internal macros.
% \end{abstract}
%
% \tableofcontents
%
% \section{Documentation}
%
% If someone wants to redefine a macro with using the old
% meaning, then one method is \TeX's command \cs{let}:
%\begin{quote}
%\begin{verbatim}
%\newcommand{\Macro}{\typeout{Test Macro}}
%\let\SavedMacro=\Macro
%\renewcommand{\Macro}{%
%  \typeout{Begin}%
%  \SavedMacro
%  \typeout{End}%
%}
%\end{verbatim}
%\end{quote}
% However, this method fails, if \cs{Macro} is defined
% by \cs{DeclareRobustCommand} and/or has an optional argument.
% In both cases \LaTeX\ defines an additional internal macro
% that is forgotten in the simple \cs{let} assignment of
% the example above.
%
% \begin{declcs}{LetLtxMacro} \M{new macro} \M{old macro}
% \end{declcs}
% Macro \cs{LetLtxMacro} behaves similar to \TeX's \cs{let}
% assignment, but it takes care of macros that are
% defined by \cs{DeclareRobustCommand} and/or have optional
% arguments. Example:
%\begin{quote}
%\begin{verbatim}
%\DeclareRobustCommand{\Macro}[1][default]{...}
%\LetLtxMacro{\SavedMacro}{\Macro}
%\end{verbatim}
%\end{quote}
% Then macro \cs{SavedMacro} only uses internal macro names
% that are derived from \cs{SavedMacro}'s macro name. Macro \cs{Macro}
% can now be redefined without affecting \cs{SavedMacro}.
%
% \begin{declcs}{GlobalLetLtxMacro} \M{new macro} \M{old macro}
% \end{declcs}
% Like \cs{LetLtxMacro}, but the \meta{new macro} is defined globally.
% Since version 2010/09/02~v1.4.
%
% \subsection{Supported macro definition commands}
%
% \begin{quote}
%   \begin{tabular}{@{}ll@{}}
%     \cs{newcommand}, \cs{renewcommand} & latex/base\\
%     \cs{newenvironment}, \cs{renewenvironment} & latex/base\\
%     \cs{DeclareRobustCommand}& latex/base\\
%     \cs{newrobustcmd}, \cs{renewrobustcmd} & etoolbox\\
%     \cs{robustify} & etoolbox 2008/06/22 v1.6\\
%   \end{tabular}
% \end{quote}
%
% \StopEventually{
% }
%
% \section{Implementation}
%
% \subsection{Show cases}
%
% \subsubsection{\xfile{letltxmacro-showcases.tex}}
%
%    \begin{macrocode}
%<*showcases>
\NeedsTeXFormat{LaTeX2e}
\makeatletter
%    \end{macrocode}
%    \begin{macro}{\Line}
%    The result is displayed by macro \cs{Line}. The percent symbol
%    at line start allows easy grepping and inserting into the DTX
%    file.
%    \begin{macrocode}
\newcommand*{\Line}[1]{%
  \typeout{\@percentchar#1}%
}
%    \end{macrocode}
%    \end{macro}
%    \begin{macrocode}
\newcommand*{\ShowCmdName}[1]{%
  \@ifundefined{#1}{}{%
    \Line{%
      \space\space(\expandafter\string\csname#1\endcsname) = %
      (\expandafter\meaning\csname#1\endcsname)%
    }%
  }%
}
\newcommand*{\ShowCmds}[1]{%
  \ShowCmdName{#1}%
  \ShowCmdName{#1 }%
  \ShowCmdName{\\#1}%
  \ShowCmdName{\\#1 }%
}
\let\\\@backslashchar
%    \end{macrocode}
%    \begin{macro}{\ShowDef}
%    \begin{macrocode}
\newcommand*{\ShowDef}[2]{%
  \begingroup
    \Line{}%
    \newcommand*{\DefString}{#2}%
    \@onelevel@sanitize\DefString
    \Line{\DefString}%
    #2%
    \ShowCmds{#1}%
  \endgroup
}
%    \end{macrocode}
%    \end{macro}
%    \begin{macrocode}
\typeout{}
\Line{* LaTeX definitions:}
\ShowDef{cmd}{%
  \newcommand{\cmd}[2][default]{}%
}
\ShowDef{cmd}{%
  \DeclareRobustCommand{\cmd}{}%
}
\ShowDef{cmd}{%
  \DeclareRobustCommand{\cmd}[2][default]{}%
}
\typeout{}
%    \end{macrocode}
% The minimal version of package \xpackage{etoolbox} is 2008/06/12 v1.6a
% because it fixes \cs{robustify}.
%    \begin{macrocode}
\RequirePackage{etoolbox}[2008/06/12]%
\Line{}
\Line{* etoolbox's robust definitions:}
\ShowDef{cmd}{%
  \newrobustcmd{\cmd}{}%
}
\ShowDef{cmd}{%
  \newrobustcmd{\cmd}[2][default]{}%
}
\Line{}
\Line{* etoolbox's \string\robustify:}
\ShowDef{cmd}{%
  \newcommand{\cmd}[2][default]{} %
  \robustify{\cmd}%
}
\ShowDef{cmd}{%
  \DeclareRobustCommand{\cmd}{} %
  \robustify{\cmd}%
}
\ShowDef{cmd}{%
  \DeclareRobustCommand{\cmd}[2][default]{} %
  \robustify{\cmd}%
}
\typeout{}
\@@end
%</showcases>
%    \end{macrocode}
%
% \subsubsection{Result}
%
% \begingroup
%   \makeatletter
%   \let\org@verbatim\@verbatim
%   \def\@verbatim{^^A
%     \org@verbatim
%     \catcode`\~=\active
%   }^^A
%   \let~\textvisiblespace
%\begin{verbatim}
%* LaTeX definitions:
%
%\newcommand {\cmd }[2][default]{}
%  (\cmd) = (macro:->\@protected@testopt \cmd \\cmd {default})
%  (\\cmd) = (\long macro:[#1]#2->)
%
%\DeclareRobustCommand {\cmd }{}
%  (\cmd) = (macro:->\protect \cmd~ )
%  (\cmd~) = (\long macro:->)
%
%\DeclareRobustCommand {\cmd }[2][default]{}
%  (\cmd) = (macro:->\protect \cmd~ )
%  (\cmd~) = (macro:->\@protected@testopt \cmd~ \\cmd~ {default})
%  (\\cmd~) = (\long macro:[#1]#2->)
%
%* etoolbox's robust definitions:
%
%\newrobustcmd {\cmd }{}
%  (\cmd) = (\protected\long macro:->)
%
%\newrobustcmd {\cmd }[2][default]{}
%  (\cmd) = (\protected macro:->\@testopt \\cmd {default})
%  (\\cmd) = (\long macro:[#1]#2->)
%
%* etoolbox's \robustify:
%
%\newcommand {\cmd }[2][default]{} \robustify {\cmd }
%  (\cmd) = (\protected macro:->\@protected@testopt \cmd \\cmd {default})
%  (\\cmd) = (\long macro:[#1]#2->)
%
%\DeclareRobustCommand {\cmd }{} \robustify {\cmd }
%  (\cmd) = (\protected macro:->)
%
%\DeclareRobustCommand {\cmd }[2][default]{} \robustify {\cmd }
%  (\cmd) = (\protected macro:->\@protected@testopt \cmd~ \\cmd~ {default})
%  (\cmd~) = (macro:->\@protected@testopt \cmd~ \\cmd~ {default})
%  (\\cmd~) = (\long macro:[#1]#2->)
%\end{verbatim}
% \endgroup
%
% \subsection{Package}
%
%    \begin{macrocode}
%<*package>
%    \end{macrocode}
%
% \subsubsection{Catcodes and identification}
%
%    \begin{macrocode}
\begingroup\catcode61\catcode48\catcode32=10\relax%
  \catcode13=5 % ^^M
  \endlinechar=13 %
  \catcode123=1 % {
  \catcode125=2 % }
  \catcode64=11 % @
  \def\x{\endgroup
    \expandafter\edef\csname llm@AtEnd\endcsname{%
      \endlinechar=\the\endlinechar\relax
      \catcode13=\the\catcode13\relax
      \catcode32=\the\catcode32\relax
      \catcode35=\the\catcode35\relax
      \catcode61=\the\catcode61\relax
      \catcode64=\the\catcode64\relax
      \catcode123=\the\catcode123\relax
      \catcode125=\the\catcode125\relax
    }%
  }%
\x\catcode61\catcode48\catcode32=10\relax%
\catcode13=5 % ^^M
\endlinechar=13 %
\catcode35=6 % #
\catcode64=11 % @
\catcode123=1 % {
\catcode125=2 % }
\def\TMP@EnsureCode#1#2{%
  \edef\llm@AtEnd{%
    \llm@AtEnd
    \catcode#1=\the\catcode#1\relax
  }%
  \catcode#1=#2\relax
}
\TMP@EnsureCode{40}{12}% (
\TMP@EnsureCode{41}{12}% )
\TMP@EnsureCode{42}{12}% *
\TMP@EnsureCode{45}{12}% -
\TMP@EnsureCode{46}{12}% .
\TMP@EnsureCode{47}{12}% /
\TMP@EnsureCode{58}{12}% :
\TMP@EnsureCode{62}{12}% >
\TMP@EnsureCode{91}{12}% [
\TMP@EnsureCode{93}{12}% ]
\edef\llm@AtEnd{%
  \llm@AtEnd
  \escapechar\the\escapechar\relax
  \noexpand\endinput
}
\escapechar=92 % `\\
%    \end{macrocode}
%
%    Package identification.
%    \begin{macrocode}
\NeedsTeXFormat{LaTeX2e}
\ProvidesPackage{letltxmacro}%
  [2010/09/02 v1.4 Let assignment for LaTeX macros (HO)]
%    \end{macrocode}
%
% \subsubsection{Main macros}
%
%    \begin{macro}{\LetLtxMacro}
%    \begin{macrocode}
\newcommand*{\LetLtxMacro}{%
  \llm@ModeLetLtxMacro{}%
}
%    \end{macrocode}
%    \end{macro}
%    \begin{macro}{\GlobalLetLtxMacro}
%    \begin{macrocode}
\newcommand*{\GlobalLetLtxMacro}{%
  \llm@ModeLetLtxMacro\global
}
%    \end{macrocode}
%    \end{macro}
%
%    \begin{macro}{\llm@ModeLetLtxMacro}
%    \begin{macrocode}
\newcommand*{\llm@ModeLetLtxMacro}[3]{%
  \edef\llm@escapechar{\the\escapechar}%
  \escapechar=-1 %
  \edef\reserved@a{%
    \noexpand\protect
    \expandafter\noexpand
    \csname\string#3 \endcsname
  }%
  \ifx\reserved@a#3\relax
    #1\edef#2{%
      \noexpand\protect
      \expandafter\noexpand
      \csname\string#2 \endcsname
    }%
    #1\expandafter\let
    \csname\string#2 \expandafter\endcsname
    \csname\string#3 \endcsname
    \expandafter\llm@LetLtxMacro
        \csname\string#2 \expandafter\endcsname
        \csname\string#3 \endcsname{#1}%
  \else
    \llm@LetLtxMacro{#2}{#3}{#1}%
  \fi
  \escapechar=\llm@escapechar\relax
}
%    \end{macrocode}
%    \end{macro}
%    \begin{macro}{\llm@LetLtxMacro}
%    \begin{macrocode}
\def\llm@LetLtxMacro#1#2#3{%
  \escapechar=92 %
  \expandafter\llm@CheckParams\meaning#2:->\@nil{%
    \begingroup
      \def\@protected@testopt{%
        \expandafter\@testopt\@gobble
      }%
      \def\@testopt##1##2{%
        \toks@={##2}%
      }%
      \let\llm@testopt\@empty
      \edef\x{%
        \noexpand\@protected@testopt
        \noexpand#2%
        \expandafter\noexpand\csname\string#2\endcsname
      }%
      \expandafter\expandafter\expandafter\def
      \expandafter\expandafter\expandafter\y
      \expandafter\expandafter\expandafter{%
        \expandafter\llm@CarThree#2{}{}{}\llm@nil
      }%
      \ifx\x\y
        #2%
        \def\llm@testopt{%
          \noexpand\@protected@testopt
          \noexpand#1%
        }%
      \else
        \edef\x{%
          \noexpand\@testopt
          \expandafter\noexpand
          \csname\string#2\endcsname
        }%
        \expandafter\expandafter\expandafter\def
        \expandafter\expandafter\expandafter\y
        \expandafter\expandafter\expandafter{%
          \expandafter\llm@CarTwo#2{}{}\llm@nil
        }%
        \ifx\x\y
          #2%
          \def\llm@testopt{%
            \noexpand\@testopt
          }%
        \fi
      \fi
      \ifx\llm@testopt\@empty
      \else
        \llm@protected\xdef\llm@GlobalTemp{%
          \llm@testopt
          \expandafter\noexpand
          \csname\string#1\endcsname
          {\the\toks@}%
        }%
      \fi
    \expandafter\endgroup\ifx\llm@testopt\@empty
      #3\let#1=#2\relax
    \else
      #3\let#1=\llm@GlobalTemp
      #3\expandafter\let
          \csname\string#1\expandafter\endcsname
          \csname\string#2\endcsname
    \fi
  }{%
    #3\let#1=#2\relax
  }%
}
%    \end{macrocode}
%    \end{macro}
%    \begin{macro}{\llm@CheckParams}
%    \begin{macrocode}
\def\llm@CheckParams#1:->#2\@nil{%
  \begingroup
    \def\x{#1}%
  \ifx\x\llm@macro
    \endgroup
    \def\llm@protected{}%
    \expandafter\@firstoftwo
  \else
    \ifx\x\llm@protectedmacro
      \endgroup
      \def\llm@protected{\protected}%
      \expandafter\expandafter\expandafter\@firstoftwo
    \else
      \endgroup
      \expandafter\expandafter\expandafter\@secondoftwo
    \fi
  \fi
}
%    \end{macrocode}
%    \end{macro}
%    \begin{macro}{\llm@macro}
%    \begin{macrocode}
\def\llm@macro{macro}
\@onelevel@sanitize\llm@macro
%    \end{macrocode}
%    \end{macro}
%    \begin{macro}{\llm@protectedmacro}
%    \begin{macrocode}
\def\llm@protectedmacro{\protected macro}
\@onelevel@sanitize\llm@protectedmacro
%    \end{macrocode}
%    \end{macro}
%    \begin{macro}{\llm@CarThree}
%    \begin{macrocode}
\def\llm@CarThree#1#2#3#4\llm@nil{#1#2#3}%
%    \end{macrocode}
%    \end{macro}
%    \begin{macro}{\llm@CarTwo}
%    \begin{macrocode}
\def\llm@CarTwo#1#2#3\llm@nil{#1#2}%
%    \end{macrocode}
%    \end{macro}
%
%    \begin{macrocode}
\llm@AtEnd%
%</package>
%    \end{macrocode}
%
% \section{Test}
%
% \subsection{Catcode checks for loading}
%
%    \begin{macrocode}
%<*test1>
%    \end{macrocode}
%    \begin{macrocode}
\NeedsTeXFormat{LaTeX2e}
\documentclass{minimal}
\makeatletter
\def\RestoreCatcodes{}
\count@=0 %
\loop
  \edef\RestoreCatcodes{%
    \RestoreCatcodes
    \catcode\the\count@=\the\catcode\count@\relax
  }%
\ifnum\count@<255 %
  \advance\count@\@ne
\repeat

\def\RangeCatcodeInvalid#1#2{%
  \count@=#1\relax
  \loop
    \catcode\count@=15 %
  \ifnum\count@<#2\relax
    \advance\count@\@ne
  \repeat
}
\def\Test{%
  \RangeCatcodeInvalid{0}{47}%
  \RangeCatcodeInvalid{58}{64}%
  \RangeCatcodeInvalid{91}{96}%
  \RangeCatcodeInvalid{123}{127}%
  \catcode`\@=12 %
  \catcode`\\=0 %
  \catcode`\{=1 %
  \catcode`\}=2 %
  \catcode`\#=6 %
  \catcode`\[=12 %
  \catcode`\]=12 %
  \catcode`\%=14 %
  \catcode`\ =10 %
  \catcode13=5 %
  \RequirePackage{letltxmacro}[2010/09/02]\relax
  \RestoreCatcodes
}
\Test
\csname @@end\endcsname
\end
%    \end{macrocode}
%    \begin{macrocode}
%</test1>
%    \end{macrocode}
%
% \subsection{Package tests}
%
%    \begin{macrocode}
%<*test2>
%    \end{macrocode}
%    \begin{macrocode}
\NeedsTeXFormat{LaTeX2e}
\nofiles
\documentclass{minimal}

\usepackage{letltxmacro}[2010/09/02]

\usepackage{qstest}
\IncludeTests{*}
\LogTests{log}{*}{*}

\makeatletter

\def\TestDef#1{%
  \begingroup
    \@makeother\\%
    \@makeother\ %
    \expandafter\@TestDef\csname #1\endcsname
}
\def\@TestDef#1#2{%
    \edef\a{\expandafter\strip@prefix\meaning#1}%
    \edef\b{\detokenize{#2}}%
    \ifx\a\b
    \else
      \typeout{1 [\a]}%
      \typeout{2 [\b]}%
    \fi
  \endgroup
  \Expect*{\expandafter\strip@prefix\meaning#1}*{\detokenize{#2}}%
}
\def\TestEquals#1#2{%
  \Expect*{%
    \expandafter\ifx\csname#1\expandafter\endcsname
                    \csname#2\endcsname
      equals%
    \else
      wrong%
    \fi
  }{equals}%
}
\def\SaveA{%
  \let\SavedA\TestA
  \expandafter\let\csname\string\SavedA\expandafter\endcsname
                  \csname\string\TestA\endcsname
  \expandafter\let\csname SavedA \expandafter\endcsname
                  \csname TestA \endcsname
}
\def\CheckA{%
  \TestEquals{SavedA}{TestA}%
  \TestEquals{\string\SavedA}{\string\TestA}%
  \TestEquals{SavedA }{TestA }%
}

\begin{qstest}{robust}{robust}
  \begingroup
    \DeclareRobustCommand*{\TestA}{Test}%
    \SaveA
    \TestDef{TestA}{\protect \TestA  }%
    \CheckA
    \DeclareRobustCommand{\TestA}{Test}%
    \SaveA
    \TestDef{TestA}{\protect \TestA  }%
    \LetLtxMacro\TestB\TestA
    \TestDef{TestB}{\protect \TestB  }%
    \TestEquals{TestB }{TestA }%
    \CheckA
    \GlobalLetLtxMacro\TestC\TestA
    \CheckA
  \endgroup
  \TestDef{TestC}{\protect \TestC  }%
  \DeclareRobustCommand{\TestA}{Test}%
  \TestEquals{TestC }{TestA }%
\end{qstest}

\begin{qstest}{default}{default}
  \begingroup
    \newcommand{\TestA}[1][\relax default \empty]{TestA #1}%
    \SaveA
    \TestDef
    {TestA}{\@protected@testopt \TestA \\TestA {\relax default \empty }}%
    \LetLtxMacro\TestB\TestA
    \TestDef
    {TestB}{\@protected@testopt \TestB \\TestB {\relax default \empty }}%
    \CheckA
    \TestEquals{\string\TestB}{\string\TestA}%
    \GlobalLetLtxMacro\TestC\TestA
    \CheckA
  \endgroup
  \TestDef
  {TestC}{\@protected@testopt \TestC \\TestC {\relax default \empty }}%
  \newcommand{\TestA}[1][\relax default \empty]{TestA #1}%
  \TestEquals{\string\TestC}{\string\TestA}%
\end{qstest}

\begin{qstest}{robustdefault}{robustdefault}
  \begingroup
    \DeclareRobustCommand{\TestA}[1][\default]{TestA}%
    \SaveA
    \LetLtxMacro\TestB\TestA
    \TestDef{TestB}{\protect \TestB  }%
    \TestDef{TestB }{\@protected@testopt \TestB  \\TestB  {\default }}%
    \CheckA
    \GlobalLetLtxMacro\TestC\TestA
    \CheckA
  \endgroup
  \TestDef{TestC}{\protect \TestC  }%
  \TestDef{TestC }{\@protected@testopt \TestC  \\TestC  {\default }}%
\end{qstest}

\begin{qstest}{plain}{plain}
  \LetLtxMacro\NewRelax\relax
  \Expect*{\meaning\relax}*{\string\relax}%
  \Expect*{\meaning\NewRelax}*{\string\relax}%
  \LetLtxMacro\NewHbox\hbox
  \Expect*{\meaning\hbox}*{\string\hbox}%
  \Expect*{\meaning\NewHbox}*{\string\hbox}%
  \LetLtxMacro\NewEmpty\empty
  \Expect*{\meaning\empty}{macro:->}%
  \Expect*{\meaning\NewEmpty}{macro:->}%
  \def\TestA{\iffalse}%
  \LetLtxMacro\TestB\TestA
  \TestDef{TestB}{\iffalse }%
  \TestEquals{TestA}{TestB}%
\end{qstest}
%    \end{macrocode}
%
%    \begin{macrocode}
\usepackage{etoolbox}[2008/06/22]

\begin{qstest}{oldnewrobustcmd}{oldnewrobustcmd}
  \begingroup
    \protected\edef\TestA{%
      \noexpand\@protected@testopt
      \noexpand\TestA
      \expandafter\noexpand\csname\string\TestA\endcsname
      {\noexpand\default}%
    }%
    \expandafter\protected\expandafter
    \def\csname\string\TestA\endcsname{TestA}%
    \SaveA
    \LetLtxMacro\TestB\TestA
    \TestDef{TestB}{\@protected@testopt \TestB \\TestB {\default }}%
    \protected\edef\TestC{%
      \noexpand\@protected@testopt
      \noexpand\TestB
      \expandafter\noexpand\csname\string\TestB\endcsname
      {\noexpand\default}%
    }%
    \TestEquals{TestB}{TestC}%
    \CheckA
    \GlobalLetLtxMacro\TestD\TestA
    \CheckA
  \endgroup
  \TestDef{TestD}{\@protected@testopt \TestD \\TestD {\default }}%
  \protected\edef\TestC{%
    \noexpand\@protected@testopt
    \noexpand\TestD
    \expandafter\noexpand\csname\string\TestD\endcsname
    {\noexpand\default}%
  }%
  \TestEquals{TestD}{TestC}%
\end{qstest}

\begin{qstest}{newrobustcmd}{newrobustcmd}
  \begingroup
    \newrobustcmd{\TestA}[1][\default]{TestA}%
    \SaveA
    \LetLtxMacro\TestB\TestA
    \TestDef{TestB}{\@testopt \\TestB {\default }}%
    \protected\edef\TestC{%
      \noexpand\@testopt
      \expandafter\noexpand\csname\string\TestB\endcsname
      {\noexpand\default}%
    }%
    \TestEquals{TestB}{TestC}%
    \CheckA
    \GlobalLetLtxMacro\TestD\TestA
    \CheckA
  \endgroup
  \TestDef{TestD}{\@testopt \\TestD {\default }}%
  \protected\edef\TestC{%
    \noexpand\@testopt
    \expandafter\noexpand\csname\string\TestD\endcsname
    {\noexpand\default}%
  }%
  \TestEquals{TestD}{TestC}%
\end{qstest}

\begin{qstest}{robustifyopt}{robustifyopt}
  \newcommand{\TestA}[2][\default]{}%
  \TestDef{TestA}{\@protected@testopt \TestA \\TestA {\default }}%
  \robustify\TestA
  \TestDef{TestA}{\@protected@testopt \TestA \\TestA {\default }}%
  \protected\edef\TestC{%
    \noexpand\@protected@testopt
    \noexpand\TestA
    \expandafter\noexpand\csname\string\TestA\endcsname
    {\noexpand\default}%
  }%
  \TestEquals{TestA}{TestC}%
  \SaveA
  \LetLtxMacro\TestB\TestA
  \TestDef{TestB}{\@protected@testopt \TestB \\TestB {\default }}%
  \protected\edef\TestC{%
    \noexpand\@protected@testopt
    \noexpand\TestB
    \expandafter\noexpand\csname\string\TestB\endcsname
    {\noexpand\default}%
  }%
  \TestEquals{TestB}{TestC}%
  \CheckA
\end{qstest}

\begin{qstest}{robustifydeclare}{robustifydeclare}
  \DeclareRobustCommand\TestA{\iffalse}%
  \robustify\TestA
  \SaveA
  \LetLtxMacro\TestB\TestA
  \TestDef{TestB}{\iffalse }%
  % before etoolbox 2008/06/22: \protected\long\def
  % since etoolbox 2008/06/22: \protected\def
  \protected\def\TestC{\iffalse}%
  \TestEquals{TestB}{TestC}%
  \CheckA
\end{qstest}

\begin{qstest}{robustifydeclarelong}{robustifydeclarelong}
  \DeclareRobustCommand\TestA[1]{\iffalse}%
  \robustify\TestA
  \SaveA
  \LetLtxMacro\TestB\TestA
  \TestDef{TestB}{\iffalse }%
  % before etoolbox 2008/06/22: \protected\long\def
  % since etoolbox 2008/06/22: \protected\def
  \protected\long\def\TestC#1{\iffalse}%
  \TestEquals{TestB}{TestC}%
  \CheckA
\end{qstest}
%    \end{macrocode}
%
%    \begin{macrocode}
\csname @@end\endcsname
%    \end{macrocode}
%    \begin{macrocode}
%</test2>
%    \end{macrocode}
%
% \section{Installation}
%
% \subsection{Download}
%
% \paragraph{Package.} This package is available on
% CTAN\footnote{\url{ftp://ftp.ctan.org/tex-archive/}}:
% \begin{description}
% \item[\CTAN{macros/latex/contrib/oberdiek/letltxmacro.dtx}] The source file.
% \item[\CTAN{macros/latex/contrib/oberdiek/letltxmacro.pdf}] Documentation.
% \end{description}
%
%
% \paragraph{Bundle.} All the packages of the bundle `oberdiek'
% are also available in a TDS compliant ZIP archive. There
% the packages are already unpacked and the documentation files
% are generated. The files and directories obey the TDS standard.
% \begin{description}
% \item[\CTAN{install/macros/latex/contrib/oberdiek.tds.zip}]
% \end{description}
% \emph{TDS} refers to the standard ``A Directory Structure
% for \TeX\ Files'' (\CTAN{tds/tds.pdf}). Directories
% with \xfile{texmf} in their name are usually organized this way.
%
% \subsection{Bundle installation}
%
% \paragraph{Unpacking.} Unpack the \xfile{oberdiek.tds.zip} in the
% TDS tree (also known as \xfile{texmf} tree) of your choice.
% Example (linux):
% \begin{quote}
%   |unzip oberdiek.tds.zip -d ~/texmf|
% \end{quote}
%
% \paragraph{Script installation.}
% Check the directory \xfile{TDS:scripts/oberdiek/} for
% scripts that need further installation steps.
% Package \xpackage{attachfile2} comes with the Perl script
% \xfile{pdfatfi.pl} that should be installed in such a way
% that it can be called as \texttt{pdfatfi}.
% Example (linux):
% \begin{quote}
%   |chmod +x scripts/oberdiek/pdfatfi.pl|\\
%   |cp scripts/oberdiek/pdfatfi.pl /usr/local/bin/|
% \end{quote}
%
% \subsection{Package installation}
%
% \paragraph{Unpacking.} The \xfile{.dtx} file is a self-extracting
% \docstrip\ archive. The files are extracted by running the
% \xfile{.dtx} through \plainTeX:
% \begin{quote}
%   \verb|tex letltxmacro.dtx|
% \end{quote}
%
% \paragraph{TDS.} Now the different files must be moved into
% the different directories in your installation TDS tree
% (also known as \xfile{texmf} tree):
% \begin{quote}
% \def\t{^^A
% \begin{tabular}{@{}>{\ttfamily}l@{ $\rightarrow$ }>{\ttfamily}l@{}}
%   letltxmacro.sty & tex/latex/oberdiek/letltxmacro.sty\\
%   letltxmacro.pdf & doc/latex/oberdiek/letltxmacro.pdf\\
%   letltxmacro-showcases.tex & doc/latex/oberdiek/letltxmacro-showcases.tex\\
%   test/letltxmacro-test1.tex & doc/latex/oberdiek/test/letltxmacro-test1.tex\\
%   test/letltxmacro-test2.tex & doc/latex/oberdiek/test/letltxmacro-test2.tex\\
%   letltxmacro.dtx & source/latex/oberdiek/letltxmacro.dtx\\
% \end{tabular}^^A
% }^^A
% \sbox0{\t}^^A
% \ifdim\wd0>\linewidth
%   \begingroup
%     \advance\linewidth by\leftmargin
%     \advance\linewidth by\rightmargin
%   \edef\x{\endgroup
%     \def\noexpand\lw{\the\linewidth}^^A
%   }\x
%   \def\lwbox{^^A
%     \leavevmode
%     \hbox to \linewidth{^^A
%       \kern-\leftmargin\relax
%       \hss
%       \usebox0
%       \hss
%       \kern-\rightmargin\relax
%     }^^A
%   }^^A
%   \ifdim\wd0>\lw
%     \sbox0{\small\t}^^A
%     \ifdim\wd0>\linewidth
%       \ifdim\wd0>\lw
%         \sbox0{\footnotesize\t}^^A
%         \ifdim\wd0>\linewidth
%           \ifdim\wd0>\lw
%             \sbox0{\scriptsize\t}^^A
%             \ifdim\wd0>\linewidth
%               \ifdim\wd0>\lw
%                 \sbox0{\tiny\t}^^A
%                 \ifdim\wd0>\linewidth
%                   \lwbox
%                 \else
%                   \usebox0
%                 \fi
%               \else
%                 \lwbox
%               \fi
%             \else
%               \usebox0
%             \fi
%           \else
%             \lwbox
%           \fi
%         \else
%           \usebox0
%         \fi
%       \else
%         \lwbox
%       \fi
%     \else
%       \usebox0
%     \fi
%   \else
%     \lwbox
%   \fi
% \else
%   \usebox0
% \fi
% \end{quote}
% If you have a \xfile{docstrip.cfg} that configures and enables \docstrip's
% TDS installing feature, then some files can already be in the right
% place, see the documentation of \docstrip.
%
% \subsection{Refresh file name databases}
%
% If your \TeX~distribution
% (\teTeX, \mikTeX, \dots) relies on file name databases, you must refresh
% these. For example, \teTeX\ users run \verb|texhash| or
% \verb|mktexlsr|.
%
% \subsection{Some details for the interested}
%
% \paragraph{Attached source.}
%
% The PDF documentation on CTAN also includes the
% \xfile{.dtx} source file. It can be extracted by
% AcrobatReader 6 or higher. Another option is \textsf{pdftk},
% e.g. unpack the file into the current directory:
% \begin{quote}
%   \verb|pdftk letltxmacro.pdf unpack_files output .|
% \end{quote}
%
% \paragraph{Unpacking with \LaTeX.}
% The \xfile{.dtx} chooses its action depending on the format:
% \begin{description}
% \item[\plainTeX:] Run \docstrip\ and extract the files.
% \item[\LaTeX:] Generate the documentation.
% \end{description}
% If you insist on using \LaTeX\ for \docstrip\ (really,
% \docstrip\ does not need \LaTeX), then inform the autodetect routine
% about your intention:
% \begin{quote}
%   \verb|latex \let\install=y% \iffalse meta-comment
%
% File: letltxmacro.dtx
% Version: 2010/09/02 v1.4
% Info: Let assignment for LaTeX macros
%
% Copyright (C) 2008, 2010 by
%    Heiko Oberdiek <heiko.oberdiek at googlemail.com>
%
% This work may be distributed and/or modified under the
% conditions of the LaTeX Project Public License, either
% version 1.3c of this license or (at your option) any later
% version. This version of this license is in
%    http://www.latex-project.org/lppl/lppl-1-3c.txt
% and the latest version of this license is in
%    http://www.latex-project.org/lppl.txt
% and version 1.3 or later is part of all distributions of
% LaTeX version 2005/12/01 or later.
%
% This work has the LPPL maintenance status "maintained".
%
% This Current Maintainer of this work is Heiko Oberdiek.
%
% This work consists of the main source file letltxmacro.dtx
% and the derived files
%    letltxmacro.sty, letltxmacro.pdf, letltxmacro.ins, letltxmacro.drv,
%    letltxmacro-showcases.tex, letltxmacro-test1.tex,
%    letltxmacro-test2.tex.
%
% Distribution:
%    CTAN:macros/latex/contrib/oberdiek/letltxmacro.dtx
%    CTAN:macros/latex/contrib/oberdiek/letltxmacro.pdf
%
% Unpacking:
%    (a) If letltxmacro.ins is present:
%           tex letltxmacro.ins
%    (b) Without letltxmacro.ins:
%           tex letltxmacro.dtx
%    (c) If you insist on using LaTeX
%           latex \let\install=y\input{letltxmacro.dtx}
%        (quote the arguments according to the demands of your shell)
%
% Documentation:
%    (a) If letltxmacro.drv is present:
%           latex letltxmacro.drv
%    (b) Without letltxmacro.drv:
%           latex letltxmacro.dtx; ...
%    The class ltxdoc loads the configuration file ltxdoc.cfg
%    if available. Here you can specify further options, e.g.
%    use A4 as paper format:
%       \PassOptionsToClass{a4paper}{article}
%
%    Programm calls to get the documentation (example):
%       pdflatex letltxmacro.dtx
%       makeindex -s gind.ist letltxmacro.idx
%       pdflatex letltxmacro.dtx
%       makeindex -s gind.ist letltxmacro.idx
%       pdflatex letltxmacro.dtx
%
% Installation:
%    TDS:tex/latex/oberdiek/letltxmacro.sty
%    TDS:doc/latex/oberdiek/letltxmacro.pdf
%    TDS:doc/latex/oberdiek/letltxmacro-showcases.tex
%    TDS:doc/latex/oberdiek/test/letltxmacro-test1.tex
%    TDS:doc/latex/oberdiek/test/letltxmacro-test2.tex
%    TDS:source/latex/oberdiek/letltxmacro.dtx
%
%<*ignore>
\begingroup
  \catcode123=1 %
  \catcode125=2 %
  \def\x{LaTeX2e}%
\expandafter\endgroup
\ifcase 0\ifx\install y1\fi\expandafter
         \ifx\csname processbatchFile\endcsname\relax\else1\fi
         \ifx\fmtname\x\else 1\fi\relax
\else\csname fi\endcsname
%</ignore>
%<*install>
\input docstrip.tex
\Msg{************************************************************************}
\Msg{* Installation}
\Msg{* Package: letltxmacro 2010/09/02 v1.4 Let assignment for LaTeX macros (HO)}
\Msg{************************************************************************}

\keepsilent
\askforoverwritefalse

\let\MetaPrefix\relax
\preamble

This is a generated file.

Project: letltxmacro
Version: 2010/09/02 v1.4

Copyright (C) 2008, 2010 by
   Heiko Oberdiek <heiko.oberdiek at googlemail.com>

This work may be distributed and/or modified under the
conditions of the LaTeX Project Public License, either
version 1.3c of this license or (at your option) any later
version. This version of this license is in
   http://www.latex-project.org/lppl/lppl-1-3c.txt
and the latest version of this license is in
   http://www.latex-project.org/lppl.txt
and version 1.3 or later is part of all distributions of
LaTeX version 2005/12/01 or later.

This work has the LPPL maintenance status "maintained".

This Current Maintainer of this work is Heiko Oberdiek.

This work consists of the main source file letltxmacro.dtx
and the derived files
   letltxmacro.sty, letltxmacro.pdf, letltxmacro.ins, letltxmacro.drv,
   letltxmacro-showcases.tex, letltxmacro-test1.tex,
   letltxmacro-test2.tex.

\endpreamble
\let\MetaPrefix\DoubleperCent

\generate{%
  \file{letltxmacro.ins}{\from{letltxmacro.dtx}{install}}%
  \file{letltxmacro.drv}{\from{letltxmacro.dtx}{driver}}%
  \usedir{tex/latex/oberdiek}%
  \file{letltxmacro.sty}{\from{letltxmacro.dtx}{package}}%
  \usedir{doc/latex/oberdiek}%
  \file{letltxmacro-showcases.tex}{\from{letltxmacro.dtx}{showcases}}%
  \usedir{doc/latex/oberdiek/test}%
  \file{letltxmacro-test1.tex}{\from{letltxmacro.dtx}{test1}}%
  \file{letltxmacro-test2.tex}{\from{letltxmacro.dtx}{test2}}%
  \nopreamble
  \nopostamble
  \usedir{source/latex/oberdiek/catalogue}%
  \file{letltxmacro.xml}{\from{letltxmacro.dtx}{catalogue}}%
}

\catcode32=13\relax% active space
\let =\space%
\Msg{************************************************************************}
\Msg{*}
\Msg{* To finish the installation you have to move the following}
\Msg{* file into a directory searched by TeX:}
\Msg{*}
\Msg{*     letltxmacro.sty}
\Msg{*}
\Msg{* To produce the documentation run the file `letltxmacro.drv'}
\Msg{* through LaTeX.}
\Msg{*}
\Msg{* Happy TeXing!}
\Msg{*}
\Msg{************************************************************************}

\endbatchfile
%</install>
%<*ignore>
\fi
%</ignore>
%<*driver>
\NeedsTeXFormat{LaTeX2e}
\ProvidesFile{letltxmacro.drv}%
  [2010/09/02 v1.4 Let assignment for LaTeX macros (HO)]%
\documentclass{ltxdoc}
\usepackage{holtxdoc}[2011/11/22]
\begin{document}
  \DocInput{letltxmacro.dtx}%
\end{document}
%</driver>
% \fi
%
% \CheckSum{990}
%
% \CharacterTable
%  {Upper-case    \A\B\C\D\E\F\G\H\I\J\K\L\M\N\O\P\Q\R\S\T\U\V\W\X\Y\Z
%   Lower-case    \a\b\c\d\e\f\g\h\i\j\k\l\m\n\o\p\q\r\s\t\u\v\w\x\y\z
%   Digits        \0\1\2\3\4\5\6\7\8\9
%   Exclamation   \!     Double quote  \"     Hash (number) \#
%   Dollar        \$     Percent       \%     Ampersand     \&
%   Acute accent  \'     Left paren    \(     Right paren   \)
%   Asterisk      \*     Plus          \+     Comma         \,
%   Minus         \-     Point         \.     Solidus       \/
%   Colon         \:     Semicolon     \;     Less than     \<
%   Equals        \=     Greater than  \>     Question mark \?
%   Commercial at \@     Left bracket  \[     Backslash     \\
%   Right bracket \]     Circumflex    \^     Underscore    \_
%   Grave accent  \`     Left brace    \{     Vertical bar  \|
%   Right brace   \}     Tilde         \~}
%
% \GetFileInfo{letltxmacro.drv}
%
% \title{The \xpackage{letltxmacro} package}
% \date{2010/09/02 v1.4}
% \author{Heiko Oberdiek\\\xemail{heiko.oberdiek at googlemail.com}}
%
% \maketitle
%
% \begin{abstract}
% \TeX's \cs{let} assignment does not work for \LaTeX\ macros
% with optional arguments or for macros that are defined
% as robust macros by \cs{DeclareRobustCommand}. This package
% defines \cs{LetLtxMacro} that also takes care of the involved
% internal macros.
% \end{abstract}
%
% \tableofcontents
%
% \section{Documentation}
%
% If someone wants to redefine a macro with using the old
% meaning, then one method is \TeX's command \cs{let}:
%\begin{quote}
%\begin{verbatim}
%\newcommand{\Macro}{\typeout{Test Macro}}
%\let\SavedMacro=\Macro
%\renewcommand{\Macro}{%
%  \typeout{Begin}%
%  \SavedMacro
%  \typeout{End}%
%}
%\end{verbatim}
%\end{quote}
% However, this method fails, if \cs{Macro} is defined
% by \cs{DeclareRobustCommand} and/or has an optional argument.
% In both cases \LaTeX\ defines an additional internal macro
% that is forgotten in the simple \cs{let} assignment of
% the example above.
%
% \begin{declcs}{LetLtxMacro} \M{new macro} \M{old macro}
% \end{declcs}
% Macro \cs{LetLtxMacro} behaves similar to \TeX's \cs{let}
% assignment, but it takes care of macros that are
% defined by \cs{DeclareRobustCommand} and/or have optional
% arguments. Example:
%\begin{quote}
%\begin{verbatim}
%\DeclareRobustCommand{\Macro}[1][default]{...}
%\LetLtxMacro{\SavedMacro}{\Macro}
%\end{verbatim}
%\end{quote}
% Then macro \cs{SavedMacro} only uses internal macro names
% that are derived from \cs{SavedMacro}'s macro name. Macro \cs{Macro}
% can now be redefined without affecting \cs{SavedMacro}.
%
% \begin{declcs}{GlobalLetLtxMacro} \M{new macro} \M{old macro}
% \end{declcs}
% Like \cs{LetLtxMacro}, but the \meta{new macro} is defined globally.
% Since version 2010/09/02~v1.4.
%
% \subsection{Supported macro definition commands}
%
% \begin{quote}
%   \begin{tabular}{@{}ll@{}}
%     \cs{newcommand}, \cs{renewcommand} & latex/base\\
%     \cs{newenvironment}, \cs{renewenvironment} & latex/base\\
%     \cs{DeclareRobustCommand}& latex/base\\
%     \cs{newrobustcmd}, \cs{renewrobustcmd} & etoolbox\\
%     \cs{robustify} & etoolbox 2008/06/22 v1.6\\
%   \end{tabular}
% \end{quote}
%
% \StopEventually{
% }
%
% \section{Implementation}
%
% \subsection{Show cases}
%
% \subsubsection{\xfile{letltxmacro-showcases.tex}}
%
%    \begin{macrocode}
%<*showcases>
\NeedsTeXFormat{LaTeX2e}
\makeatletter
%    \end{macrocode}
%    \begin{macro}{\Line}
%    The result is displayed by macro \cs{Line}. The percent symbol
%    at line start allows easy grepping and inserting into the DTX
%    file.
%    \begin{macrocode}
\newcommand*{\Line}[1]{%
  \typeout{\@percentchar#1}%
}
%    \end{macrocode}
%    \end{macro}
%    \begin{macrocode}
\newcommand*{\ShowCmdName}[1]{%
  \@ifundefined{#1}{}{%
    \Line{%
      \space\space(\expandafter\string\csname#1\endcsname) = %
      (\expandafter\meaning\csname#1\endcsname)%
    }%
  }%
}
\newcommand*{\ShowCmds}[1]{%
  \ShowCmdName{#1}%
  \ShowCmdName{#1 }%
  \ShowCmdName{\\#1}%
  \ShowCmdName{\\#1 }%
}
\let\\\@backslashchar
%    \end{macrocode}
%    \begin{macro}{\ShowDef}
%    \begin{macrocode}
\newcommand*{\ShowDef}[2]{%
  \begingroup
    \Line{}%
    \newcommand*{\DefString}{#2}%
    \@onelevel@sanitize\DefString
    \Line{\DefString}%
    #2%
    \ShowCmds{#1}%
  \endgroup
}
%    \end{macrocode}
%    \end{macro}
%    \begin{macrocode}
\typeout{}
\Line{* LaTeX definitions:}
\ShowDef{cmd}{%
  \newcommand{\cmd}[2][default]{}%
}
\ShowDef{cmd}{%
  \DeclareRobustCommand{\cmd}{}%
}
\ShowDef{cmd}{%
  \DeclareRobustCommand{\cmd}[2][default]{}%
}
\typeout{}
%    \end{macrocode}
% The minimal version of package \xpackage{etoolbox} is 2008/06/12 v1.6a
% because it fixes \cs{robustify}.
%    \begin{macrocode}
\RequirePackage{etoolbox}[2008/06/12]%
\Line{}
\Line{* etoolbox's robust definitions:}
\ShowDef{cmd}{%
  \newrobustcmd{\cmd}{}%
}
\ShowDef{cmd}{%
  \newrobustcmd{\cmd}[2][default]{}%
}
\Line{}
\Line{* etoolbox's \string\robustify:}
\ShowDef{cmd}{%
  \newcommand{\cmd}[2][default]{} %
  \robustify{\cmd}%
}
\ShowDef{cmd}{%
  \DeclareRobustCommand{\cmd}{} %
  \robustify{\cmd}%
}
\ShowDef{cmd}{%
  \DeclareRobustCommand{\cmd}[2][default]{} %
  \robustify{\cmd}%
}
\typeout{}
\@@end
%</showcases>
%    \end{macrocode}
%
% \subsubsection{Result}
%
% \begingroup
%   \makeatletter
%   \let\org@verbatim\@verbatim
%   \def\@verbatim{^^A
%     \org@verbatim
%     \catcode`\~=\active
%   }^^A
%   \let~\textvisiblespace
%\begin{verbatim}
%* LaTeX definitions:
%
%\newcommand {\cmd }[2][default]{}
%  (\cmd) = (macro:->\@protected@testopt \cmd \\cmd {default})
%  (\\cmd) = (\long macro:[#1]#2->)
%
%\DeclareRobustCommand {\cmd }{}
%  (\cmd) = (macro:->\protect \cmd~ )
%  (\cmd~) = (\long macro:->)
%
%\DeclareRobustCommand {\cmd }[2][default]{}
%  (\cmd) = (macro:->\protect \cmd~ )
%  (\cmd~) = (macro:->\@protected@testopt \cmd~ \\cmd~ {default})
%  (\\cmd~) = (\long macro:[#1]#2->)
%
%* etoolbox's robust definitions:
%
%\newrobustcmd {\cmd }{}
%  (\cmd) = (\protected\long macro:->)
%
%\newrobustcmd {\cmd }[2][default]{}
%  (\cmd) = (\protected macro:->\@testopt \\cmd {default})
%  (\\cmd) = (\long macro:[#1]#2->)
%
%* etoolbox's \robustify:
%
%\newcommand {\cmd }[2][default]{} \robustify {\cmd }
%  (\cmd) = (\protected macro:->\@protected@testopt \cmd \\cmd {default})
%  (\\cmd) = (\long macro:[#1]#2->)
%
%\DeclareRobustCommand {\cmd }{} \robustify {\cmd }
%  (\cmd) = (\protected macro:->)
%
%\DeclareRobustCommand {\cmd }[2][default]{} \robustify {\cmd }
%  (\cmd) = (\protected macro:->\@protected@testopt \cmd~ \\cmd~ {default})
%  (\cmd~) = (macro:->\@protected@testopt \cmd~ \\cmd~ {default})
%  (\\cmd~) = (\long macro:[#1]#2->)
%\end{verbatim}
% \endgroup
%
% \subsection{Package}
%
%    \begin{macrocode}
%<*package>
%    \end{macrocode}
%
% \subsubsection{Catcodes and identification}
%
%    \begin{macrocode}
\begingroup\catcode61\catcode48\catcode32=10\relax%
  \catcode13=5 % ^^M
  \endlinechar=13 %
  \catcode123=1 % {
  \catcode125=2 % }
  \catcode64=11 % @
  \def\x{\endgroup
    \expandafter\edef\csname llm@AtEnd\endcsname{%
      \endlinechar=\the\endlinechar\relax
      \catcode13=\the\catcode13\relax
      \catcode32=\the\catcode32\relax
      \catcode35=\the\catcode35\relax
      \catcode61=\the\catcode61\relax
      \catcode64=\the\catcode64\relax
      \catcode123=\the\catcode123\relax
      \catcode125=\the\catcode125\relax
    }%
  }%
\x\catcode61\catcode48\catcode32=10\relax%
\catcode13=5 % ^^M
\endlinechar=13 %
\catcode35=6 % #
\catcode64=11 % @
\catcode123=1 % {
\catcode125=2 % }
\def\TMP@EnsureCode#1#2{%
  \edef\llm@AtEnd{%
    \llm@AtEnd
    \catcode#1=\the\catcode#1\relax
  }%
  \catcode#1=#2\relax
}
\TMP@EnsureCode{40}{12}% (
\TMP@EnsureCode{41}{12}% )
\TMP@EnsureCode{42}{12}% *
\TMP@EnsureCode{45}{12}% -
\TMP@EnsureCode{46}{12}% .
\TMP@EnsureCode{47}{12}% /
\TMP@EnsureCode{58}{12}% :
\TMP@EnsureCode{62}{12}% >
\TMP@EnsureCode{91}{12}% [
\TMP@EnsureCode{93}{12}% ]
\edef\llm@AtEnd{%
  \llm@AtEnd
  \escapechar\the\escapechar\relax
  \noexpand\endinput
}
\escapechar=92 % `\\
%    \end{macrocode}
%
%    Package identification.
%    \begin{macrocode}
\NeedsTeXFormat{LaTeX2e}
\ProvidesPackage{letltxmacro}%
  [2010/09/02 v1.4 Let assignment for LaTeX macros (HO)]
%    \end{macrocode}
%
% \subsubsection{Main macros}
%
%    \begin{macro}{\LetLtxMacro}
%    \begin{macrocode}
\newcommand*{\LetLtxMacro}{%
  \llm@ModeLetLtxMacro{}%
}
%    \end{macrocode}
%    \end{macro}
%    \begin{macro}{\GlobalLetLtxMacro}
%    \begin{macrocode}
\newcommand*{\GlobalLetLtxMacro}{%
  \llm@ModeLetLtxMacro\global
}
%    \end{macrocode}
%    \end{macro}
%
%    \begin{macro}{\llm@ModeLetLtxMacro}
%    \begin{macrocode}
\newcommand*{\llm@ModeLetLtxMacro}[3]{%
  \edef\llm@escapechar{\the\escapechar}%
  \escapechar=-1 %
  \edef\reserved@a{%
    \noexpand\protect
    \expandafter\noexpand
    \csname\string#3 \endcsname
  }%
  \ifx\reserved@a#3\relax
    #1\edef#2{%
      \noexpand\protect
      \expandafter\noexpand
      \csname\string#2 \endcsname
    }%
    #1\expandafter\let
    \csname\string#2 \expandafter\endcsname
    \csname\string#3 \endcsname
    \expandafter\llm@LetLtxMacro
        \csname\string#2 \expandafter\endcsname
        \csname\string#3 \endcsname{#1}%
  \else
    \llm@LetLtxMacro{#2}{#3}{#1}%
  \fi
  \escapechar=\llm@escapechar\relax
}
%    \end{macrocode}
%    \end{macro}
%    \begin{macro}{\llm@LetLtxMacro}
%    \begin{macrocode}
\def\llm@LetLtxMacro#1#2#3{%
  \escapechar=92 %
  \expandafter\llm@CheckParams\meaning#2:->\@nil{%
    \begingroup
      \def\@protected@testopt{%
        \expandafter\@testopt\@gobble
      }%
      \def\@testopt##1##2{%
        \toks@={##2}%
      }%
      \let\llm@testopt\@empty
      \edef\x{%
        \noexpand\@protected@testopt
        \noexpand#2%
        \expandafter\noexpand\csname\string#2\endcsname
      }%
      \expandafter\expandafter\expandafter\def
      \expandafter\expandafter\expandafter\y
      \expandafter\expandafter\expandafter{%
        \expandafter\llm@CarThree#2{}{}{}\llm@nil
      }%
      \ifx\x\y
        #2%
        \def\llm@testopt{%
          \noexpand\@protected@testopt
          \noexpand#1%
        }%
      \else
        \edef\x{%
          \noexpand\@testopt
          \expandafter\noexpand
          \csname\string#2\endcsname
        }%
        \expandafter\expandafter\expandafter\def
        \expandafter\expandafter\expandafter\y
        \expandafter\expandafter\expandafter{%
          \expandafter\llm@CarTwo#2{}{}\llm@nil
        }%
        \ifx\x\y
          #2%
          \def\llm@testopt{%
            \noexpand\@testopt
          }%
        \fi
      \fi
      \ifx\llm@testopt\@empty
      \else
        \llm@protected\xdef\llm@GlobalTemp{%
          \llm@testopt
          \expandafter\noexpand
          \csname\string#1\endcsname
          {\the\toks@}%
        }%
      \fi
    \expandafter\endgroup\ifx\llm@testopt\@empty
      #3\let#1=#2\relax
    \else
      #3\let#1=\llm@GlobalTemp
      #3\expandafter\let
          \csname\string#1\expandafter\endcsname
          \csname\string#2\endcsname
    \fi
  }{%
    #3\let#1=#2\relax
  }%
}
%    \end{macrocode}
%    \end{macro}
%    \begin{macro}{\llm@CheckParams}
%    \begin{macrocode}
\def\llm@CheckParams#1:->#2\@nil{%
  \begingroup
    \def\x{#1}%
  \ifx\x\llm@macro
    \endgroup
    \def\llm@protected{}%
    \expandafter\@firstoftwo
  \else
    \ifx\x\llm@protectedmacro
      \endgroup
      \def\llm@protected{\protected}%
      \expandafter\expandafter\expandafter\@firstoftwo
    \else
      \endgroup
      \expandafter\expandafter\expandafter\@secondoftwo
    \fi
  \fi
}
%    \end{macrocode}
%    \end{macro}
%    \begin{macro}{\llm@macro}
%    \begin{macrocode}
\def\llm@macro{macro}
\@onelevel@sanitize\llm@macro
%    \end{macrocode}
%    \end{macro}
%    \begin{macro}{\llm@protectedmacro}
%    \begin{macrocode}
\def\llm@protectedmacro{\protected macro}
\@onelevel@sanitize\llm@protectedmacro
%    \end{macrocode}
%    \end{macro}
%    \begin{macro}{\llm@CarThree}
%    \begin{macrocode}
\def\llm@CarThree#1#2#3#4\llm@nil{#1#2#3}%
%    \end{macrocode}
%    \end{macro}
%    \begin{macro}{\llm@CarTwo}
%    \begin{macrocode}
\def\llm@CarTwo#1#2#3\llm@nil{#1#2}%
%    \end{macrocode}
%    \end{macro}
%
%    \begin{macrocode}
\llm@AtEnd%
%</package>
%    \end{macrocode}
%
% \section{Test}
%
% \subsection{Catcode checks for loading}
%
%    \begin{macrocode}
%<*test1>
%    \end{macrocode}
%    \begin{macrocode}
\NeedsTeXFormat{LaTeX2e}
\documentclass{minimal}
\makeatletter
\def\RestoreCatcodes{}
\count@=0 %
\loop
  \edef\RestoreCatcodes{%
    \RestoreCatcodes
    \catcode\the\count@=\the\catcode\count@\relax
  }%
\ifnum\count@<255 %
  \advance\count@\@ne
\repeat

\def\RangeCatcodeInvalid#1#2{%
  \count@=#1\relax
  \loop
    \catcode\count@=15 %
  \ifnum\count@<#2\relax
    \advance\count@\@ne
  \repeat
}
\def\Test{%
  \RangeCatcodeInvalid{0}{47}%
  \RangeCatcodeInvalid{58}{64}%
  \RangeCatcodeInvalid{91}{96}%
  \RangeCatcodeInvalid{123}{127}%
  \catcode`\@=12 %
  \catcode`\\=0 %
  \catcode`\{=1 %
  \catcode`\}=2 %
  \catcode`\#=6 %
  \catcode`\[=12 %
  \catcode`\]=12 %
  \catcode`\%=14 %
  \catcode`\ =10 %
  \catcode13=5 %
  \RequirePackage{letltxmacro}[2010/09/02]\relax
  \RestoreCatcodes
}
\Test
\csname @@end\endcsname
\end
%    \end{macrocode}
%    \begin{macrocode}
%</test1>
%    \end{macrocode}
%
% \subsection{Package tests}
%
%    \begin{macrocode}
%<*test2>
%    \end{macrocode}
%    \begin{macrocode}
\NeedsTeXFormat{LaTeX2e}
\nofiles
\documentclass{minimal}

\usepackage{letltxmacro}[2010/09/02]

\usepackage{qstest}
\IncludeTests{*}
\LogTests{log}{*}{*}

\makeatletter

\def\TestDef#1{%
  \begingroup
    \@makeother\\%
    \@makeother\ %
    \expandafter\@TestDef\csname #1\endcsname
}
\def\@TestDef#1#2{%
    \edef\a{\expandafter\strip@prefix\meaning#1}%
    \edef\b{\detokenize{#2}}%
    \ifx\a\b
    \else
      \typeout{1 [\a]}%
      \typeout{2 [\b]}%
    \fi
  \endgroup
  \Expect*{\expandafter\strip@prefix\meaning#1}*{\detokenize{#2}}%
}
\def\TestEquals#1#2{%
  \Expect*{%
    \expandafter\ifx\csname#1\expandafter\endcsname
                    \csname#2\endcsname
      equals%
    \else
      wrong%
    \fi
  }{equals}%
}
\def\SaveA{%
  \let\SavedA\TestA
  \expandafter\let\csname\string\SavedA\expandafter\endcsname
                  \csname\string\TestA\endcsname
  \expandafter\let\csname SavedA \expandafter\endcsname
                  \csname TestA \endcsname
}
\def\CheckA{%
  \TestEquals{SavedA}{TestA}%
  \TestEquals{\string\SavedA}{\string\TestA}%
  \TestEquals{SavedA }{TestA }%
}

\begin{qstest}{robust}{robust}
  \begingroup
    \DeclareRobustCommand*{\TestA}{Test}%
    \SaveA
    \TestDef{TestA}{\protect \TestA  }%
    \CheckA
    \DeclareRobustCommand{\TestA}{Test}%
    \SaveA
    \TestDef{TestA}{\protect \TestA  }%
    \LetLtxMacro\TestB\TestA
    \TestDef{TestB}{\protect \TestB  }%
    \TestEquals{TestB }{TestA }%
    \CheckA
    \GlobalLetLtxMacro\TestC\TestA
    \CheckA
  \endgroup
  \TestDef{TestC}{\protect \TestC  }%
  \DeclareRobustCommand{\TestA}{Test}%
  \TestEquals{TestC }{TestA }%
\end{qstest}

\begin{qstest}{default}{default}
  \begingroup
    \newcommand{\TestA}[1][\relax default \empty]{TestA #1}%
    \SaveA
    \TestDef
    {TestA}{\@protected@testopt \TestA \\TestA {\relax default \empty }}%
    \LetLtxMacro\TestB\TestA
    \TestDef
    {TestB}{\@protected@testopt \TestB \\TestB {\relax default \empty }}%
    \CheckA
    \TestEquals{\string\TestB}{\string\TestA}%
    \GlobalLetLtxMacro\TestC\TestA
    \CheckA
  \endgroup
  \TestDef
  {TestC}{\@protected@testopt \TestC \\TestC {\relax default \empty }}%
  \newcommand{\TestA}[1][\relax default \empty]{TestA #1}%
  \TestEquals{\string\TestC}{\string\TestA}%
\end{qstest}

\begin{qstest}{robustdefault}{robustdefault}
  \begingroup
    \DeclareRobustCommand{\TestA}[1][\default]{TestA}%
    \SaveA
    \LetLtxMacro\TestB\TestA
    \TestDef{TestB}{\protect \TestB  }%
    \TestDef{TestB }{\@protected@testopt \TestB  \\TestB  {\default }}%
    \CheckA
    \GlobalLetLtxMacro\TestC\TestA
    \CheckA
  \endgroup
  \TestDef{TestC}{\protect \TestC  }%
  \TestDef{TestC }{\@protected@testopt \TestC  \\TestC  {\default }}%
\end{qstest}

\begin{qstest}{plain}{plain}
  \LetLtxMacro\NewRelax\relax
  \Expect*{\meaning\relax}*{\string\relax}%
  \Expect*{\meaning\NewRelax}*{\string\relax}%
  \LetLtxMacro\NewHbox\hbox
  \Expect*{\meaning\hbox}*{\string\hbox}%
  \Expect*{\meaning\NewHbox}*{\string\hbox}%
  \LetLtxMacro\NewEmpty\empty
  \Expect*{\meaning\empty}{macro:->}%
  \Expect*{\meaning\NewEmpty}{macro:->}%
  \def\TestA{\iffalse}%
  \LetLtxMacro\TestB\TestA
  \TestDef{TestB}{\iffalse }%
  \TestEquals{TestA}{TestB}%
\end{qstest}
%    \end{macrocode}
%
%    \begin{macrocode}
\usepackage{etoolbox}[2008/06/22]

\begin{qstest}{oldnewrobustcmd}{oldnewrobustcmd}
  \begingroup
    \protected\edef\TestA{%
      \noexpand\@protected@testopt
      \noexpand\TestA
      \expandafter\noexpand\csname\string\TestA\endcsname
      {\noexpand\default}%
    }%
    \expandafter\protected\expandafter
    \def\csname\string\TestA\endcsname{TestA}%
    \SaveA
    \LetLtxMacro\TestB\TestA
    \TestDef{TestB}{\@protected@testopt \TestB \\TestB {\default }}%
    \protected\edef\TestC{%
      \noexpand\@protected@testopt
      \noexpand\TestB
      \expandafter\noexpand\csname\string\TestB\endcsname
      {\noexpand\default}%
    }%
    \TestEquals{TestB}{TestC}%
    \CheckA
    \GlobalLetLtxMacro\TestD\TestA
    \CheckA
  \endgroup
  \TestDef{TestD}{\@protected@testopt \TestD \\TestD {\default }}%
  \protected\edef\TestC{%
    \noexpand\@protected@testopt
    \noexpand\TestD
    \expandafter\noexpand\csname\string\TestD\endcsname
    {\noexpand\default}%
  }%
  \TestEquals{TestD}{TestC}%
\end{qstest}

\begin{qstest}{newrobustcmd}{newrobustcmd}
  \begingroup
    \newrobustcmd{\TestA}[1][\default]{TestA}%
    \SaveA
    \LetLtxMacro\TestB\TestA
    \TestDef{TestB}{\@testopt \\TestB {\default }}%
    \protected\edef\TestC{%
      \noexpand\@testopt
      \expandafter\noexpand\csname\string\TestB\endcsname
      {\noexpand\default}%
    }%
    \TestEquals{TestB}{TestC}%
    \CheckA
    \GlobalLetLtxMacro\TestD\TestA
    \CheckA
  \endgroup
  \TestDef{TestD}{\@testopt \\TestD {\default }}%
  \protected\edef\TestC{%
    \noexpand\@testopt
    \expandafter\noexpand\csname\string\TestD\endcsname
    {\noexpand\default}%
  }%
  \TestEquals{TestD}{TestC}%
\end{qstest}

\begin{qstest}{robustifyopt}{robustifyopt}
  \newcommand{\TestA}[2][\default]{}%
  \TestDef{TestA}{\@protected@testopt \TestA \\TestA {\default }}%
  \robustify\TestA
  \TestDef{TestA}{\@protected@testopt \TestA \\TestA {\default }}%
  \protected\edef\TestC{%
    \noexpand\@protected@testopt
    \noexpand\TestA
    \expandafter\noexpand\csname\string\TestA\endcsname
    {\noexpand\default}%
  }%
  \TestEquals{TestA}{TestC}%
  \SaveA
  \LetLtxMacro\TestB\TestA
  \TestDef{TestB}{\@protected@testopt \TestB \\TestB {\default }}%
  \protected\edef\TestC{%
    \noexpand\@protected@testopt
    \noexpand\TestB
    \expandafter\noexpand\csname\string\TestB\endcsname
    {\noexpand\default}%
  }%
  \TestEquals{TestB}{TestC}%
  \CheckA
\end{qstest}

\begin{qstest}{robustifydeclare}{robustifydeclare}
  \DeclareRobustCommand\TestA{\iffalse}%
  \robustify\TestA
  \SaveA
  \LetLtxMacro\TestB\TestA
  \TestDef{TestB}{\iffalse }%
  % before etoolbox 2008/06/22: \protected\long\def
  % since etoolbox 2008/06/22: \protected\def
  \protected\def\TestC{\iffalse}%
  \TestEquals{TestB}{TestC}%
  \CheckA
\end{qstest}

\begin{qstest}{robustifydeclarelong}{robustifydeclarelong}
  \DeclareRobustCommand\TestA[1]{\iffalse}%
  \robustify\TestA
  \SaveA
  \LetLtxMacro\TestB\TestA
  \TestDef{TestB}{\iffalse }%
  % before etoolbox 2008/06/22: \protected\long\def
  % since etoolbox 2008/06/22: \protected\def
  \protected\long\def\TestC#1{\iffalse}%
  \TestEquals{TestB}{TestC}%
  \CheckA
\end{qstest}
%    \end{macrocode}
%
%    \begin{macrocode}
\csname @@end\endcsname
%    \end{macrocode}
%    \begin{macrocode}
%</test2>
%    \end{macrocode}
%
% \section{Installation}
%
% \subsection{Download}
%
% \paragraph{Package.} This package is available on
% CTAN\footnote{\url{ftp://ftp.ctan.org/tex-archive/}}:
% \begin{description}
% \item[\CTAN{macros/latex/contrib/oberdiek/letltxmacro.dtx}] The source file.
% \item[\CTAN{macros/latex/contrib/oberdiek/letltxmacro.pdf}] Documentation.
% \end{description}
%
%
% \paragraph{Bundle.} All the packages of the bundle `oberdiek'
% are also available in a TDS compliant ZIP archive. There
% the packages are already unpacked and the documentation files
% are generated. The files and directories obey the TDS standard.
% \begin{description}
% \item[\CTAN{install/macros/latex/contrib/oberdiek.tds.zip}]
% \end{description}
% \emph{TDS} refers to the standard ``A Directory Structure
% for \TeX\ Files'' (\CTAN{tds/tds.pdf}). Directories
% with \xfile{texmf} in their name are usually organized this way.
%
% \subsection{Bundle installation}
%
% \paragraph{Unpacking.} Unpack the \xfile{oberdiek.tds.zip} in the
% TDS tree (also known as \xfile{texmf} tree) of your choice.
% Example (linux):
% \begin{quote}
%   |unzip oberdiek.tds.zip -d ~/texmf|
% \end{quote}
%
% \paragraph{Script installation.}
% Check the directory \xfile{TDS:scripts/oberdiek/} for
% scripts that need further installation steps.
% Package \xpackage{attachfile2} comes with the Perl script
% \xfile{pdfatfi.pl} that should be installed in such a way
% that it can be called as \texttt{pdfatfi}.
% Example (linux):
% \begin{quote}
%   |chmod +x scripts/oberdiek/pdfatfi.pl|\\
%   |cp scripts/oberdiek/pdfatfi.pl /usr/local/bin/|
% \end{quote}
%
% \subsection{Package installation}
%
% \paragraph{Unpacking.} The \xfile{.dtx} file is a self-extracting
% \docstrip\ archive. The files are extracted by running the
% \xfile{.dtx} through \plainTeX:
% \begin{quote}
%   \verb|tex letltxmacro.dtx|
% \end{quote}
%
% \paragraph{TDS.} Now the different files must be moved into
% the different directories in your installation TDS tree
% (also known as \xfile{texmf} tree):
% \begin{quote}
% \def\t{^^A
% \begin{tabular}{@{}>{\ttfamily}l@{ $\rightarrow$ }>{\ttfamily}l@{}}
%   letltxmacro.sty & tex/latex/oberdiek/letltxmacro.sty\\
%   letltxmacro.pdf & doc/latex/oberdiek/letltxmacro.pdf\\
%   letltxmacro-showcases.tex & doc/latex/oberdiek/letltxmacro-showcases.tex\\
%   test/letltxmacro-test1.tex & doc/latex/oberdiek/test/letltxmacro-test1.tex\\
%   test/letltxmacro-test2.tex & doc/latex/oberdiek/test/letltxmacro-test2.tex\\
%   letltxmacro.dtx & source/latex/oberdiek/letltxmacro.dtx\\
% \end{tabular}^^A
% }^^A
% \sbox0{\t}^^A
% \ifdim\wd0>\linewidth
%   \begingroup
%     \advance\linewidth by\leftmargin
%     \advance\linewidth by\rightmargin
%   \edef\x{\endgroup
%     \def\noexpand\lw{\the\linewidth}^^A
%   }\x
%   \def\lwbox{^^A
%     \leavevmode
%     \hbox to \linewidth{^^A
%       \kern-\leftmargin\relax
%       \hss
%       \usebox0
%       \hss
%       \kern-\rightmargin\relax
%     }^^A
%   }^^A
%   \ifdim\wd0>\lw
%     \sbox0{\small\t}^^A
%     \ifdim\wd0>\linewidth
%       \ifdim\wd0>\lw
%         \sbox0{\footnotesize\t}^^A
%         \ifdim\wd0>\linewidth
%           \ifdim\wd0>\lw
%             \sbox0{\scriptsize\t}^^A
%             \ifdim\wd0>\linewidth
%               \ifdim\wd0>\lw
%                 \sbox0{\tiny\t}^^A
%                 \ifdim\wd0>\linewidth
%                   \lwbox
%                 \else
%                   \usebox0
%                 \fi
%               \else
%                 \lwbox
%               \fi
%             \else
%               \usebox0
%             \fi
%           \else
%             \lwbox
%           \fi
%         \else
%           \usebox0
%         \fi
%       \else
%         \lwbox
%       \fi
%     \else
%       \usebox0
%     \fi
%   \else
%     \lwbox
%   \fi
% \else
%   \usebox0
% \fi
% \end{quote}
% If you have a \xfile{docstrip.cfg} that configures and enables \docstrip's
% TDS installing feature, then some files can already be in the right
% place, see the documentation of \docstrip.
%
% \subsection{Refresh file name databases}
%
% If your \TeX~distribution
% (\teTeX, \mikTeX, \dots) relies on file name databases, you must refresh
% these. For example, \teTeX\ users run \verb|texhash| or
% \verb|mktexlsr|.
%
% \subsection{Some details for the interested}
%
% \paragraph{Attached source.}
%
% The PDF documentation on CTAN also includes the
% \xfile{.dtx} source file. It can be extracted by
% AcrobatReader 6 or higher. Another option is \textsf{pdftk},
% e.g. unpack the file into the current directory:
% \begin{quote}
%   \verb|pdftk letltxmacro.pdf unpack_files output .|
% \end{quote}
%
% \paragraph{Unpacking with \LaTeX.}
% The \xfile{.dtx} chooses its action depending on the format:
% \begin{description}
% \item[\plainTeX:] Run \docstrip\ and extract the files.
% \item[\LaTeX:] Generate the documentation.
% \end{description}
% If you insist on using \LaTeX\ for \docstrip\ (really,
% \docstrip\ does not need \LaTeX), then inform the autodetect routine
% about your intention:
% \begin{quote}
%   \verb|latex \let\install=y\input{letltxmacro.dtx}|
% \end{quote}
% Do not forget to quote the argument according to the demands
% of your shell.
%
% \paragraph{Generating the documentation.}
% You can use both the \xfile{.dtx} or the \xfile{.drv} to generate
% the documentation. The process can be configured by the
% configuration file \xfile{ltxdoc.cfg}. For instance, put this
% line into this file, if you want to have A4 as paper format:
% \begin{quote}
%   \verb|\PassOptionsToClass{a4paper}{article}|
% \end{quote}
% An example follows how to generate the
% documentation with pdf\LaTeX:
% \begin{quote}
%\begin{verbatim}
%pdflatex letltxmacro.dtx
%makeindex -s gind.ist letltxmacro.idx
%pdflatex letltxmacro.dtx
%makeindex -s gind.ist letltxmacro.idx
%pdflatex letltxmacro.dtx
%\end{verbatim}
% \end{quote}
%
% \section{Catalogue}
%
% The following XML file can be used as source for the
% \href{http://mirror.ctan.org/help/Catalogue/catalogue.html}{\TeX\ Catalogue}.
% The elements \texttt{caption} and \texttt{description} are imported
% from the original XML file from the Catalogue.
% The name of the XML file in the Catalogue is \xfile{letltxmacro.xml}.
%    \begin{macrocode}
%<*catalogue>
<?xml version='1.0' encoding='us-ascii'?>
<!DOCTYPE entry SYSTEM 'catalogue.dtd'>
<entry datestamp='$Date$' modifier='$Author$' id='letltxmacro'>
  <name>letltxmacro</name>
  <caption>Let assignment for LaTeX macros.</caption>
  <authorref id='auth:oberdiek'/>
  <copyright owner='Heiko Oberdiek' year='2008,2010'/>
  <license type='lppl1.3'/>
  <version number='1.4'/>
  <description>
    TeX&#x2019;s <tt>\let</tt> assignment does not work for LaTeX macros
    with optional arguments or for macros that are defined
    as robust macros by <tt>\DeclareRobustCommand</tt>.  This package
    defines <tt>\LetLtxMacro</tt> that also takes care of the involved
    internal macros.
    <p/>
    The package is part of the <xref refid='oberdiek'>oberdiek</xref>
    bundle.
  </description>
  <documentation details='Package documentation'
      href='ctan:/macros/latex/contrib/oberdiek/letltxmacro.pdf'/>
  <ctan file='true' path='/macros/latex/contrib/oberdiek/letltxmacro.dtx'/>
  <miktex location='oberdiek'/>
  <texlive location='oberdiek'/>
  <install path='/macros/latex/contrib/oberdiek/oberdiek.tds.zip'/>
</entry>
%</catalogue>
%    \end{macrocode}
%
% \begin{History}
%   \begin{Version}{2008/06/09 v1.0}
%   \item
%     First version.
%   \end{Version}
%   \begin{Version}{2008/06/12 v1.1}
%   \item
%     Support for \xpackage{etoolbox}'s \cs{newrobustcmd} added.
%   \end{Version}
%   \begin{Version}{2008/06/13 v1.2}
%   \item
%     Support for \xpackage{etoolbox}'s \cs{robustify} added.
%   \end{Version}
%   \begin{Version}{2008/06/24 v1.3}
%   \item
%     Test file adapted for etoolbox 2008/06/22 v1.6.
%   \end{Version}
%   \begin{Version}{2010/09/02 v1.4}
%   \item
%     \cs{GlobalLetLtxMacro} added.
%   \end{Version}
% \end{History}
%
% \PrintIndex
%
% \Finale
\endinput
|
% \end{quote}
% Do not forget to quote the argument according to the demands
% of your shell.
%
% \paragraph{Generating the documentation.}
% You can use both the \xfile{.dtx} or the \xfile{.drv} to generate
% the documentation. The process can be configured by the
% configuration file \xfile{ltxdoc.cfg}. For instance, put this
% line into this file, if you want to have A4 as paper format:
% \begin{quote}
%   \verb|\PassOptionsToClass{a4paper}{article}|
% \end{quote}
% An example follows how to generate the
% documentation with pdf\LaTeX:
% \begin{quote}
%\begin{verbatim}
%pdflatex letltxmacro.dtx
%makeindex -s gind.ist letltxmacro.idx
%pdflatex letltxmacro.dtx
%makeindex -s gind.ist letltxmacro.idx
%pdflatex letltxmacro.dtx
%\end{verbatim}
% \end{quote}
%
% \section{Catalogue}
%
% The following XML file can be used as source for the
% \href{http://mirror.ctan.org/help/Catalogue/catalogue.html}{\TeX\ Catalogue}.
% The elements \texttt{caption} and \texttt{description} are imported
% from the original XML file from the Catalogue.
% The name of the XML file in the Catalogue is \xfile{letltxmacro.xml}.
%    \begin{macrocode}
%<*catalogue>
<?xml version='1.0' encoding='us-ascii'?>
<!DOCTYPE entry SYSTEM 'catalogue.dtd'>
<entry datestamp='$Date$' modifier='$Author$' id='letltxmacro'>
  <name>letltxmacro</name>
  <caption>Let assignment for LaTeX macros.</caption>
  <authorref id='auth:oberdiek'/>
  <copyright owner='Heiko Oberdiek' year='2008,2010'/>
  <license type='lppl1.3'/>
  <version number='1.4'/>
  <description>
    TeX&#x2019;s <tt>\let</tt> assignment does not work for LaTeX macros
    with optional arguments or for macros that are defined
    as robust macros by <tt>\DeclareRobustCommand</tt>.  This package
    defines <tt>\LetLtxMacro</tt> that also takes care of the involved
    internal macros.
    <p/>
    The package is part of the <xref refid='oberdiek'>oberdiek</xref>
    bundle.
  </description>
  <documentation details='Package documentation'
      href='ctan:/macros/latex/contrib/oberdiek/letltxmacro.pdf'/>
  <ctan file='true' path='/macros/latex/contrib/oberdiek/letltxmacro.dtx'/>
  <miktex location='oberdiek'/>
  <texlive location='oberdiek'/>
  <install path='/macros/latex/contrib/oberdiek/oberdiek.tds.zip'/>
</entry>
%</catalogue>
%    \end{macrocode}
%
% \begin{History}
%   \begin{Version}{2008/06/09 v1.0}
%   \item
%     First version.
%   \end{Version}
%   \begin{Version}{2008/06/12 v1.1}
%   \item
%     Support for \xpackage{etoolbox}'s \cs{newrobustcmd} added.
%   \end{Version}
%   \begin{Version}{2008/06/13 v1.2}
%   \item
%     Support for \xpackage{etoolbox}'s \cs{robustify} added.
%   \end{Version}
%   \begin{Version}{2008/06/24 v1.3}
%   \item
%     Test file adapted for etoolbox 2008/06/22 v1.6.
%   \end{Version}
%   \begin{Version}{2010/09/02 v1.4}
%   \item
%     \cs{GlobalLetLtxMacro} added.
%   \end{Version}
% \end{History}
%
% \PrintIndex
%
% \Finale
\endinput
|
% \end{quote}
% Do not forget to quote the argument according to the demands
% of your shell.
%
% \paragraph{Generating the documentation.}
% You can use both the \xfile{.dtx} or the \xfile{.drv} to generate
% the documentation. The process can be configured by the
% configuration file \xfile{ltxdoc.cfg}. For instance, put this
% line into this file, if you want to have A4 as paper format:
% \begin{quote}
%   \verb|\PassOptionsToClass{a4paper}{article}|
% \end{quote}
% An example follows how to generate the
% documentation with pdf\LaTeX:
% \begin{quote}
%\begin{verbatim}
%pdflatex letltxmacro.dtx
%makeindex -s gind.ist letltxmacro.idx
%pdflatex letltxmacro.dtx
%makeindex -s gind.ist letltxmacro.idx
%pdflatex letltxmacro.dtx
%\end{verbatim}
% \end{quote}
%
% \section{Catalogue}
%
% The following XML file can be used as source for the
% \href{http://mirror.ctan.org/help/Catalogue/catalogue.html}{\TeX\ Catalogue}.
% The elements \texttt{caption} and \texttt{description} are imported
% from the original XML file from the Catalogue.
% The name of the XML file in the Catalogue is \xfile{letltxmacro.xml}.
%    \begin{macrocode}
%<*catalogue>
<?xml version='1.0' encoding='us-ascii'?>
<!DOCTYPE entry SYSTEM 'catalogue.dtd'>
<entry datestamp='$Date$' modifier='$Author$' id='letltxmacro'>
  <name>letltxmacro</name>
  <caption>Let assignment for LaTeX macros.</caption>
  <authorref id='auth:oberdiek'/>
  <copyright owner='Heiko Oberdiek' year='2008,2010'/>
  <license type='lppl1.3'/>
  <version number='1.4'/>
  <description>
    TeX&#x2019;s <tt>\let</tt> assignment does not work for LaTeX macros
    with optional arguments or for macros that are defined
    as robust macros by <tt>\DeclareRobustCommand</tt>.  This package
    defines <tt>\LetLtxMacro</tt> that also takes care of the involved
    internal macros.
    <p/>
    The package is part of the <xref refid='oberdiek'>oberdiek</xref>
    bundle.
  </description>
  <documentation details='Package documentation'
      href='ctan:/macros/latex/contrib/oberdiek/letltxmacro.pdf'/>
  <ctan file='true' path='/macros/latex/contrib/oberdiek/letltxmacro.dtx'/>
  <miktex location='oberdiek'/>
  <texlive location='oberdiek'/>
  <install path='/macros/latex/contrib/oberdiek/oberdiek.tds.zip'/>
</entry>
%</catalogue>
%    \end{macrocode}
%
% \begin{History}
%   \begin{Version}{2008/06/09 v1.0}
%   \item
%     First version.
%   \end{Version}
%   \begin{Version}{2008/06/12 v1.1}
%   \item
%     Support for \xpackage{etoolbox}'s \cs{newrobustcmd} added.
%   \end{Version}
%   \begin{Version}{2008/06/13 v1.2}
%   \item
%     Support for \xpackage{etoolbox}'s \cs{robustify} added.
%   \end{Version}
%   \begin{Version}{2008/06/24 v1.3}
%   \item
%     Test file adapted for etoolbox 2008/06/22 v1.6.
%   \end{Version}
%   \begin{Version}{2010/09/02 v1.4}
%   \item
%     \cs{GlobalLetLtxMacro} added.
%   \end{Version}
% \end{History}
%
% \PrintIndex
%
% \Finale
\endinput
|
% \end{quote}
% Do not forget to quote the argument according to the demands
% of your shell.
%
% \paragraph{Generating the documentation.}
% You can use both the \xfile{.dtx} or the \xfile{.drv} to generate
% the documentation. The process can be configured by the
% configuration file \xfile{ltxdoc.cfg}. For instance, put this
% line into this file, if you want to have A4 as paper format:
% \begin{quote}
%   \verb|\PassOptionsToClass{a4paper}{article}|
% \end{quote}
% An example follows how to generate the
% documentation with pdf\LaTeX:
% \begin{quote}
%\begin{verbatim}
%pdflatex letltxmacro.dtx
%makeindex -s gind.ist letltxmacro.idx
%pdflatex letltxmacro.dtx
%makeindex -s gind.ist letltxmacro.idx
%pdflatex letltxmacro.dtx
%\end{verbatim}
% \end{quote}
%
% \section{Catalogue}
%
% The following XML file can be used as source for the
% \href{http://mirror.ctan.org/help/Catalogue/catalogue.html}{\TeX\ Catalogue}.
% The elements \texttt{caption} and \texttt{description} are imported
% from the original XML file from the Catalogue.
% The name of the XML file in the Catalogue is \xfile{letltxmacro.xml}.
%    \begin{macrocode}
%<*catalogue>
<?xml version='1.0' encoding='us-ascii'?>
<!DOCTYPE entry SYSTEM 'catalogue.dtd'>
<entry datestamp='$Date$' modifier='$Author$' id='letltxmacro'>
  <name>letltxmacro</name>
  <caption>Let assignment for LaTeX macros.</caption>
  <authorref id='auth:oberdiek'/>
  <copyright owner='Heiko Oberdiek' year='2008,2010'/>
  <license type='lppl1.3'/>
  <version number='1.4'/>
  <description>
    TeX&#x2019;s <tt>\let</tt> assignment does not work for LaTeX macros
    with optional arguments or for macros that are defined
    as robust macros by <tt>\DeclareRobustCommand</tt>.  This package
    defines <tt>\LetLtxMacro</tt> that also takes care of the involved
    internal macros.
    <p/>
    The package is part of the <xref refid='oberdiek'>oberdiek</xref>
    bundle.
  </description>
  <documentation details='Package documentation'
      href='ctan:/macros/latex/contrib/oberdiek/letltxmacro.pdf'/>
  <ctan file='true' path='/macros/latex/contrib/oberdiek/letltxmacro.dtx'/>
  <miktex location='oberdiek'/>
  <texlive location='oberdiek'/>
  <install path='/macros/latex/contrib/oberdiek/oberdiek.tds.zip'/>
</entry>
%</catalogue>
%    \end{macrocode}
%
% \begin{History}
%   \begin{Version}{2008/06/09 v1.0}
%   \item
%     First version.
%   \end{Version}
%   \begin{Version}{2008/06/12 v1.1}
%   \item
%     Support for \xpackage{etoolbox}'s \cs{newrobustcmd} added.
%   \end{Version}
%   \begin{Version}{2008/06/13 v1.2}
%   \item
%     Support for \xpackage{etoolbox}'s \cs{robustify} added.
%   \end{Version}
%   \begin{Version}{2008/06/24 v1.3}
%   \item
%     Test file adapted for etoolbox 2008/06/22 v1.6.
%   \end{Version}
%   \begin{Version}{2010/09/02 v1.4}
%   \item
%     \cs{GlobalLetLtxMacro} added.
%   \end{Version}
% \end{History}
%
% \PrintIndex
%
% \Finale
\endinput
